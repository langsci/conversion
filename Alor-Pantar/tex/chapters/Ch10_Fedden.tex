
\clearpage\setcounter{page}{1}\pagestyle{Standard}
{\centering\itshape
Chapter 10
\par}

{\centering
Variation in pronominal indexing: lexical stipulation vs. referential properties
\par}

{\centering\itshape
Sebastian Fedden, Dunstan Brown, Franti\v{s}ek Kratochv\'il, 
\par}

{\centering\itshape
Laura C. Robinson and Antoinette Schapper
\par}

\clearpage{\centering\itshape
Chapter 10
\par}

{\centering\bfseries
Variation in pronominal indexing: lexical stipulation vs. referential properties
\par}

{\centering\itshape
Sebastian Fedden, Dunstan Brown, Franti\v{s}ek Kratochv\'il, 
\par}

{\centering\itshape
Laura C. Robinson and Antoinette Schapper
\par}

Abstract: We examine the role of referential properties and lexical stipulation in three closely related languages of eastern Indonesia, the Alor-Pantar languages Abui, Kamang and Teiwa. Our focus is on the continuum where event properties (e.g., volitionality, affectedness) are highly important at one extreme or play virtually no role at the other. These languages occupy different points on this continuum. In Abui event semantics play the greatest role, while in Teiwa they play the least role (the lexical property animacy being dominant in the formation of verb classes). Kamang occupies an intermediate position. Teiwa has conventionalized the relation between a verb and its class along the lines of animacy so that classes become associated with the animacy value of the objects with which the verbs in a given class typically occur. Paying attention to a lexical property like animacy, in contrast with event properties, has meant greater potential for arbitrary classes to emerge.

\section[Introduction]{Introduction\footnotemark{}}
\footnotetext{Sebastian Fedden and Dunstan Brown wrote the paper. Franti\v{s}ek Kratochv\'il did the corpus search for Abui and contributed expertise on the language. Laura C. Robinson contributed data on Teiwa. Antoinette Schapper provided examples and analysis for Abui, did the corpus search for Kamang and contributed expertise on the language. We are grateful to Marian Klamer, the editorial reviewer, and two anonymous reviewers. This paper was presented at the 5th Austronesian and Papuan Languages and Linguistics (APLL 5) conference at the School of Oriental and African Studies (SOAS), University of London, May 4-5, 2011, at the 44th meeting of the Societas Linguistica Europaea (SLE) in Logro\~no, Spain, September 8-11, 2011, and at the University of Zurich on March 30, 2012. We would like to thank the respective audiences for helpful comments and discussion. The work reported here was supported under the European Science Foundation{\textquoteright}s EuroBABEL programme (project {\textquoteleft}Alor-
Pantar languages: origin and theoretical impact{\textquoteright}). Fedden and Brown were funded by the Arts and Humanities Research Council (UK) under grant AH/H500251/1. Since April 2013 Fedden and Brown were funded by the Arts and Humanities Research Council (UK) under grant AH/K003194/1. Robinson was funded by the National Science Foundation (US) under BCS Grant No. 0936887. Schapper was funded by the Netherlands Organisation for Scientific Research (NWO). We thank these funding bodies for their support. A version of this paper appeared in \textit{Studies in Language} [{\dots}(to be completed)]. Correspondence address: Sebastian Fedden, Surrey Morphology Group, School of English and Languages, University of Surrey, Guildford GU2 7XH, UK. E-mail: s.fedden@surrey.ac.uk.}
In this chapter we examine in detail the role of referential properties in pronominal indexing and the degree of lexical stipulation within the verbal lexicon. By degree of lexical stipulation we mean the proportion of verbs for which the indexing patterns are lexically specified rather than dependent on semantic or pragmatic factors.

We examine three languages from the Alor-Pantar family: Teiwa, Kamang and Abui. For a map, see the introduction to this volume. We focus especially on the difference between properties expressing a relationship between participants and events (e.g., affectedness, volitionality) in Kamang and Abui, on the one hand, and the lexical properties of words (animacy, verb classes) in Teiwa, on the other hand. We find that Abui, Kamang and Teiwa are located at different points on a continuum of lexical stipulation: Abui is at one end, where event semantics play the greatest role, and Teiwa is at the other end, where lexical properties play the greatest role, with Kamang located somewhere between these two extremes.

It is well-known from the literature on the effect of semantic factors on case marking that similar factors to those found in the Alor-Pantar languages are involved in differential object marking, including: animacy (Croft 1988; Bossong 1991; Aissen 2003) and affectedness (Hopper and Thompson 1980; Tsunoda 1981, 1985; von Heusinger and Kaiser 2011). Volitionality is, among other things, argued to play a role in differential subject marking in Hindi (Mohanan 1990). It is important to bear in mind that we are not dealing with morphological case in the Alor-Pantar languages but with indexing of arguments on the verb.

The languages under investigation can be contrasted along at least three further dimensions: \textit{alignment type}, \textit{multifactoriality} (multiple conditions determining indexing), and \textit{number of prefix series}. Although logically independent, there may be a connection between the dimensions. Abui and Kamang have semantic alignment, Abui being more fluid in its alignment than Kamang, as we will see in the course of this chapter. Teiwa, on the other hand, has accusative syntactic alignment. Furthermore, in terms of the effects on pronominal marking, Abui and Kamang are multifactorial, with volitionality, affectedness and, marginally, animacy playing a role, while animacy in Teiwa constitutes the main factor according to which the verb classes in the language are defined (Fedden et al. 2013). Finally, the Alor languages Abui and Kamang have multiple prefix series, five and six, respectively, while the Pantar language Teiwa has only one.

In this chapter we use the term {\textquoteleft}pronominal indexing{\textquoteright} to describe a structure where there is a pronominal affix on the verb\footnote{In the Alor-Pantar languages, pronominal indices are exclusively prefixal.} and a co-referent noun phrase or free pronoun optionally (indicated by brackets) in the same clause, as in (). Co-reference is indicated by the index \textit{k}. There is no pronominal indexing in (). As the Alor-Pantar languages have APV and SV word order, any overt A or S argument precedes the verb.\footnote{We use the following primitives for core participants: S for the single argument of an intransitive verb, A for the more agent-like argument of a transitive verb, and P for the more patient-like argument of a transitive verb.}

\label{bkm:Ref306281876}()\ \ \ \ (noun phrase\textit{\textsubscript{k }}/free pronoun\textit{\textsubscript{k}})\ \ \ \ prefix\textit{\textsubscript{k}}{}-verb\footnote{The co-occurrence of a pronominal prefix and a co-referent free pronoun is generally restricted in the Alor-Pantar languages but the constraints for this differ between the languages. In some languages the co-occurrence of the free pronoun and pronominal prefix is possible under certain circumstances, but we do not address the issue here. We take up this issue in Brown and Fedden (in prep.).}

\label{bkm:Ref306281880}()\ \ \ \ (noun phrase/free pronoun)\ \ \ \ verb

The chapter has four parts. In {\S}2 and {\S}3, we briefly sketch the systems of syntactic and semantic alignment in Abui, Kamang and Teiwa, and discuss the number of prefix series that one finds in these languages, respectively. In {\S}4, we take a detailed look at Abui, Kamang, and Teiwa and show that Teiwa does not use indexing to directly represent information about events and participants but relies strongly on verb classes, with a high degree of arbitrary stipulation. Although verb classes also play a role in Abui and Kamang, indexing in these languages is used to directly encode information about events and participants, such as volitionality and affectedness in Abui, and affectedness in Kamang. Finally in {\S}5, we summarize and give a conclusion of our findings.

\section[Alignment]{Alignment}
The person prefixes found on the verbs in the Alor-Pantar languages are all very similar in form, pointing to a common historical origin.\footnote{Similar prefixes occur on nouns to mark possession. There are parallels, particularly because inalienable possession usually involves possessors linearly preceding the possessed in the same way that arguments linearly precede the verb.} However, pronominal indexing is conditioned by a variety of constraints which differ between the languages. We are concentrating on three languages from the Alor-Pantar family, Teiwa (Pantar; Klamer 2010a; Klamer, fieldnotes; Robinson, fieldnotes), Kamang (Eastern Alor, Schapper and Manimau 2011; Schapper, fieldnotes) and Abui (Central-Western Alor; Kratochv\'il 2007, 2011; Kratochv\'il, fieldnotes; Schapper, fieldnotes). For all languages we used experimental techniques (see Fedden et al. 2013), including a set of video stimuli for field elicitation (Fedden et al., 2010), availalbe corpora and elicited data. In the following we 
deal with alignment in our sample languages and then look at the number of prefix series ({\S}3). 

Teiwa, a language of Pantar, has syntactic alignment (accusative), whereas both Kamang, from eastern Alor, and Abui, from central western Alor, have semantic alignment. Although there is no case marking on noun phrases, alignment can be defined relative to pronominal indexing. 

For almost all Teiwa verbs the following holds: only Ps are indexed whereas Ss and As are never indexed. There is a small subset of three reflexive-like verbs which index the S (see below). Generally, therefore, Teiwa treats S like A and unlike P and can be said to have syntactic alignment of the accusative type. In Abui and Kamang, Ps are also indexed, as are more patient-like Ss (S\textsubscript{P}), while more agent-like Ss (S\textsubscript{A}) are not indexed. As in Teiwa, As are not indexed. Such systems in which Ss behave differently depending on semantic factors are generally called semantic alignment systems (Donohue and Wichmann 2008) or active/agentive systems (Mithun 1991).

The Alor-Pantar languages are of interest at the macro-typological level for a number of reasons. First, the nominative-accusative alignment system in Teiwa{\textquoteright}s prefixal marking is typologically the most common (Siewierska 2004: 53), yet in Teiwa it is associated with the rare property of marking only the person of the P argument on the verb (Siewierska 2011). Second, the Alor-Pantar languages which have semantic alignment are subject to differing semantic factors in determining their pronominal indexing, including animacy, volitionality and affectedness. These are, of course, implicated in many phenomena of a wider macro-typological interest, and it is worthwhile looking at the impact of the various semantic factors on pronominal indexing to consider how the languages with the more semantically fluid systems, where volitionality and affectedness play a role, may be associated with those languages with systems that make use of more lexically determined verb classes or specific features, such as 
animacy. Teiwa is of the latter type. For almost all verbs, Ss are encoded with a free pronoun, as illustrated in ():

Teiwa (Klamer 2010a: 169)

\begin{flushleft}
\tablehead{}
\begin{supertabular}{m{0.35185984in}m{0.40665984in}m{0.71155983in}}
\label{bkm:Ref353455416}()\ \  &
\itshape A &
\itshape her.\\
 &
3\textsc{sg} &
climb\\
 &
\multicolumn{2}{m{1.1969599in}}{{\textquoteleft}He climbs up.{\textquoteright}}\\
\end{supertabular}
\end{flushleft}
An example of an indexed S is provided in (). Teiwa has only three verbs which follow this pattern. These are \textit{{}-o{\textquoteright}on} {\textquoteleft}hide{\textquoteright}, \textit{{}-ewar} {\textquoteleft}return{\textquoteright} and \textit{{}-ufan} {\textquoteleft}forget{\textquoteright}. 

Teiwa (Klamer 2010a: 98)

\begin{flushleft}
\tablehead{}
\begin{supertabular}{m{0.35115984in}m{0.40665984in}m{0.75045985in}}
\label{bkm:Ref357868399}()\ \  &
\itshape Ha &
\itshape h-o{\textquoteright}on.\\
 &
2\textsc{sg} &
2\textsc{sg}{}-hide\\
 &
\multicolumn{2}{m{1.2358599in}}{{\textquoteleft}You hide.{\textquoteright}}\\
\end{supertabular}
\end{flushleft}
Indexation of P on the Teiwa verb is associated with animacy of P. In the Teiwa corpus (Klamer, Teiwa corpus) indexing is restricted to 49 out of 224 transitive verbs (types), i.e., \~{}22\%, comprising 44 verbs which always index P and five verbs in which the presence of the index depends on the animacy value of P. The rest of the transitive verbs never index their object. This is illustrated in () below for the prefixing transitive verb \textit{{}-}\textit{unba{\textquoteright}} {\textquoteleft}meet{\textquoteright}, where the object is animate and in the third person singular, while the subject is in the second person singular. In (), we see the non-prefixing transitive verb \textit{ari{\textquoteright}} {\textquoteleft}break{\textquoteright}, which typically takes an inanimate object:

Teiwa (Klamer 2010a: 159)

\begin{flushleft}
\tablehead{}
\begin{supertabular}{m{0.35115984in}m{0.58515984in}m{0.54345983in}m{1.1518599in}m{0.7670598in}}
\label{bkm:Ref306280773}()\ \  &
\itshape Name, &
\itshape ha{\textquoteright}an &
\itshape n-oqai &
\itshape g-unba{\textquoteright}?\\
 &
sir &
2\textsc{sg} &
1\textsc{sg}.\textsc{poss}{}-child &
3-meet\\
 &
\multicolumn{4}{m{3.2837598in}}{{\textquoteleft}Sir, did you see (lit. meet) my child? {\textquoteright}}\\
\end{supertabular}
\end{flushleft}
Teiwa (Klamer 2010a: 101)

\begin{flushleft}
\tablehead{}
\begin{supertabular}{m{0.34835985in}m{0.5809598in}m{0.47815987in}m{0.85455984in}m{0.5261598in}}
\label{bkm:Ref306280777}() &
\itshape Ha{\textquoteright}an &
\itshape meja &
\itshape ga-fat &
\itshape ari{\textquoteright}.\\
 &
2\textsc{sg} &
table &
3.\textsc{poss}{}-leg &
break\\
 &
\multicolumn{4}{m{2.6760597in}}{{\textquoteleft}You broke that table leg!{\textquoteright} }\\
\end{supertabular}
\end{flushleft}
Kamang, on the other hand, has semantic alignment, where the single argument of an intransitive verb (S) is coded like the agentive argument of a transitive verb (A) or like the patientive argument of a transitive verb (P), if the S is affected. This is shown in () and (). An affected single participant of an intransitive verb is indexed with a prefix.

Kamang (Schapper, to appear)

\begin{flushleft}
\tablehead{}
\begin{supertabular}{m{0.35115984in}m{1.5580599in}}
\label{bkm:Ref353455458}()\ \  &
\itshape Na-maitan-si.\\
 &
\textsc{1sg.pat-}hunger-\textsc{ipfv}\\
 &
{\textquoteleft}I{\textquoteright}m hungry.{\textquoteright}\\
\end{supertabular}
\end{flushleft}
Kamang (Response to video clip C03\_dance\_05, SP13)

\begin{flushleft}
\tablehead{}
\begin{supertabular}{m{0.35115984in}m{1.0268599in}m{1.2552599in}m{0.85455984in}m{0.5497598in}}
\label{bkm:Ref353455450}()\ \  &
\itshape Almakang=a &
\itshape pilan. &
 &
\\
 &
people=\textsc{spec} &
dance\_lego-lego &
 &
\\
 &
\multicolumn{4}{m{3.9226599in}}{{\textquoteleft}The people are dancing a \textit{lego-lego} (traditional dance).{\textquoteright}}\\
\end{supertabular}
\end{flushleft}
Some verbs allow alternation between having a prefix and an affected S and having no prefix and a non-affected S. This can be seen in (), where the dog runs off because it was chased away, whereas () does not have this affected meaning.

Kamang (Schapper, to appear)

\begin{flushleft}
\tablehead{}
\begin{supertabular}{m{0.35185984in}m{0.40455985in}m{0.8281598in}m{0.85455984in}m{0.47955987in}}
\label{bkm:Ref353455505}()\ \  &
\itshape Kui &
\itshape ge-tak. &
 &
\\
 &
dog &
3.\textsc{gen}{}-run &
 &
\\
 &
\multicolumn{4}{m{2.80306in}}{{\textquoteleft}The dog ran off (was forced to run).{\textquoteright}}\\
\end{supertabular}
\end{flushleft}
Kamang (Schapper, to appear)

\begin{flushleft}
\tablehead{}
\begin{supertabular}{m{0.35185984in}m{0.40455985in}m{0.40045986in}m{0.19065985in}}
\label{bkm:Ref324857914}() &
\itshape Kui &
\itshape tak. &
\\
 &
dog &
run &
\\
 &
\multicolumn{3}{m{1.1531599in}}{{\textquoteleft}The dog ran.{\textquoteright}}\\
\end{supertabular}
\end{flushleft}
\clearpage
Kamang indexes P{\textquoteright}s, for instance on the verb \textit{{}-tan} {\textquoteleft}wake someone up{\textquoteright} in example ():

Kamang (Response to video clip P07\_wake\_up\_person\_19, SP15)

\begin{flushleft}
\tablehead{}
\begin{supertabular}{m{0.43165985in}m{0.43235984in}m{1.8372599in}m{0.54345983in}m{1.5143598in}m{0.26635984in}}
\label{bkm:Ref357868094}() &
\itshape [{\dots}] &
\itshape ge-pa-l &
\itshape sue &
\textit{ga-tan}. &
\\
 &
[{\dots}] &
3.\textsc{gen}{}-father-\textsc{contr\_foc} &
arrive &
3.\textsc{pat}{}-wake\_up &
\\
 &
\multicolumn{5}{m{4.9087596in}}{{\textquoteleft}[{\dots}] his father comes and wakes him.{\textquoteright}}\\
\end{supertabular}
\end{flushleft}
Abui also has semantic alignment. An important semantic factor in the indexing of Ss is volitionality. Volitional Ss are expressed with a free pronoun, as in (), and non-volitional Ss are indexed on the verb, as in (). The free translations try to capture the difference in volitionality involved here.

Abui (Kratochv\'il 2007: 15)

\begin{flushleft}
\tablehead{}
\begin{supertabular}{m{0.42195985in}m{0.42265984in}m{0.42265984in}}
\label{bkm:Ref306280914}() &
\itshape Na\ \  &
\itshape laak.\\
 &
1\textsc{sg} &
leave\\
 &
\multicolumn{2}{m{0.9240598in}}{{\textquoteleft}I go away.{\textquoteright}}\\
\end{supertabular}
\end{flushleft}
Abui (Kratochv\'il 2007: 15)

\begin{flushleft}
\tablehead{}
\begin{supertabular}{m{0.43165985in}m{2.68936in}}
\label{bkm:Ref306280918}() &
\itshape No-laak.\\
 &
1\textsc{sg.rec}{}-leave\\
 &
{\textquoteleft}I (am forced to) retreat.{\textquoteright}\\
\end{supertabular}
\end{flushleft}
Abui indexes Ps. There are no verbs in the corpus which are never prefixed. An example of a prefixed transitive verb indexing a P is (). Animacy is much less important in Abui; both \textit{fik} {\textquoteleft}pull{\textquoteright} and \textit{{}-bel} {\textquoteleft}pull{\textquoteright} in () would be prefixed, even if their Ps were inanimate. 

Abui (Response to video clip C01\_pull\_person\_25, SP8)

\begin{flushleft}
\tablehead{}
\begin{supertabular}{m{0.43165985in}m{0.48725984in}m{0.47885987in}m{0.47885987in}m{0.50185984in}m{1.2101599in}m{0.84625983in}}
\label{bkm:Ref306280933}() &
\itshape Wiil &
\itshape neng &
\itshape nuku &
\itshape di &
\itshape de-feela &
\itshape ha-fik\\
 &
child &
male &
one &
3\textsc{act} &
3.\textsc{al.poss}{}-friend &
3.\textsc{pat}{}-pull\\
 &
\multicolumn{3}{m{1.6024599in}}{\itshape ha-bel-e.} &
 &
 &
\\
 &
\multicolumn{3}{m{1.6024599in}}{3\textsc{.pat}{}-pull-\textsc{ipfv}} &
 &
 &
\\
 &
\multicolumn{4}{m{2.18306in}}{{\textquoteleft}A boy is pulling his friend.{\textquoteright} } &
 &
\\
\end{supertabular}
\end{flushleft}
For some Abui verbs a difference of affectedness in the P can be encoded by the choice of prefix, namely a prefix from the \textsc{loc} series for a lower degree of affectedness and a prefix from the \textsc{pat} series for a higher degree of affectedness; for example, \textit{he-dik} [3.\textsc{loc}{}-pierce] {\textquoteleft}stab at it{\textquoteright} vs. \textit{ha-dik} [3.\textsc{pat}{}-pierce] {\textquoteleft}pierce it through{\textquoteright}. We take this up in {\S}4.1.2 below.

To sum up the role of conditions, Abui and Kamang index P arguments and some S arguments of the verb. This is in part determined by affectedness (in Abui and Kamang) and volitionality (in Abui), and to a lesser degree by animacy (see {\S}4.3 below). In both languages lexical verb classes also play a role to some degree, in Kamang more than in Abui. Teiwa indexes P arguments of the verb, in part determined by animacy. The role of animacy in Teiwa in the formation of verb classes will be taken up in {\S}4.4.

\clearpage\section[Number of person prefix series]{Number of person prefix series}
All Alor-Pantar languages have at least one series of person prefixes. In the languages which have only one series, like Teiwa and Western Pantar (Holton 2010), this is always the series which is identified by \textit{a}{}-vowels in the singular and \textit{i}{}-vowels in the plural.\footnote{Teiwa actually has four verbs for which a difference between an animate and an inanimate object can be encoded by the choice of two different prefixes in the third person only. We discuss this alternation in detail in section 3.4.} The Teiwa prefixes are given in Table 1.

Table 1: Teiwa person prefixes (Klamer 2010a: 77, 78)

\begin{flushleft}
\tablehead{}
\begin{supertabular}{|m{0.90875983in}|m{0.70945984in}|}
\hline
 &
Prefix\\\hline
\scshape 1sg &
\itshape n(a)-\\\hline
\scshape 2sg &
\itshape h(a)-\\\hline
\scshape 3sg &
\itshape g(a)-\\\hline
\scshape 1pl.excl &
\itshape n(i)-\\\hline
\scshape 1pl.incl &
\itshape p(i)-\\\hline
\scshape 2pl &
\itshape y(i)-\\\hline
\scshape 3pl &
\itshape g(i)-, ga-\\\hline
\end{supertabular}
\end{flushleft}
This \textit{a}{}-vowel series is the only series that can be reconstructed back to proto-Alor-Pantar (Holton et al. 2012: 115; Holton and Robinson, this volume). The second series of person prefixes~is only reconstructed as indexing possessors (genitive). The reconstructed prefix forms for Alor-Pantar are given in Table 2.

Table 2: Proto-Alor-Pantar verb prefixes

\begin{flushleft}
\tablehead{}
\begin{supertabular}{|m{0.9038598in}|m{0.60455984in}|}
\hline
 &
pAP\\\hline
\scshape 1sg &
*na- \\\hline
\scshape 2sg &
*(h)a- \\\hline
\scshape 3sg &
*ga- \\\hline
\scshape 1pl.excl &
*ni- \\\hline
\scshape 1pl.incl &
*pi- \\\hline
\scshape 2pl &
*(h)i- \\\hline
\scshape 3pl &
*gi- \\\hline
\end{supertabular}
\end{flushleft}
Abui and Kamang are innovative in that they developed multiple prefix series. The Abui prefixes are given in Table 3.

Table 3: Abui person prefixes

\begin{flushleft}
\tablehead{}
\begin{supertabular}{|m{1.6726599in}|m{0.6150598in}m{0.23235986in}|m{1.2045599in}|m{0.7962598in}|m{1.4247599in}|}
\hhline{~-~~~~}
\multicolumn{1}{m{1.6726599in}|}{} &
\multicolumn{1}{m{0.6150598in}|}{\centering Prefixes} &
\multicolumn{4}{m{3.8941598in}}{}\\\hline
 &
\multicolumn{2}{m{0.9261598in}|}{\scshape pat} &
\scshape rec &
\scshape loc &
\scshape goal\\\hline
\scshape 1sg &
\multicolumn{2}{m{0.9261598in}|}{\itshape n(a)-} &
\itshape no- &
\itshape ne- &
\itshape noo-\\\hline
\scshape 2sg &
\multicolumn{2}{m{0.9261598in}|}{\textit{a-}\footnotemark{} } &
\itshape o- &
\itshape e- &
\itshape oo-\\\hline
3 &
\multicolumn{2}{m{0.9261598in}|}{\itshape h(a)-} &
\itshape ho- &
\itshape he- &
\itshape hoo-\\\hline
\scshape 1pl.excl &
\multicolumn{2}{m{0.9261598in}|}{\itshape ni-} &
\itshape nu- &
\itshape ni- &
\itshape nuu-\\\hline
\scshape 1pl.incl &
\multicolumn{2}{m{0.9261598in}|}{\itshape pi-} &
\itshape po-/pu- &
\itshape pi- &
\itshape puu-/poo-\\\hline
\scshape 2pl &
\multicolumn{2}{m{0.9261598in}|}{\itshape ri-} &
\itshape ro-/ru- &
\itshape ri- &
\itshape ruu-/roo-\\\hline
\end{supertabular}
\end{flushleft}
\footnotetext{\textit{{\O}- }before vowel.}
In Kamang, which has six prefix series, the \textsc{gen} series can be related to the free alienable possessive marker *\textit{e} plus person marking prefix, the combination of which was then attached to the front of verbs. The \textsc{loc} series comes from a free element, probably the postposition {\textquoteleft}at{\textquoteright}, which carried the person marker at some point and which was reinterpreted as a prefix of a different series. The origin of the other three prefix series in Kamang is unclear. The Kamang prefixes are given in Table 4.

Table 4: Kamang person prefixes\footnote{In the third person prefixes, /g/ can be realized as [j] before front vowels, i.e., in the \textsc{gen} and \textsc{dat} series.}

\begin{flushleft}
\tablehead{}
\begin{supertabular}{|m{1.7518599in}|m{0.20735985in}m{0.46775988in}|m{0.8379598in}|m{0.8219598in}|m{0.8719598in}|m{0.90875983in}|}
\hhline{~-~~~~~}
\multicolumn{1}{m{1.7518599in}|}{} &
\multicolumn{1}{m{0.20735985in}|}{\centering Prefixes} &
\multicolumn{5}{m{4.22336in}}{}\\\hline
 &
\multicolumn{2}{m{0.7538598in}|}{\scshape pat} &
\scshape loc &
\scshape gen &
\textsc{ast}\footnotemark{} &
\scshape dat\\\hline
\scshape 1sg &
\multicolumn{2}{m{0.7538598in}|}{\itshape na-} &
\itshape no- &
\itshape ne- &
\itshape noo- &
\itshape nee-\\\hline
\scshape 2sg &
\multicolumn{2}{m{0.7538598in}|}{\itshape a-} &
\itshape o- &
\itshape e- &
\itshape oo- &
\itshape ee-\\\hline
3 &
\multicolumn{2}{m{0.7538598in}|}{\itshape ga-} &
\itshape wo- &
\itshape ge- &
\itshape woo- &
\itshape gee-\\\hline
\scshape 1pl.excl &
\multicolumn{2}{m{0.7538598in}|}{\itshape ni-} &
\itshape nio- &
\itshape ni- &
\itshape nioo- &
\itshape nii-\\\hline
\scshape 1pl.incl &
\multicolumn{2}{m{0.7538598in}|}{\itshape si-} &
\itshape sio- &
\itshape si- &
\itshape sioo- &
\itshape sii-\\\hline
\scshape 2pl &
\multicolumn{2}{m{0.7538598in}|}{\itshape i-} &
\itshape io- &
\itshape i- &
\itshape ioo- &
\itshape ii-\\\hline
\end{supertabular}
\end{flushleft}
\footnotetext{The assistive (\textsc{ast}) refers to the participant who assists in the action.}
Having multiple person prefix series is not restricted to Alor-Pantar languages with semantic alignment. For example Adang (Haan 2001) has three series. Adang has syntactic alignment like Teiwa (i.e., only P{\textquoteright}s are indexed with a prefix) but multiple prefix series like Abui and Kamang.

\section[Lexical stipulation and referential properties]{Lexical stipulation and referential properties}
In the following section, we take a detailed look at the three sample languages Abui, Kamang and Teiwa. We focus on the difference between properties expressing a relationship between participants and events (affectedness, volitionality) on the one hand and lexical properties (animacy, verb classes) on the other.

\ \ A wide range of different semantic factors has been implicated in the literature in playing a role in argument realization in semantic alignment systems (Mithun 1991, Arkadiev 2008, Klamer 2008). For instance, the active/stative distinction (in Loma and Classical Guaran\'i) and the agentive/patientive distinction (in Lakhota), telicity (in Georgian), volitionality (in Tabassaran and Kambera), and affectedness (in Central Pomo and Mohawk). The patterns of argument marking one finds based on these semantic factors represent the grammaticalization of semantic relations between predicates and arguments (Mithun 1991: 542). Animacy is not among the factors typically identified in semantic alignment systems because it does not describe a relationship between arguments and the predicate, but rather is a lexical property. As Hurford (2007: 43) notes in his discussion of the pre-linguistic basis for semantics, animacy is a more permanent property and is {\textquoteleft}less perception dependent{\textquoteright}. 
This permanence means that it can be a lexical property which is not dependent on the particular details of a given event. It seems that if animacy plays a role at all in those Alor-Pantar languages with semantic alignment, it is a subservient one, as shown for Abui and Kamang in {\S}3.3.

\subsection[Abui]{Abui}
Of the three languages in our sample Abui shows the greatest flexibility of combining verbs with prefixes from different series. However, the \textsc{pat} prefix series is much more lexically limited than the other inflections in Abui. These are the only verbs showing lexical classing, i.e., the absence of alternation. An example of such a verb is provided in ():

Abui (Kratochv\'il 2007: 463)

\begin{flushleft}
\tablehead{}
\begin{supertabular}{m{0.42265984in}m{0.47815987in}m{0.40455985in}m{1.1406599in}}
\label{bkm:Ref323745334}() &
\itshape Kaai &
\itshape afu &
\itshape ha-ful.\\
 &
dog &
fish &
3.\textsc{pat}{}-swallow\\
 &
\multicolumn{3}{m{2.18086in}}{{\textquoteleft}The dog swallowed the fish.{\textquoteright}}\\
\end{supertabular}
\end{flushleft}
\subsubsection[Inflection classes in Abui]{\bfseries Inflection classes in Abui}
The discussion here is based on a detailed examination of the prefixal behaviour of 210 verbs. The numbers reflect the state of the documentation and analysis of the language at present. 

For 33 Abui verbs, inflection with a \textsc{pat} prefix is either obligatory or optional.\footnote{The difference between 33 and 36 is made up of three verbs: \textit{{}-l} {\textquoteleft}give{\textquoteright}, \textit{{}-maria} {\textquoteleft}leak fluid{\textquoteright} and \textit{mpang} {\textquoteleft}think{\textquoteright}, which require a prefix but this prefix cannot come from the \textsc{pat} series.} Table 5 below presents the distribution of the \textsc{pat} prefix across the whole sample (all percentages rounded to whole numbers). Obligatory inflection with a \textsc{pat} prefix means that a verb has to have a prefix and that the prefix has to be from the \textsc{pat} prefix series. In other words, these verbs exclusively appear with the \textsc{pat} prefix. This is the case for 14\% of the verbs in the sample (29 out of 210 verbs). Within optional \textsc{pat} verbs we distinguish two cases. First, the verbs that can occur with a \textsc{pat} prefix or with a prefix from one or more of the 
other series but which always require a prefix. These are a minority (4 out of 210 verbs or 2\% in table 5). Second, the verbs that can occur with \textsc{pat} or with a prefix from one or more of the other series but which also allow occurrence without a prefix. These form a substantial subset (68 out of 210 verbs or 32\% in table 5).

Table 5: Distribution of the Abui \textsc{pat} prefixes

\begin{flushleft}
\tablehead{}
\begin{supertabular}{|m{2.1719599in}|m{1.6962599in}|m{1.9740598in}|m{0.26085985in}|}
\hline
 &
\textsc{pat} obligatory &
\multicolumn{2}{m{2.31366in}|}{\centering \textsc{pat} optional}\\\hline
 &
Prefix required &
A prefix is required &
A prefix is not required\\\hline
 &
29 verbs &
4 verbs &
68 verbs\\\hline
Total (of 210 verbs) &
14\% (29/210) &
2\% (4/210)  &
32\% (68/210)\\\hline
\end{supertabular}
\end{flushleft}
The 29 verbs in our sample which obligatorily occur with the \textsc{pat} prefix are set out in the column on the left in (). The column on the right contains verbs which optionally take \textsc{pat}. The first four verbs are those which optionally take \textsc{pat} and always require a prefix, i.e., the verb cannot occur without a prefix. The remaining verbs in the column on the right are examples of verbs which optionally take \textsc{pat} but where occurrence without a prefix is also possible. (Optional prefixing is indicated by brackets.) The addition of \textit{someone} (s.o.) and/or \textit{something} (sth.) in the glosses indicated whether a verb can appear with an animate or an inanimate O in the corpus. If there is no such addition, e.g., \textit{maha} {\textquoteleft}want{\textquoteright} the prefix indexes the S.

\begin{flushleft}
\tablehead{}
\begin{supertabular}{|m{0.6011598in}|m{2.7900598in}|m{2.79136in}|}
\hline
\label{bkm:Ref372878878}() &
{}-\textit{al} {\textquoteleft}burn sth.{\textquoteright} &
\textit{{}-dak} {\textquoteleft}grab firmly s.o./sth.{\textquoteright}\\\hline
 &
\textit{{}-ai }{\textquoteleft}put at, add sth.{\textquoteright} &
\textit{{}-luol} {\textquoteleft}follow, collect s.o./sth.{\textquoteright}\\\hline
 &
{}-\textit{balak} {\textquoteleft}hit, punch s.o./sth.{\textquoteright} &
\textit{{}-k} {\textquoteleft}throw at s.o./sth.; feed s.o./sth.{\textquoteright}\\\hline
 &
{}-\textit{basa} {\textquoteleft}bind, hit, punch sth.{\textquoteright} &
\textit{{}-maha} {\textquoteleft}want{\textquoteright}\\\hline
 &
{}-\textit{buk} {\textquoteleft}tie together s.o./sth.{\textquoteright} &
\textit{(-)}\textit{aahi} {\textquoteleft}take away sth.{\textquoteright}\\\hline
 &
\textit{{}-bV} {\textquoteleft}join sth., lean against sth.{\textquoteright} &
\textit{(-)}\textit{afui} {\textquoteleft}scoop up sth.{\textquoteright}\\\hline
 &
\textit{ful} {\textquoteleft}swallow s.o./sth.{\textquoteright} &
\textit{(-)}\textit{ahelri} {\textquoteleft}tired{\textquoteright}\\\hline
 &
{}-\textit{iel} {\textquoteleft}roast, burn sth.{\textquoteright} &
\textit{(-)}\textit{bel/}\textit{(-)}\textit{ber} {\textquoteleft}pluck, pull out s.o./sth.{\textquoteright}\\\hline
 &
{}-\textit{ieng}~{\textquoteleft}see s.o./sth.{\textquoteright} &
\textit{(-)}\textit{dik} {\textquoteleft}stab s.o./sth.{\textquoteright}\\\hline
 &
{}-\textit{iengria} {\textquoteleft}show s.o.{\textquoteright} &
\textit{(-)}\textit{fanga} {\textquoteleft}say sth./to s.o.{\textquoteright}\\\hline
 &
{}-\textit{fik} {\textquoteleft}pull s.o./sth.{\textquoteright} &
\textit{(-)}\textit{keila} {\textquoteleft}block sth./s.o.{\textquoteright}\\\hline
 &
{}-\textit{kai} {\textquoteleft}drop s.o./sth.{\textquoteright} &
\textit{(-)}\textit{kol} {\textquoteleft}trick, cheat s.o.{\textquoteright}\\\hline
 &
{}-\textit{kawalia} {\textquoteleft}protect sth.{\textquoteright} &
\textit{(-)}\textit{kol} {\textquoteleft}tie up, bind s.o./sth.{\textquoteright}\\\hline
 &
{}-\textit{kuoila} {\textquoteleft}make fall, topple s.o./sth.{\textquoteright} &
\textit{(-)}\textit{komang} {\textquoteleft}blunt sth.{\textquoteright}\\\hline
 &
\textit{{}-lal} {\textquoteleft}laugh{\textquoteright} &
\textit{(-)}\textit{l\'ak} {\textquoteleft}break sth.; break out s.o.{\textquoteright}\\\hline
 &
{}-\textit{langa} {\textquoteleft}harass s.o.{\textquoteright} &
\textit{(-)}\textit{likda} {\textquoteleft}bend sth.; blame s.o.{\textquoteright}\\\hline
 &
{}-\textit{minang} {\textquoteleft}remember{\textquoteright}  &
\textit{(-)}\textit{miti} {\textquoteleft}sit{\textquoteright}\\\hline
 &
\textit{{}-mintaai} {\textquoteleft}pray to s.o.{\textquoteright} &
\textit{(-)}\textit{muila} {\textquoteleft}play sth.{\textquoteright}\\\hline
 &
{}-\textit{moida} {\textquoteleft}sound{\textquoteright} &
\textit{(-)}\textit{poku} {\textquoteleft}broken{\textquoteright}\\\hline
 &
{}-\textit{pai} {\textquoteleft}keep s.o./sth.{\textquoteright} &
\textit{(-)}\textit{rel} {\textquoteleft}hit the ground{\textquoteright}\\\hline
 &
{}-\textit{pakda} {\textquoteleft}throw sth.{\textquoteright} &
\textit{(-)}\textit{rumaidia} {\textquoteleft}strengthen s.o./sth.{\textquoteright}\\\hline
 &
{}-\textit{patingdi} {\textquoteleft}advise s.o.{\textquoteright} &
\textit{(-)}\textit{sik} {\textquoteleft}sever, split sth./from s.o.{\textquoteright}\\\hline
 &
{}-\textit{rik} {\textquoteleft}hurt s.o.{\textquoteright} &
\textit{(-)}\textit{tak} {\textquoteleft}drop sth.; stop s.o.{\textquoteright}\\\hline
 &
{}-\textit{reng} {\textquoteleft}face, turn to s.o./sth.{\textquoteright} &
\textit{(-)}\textit{tek} {\textquoteleft}move down sth./on s.o.{\textquoteright}\\\hline
 &
{}-\textit{tamadia} {\textquoteleft}repair sth.{\textquoteright} &
\textit{(-)}\textit{tilia }{\textquoteleft}hang{\textquoteright}\\\hline
 &
{}-\textit{tuokda} {\textquoteleft}jump{\textquoteright} &
\textit{(-)}\textit{took }{\textquoteleft}drop, put to sth./for s.o.{\textquoteright}\\\hline
 &
{}-\textit{wel} {\textquoteleft}bathe s.o.{\textquoteright} &
\textit{(-)}\textit{tuk }{\textquoteleft}stick, measure s.o./sth.{\textquoteright}\\\hline
 &
{}-\textit{yaal} {\textquoteleft}give birth to s.o./to sth.{\textquoteright} (e.g., a banana blossom in a story) &
\textit{(-)}\textit{wik }{\textquoteleft}carry s.o./sth.{\textquoteright}\\\hline
 &
{}-\textit{yongfa} {\textquoteleft}forget{\textquoteright} &
\textit{(-)}\textit{yok }{\textquoteleft}cover s.o./sth.{\textquoteright}\\\hline
\end{supertabular}
\end{flushleft}
As the verbs which obligatorily take a \textsc{pat} prefix (left-hand column in () above) do not form an obvious semantic class, we treat them as an inflectional class, defined by the fact that these verbs can only occur with a \textsc{pat} prefix. The verbs which optionally occur with a \textsc{pat} prefix (right-hand column in () above) can take a prefix from at least one other series instead of \textsc{pat}. The majority of these verbs can alternate between the \textsc{rec}, \textsc{loc}, \textsc{goal}, and \textsc{ben} inflections, whereby semantic differences in the indexed participant are observable when alternating one inflection with another.

To sum up, apart from lexical classing found in verbs which only occur with the \textsc{pat} prefix, the Abui prefix system is highly fluid and verbs can occur with most, perhaps all of the prefixes, or be unprefixed. In the following sections, we deal in turn with affectedness and volitionality as factors which impact on the prefixation patterns.

\subsubsection[Affectedness]{\bfseries Affectedness}
Affectedness is one of the factors that has impact on pronominal indexing in Abui. Affected participants undergo a persistent change. On affectedness as a criterion for high transitivity, see Hopper and Thompson (1980), Tsunoda (1981, 1985). On affectedness as a parameter of semantic distinctness between the two participants of a transitive clause, see N{\ae}ss (2004, 2006, 2007).

Affectedness is clearly a relation between a participant and an event because, while the participant is the affected entity, the predicate contains the information whether the change of state is entailed (Beavers 2011: 337). Consider the examples from English in () and ():

\label{bkm:Ref357868137}()\ \ \textit{He breaks the wooden board.}

\label{bkm:Ref357868148}()\ \ \textit{He hits the wooden board.}

In each case the \textit{wooden board }is the patient. Only the predicate specifies the degree of affectedness, which is higher in () than in (). In (), with the predicate \textit{break}, the change in the affected participant is entailed; one cannot break a board without effecting a change of state in the board. However, in () with the predicate \textit{hit},\textit{ }this is not the case. The fact that the agent makes contact with the wooden board means that it is impinged upon but this does not entail a change of state; one can hit a board without effecting a change in it.

Abui allows the expression of different degrees of affectedness by choosing between the \textsc{pat} and the \textsc{loc} prefix series for P, as illustrated in (19):

Abui (Kratochv\'il 2011: 596; p.c.)

\begin{flushleft}
\tablehead{}
\begin{supertabular}{|m{0.53165984in}|m{2.73236in}|m{2.91776in}|}
\hline
() &
Lower degree of affectedness: \textsc{loc} prefix &
Higher degree of affectedness: \textsc{pat} prefix\\\hline
 &
\textit{he-dik} {\textquoteleft}stab at it{\textquoteright} &
\textit{ha-dik} {\textquoteleft}pierce it through{\textquoteright}\\\hline
 &
\textit{he-akung} {\textquoteleft}cover it{\textquoteright} &
\textit{h-akung} {\textquoteleft}extinguish it{\textquoteright}\\\hline
 &
\textit{he-pung} {\textquoteleft}hold it{\textquoteright} &
\textit{ha-pung} {\textquoteleft}catch it{\textquoteright}\\\hline
 &
\textit{he-komangdi} {\textquoteleft}make it less sharp{\textquoteright} &
\textit{ha-komangdi }{\textquoteleft}make it completely blunt{\textquoteright}\\\hline
 &
\textit{he-lilri }{\textquoteleft}warm it up (water){\textquoteright} &
\textit{ha-lilri }{\textquoteleft}boil it (water){\textquoteright}\\\hline
 &
\textit{he-lak }{\textquoteleft}take it apart{\textquoteright} &
\textit{ha-lak }{\textquoteleft}demolish it{\textquoteright}\\\hline
\end{supertabular}
\end{flushleft}
The \textsc{loc} series is chosen if the change of state in P is either not entailed, e.g., \textit{he-pung} {\textquoteleft}hold it{\textquoteright} vs. \textit{ha-pung} {\textquoteleft}catch it{\textquoteright}, or if it is that P is less strongly affected \textit{he-lak }{\textquoteleft}take apart{\textquoteright} vs. \textit{ha-lak }{\textquoteleft}demolish{\textquoteright}. The \textsc{pat} series on the other hand is chosen if P is highly affected and a change of state in P is entailed. These Abui examples show the impact of different degrees of affectedness depending on which prefix series is chosen for the indexing of P.

\subsubsection[Volitionality]{\bfseries Volitionality}
Next we deal with the factor of volitionality. Volitionality in a linguistic context has been defined in various ways in the literature which make sense intuitively, but to our knowledge there has been no serious attempt to formalize volitionality in a way that Beavers (2011) did for affectedness. Hopper and Thompson (1980: 286) define volitionality as the {\textquotedblleft}degree of planned involvement of an A[gent] in the activity of the verb{\textquotedblright}. DeLancey (1985: 52) equates volitionality with conscious control over the activity of the verb. Furthermore, it has been observed in the literature that control and volition often coincide (Tsunoda 1985: 392, DeLancey 1985: 56)\footnote{Indeed, some scholars use the term {\textquoteleft}control{\textquoteright} rather than volitionality (Mithun 1991, Holton 2010).} and that instigation is sometimes used interchangeably with control (N{\ae}ss 2007: 45). On volition as an entailment which identifies (Proto-)Agents, see Dowty (1991).

There are finer grained distinctions in the literature, for example, both Mithun (1991) and Kratochv\'il (2011) differentiate between instigation, i.e., the responsibility for the onset of an event, and control, i.e., the responsibility for its execution. We are using the term {\textquoteleft}volitionality{\textquoteright} as a cover term to include both instigation and conscious control.

While volitionality as a term suggests that it is exclusively a property of a human (or at least an animate) participant it is typically not a property of the lexical semantics of nouns that they are volitional or non-volitional agents, apart from individual items like {\textquoteleft}volunteer{\textquoteright}. In other words, for many nouns in the lexicon, they are non-committal as to volitionality. Nouns such as \textit{person, child,} or \textit{man} can be used in contexts in which they may be subject to non-volitional acts (e.g., fall) or volitional ones (e.g., walk), while they remain constant in their values for animacy. This means that a distinction on the basis of volitionality would not yield an exhaustive partition of the lexicon in the way that the animate-inanimate opposition would. Typically, volitionality is a property of a participant which is observed in the context of an event. In this sense we can attribute it to the event as a whole. Volitionality as we use it here (or the absence 
thereof) is more likely a part of the lexical semantics of verbs, as can be seen in examples like {\textquoteleft}stumble{\textquoteright}, {\textquoteleft}trip{\textquoteright}, {\textquoteleft}fall{\textquoteright}, and {\textquoteleft}vomit{\textquoteright}. But as with the noun examples mentioned, it is possible to find verbs where there is no requirement that their lexical semantics are committed to a value for volitionality. This entails that, while volitionality may be relevant for some verbs such as the ones we mention, it does not partition the verb lexicon in the way that animacy partitions the noun lexicon.

Volitionality is a key semantic factor in determining whether an S is indexed in Abui. There is no relationship between the choice of prefix and the degree of volitionality of the S. The absence of a prefix signals volitional Ss, whereas free pronouns are outside the system of volitionality and non-volitionality. This is illustrated with the following pair: \textit{na laak} [1\textsc{sg }leave] {\textquoteleft}I go away{\textquoteright} vs. \textit{(na)} \textit{no-laak }[(1\textsc{sg}) 1\textsc{sg.rec}{}-leave] {\textquoteleft}I (am forced to) retreat{\textquoteright}. These examples illustrate this with the first person, which has the potential to differ in terms of volitionality. We can therefore identify a relative scale with respect to the factors, where affectedness is about the event and volitionality can be about the event, but where the lexical semantics of certain items restricts the possibilities for its application.

\subsubsection[Other examples of semantically determined functions in Abui]{\bfseries Other examples of semantically determined functions in Abui}
Due to the high degree of fluidity of the Abui prefixation system, we can only highlight a few cases here where the choice of the prefix is determined by semantics. Naturally, given the fluid nature of the system, it is possible for a verb to alternate in the number of arguments it takes, and this may be reflected in the choice of prefix. The \textsc{loc} prefix indexes a P argument in a clause. It may add an argument to verbs which normally have only one. That is, it adds a P. This is illustrated with the verb \textit{aisa} {\textquoteleft}urinate{\textquoteright} in () and the associated verb \textit{he-aisa} {\textquoteleft}urinate on{\textquoteright} in (), which has two arguments:

Abui (Schapper, fieldnotes)

\begin{flushleft}
\tablehead{}
\begin{supertabular}{m{0.43165985in}m{0.5712598in}m{0.96015984in}}
\label{bkm:Ref324775992}() &
\itshape Simon &
\itshape ais-a.\\
 &
Simon &
urinate-\textsc{ipfv}\\
 &
\multicolumn{2}{m{1.6101599in}}{{\textquoteleft}Simon is urinating.{\textquoteright}}\\
\end{supertabular}
\end{flushleft}
Abui (Schapper, fieldnotes)

\begin{flushleft}
\tablehead{}
\begin{supertabular}{m{0.43165985in}m{0.5705598in}m{1.2573599in}m{1.4031599in}m{1.4455599in}}
\label{bkm:Ref324776040}() &
\itshape Simon &
\itshape he-kaik &
\itshape he-ais-a. &
\\
 &
Simon\textsc{ } &
3.\textsc{al}.\textsc{poss}{}-arrow &
3.\textsc{loc}{}-urinate-\textsc{ipfv} &
\\
 &
\multicolumn{4}{m{4.91286in}}{{\textquoteleft}Simon is urinating on his arrows (to infuse them with magical power in preparation for battle).{\textquoteright}}\\
\end{supertabular}
\end{flushleft}
On verbs which have two arguments associated with them the \textsc{loc} prefix can be used for specific locations. Here the prefix does not increase the number of arguments, but rather it provides additional information about the P. For instance, in () where the speaker is struck in general, there is no prefix on \textit{balas} {\textquoteleft}strike{\textquoteright}, whereas with the prefix in () it is specifically the speaker{\textquoteright}s leg that is struck.

\clearpage
Abui (Schapper, fieldnotes)

\begin{flushleft}
\tablehead{}
\begin{supertabular}{m{0.43165985in}m{0.6643598in}m{0.40665984in}m{0.36775985in}m{0.40665984in}m{0.8670598in}}
\label{bkm:Ref324784430}() &
\itshape Markus &
\itshape nel &
\itshape bol &
\itshape nel &
\itshape balas-a.\\
 &
Markus\textsc{ } &
1\textsc{sg} &
hit\ \  &
1\textsc{sg} &
strike-\textsc{ipfv}\\
 &
\multicolumn{5}{m{3.0274599in}}{{\textquoteleft}Markus hit and bashed me.{\textquoteright}}\\
\end{supertabular}
\end{flushleft}
Abui (Schapper, fieldnotes)

\begin{flushleft}
\tablehead{}
\begin{supertabular}{m{0.42055985in}m{0.6150598in}m{1.2101599in}m{0.53235984in}m{1.2663599in}m{0.41435984in}m{0.79205984in}}
\label{bkm:Ref324784436}() &
\itshape Baloku &
\itshape ne-toku &
\itshape beeka &
\itshape he-balas-i &
\itshape ba &
\itshape wea-di.\\
 &
grass &
1\textsc{sg}.\textsc{al}.\textsc{poss}{}-leg &
bad &
3.\textsc{loc}{}-strike-\textsc{pfv} &
\scshape lnk &
bleed-\textsc{pfv}\\
 &
\multicolumn{5}{m{4.35326in}}{{\textquoteleft}The grass struck my bad leg and it became bloody.{\textquoteright}} &
\\
\end{supertabular}
\end{flushleft}
The \textsc{goal} series contrasts with the \textsc{loc} series in that the latter denotes a location which is a semantic patient of the verb, as in (), whereas the former denotes a location at or towards which the action of the verb occurs, as in ():

Abui (Schapper, fieldnotes)

\begin{flushleft}
\tablehead{}
\begin{supertabular}{m{0.43165985in}m{0.40665984in}m{1.4066598in}}
\label{bkm:Ref324777205}() &
\itshape Na &
\itshape he-ais-a.\\
 &
1\textsc{sg} &
3.\textsc{loc}{}-urinate-\textsc{ipfv}\\
 &
\multicolumn{2}{m{1.8920598in}}{{\textquoteleft}I{\textquoteright}m urinating on it.{\textquoteright}}\\
\end{supertabular}
\end{flushleft}
Abui (Schapper, fieldnotes)

\begin{flushleft}
\tablehead{}
\begin{supertabular}{m{0.43165985in}m{0.27265984in}m{0.78305984in}m{0.78375983in}}
\multicolumn{2}{m{0.7830598in}}{\label{bkm:Ref324777215}()} &
\itshape Na &
\itshape hoo-ais-a.\\
\multicolumn{2}{m{0.7830598in}}{} &
1\textsc{sg} &
3.\textsc{goal}{}-urinate-\textsc{ipfv}\\
 &
\multicolumn{3}{m{1.9969599in}}{{\textquoteleft}I{\textquoteright}m urinating at it.{\textquoteright}}\\
\end{supertabular}
\end{flushleft}
The \textsc{ben} prefix indexes a P argument in a clause. That is, on a verb which typically has one argument, it adds a P, while on a verb which typically has two arguments, it replaces the P-like argument with a benefactive. A very common use of the \textsc{ben} series is to add a participant which is the reason for the action denoted by the verb or on whose account the action denoted by the verb occurs. This is the case for verbs which typically represent one-place or two-place predicates. In () \textit{burook} {\textquoteleft}move{\textquoteright} is a verb with typically one argument, but with a \textsc{ben} inflection it has a second participant denoting the reason for the action, as in (). 

Abui (Schapper, fieldnotes)

\begin{flushleft}
\tablehead{}
\begin{supertabular}{m{0.42885986in}m{0.5511598in}m{1.2663599in}m{0.46775988in}m{0.7483598in}m{0.66775984in}}
\label{bkm:Ref324776788}() &
\itshape Bataa &
\itshape ha-t\'ang &
\itshape dara &
\itshape oro &
\itshape burook.\\
 &
tree &
3.\textsc{inal}.\textsc{poss}{}-arm &
still &
\scshape across &
move\\
 &
\multicolumn{4}{m{3.26986in}}{{\textquoteleft}A tree branch is still moving over there.{\textquoteright}} &
\\
\end{supertabular}
\end{flushleft}
Abui (Schapper, fieldnotes)

\begin{flushleft}
\tablehead{}
\begin{supertabular}{m{0.43165985in}m{0.7122598in}m{0.40665984in}m{1.1386598in}m{0.48725984in}m{1.1663599in}}
\label{bkm:Ref353455604}() &
\itshape Na &
\itshape edo &
\itshape ee-burook &
\itshape naha &
\itshape do!\\
 &
1\textsc{sg}.\textsc{act} &
2\textsc{sg} &
2\textsc{sg.ben}{}-move &
\scshape neg &
\scshape dem\\
 &
\multicolumn{4}{m{2.9810598in}}{{\textquoteleft}I{\textquoteright}m not moving on your account.{\textquoteright}} &
\\
\end{supertabular}
\end{flushleft}
In () below the verb \textit{akeen}\textit{(}\textit{g)} {\textquoteleft}struggle, fight{\textquoteright} occurs without a prefix and with a P denoting a patient, i.e., the one who is being fought. However, with the \textsc{ben} inflection the P is the reason for the struggling, as in ().

Abui (Schapper, fieldnotes)

\begin{flushleft}
\tablehead{}
\begin{supertabular}{m{0.43165985in}m{0.62755984in}m{0.66775984in}m{1.0268599in}m{2.0837598in}}
\label{bkm:Ref324777275}() &
\itshape Tafuda &
\itshape oro &
\itshape Kafola=ng &
\itshape akeen-i.\\
 &
All &
\scshape across &
Kabola\textsc{ =loc} &
struggle-\textsc{pfv}\\
 &
\multicolumn{4}{m{4.64216in}}{{\textquoteleft}All fought against Kabola over there.{\textquoteright}}\\
\end{supertabular}
\end{flushleft}
Abui (Schapper, fieldnotes)

\begin{flushleft}
\tablehead{}
\begin{supertabular}{m{0.43165985in}m{0.47815987in}m{0.50185984in}m{0.36775985in}m{3.75046in}}
\label{bkm:Ref324777283}() &
\itshape Kaai &
\itshape di &
\itshape fe &
\itshape hee-akeeng.\\
 &
dog &
3.\textsc{act} &
pig &
3.\textsc{ben}{}-struggle\\
 &
\multicolumn{4}{m{5.3344603in}}{{\textquoteleft}The dog is struggling on account of the pig (i.e., fighting to get free so as to be able to attack the pig).{\textquoteright}}\\
\end{supertabular}
\end{flushleft}
In sum, Abui has a high degree of semantic fluidity, and prefixation patterns depend on the factors affectedness and volitionality. We now turn to the neighboring language Kamang, in which arbitrary inflection classes (at least synchronically) play a larger role than in Abui.

\subsection[Kamang]{Kamang}
Kamang, like Abui, has semantic alignment and several prefix series. However, in Kamang the actual use of prefixes differs radically from Abui. Kamang is more restricted in terms of the possible combinations of verbs with prefixes than Abui. More than in Abui, lexical classes in Kamang play an important role in determining prefixation patterns of S in intransitive clauses and P in transitive clauses. We have based our analysis of Kamang on a corpus of 510 verbs. In Kamang the primary verb class divide is between: 

(i)\ \ \textit{Obligatorily prefixed verbs}: These require a prefix on the verb in order to be well-formed. The prefix comes from one of the six series, is lexically fixed for each verb and does not alternate. For verbs in this group the different prefixal inflections have no obvious semantic functions, but rather define arbitrary inflection classes. Of the 510 verbs, 166 are obligatorily prefixed (approx. 33\%).

(ii) \textit{Non-obligatorily prefixed verbs}: These do not require a prefix. Where prefixes are added to these verbs they have semantically transparent functions. Prefixation can either be argument-preserving, whereby prefixation of the verb does not add another argument or alter the valency of the verb, or argument-adding, whereby the prefix indexes an additional argument. 344 verbs belong into this class (approx. 67\%).

We see in Table 6 below that there is a substantial difference in the prefixal requirements of transitive and intransitive verbs (all percentages rounded to whole numbers). In the classification of verbs as either intransitive or transitive we follow Schapper and Manimau (2011).

Table 6: Kamang verbs (obligatorily prefixed and non-obligatorily prefixed)

\begin{flushleft}
\tablehead{}
\begin{supertabular}{|m{1.7518599in}|m{2.3295598in}|m{2.1004598in}|}
\hline
 &
Obligatorily prefixed &
Non-obligatorily prefixed\\\hline
Transitive &
45\% (113/250 verbs) &
55\% (137/250 verbs)\\\hline
Intransitive  &
20\% (53/260 verbs) &
80\% (207/260 verbs)\\\hline
Total (of 510 verbs) &
33\% (166/510 verbs) &
67\% (344/510 verbs)\\\hline
\end{supertabular}
\end{flushleft}
Almost half of the transitive verbs that we sampled from the corpus are obligatorily prefixed, whereas substantially fewer of the intransitive verbs (only 20\%) are. 

\subsubsection[Inflection classes in Kamang]{\bfseries Inflection classes in Kamang}
About one third of the verbs in Kamang are obligatorily prefixed and fall into arbitrary inflection classes. All of these verbs require a prefix and the prefix series is lexically fixed and independent of verb semantics. 

Table 7 presents the percentages of obligatorily prefixed intransitive verbs across inflection classes (rounded to whole numbers). The prefix indexes S. Well over half occur in the \textsc{pat} inflection, whereas less than one fifth goes in each of the \textsc{loc} and \textsc{gen} inflection classes. The remainder is made up of the \textsc{ast} class. There are no instances of obligatorily prefixed intransitive verbs outside these four inflection classes.

Table 7: Proportion of obligatorily prefixed intransitive verbs by prefix class

\begin{flushleft}
\tablehead{}
\begin{supertabular}{|m{1.6226599in}|m{1.5337598in}|m{1.5337598in}|m{1.4136599in}|}
\hline
\scshape pat &
\textsc{loc}  &
\textsc{gen}  &
\scshape ast\\\hline
65\% (33 verbs) &
15\% (8 verbs) &
18\% (11 verbs) &
{\textless}2\% (1 verb)\\\hline
\end{supertabular}
\end{flushleft}
Table 8 presents the percentages of obligatorily prefixed transitive verbs across inflection classes (rounded to whole numbers). The prefix indexes P. Over half of these verbs belong to the \textsc{loc} inflection, while roughly 35\% are in the \textsc{pat} inflection. The remainder is made up by a handful of transitive verbs from the other four inflections.

Table 8: Proportion of obligatorily prefixed transitive verbs by prefix class

\begin{flushleft}
\tablehead{}
\begin{supertabular}{|m{2.1511598in}|m{2.15256in}|m{1.8781599in}|}
\hline
\textsc{pat}  &
\textsc{loc}  &
Other\\\hline
35\% (46 verbs) &
60\% (82 verbs) &
{\textless}5\% (9 verbs)\\\hline
\end{supertabular}
\end{flushleft}
The distribution of verbs over these classes is independent of verb semantics. Within the obligatorily prefixed intransitive verbs {}-\textit{waawang} {\textquoteleft}remember{\textquoteright}, -\textit{mitan} {\textquoteleft}understand{\textquoteright} and \textit{{}-pan} {\textquoteleft}forget{\textquoteright} have similar semantics, yet they belong to the inflection classes \textsc{pat}, \textsc{gen} and \textsc{ast}, respectively. Similarly, {}-\textit{iwei} {\textquoteleft}vomit{\textquoteright}, \textit{{}-tasusin} {\textquoteleft}be sweaty{\textquoteright} and -\textit{wilii} {\textquoteleft}defecate{\textquoteright} belong to the classes \textsc{pat,} \textsc{loc} and \textsc{gen}. Within the obligatorily prefixed transitive verbs -\textit{set} {\textquoteleft}shake up and down{\textquoteright} belongs to \textsc{pat}, while \textit{gaook} {\textquoteleft}shake back and forth{\textquoteright} belongs to \textsc{loc}. Similarly, -\textit{kut} {\textquoteleft}stab{\textquoteright} belongs to \textsc{
pat} and \textit{{}-fanee} {\textquoteleft}strike, shoot{\textquoteright} to \textsc{gen}. The inflection classes \textsc{dat} and \textsc{dir} contain one verb each and are therefore too small for any common semantics to be discernible.

In the following examples we illustrate the inflection classes in Kamang. For each class we give an intransitive and a transitive example and provide a list of verbs so the reader can further appreciate that classing is independent of verb semantics.

Examples () and () show an intransitive verb encoding S with a \textsc{pat} prefix and a transitive verb encoding P with a \textsc{pat} prefix, respectively:

Kamang (Schapper and Manimau 2011:113)

\begin{flushleft}
\tablehead{}
\begin{supertabular}{m{0.43165985in}m{1.2275599in}m{1.7386599in}}
\multicolumn{2}{m{1.7379599in}}{\label{bkm:Ref324337569}()} &
\itshape Na-serang-si\\
\multicolumn{2}{m{1.7379599in}}{} &
1\textsc{sg}.\textsc{pat}{}-get\_up-\textsc{ipfv}\\
 &
\multicolumn{2}{m{3.04496in}}{{\textquoteleft}I{\textquoteright}m getting up.{\textquoteright}}\\
\end{supertabular}
\end{flushleft}
Kamang (Schapper and Manimau 2011:73)

\begin{flushleft}
\tablehead{}
\begin{supertabular}{m{0.43165985in}m{0.40455985in}m{1.0219599in}m{0.085159846in}m{0.17195985in}}
\label{bkm:Ref324338165}() &
\itshape Gal &
\itshape na-kut. &
 &
\\
 &
3 &
1\textsc{sg.pat-}stab &
 &
\\
 &
\multicolumn{4}{m{1.9198599in}}{{\textquoteleft}He stabbed me.{\textquoteright}}\\
\end{supertabular}
\end{flushleft}
\clearpage
Examples of intransitive verbs in the \textsc{pat} inflectional class are given in (): 

Kamang (Schapper and Manimau 2011)

\begin{flushleft}
\tablehead{}
\begin{supertabular}{|m{0.43165985in}|m{3.0281599in}|m{2.67886in}|}
\hline
\label{bkm:Ref353454134}() &
\textit{ah} {\textquoteleft}eat{\textquoteright} &
\textit{mantei} {\textquoteleft}thirsty{\textquoteright}\\\hline
 &
\textit{bo{\textquoteright}ra} {\textquoteleft}die (of humans){\textquoteright} &
\textit{mara} {\textquoteleft}sound, make a sound{\textquoteright}\\\hline
 &
\textit{iloi} {\textquoteleft}feel nauseous{\textquoteright} &
\textit{ook} {\textquoteleft}shiver, tremble{\textquoteright}\\\hline
 &
\textit{iwei} {\textquoteleft}vomit{\textquoteright} &
\textit{pa }{\textquoteleft}have fun{\textquoteright}\\\hline
 &
\textit{leeng} {\textquoteleft}slow, careful{\textquoteright} &
\textit{serang }{\textquoteleft}get up{\textquoteright}\\\hline
 &
\textit{maai} {\textquoteleft}sink{\textquoteright} &
\textit{tan }{\textquoteleft}collapse, fall over{\textquoteright}\\\hline
 &
\textit{maaung} {\textquoteleft}want, like{\textquoteright} &
\textit{waawang }{\textquoteleft}remember{\textquoteright}\\\hline
 &
\textit{maitang} {\textquoteleft}hungry{\textquoteright} &
\\\hline
\end{supertabular}
\end{flushleft}
Examples of transitive verbs in the \textsc{pat} inflectional class are provided in (): 

Kamang (Schapper and Manimau 2011)

\begin{flushleft}
\tablehead{}
\begin{supertabular}{|m{0.43165985in}|m{3.0281599in}|m{2.67886in}|}
\hline
\label{bkm:Ref353454143}() &
\textit{asui} {\textquoteleft}disturb s.o./sth.{\textquoteright} &
\textit{sama} {\textquoteleft}be considerate of s.o./sth.{\textquoteright}\\\hline
 &
\textit{beh} {\textquoteleft}order s.o./sth.{\textquoteright} &
\textit{sara} {\textquoteleft}scatter sth.{\textquoteright}\\\hline
 &
\textit{bei} {\textquoteleft}abuse s.o./sth. verbally{\textquoteright} &
\textit{set} {\textquoteleft}shake s.o./sth. up and down{\textquoteright}\\\hline
 &
\textit{engda} {\textquoteleft}reply to s.o./sth.{\textquoteright} &
\textit{sooran} {\textquoteleft}push s.o./sth.{\textquoteright}\\\hline
 &
\textit{feesa} {\textquoteleft}restrict, put pressure on s.o./sth.{\textquoteright} &
\textit{suh} {\textquoteleft}collide with, push s.o./sth. together{\textquoteright}\\\hline
 &
\textit{gai} {\textquoteleft}associate with s.o./sth.{\textquoteright} &
\textit{suka} {\textquoteleft}have sex with s.o./sth.{\textquoteright} \\\hline
 &
\textit{kila} {\textquoteleft}differ from s.o./sth.{\textquoteright} &
\textit{suma} {\textquoteleft}compare to s.o./sth.{\textquoteright}\\\hline
 &
\textit{kosilaai} {\textquoteleft}rub (a pig{\textquoteright}s) stomach slowly so that it sleeps{\textquoteright} &
\textit{tafanee} {\textquoteleft}receive from s.o./sth.{\textquoteright}\\\hline
 &
\textit{kut} {\textquoteleft}stab s.o./sth.{\textquoteright} &
\textit{tak} {\textquoteleft}see s.o./sth.{\textquoteright}\\\hline
 &
\textit{reide} {\textquoteleft}wait for s.o./sth.{\textquoteright} &
\textit{tan} {\textquoteleft}wake s.o./sth. up{\textquoteright}\\\hline
 &
\textit{rot} {\textquoteleft}cut off (piece of cloth){\textquoteright} &
\textit{toka} {\textquoteleft}peck, lunge at s.o./sth. with head{\textquoteright}\\\hline
 &
\textit{saa} {\textquoteleft}pour sth. (liquid){\textquoteright} &
\textit{tota} {\textquoteleft}support s.o./sth.{\textquoteright}\\\hline
\end{supertabular}
\end{flushleft}
Examples () and () below show an intransitive verb encoding S with a \textsc{loc} prefix and a transitive verb encoding P with a \textsc{loc} prefix respectively:

Kamang (Schapper and Manimau 2011: 286)

\begin{flushleft}
\tablehead{}
\begin{supertabular}{m{0.43165985in}m{1.1476599in}m{0.87755984in}}
\label{bkm:Ref324338359}() &
\itshape No-tasusing. &
\\
 &
\textsc{1sg.loc}{}-sweat &
\\
 &
\multicolumn{2}{m{2.10396in}}{{\textquoteleft}I{\textquoteright}m sweaty.{\textquoteright}}\\
\end{supertabular}
\end{flushleft}
Kamang (Schapper and Manimau 2011: 50)

\begin{flushleft}
\tablehead{}
\begin{supertabular}{m{0.43165985in}m{0.5094598in}m{0.81435984in}m{2.06636in}m{0.5691598in}}
\label{bkm:Ref324758916}() &
\itshape Ga &
\itshape bong=a &
\itshape wo-gaook. &
\\
 &
\scshape 3.agt &
tree=\textsc{spec} &
3.\textsc{loc}{}-shake\_back\_and\_forth &
\\
 &
\multicolumn{4}{m{4.19556in}}{{\textquoteleft}He shook the tree.{\textquoteright}}\\
\end{supertabular}
\end{flushleft}
\clearpage
Examples of intransitive verbs in the \textsc{loc} inflectional class are given in (): 

Kamang (Schapper and Manimau 2011)

\begin{flushleft}
\tablehead{}
\begin{supertabular}{|m{0.43165985in}|m{2.7386599in}|}
\hline
\label{bkm:Ref353451748}() &
\textit{biee} {\textquoteleft}angry{\textquoteright}\\\hline
 &
\textit{garit} {\textquoteleft}be caught by surprise, shocked{\textquoteright}\\\hline
 &
\textit{moosa} {\textquoteleft}half dead, mortally wounded{\textquoteright}\\\hline
 &
\textit{patak} {\textquoteleft}break away{\textquoteright}\\\hline
 &
\textit{tasusin} {\textquoteleft}sweaty{\textquoteright}\\\hline
 &
\textit{uka} {\textquoteleft}dry (of animates){\textquoteright}\\\hline
 &
\textit{waai} {\textquoteleft}worn out, fed up, tired{\textquoteright}\\\hline
\end{supertabular}
\end{flushleft}
Examples of transitive verbs in the \textsc{loc} inflectional class are provided in (): 

Kamang (Schapper and Manimau 2011)

\begin{flushleft}
\tablehead{}
\begin{supertabular}{|m{0.48795983in}|m{2.5781598in}|m{2.54066in}|}
\hline
\label{bkm:Ref353454210}() &
\textit{aakai} {\textquoteleft}trap, trick s.o./sth.{\textquoteright} &
\textit{musan} {\textquoteleft}sniff, smell s.o./sth.{\textquoteright}\\\hline
 &
\textit{baa }{\textquoteleft}make sth.{\textquoteright} &
\textit{o} {\textquoteleft}follow, hunt s.o./sth.{\textquoteright}\\\hline
 &
\textit{baleela }{\textquoteleft}wrapped around sth.{\textquoteright} &
\textit{paasa} {\textquoteleft}stick on/to s.o./sth.{\textquoteright}\\\hline
 &
\textit{balkei} {\textquoteleft}sell s.o./sth.{\textquoteright} &
\textit{pak} {\textquoteleft}cover up/over s.o./sth.{\textquoteright}\\\hline
 &
\textit{eh} {\textquoteleft}measure sth.{\textquoteright} &
\textit{pakah} {\textquoteleft}carry sth. heavy using two hands{\textquoteright}\\\hline
 &
\textit{fale{\textquoteright}la} {\textquoteleft}be stuck to/on (s.o./sth.){\textquoteright} &
\textit{pan} {\textquoteleft}climb (sth.){\textquoteright}\\\hline
 &
\textit{foina} {\textquoteleft}dream about (s.o./sth.){\textquoteright} &
\textit{pukan} {\textquoteleft}guard (s.o./sth.){\textquoteright}\\\hline
 &
\textit{furat} {\textquoteleft}quickly sip at (a hot liquid){\textquoteright} &
\textit{ra} {\textquoteleft}carry (s.o./sth.){\textquoteright}\\\hline
 &
\textit{gasam} {\textquoteleft}store (sth.){\textquoteright} &
\textit{rida} {\textquoteleft}point out, indicate (s.o./sth.){\textquoteright}\\\hline
 &
\textit{gatan} {\textquoteleft}put on, load (s.o./sth.){\textquoteright} &
\textit{saidi} {\textquoteleft}brush, sweep off (sth.){\textquoteright}\\\hline
 &
\textit{gayat} {\textquoteleft}store with (s.o.){\textquoteright} &
\textit{sak} {\textquoteleft}dry (sth.) in sun{\textquoteright}\\\hline
 &
\textit{gayau} {\textquoteleft}collect up (sth.){\textquoteright} &
\textit{sire} {\textquoteleft}wash, clean (s.o./sth.) with water{\textquoteright}\\\hline
 &
\textit{ilam} {\textquoteleft}pay attention to (s.o./sth.){\textquoteright} &
\textit{subau} {\textquoteleft}blow on (sth.){\textquoteright}\\\hline
 &
\textit{kawai} {\textquoteleft}massage (s.o.){\textquoteright} &
\textit{sukui} {\textquoteleft}make a hole in (sth.){\textquoteright}\\\hline
 &
\textit{kik} {\textquoteleft}winnow (corn, rice, grain etc.){\textquoteright} &
\textit{sumee} {\textquoteleft}imitate, copy (s.o./sth.){\textquoteright}\\\hline
 &
\textit{kilai} {\textquoteleft}split up, divide (sth.){\textquoteright} &
\textit{taabe} {\textquoteleft}honour (s.o./sth.){\textquoteright}\\\hline
 &
\textit{lakuui} {\textquoteleft}razor, cut (hair){\textquoteright} &
\textit{tuk} {\textquoteleft}cover head with (sth.){\textquoteright}\\\hline
 &
\textit{loh} {\textquoteleft}swarm over (s.o./sth.){\textquoteright} &
\textit{um} {\textquoteleft}gossip with/to (s.o.){\textquoteright}\\\hline
 &
\textit{masela} {\textquoteleft}kiss, rub nose/cheek of (s.o.){\textquoteright} &
\textit{waa} {\textquoteleft}inspect{\textquoteright}\\\hline
 &
\textit{mota} {\textquoteleft}lean on (s.o./sth.){\textquoteright} &
\\\hline
\end{supertabular}
\end{flushleft}
Examples () and () below show an intransitive verb encoding S with a \textsc{gen} prefix and a transitive verb encoding P with a \textsc{gen} prefix respectively:

Kamang (Schapper, fieldnotes)

\begin{flushleft}
\tablehead{}
\begin{supertabular}{m{0.43165985in}m{1.8038598in}m{0.14835984in}}
\label{bkm:Ref353451839}() &
\itshape Ne-soona-ma. &
\\
 &
1\textsc{sg.gen}{}-slip-\textsc{pfv} &
\\
 &
\multicolumn{2}{m{2.0309598in}}{{\textquoteleft}I slipped over.{\textquoteright}}\\
\end{supertabular}
\end{flushleft}
\clearpage
Kamang (Schapper, fieldnotes)

\begin{flushleft}
\tablehead{}
\begin{supertabular}{m{0.43165985in}m{0.49765983in}m{1.4788599in}}
\label{bkm:Ref324339697}() &
\itshape Leon &
\itshape ne-fanee-si.\\
 &
Leon &
1\textsc{sg}.\textsc{gen}{}-shoot-\textsc{ipfv}\\
 &
\multicolumn{2}{m{2.05526in}}{{\textquoteleft}Leon shoots at me.{\textquoteright}}\\
\end{supertabular}
\end{flushleft}
Examples of intransitive verbs in the \textsc{gen} inflectional class are: \textit{biee }{\textquoteleft}afraid{\textquoteright}, \textit{{}-foi} {\textquoteleft}dream{\textquoteright}, \textit{iyaa} {\textquoteleft}go home{\textquoteright}, \textit{laita} {\textquoteleft}shy{\textquoteright}, \textit{laka} {\textquoteleft}be naked{\textquoteright}, \textit{reitang} {\textquoteleft}be troubled, feel sick{\textquoteright}, \textit{soona }{\textquoteleft}slip over{\textquoteright}, \textit{{}-suu} {\textquoteleft}go first, lead the way{\textquoteright}, \textit{{}-taiyai} {\textquoteleft}cooperate, work together{\textquoteright},\textit{ wilii} {\textquoteleft}defecate{\textquoteright}.

\ \ There are only two transitive verbs in the \textsc{gen} inflectional class, namely \textit{fanee} {\textquoteleft}strike, shoot{\textquoteright} and \textit{towan} {\textquoteleft}carry on a pole between two people{\textquoteright}.

Examples () and () below show an intransitive verb encoding S with an \textsc{ast} prefix and a transitive verb encoding P with an \textsc{ast} prefix respectively:

Kamang (Schapper and Manimau 2011: 103)

\begin{flushleft}
\tablehead{}
\begin{supertabular}{m{0.43165985in}m{1.4927598in}m{0.44695982in}}
\label{bkm:Ref324340307}\label{bkm:Ref372879178}() &
\itshape Oo-pan-si &
\itshape naa.\\
 &
2\textsc{sg.ast}{}-forget-\textsc{ipfv} &
\scshape neg\\
 &
\multicolumn{2}{m{2.0184598in}}{{\textquoteleft}Don{\textquoteright}t you forget.{\textquoteright}}\\
\end{supertabular}
\end{flushleft}
Kamang\textsc{ }(Schapper and Manimau 2011: 131)

\begin{flushleft}
\tablehead{}
\begin{supertabular}{m{0.43165985in}m{0.48795983in}m{0.9254598in}m{0.9656598in}m{1.0483599in}}
\label{bkm:Ref324340314}() &
\itshape Dum &
\itshape kiding=a &
\itshape ga-filing &
\itshape woo-tee.\\
 &
child &
small=\textsc{spec} &
\textsc{3.poss}{}-head &
3.\textsc{ast}{}-protect\\
 &
\multicolumn{4}{m{3.6636598in}}{{\textquoteleft}The child protected his head.{\textquoteright}}\\
\end{supertabular}
\end{flushleft}
The number of verbs in the \textsc{ast} inflectional class is very small. Transitive verbs are \textit{sui} {\textquoteleft}dry off{\textquoteright}, \textit{tee} {\textquoteleft}protect{\textquoteright}, and \textit{waai} {\textquoteleft}be facing, look out onto{\textquoteright}. There is only one intransitive verb \textit{pan} {\textquoteleft}forget{\textquoteright}. Like other cognition verbs and sensory perception verbs in Kamang (e.g., \textit{mitan} {\textquoteleft}understand{\textquoteright}, \textit{{}-mai} {\textquoteleft}hear{\textquoteright}) this verb is intransitive. This is seen by the fact that it is unable to occur with an NP encoding the stimulus in its basic form, such as that in example () above, as shown in (). The stimulus or better said that which is to be remembered must be retrieved simply from the discourse context. To explicitly include an extra participant with such a verb is possible in two ways: (i) by using an applicative morpheme, such as \textit{wo-} in (), or (ii) by having a 
complement clause following the clause with the cognition verb, as in (). 

Kamang (Schapper fieldnotes)

\begin{flushleft}
\tablehead{}
\begin{supertabular}{m{0.5344598in}m{0.26705986in}m{0.66775984in}m{0.13375986in}m{0.8809598in}m{0.6802598in}m{0.121259846in}m{0.07815985in}}
\label{bkm:Ref372879210}() &
\multicolumn{2}{m{1.0135599in}}{*\textit{Mooi}} &
\multicolumn{3}{m{1.8524599in}}{\itshape oo-pan-si} &
\itshape naa. &
\\
 &
\multicolumn{2}{m{1.0135599in}}{banana} &
\multicolumn{3}{m{1.8524599in}}{\textsc{2sg.ast}{}-forget-\textsc{ipfv}} &
\scshape neg &
\\
\multicolumn{2}{m{0.8802598in}}{} &
\multicolumn{5}{m{2.79896in}}{{\textquoteleft}Don{\textquoteright}t you forget the bananas.{\textquoteright}} &
\\
\multicolumn{2}{m{0.8802598in}}{} &
\multicolumn{5}{m{2.79896in}}{} &
\\
\multicolumn{2}{m{0.8802598in}}{} &
\multicolumn{5}{m{2.79896in}}{} &
\\
\label{bkm:Ref372879221}() &
\multicolumn{2}{m{1.0135599in}}{\itshape Mooi} &
\multicolumn{5}{m{2.20936in}}{\itshape wo-oo-pan-si}\\
 &
\multicolumn{2}{m{1.0135599in}}{banana} &
\multicolumn{5}{m{2.20936in}}{\textsc{appl-2sg.ast}{}-forget-\textsc{ipfv}}\\
\multicolumn{2}{m{0.8802598in}}{} &
\multicolumn{5}{m{2.79896in}}{{\textquoteleft}Don{\textquoteright}t you forget the bananas.{\textquoteright}} &
\\
\multicolumn{2}{m{0.8802598in}}{} &
\multicolumn{5}{m{2.79896in}}{} &
\\
\multicolumn{2}{m{0.8802598in}}{} &
\multicolumn{5}{m{2.79896in}}{} &
\\
\multicolumn{2}{m{0.8802598in}}{\label{bkm:Ref372879227}()} &
\multicolumn{2}{m{0.8802598in}}{\itshape Oo-pan-si} &
\itshape naa &
\multicolumn{2}{m{0.8802598in}}{\itshape mooi} &
\itshape met.\\
\multicolumn{2}{m{0.8802598in}}{} &
\multicolumn{2}{m{0.8802598in}}{\textsc{2sg.ast}{}-forget-\textsc{ipfv}} &
\scshape neg &
\multicolumn{2}{m{0.8802598in}}{banana} &
take\\
\multicolumn{2}{m{0.8802598in}}{} &
\multicolumn{5}{m{2.79896in}}{{\textquoteleft}Don{\textquoteright}t you forget to bring the bananas.{\textquoteright}} &
\\
\end{supertabular}
\end{flushleft}
Class size decreases even further in the inflection classes \textsc{dat} with \textit{sah} {\textquoteleft}block, prohibit{\textquoteright} and \textsc{dir} with \textit{surut} {\textquoteleft}chase{\textquoteright}. They each include a single transitive verb only. There are no intransitive verbs in either \textsc{dat} or \textsc{dir}.

\ \ To sum up, obligatorily prefixed verbs in Kamang fall into inflection classes. Synchronically, there is no semantically transparent reason why one prefixal inflection is used with one verb and another inflection with another one. The relation between prefix and verb is simply lexically fixed. None of these verbs can ever occur without a prefix. 

We now turn to prefixation in non-obligatorily prefixed verbs and the semantic factor of affectedness which has an effect on the prefixation patterns. 

\subsubsection[Non{}-obligatorily prefixed verbs in Kamang]{\bfseries Non-obligatorily prefixed verbs in Kamang}
Non-obligatorily prefixed Kamang verbs allow the full range of prefixes except \textsc{pat} and \textsc{gen}. They can also appear without a prefix. The addition of a prefix always adds an argument. For instance, the intransitive verb \textit{silanta} {\textquoteleft}wail{\textquoteright} can appear without a prefix, as in (), or with different prefixes encoding different semantic kinds of P participants, for instance a location (), a beneficiary () or an assisted participant (). 

Kamang (Schapper, to appear)

\begin{flushleft}
\tablehead{}
\begin{supertabular}{m{0.43165985in}m{0.6643598in}m{0.8302598in}}
\label{bkm:Ref324766124}() &
\itshape Markus &
\itshape silanta.\\
 &
Markus\textsc{ } &
wail\\
 &
\multicolumn{2}{m{1.5733598in}}{{\textquoteleft}Markus wails.{\textquoteright}}\\
\end{supertabular}
\end{flushleft}
Kamang (Schapper, to appear)

\begin{flushleft}
\tablehead{}
\begin{supertabular}{m{0.43165985in}m{0.6643598in}m{1.0594599in}}
\label{bkm:Ref324766139}() &
\itshape Markus &
\itshape no-silanta.\\
 &
Markus\textsc{ } &
1\textsc{sg.loc}{}-wail\\
 &
\multicolumn{2}{m{1.8025599in}}{{\textquoteleft}Markus wails over me.{\textquoteright}}\\
\end{supertabular}
\end{flushleft}
Kamang (Schapper, to appear)

\begin{flushleft}
\tablehead{}
\begin{supertabular}{m{0.43165985in}m{0.6643598in}m{1.4309598in}}
\label{bkm:Ref324766148}() &
\itshape Markus &
\textit{nee-silanta.}\ \ \\
 &
Markus\textsc{ } &
1\textsc{sg.dat}{}-wail\ \ \\
 &
\multicolumn{2}{m{2.1740599in}}{{\textquoteleft}Markus wails in want of me.{\textquoteright}}\\
\end{supertabular}
\end{flushleft}
Kamang (Schapper, to appear)

\begin{flushleft}
\tablehead{}
\begin{supertabular}{m{0.43165985in}m{0.6643598in}m{1.0400599in}m{0.59345984in}}
\label{bkm:Ref324766154}() &
\itshape Markus &
\itshape noo-silanta. &
\\
 &
Markus\textsc{ } &
1\textsc{sg.ast}{}-wail &
\\
 &
\multicolumn{3}{m{2.45536in}}{{\textquoteleft}Markus wails with my assistance.{\textquoteright}}\\
\end{supertabular}
\end{flushleft}
Affectedness can be identified as a semantic factor which plays a role in argument indexing in non-obligatorily prefixed verbs in Kamang. It is a property expressing a relationship between participants and events and it is always expressed by argument-preserving prefixes, i.e., prefixation of the verb does not add another argument or alter the valency of the verb. The prefix indexes the S of intransitive verbs or the P of transitive verbs. Stative verbs like \textit{saara} {\textquoteleft}burn{\textquoteright} or \textit{suusa} {\textquoteleft}be in difficulty{\textquoteright} take a \textsc{loc} prefix to express that the S is affected. In (), the S is affected in its entirety. Kamang expresses this by indexing the S with a \textsc{loc} prefix on the verb. On the other hand, in (), where the S is less affected, the prefix is absent.

\clearpage
Kamang (Schapper, to appear)

\begin{flushleft}
\tablehead{}
\begin{supertabular}{m{0.43165985in}m{0.7559598in}m{0.39555985in}m{2.06636in}m{0.5698598in}}
\label{bkm:Ref306280872}() &
\itshape Kik &
\itshape nok &
\itshape wo-saara. &
\\
 &
palm\_rib &
one &
3\textsc{.loc-}burn &
\\
 &
\multicolumn{4}{m{4.02396in}}{{\textquoteleft}A palm rib burns down/on (i.e., is consumed over time).{\textquoteright}}\\
\end{supertabular}
\end{flushleft}
Kamang (Schapper, to appear)

\begin{flushleft}
\tablehead{}
\begin{supertabular}{m{0.43165985in}m{0.7559598in}m{0.39555985in}m{2.06636in}m{0.5698598in}}
\label{bkm:Ref353451952}() &
\itshape Kik &
\itshape nok &
\itshape saara. &
\\
 &
palm\_rib &
one &
burn &
\\
 &
\multicolumn{4}{m{4.02396in}}{{\textquoteleft}A palm rib burns.{\textquoteright}}\\
\end{supertabular}
\end{flushleft}
\ \ The possibility of indexing affected participants with a prefix is not restricted to inanimates. Compare (), with an inanimate, and (), with an animate participant:

Kamang (Schapper, fieldnotes)

\begin{flushleft}
\tablehead{}
\begin{supertabular}{m{0.43165985in}m{0.78375983in}m{0.36775985in}m{0.60385984in}m{2.2136598in}}
\label{bkm:Ref306280880}() &
\itshape Buk &
\itshape taa &
\itshape kamal. &
\\
 &
mountain &
top &
cold &
\\
 &
\multicolumn{4}{m{4.20526in}}{{\textquoteleft}The mountains are cold.{\textquoteright} (i.e., {\textquoteleft}In the mountains, it is cold.{\textquoteright})}\\
\end{supertabular}
\end{flushleft}
Kamang (Schapper, fieldnotes)

\begin{flushleft}
\tablehead{}
\begin{supertabular}{m{0.43165985in}m{1.6941599in}m{2.0656598in}m{0.5698598in}}
\label{bkm:Ref306280885}() &
\itshape No-kamal-da-ma. &
 &
\\
 &
1\textsc{sg}.\textsc{loc}{}-cold-\textsc{aux-pfv} &
 &
\\
 &
\multicolumn{3}{m{4.4871597in}}{{\textquoteleft}I have cooled.{\textquoteright} (i.e., {\textquoteleft}My fever has come down.{\textquoteright})}\\
\end{supertabular}
\end{flushleft}
In () \textit{kamal }{\textquoteleft}cold{\textquoteright} describes a constant property, whereas in () it denotes a change of state in an (animate) participant affected by the process of the dropping of their body temperature.

\ \ For affected (or more patientive) Ss of motion and posture verbs, the \textsc{gen} series of prefixes is used. Compare examples () and (), repeated from () and ():

Kamang (Schapper, to appear)

\begin{flushleft}
\tablehead{}
\begin{supertabular}{m{0.43165985in}m{0.40455985in}m{0.87755984in}m{1.2240599in}}
\label{bkm:Ref323654778}() &
\itshape Kui &
\itshape ye-tak. &
\\
 &
dog &
3.\textsc{gen}{}-run &
\\
 &
\multicolumn{3}{m{2.66366in}}{{\textquoteleft}The dog ran off (was forced to run).{\textquoteright}}\\
\end{supertabular}
\end{flushleft}
Kamang (Schapper, to appear)

\begin{flushleft}
\tablehead{}
\begin{supertabular}{m{0.43165985in}m{0.40455985in}m{0.6302598in}}
\label{bkm:Ref323655171}() &
\itshape Kui &
\itshape tak.\\
 &
dog &
run\\
 &
\multicolumn{2}{m{1.1135598in}}{{\textquoteleft}The dog ran.{\textquoteright}}\\
\end{supertabular}
\end{flushleft}
The presence of the prefix in () indicates the dog was affected by an external event, such as someone kicking at it, whereas () expresses that there is no specific cause for the dog{\textquoteright}s running.

In sum, affectedness plays an important role in the indexing patterns in Kamang. In contrast to Abui, the degree of lexical stipulation is much higher. While Abui coerces only one sixth of its verbs into one fixed inflection, namely the \textsc{pat} inflection, Kamang (unevenly) assigns one third of its verbal vocabulary to six inflection classes. Because of practical constraints we have sampled a larger number of Kamang verbs than is the case for Abui or Teiwa. It is a reasonable expectation that a larger sample size would give us the opportunity to see the verbs more evenly distributed across the classes, and yet Kamang does not show this. This suggests that this contrast between Abui and Kamang is a real and important factor.

\ \ Before turning to the importance of animacy as a factor in Teiwa, we briefly review the role animacy plays in the semantically aligned languages Abui and Kamang.

\subsection[Animacy in Abui and Kamang]{Animacy in Abui and Kamang}
Animacy is not identified as a typical factor in semantic alignment systems. The factor animacy plays is a marginal role in Kamang and Abui. In Teiwa, on the other hand, animacy is the core semantic factor.

Kratochv\'il (2011) does not identify animacy as a relevant factor for argument realization in Abui. All semantic distinctions that the Abui system makes seem to apply to animates as well as to inanimates. In Abui transitives, the P is generally indexed regardless of its animacy value. The verb form \textit{ha-fik }[3.\textsc{pat}{}-pull] {\textquoteleft}pull s.o./sth.{\textquoteright} can be used to describe a person being pulled or a log being pulled. This is illustrated in example (), repeated from (), and example ():

Abui (Response to video clip C01\_pull\_person\_25, SP8)

\begin{flushleft}
\tablehead{}
\begin{supertabular}{m{0.43165985in}m{0.48795983in}m{0.5615598in}m{0.39555985in}m{0.50185984in}m{1.2226598in}m{0.5698598in}}
\label{bkm:Ref353454421}() &
\itshape Wiil &
\itshape neng &
\itshape nuku &
\itshape di &
\itshape de-fela &
\itshape ha-fik\\
 &
child &
male &
one &
3\textsc{act} &
3.\textsc{al.poss}{}-friend &
3.\textsc{i}{}-pull\\
 &
\multicolumn{2}{m{1.1282599in}}{\itshape ha-bel-e.} &
 &
 &
 &
\\
 &
\multicolumn{2}{m{1.1282599in}}{3\textsc{.pat-}pull-\textsc{ipfv}} &
 &
 &
 &
\\
 &
\multicolumn{4}{m{2.1831598in}}{{\textquoteleft}A boy is pulling his friend.{\textquoteright}} &
 &
\\
\end{supertabular}
\end{flushleft}
Abui (Response to video clip C18\_pull\_log\_29, SP8)

\begin{flushleft}
\tablehead{}
\begin{supertabular}{m{0.43165985in}m{0.62545985in}m{0.50115985in}m{0.48445985in}m{0.5802598in}m{0.5150598in}m{0.5497598in}m{1.1872599in}}
\label{bkm:Ref353454429}() &
\itshape Mayol &
\itshape fila &
\itshape di &
\itshape maha &
\itshape bataa &
\itshape takata &
\itshape ha-fik-e.\\
 &
woman &
small &
3\textsc{act} &
maybe &
wood &
dry &
3.\textsc{pat}{}-pull-\textsc{ipfv}\\
 &
\multicolumn{7}{m{4.91586in}}{{\textquoteleft}A girl is pulling a dry log.{\textquoteright}}\\
\end{supertabular}
\end{flushleft}
In intransitive verbs, however, animacy has an effect in Abui. Fedden et al. (2013) show that the frequency of Ss being indexed on the verb were highest for non-volitional animate Ss. This means that the role of animacy in Abui is very different from what we find in Teiwa (see 4.4 below), where the effect of animacy is seen in transitive verbs in which objects that are indexed with a prefix are typically animate.

In Kamang, animacy has a marginal effect at best. There are a few instances of lexicalization of a transitive verb with a prefix. In these cases a transitive verb appears both without and with a prefix whereby the prefixed form has been reinterpreted as a verb with a slightly different or metaphorical meaning, which typically takes an animate P, but whose use also extends to inanimate Ps. 

For instance, the unprefixed verb \textit{buh} means {\textquoteleft}lift with two hands{\textquoteright} and can be used with an animate P () or an inanimate P ():

Kamang (Schapper, fieldnotes)

\begin{flushleft}
\tablehead{}
\begin{supertabular}{m{0.43165985in}m{0.40665984in}m{0.5712598in}m{0.39975986in}}
\label{bkm:Ref353452283}() &
\itshape Nal &
\itshape woi &
\itshape buh.\\
 &
1\textsc{sg} &
stones &
lift\\
 &
\multicolumn{3}{m{1.5351598in}}{{\textquoteleft}I pick up stones.{\textquoteright}}\\
\end{supertabular}
\end{flushleft}
Kamang (Schapper, fieldnotes)

\begin{flushleft}
\tablehead{}
\begin{supertabular}{m{0.43165985in}m{0.91565984in}m{0.37755984in}m{0.5156598in}m{0.5900598in}m{0.6538598in}}
\label{bkm:Ref353452217}() &
\itshape Kili=a &
\itshape lila &
\itshape se &
\itshape sibe &
\itshape buh.\\
 &
eagle=\textsc{spec} &
fly &
come &
chicken &
lift\\
 &
\multicolumn{5}{m{3.36776in}}{{\textquoteleft}An eagle flies in and lifts off with the chicken.{\textquoteright}}\\
\end{supertabular}
\end{flushleft}
The prefixed verb \textit{{}-buh} has been reinterpreted as a verb meaning {\textquoteleft}cradle{\textquoteright} which occurs typically with a human P, as in ():

Kamang (Schapper, fieldnotes)

\begin{flushleft}
\tablehead{}
\begin{supertabular}{m{0.43165985in}m{0.40665984in}m{0.9844598in}m{0.7795598in}}
\label{bkm:Ref353452252}() &
\itshape Nal &
\itshape ge-dum &
\itshape ga-buh.\\
 &
1\textsc{sg} &
3.\textsc{poss}{}-child &
3.\textsc{pat}{}-lift\\
 &
\multicolumn{3}{m{2.3281598in}}{{\textquoteleft}I cradle the child.{\textquoteright}}\\
\end{supertabular}
\end{flushleft}
Other Kamang verbs for which such a reinterpretation has taken place are \textit{fah} {\textquoteleft}search for (inanimate){\textquoteright} vs. \textit{{}-fah} {\textquoteleft}search for (animate){\textquoteright}, \textit{tat} {\textquoteleft}cut up, dice (meat){\textquoteright} vs. \textit{{}-tat} {\textquoteleft}cut off (the path of s.o.){\textquoteright} and \textit{wita} {\textquoteleft}carry by bag or basket on the back hung from forehead{\textquoteright} vs. \textit{{}-wita }{\textquoteleft}carry (a child) on the lower back with a cloth tied around the head{\textquoteright}. There are only a handful of those; this pattern is not productive. 

\ \ In some cases the use of the prefixed verb extends to include inanimate Ps. For example, \newline
{}-\textit{buh} {\textquoteleft}cradle{\textquoteright} can occur with \textit{pop} {\textquoteleft}doll{\textquoteright} ().

Kamang (Schapper, fieldnotes)

\begin{flushleft}
\tablehead{}
\begin{supertabular}{m{0.43165985in}m{0.40665984in}m{0.41435984in}m{1.1983598in}}
\label{bkm:Ref353452580}() &
\itshape Nal &
\itshape pop &
\itshape ga-buh.\\
 &
1\textsc{sg} &
doll &
3.\textsc{pat}{}-cradle\\
 &
\multicolumn{3}{m{2.1768599in}}{{\textquoteleft}I cradle the doll.{\textquoteright}}\\
\end{supertabular}
\end{flushleft}
Hence, the prefix does not carry information about the animacy of the P per se, but instead selects particular verbs, regardless of the actual animacy of the P. These prefixed verbs are typically associated with animate Ps through usage. In this sense, Kamang shows a marginal effect of animacy. 

We will see a much stronger impact of animacy in Teiwa below (section 3.4), where verbs are divided into lexical classes (prefixing vs. not prefixing) depending on whether they typically appear with an animate or inanimate object, respectively.

What we have seen so far is that factors which relate to the event, namely affectedness and volitionality play a greater role in Abui and Kamang, with animacy being marginal at best. However, examples () to () suggest an important link between the verb semantics and the animacy of the argument. We now turn to Teiwa, in which animacy and its relationship with verb classes plays a more important role.

\subsection[Animacy and verb classes in Teiwa]{Animacy and verb classes in Teiwa}
Teiwa has syntactic alignment, whereby only Ps are indexed on the verb. This is a rare type cross-linguistically, occurring in only 7\% of the languages from Siewierska{\textquoteright}s (2011) WALS sample. Animacy is the core semantic factor which plays a role in whether an object is indexed on the verb. It has often been observed in the literature that objects are typically not animate, definite or specific and that it is marked, if they are animate, definite or specific in a given context (see for example, Giv\'on 1976, Aissen 2003; also see Bickel 2008: 204-205). There is a cross-linguistically robust association between marked objects and topicality. This association may have been obscured by grammaticalization, but what we still find in some languages is that marked objects are associated with semantic features typical of topics, such as animacy (Dalrymple and Nikolaeva 2011: 2). 

However, the realization of the animate-inanimate distinction is not absolute in Teiwa. Given that only approximately 22\% of transitive verbs allow prefixation (in absolute numbers this is 49 of 224 transitive verbs in our corpus), it is worth checking whether object indexing in Teiwa is at all productive. Fedden et al. (2013) present a corpus search of transitive verb hapaxes, inspired by the quantitative method in Baayen (1992) and later papers. The Teiwa corpus consists of approximately 11,000 words of spontaneous text. The assumption is that if a morphological process is productive in a language the hapax legomena in the corpus, i.e., those items which only occur once, will exhibit it. Lower frequency items will need to rely on the creativity associated with rules, whereas memory will have a greater role in relation to high frequency items. The results strongly indicate that prefixation of animate objects is indeed productive in Teiwa: 85.7\% of transitive verb hapaxes with an animate object actually 
also have a prefix.\footnote{Bearing in mind the caveat that the Teiwa corpus is nowhere nearly as massive as the corpora Baayen used.}

If prefixation in Teiwa were purely a matter of sensitivity to the animacy property of the argument, rather than a manifestation of the class to which a verb belongs, we would expect one and the same verb to alternate between prefixation and non-prefixation, depending on the animacy of the object it happened to be taking. This, however, is typically not the case. There are cases where the same verb does (or doesn{\textquoteright}t) have a prefix regardless of the animacy value of the object. This is illustrated for the verb \textit{{}-uyan}, which is prefixing in (), where it appears with an animate P, and also prefixing in (), where it appears with an inanimate P:

Teiwa (Klamer 2010a: 88)

\begin{flushleft}
\tablehead{}
\begin{supertabular}{m{0.43165985in}m{0.40665984in}m{0.5490598in}m{0.71915984in}m{0.32125986in}m{0.50115985in}}
\label{bkm:Ref353452306}() &
\itshape A &
\itshape qavif\ \  &
\itshape ga-uyan &
\itshape gi &
\itshape si\ \ {\dots}\\
 &
3\textsc{sg} &
goat &
3-search &
go &
\scshape sim\\
 &
\multicolumn{5}{m{2.81226in}}{{\textquoteleft}He went searching for a goat, [{\dots}]{\textquoteright}}\\
\end{supertabular}
\end{flushleft}
Teiwa (Klamer 2010a: 340)

\begin{flushleft}
\tablehead{}
\begin{supertabular}{m{0.43165985in}m{0.30255985in}m{0.40665984in}m{0.35185984in}m{0.9650598in}m{0.37755984in}m{0.39485985in}m{0.71915984in}m{0.45045987in}m{0.54345983in}}
\label{bkm:Ref353454583}() &
\itshape {\dots} &
\itshape ha &
\itshape gi &
\itshape ya{\textquoteright} &
\itshape siis &
\itshape nuk &
\itshape ga-uyan &
\itshape pin &
\itshape aria{\textquoteright}.\\
 &
 &
2\textsc{sg} &
go\ \  &
bamboo\_sp. &
dry &
one &
3-search &
hold &
arrive\\
 &
\multicolumn{7}{m{3.99016in}}{{\textquoteleft}[{\dots}] You go look for dry bamboo to bring here.{\textquoteright}} &
 &
\\
\end{supertabular}
\end{flushleft}
The converse case is more frequent. There are many transitive verbs that never index their P, regardless of its animacy value. This is illustrated in () and () where the verb \textit{tumah} occurs with an animate and an inanimate P, respectively:

Teiwa (Response to video clip C13\_bump\_into\_person\_38, SP4)

\begin{flushleft}
\tablehead{}
\begin{supertabular}{m{0.43165985in}m{0.5990598in}m{0.5719598in}m{0.39625984in}m{0.35185984in}m{0.7372598in}m{0.54065984in}}
\label{bkm:Ref306281423}() &
\itshape Uy\ \  &
\itshape masar &
\itshape nuk &
\itshape wa\ \  &
\itshape kri &
\itshape tumah.\\
 &
person &
male &
one &
go &
old\_man &
bump\\
 &
\multicolumn{6}{m{3.59076in}}{{\textquoteleft}A man is going and bumps into an old man.{\textquoteright}}\\
\end{supertabular}
\end{flushleft}
Teiwa (Response to video clip C16\_bump\_into\_tree\_42, SP4)

\begin{flushleft}
\tablehead{}
\begin{supertabular}{m{0.43165985in}m{0.7372598in}m{0.39555985in}m{0.5344598in}m{0.34975985in}m{0.40455985in}m{0.53095984in}}
\label{bkm:Ref353452853}() &
\itshape Kri &
\itshape nuk &
\itshape tewar &
\itshape wa &
\itshape tei &
\itshape tumah.\\
 &
old\_man &
one &
walk &
go &
tree &
bump\\
 &
\multicolumn{6}{m{3.3462598in}}{{\textquoteleft}An old man walks and bumps into a tree.{\textquoteright}}\\
\end{supertabular}
\end{flushleft}
In Teiwa we find the formation of a class of prefixed vs. a class of not prefixed verbs) based on the animacy value of the objects a verb typically occurs with. There are four classes of verbs. 

The first class of transitive verbs consists of prefixed verbs. These always index their P with a prefix and they typically occur with animate objects. A separate noun phrase constituent may optionally be present. The full set of verbs that belong to this class is given in (a). The five verbs that are obligatorily applicativized are given in (b). The addition of \textit{someone} (s.o.) and/or \textit{something} (sth.) in the glosses indicate whether a verb can appear with an animate or an inanimate object in the corpus. 

\begin{flushleft}
\tablehead{}
\begin{supertabular}{m{0.48795983in}m{2.5781598in}m{3.07266in}}
\label{bkm:Ref324859969}() &
a. Prefixed transitive verbs (class 1) &
\\
 &
\textit{{}-adiman} {\textquoteleft}drown s.o.{\textquoteright} &
\textit{{}-mis} {\textquoteleft}marry s.o.{\textquoteright}\\
 &
\textit{{}-an} {\textquoteleft}give to s.o.{\textquoteright} &
{}-\textit{pak} {\textquoteleft}call s.o.{\textquoteright}\\
 &
\textit{{}-{\textquoteright}an} {\textquoteleft}sell to s.o.{\textquoteright} &
{}-\textit{panaat }{\textquoteleft}send to s.o.{\textquoteright}\\
 &
\textit{{}-arar} {\textquoteleft}be afraid of s.o.{\textquoteright} &
{}-\textit{regan }{\textquoteleft}ask s.o.{\textquoteright}\\
 &
{}-\textit{ayas} {\textquoteleft}throw at s.o.{\textquoteright} &
{}-\textit{rian} {\textquoteleft}look after s.o.{\textquoteright}\\
 &
\textit{{}-bir} {\textquoteleft}run away with s.o.{\textquoteright} &
\textit{{}-sar }{\textquoteleft}notice, find s.o./sth.{\textquoteright}\\
 &
{}-\textit{bun} {\textquoteleft}answer s.o.{\textquoteright} &
{}-\textit{sas} {\textquoteleft}feed s.o.{\textquoteright}\\
 &
\textit{{}-buri} {\textquoteleft}fix sth.{\textquoteright} &
{}-\textit{soi} {\textquoteleft}order s.o.{\textquoteright}\\
 &
{}-\textit{fai} {\textquoteleft}swear at s.o.{\textquoteright} &
\textit{{}-tan tup} {\textquoteleft}wake s.o. up{\textquoteright}\\
 &
\textit{{}-far} {\textquoteleft}kill s.o.{\textquoteright} &
\textit{{}-tane{\textquoteright} }{\textquoteleft}kick sth. to the side{\textquoteright}\\
 &
{}-\textit{fin} {\textquoteleft}catch s.o.{\textquoteright} &
{}-\textit{tiar} {\textquoteleft}chase s.o.{\textquoteright}\\
 &
\textit{{}-fur} {\textquoteleft}turn s.o.{\textquoteright} &
\textit{{}-u{\textquoteright}an} {\textquoteleft}carry s.o.{\textquoteright}\\
 &
\textit{{}-honan} {\textquoteleft}come to s.o.{\textquoteright} &
{}-\textit{ua{\textquoteright}} {\textquoteleft}hit s.o.{\textquoteright}\\
 &
{}-\textit{lal }{\textquoteleft}show to s.o.{\textquoteright} &
{}-\textit{{\textquoteright}uam} {\textquoteleft}teach s.o.{\textquoteright}\\
 &
\textit{{}-laman} {\textquoteleft}oppose s.o., negotiate sth. (e.g., a road){\textquoteright} &
{}-\textit{uyan} {\textquoteleft}search for s.o./sth.{\textquoteright}\\
 &
{}-\textit{liin} {\textquoteleft}invite s.o.{\textquoteright} &
{}-\textit{wei} {\textquoteleft}bathe s.o.{\textquoteright}\\
 &
\textit{{}-miar }{\textquoteleft}play with sth.{\textquoteright} &
\textit{{}-yix} {\textquoteleft}descend with s.o.{\textquoteright}\\
 &
{}-\textit{mir} {\textquoteleft}ascend to s.o.{\textquoteright} &
\\
 &
 &
\\
 &
\multicolumn{2}{m{5.72956in}}{b. Prefixed transitive verbs with obligatory applicative \textit{{}-un}}\\
 &
\multicolumn{2}{m{5.72956in}}{\textit{{}-unba{\textquoteright}} {\textquoteleft}meet s.o.{\textquoteright}}\\
 &
\multicolumn{2}{m{5.72956in}}{\textit{{}-unbungan} {\textquoteleft}ask s.o.{\textquoteright}}\\
 &
\multicolumn{2}{m{5.72956in}}{{}-\textit{undagar} {\textquoteleft}turn towards s.o.{\textquoteright}}\\
 &
\multicolumn{2}{m{5.72956in}}{{}-\textit{unmulax} {\textquoteleft}help s.o.{\textquoteright}}\\
 &
\multicolumn{2}{m{5.72956in}}{{}-\textit{unpaxai} {\textquoteleft}share with s.o., divide sth.{\textquoteright}}\\
\end{supertabular}
\end{flushleft}
The second class of transitive verbs consists of unprefixed verbs. These never index their P and typically occur with inanimate objects. A separate noun phrase constituent may optionally be present. As this class is rather large, we only give examples in (). 

\begin{flushleft}
\tablehead{}
\begin{supertabular}{m{0.77125984in}m{3.76086in}m{1.0163599in}}
\label{bkm:Ref353453047}() &
\multicolumn{2}{m{4.85596in}}{Examples of unprefixed transitive verbs (class 2)}\\
 &
\textit{bali} {\textquoteleft}see s.o./sth.{\textquoteright} &
\textit{na} {\textquoteleft}eat sth.{\textquoteright}\\
 &
\textit{bangan} {\textquoteleft}ask for sth.{\textquoteright} &
\textit{ol} {\textquoteleft}buy sth.{\textquoteright}\\
 &
\textit{boqai} {\textquoteleft}cut sth. up{\textquoteright} &
\textit{pin} {\textquoteleft}hold s.o./sth.{\textquoteright}\\
 &
\textit{dumar} {\textquoteleft}push sth. away{\textquoteright} &
\textit{qas} {\textquoteleft}split sth.{\textquoteright}\\
 &
\textit{hela} {\textquoteleft}pull sth.{\textquoteright} &
\textit{si{\textquoteright} }{\textquoteleft}wash sth.{\textquoteright}\\
 &
\textit{mat} {\textquoteleft}take sth.{\textquoteright} &
\textit{taxar} {\textquoteleft}cut sth. in two{\textquoteright}\\
 &
\textit{me{\textquoteright}} {\textquoteleft}be in sth.{\textquoteright} &
\textit{tian} {\textquoteleft}carry sth. on head or shoulder{\textquoteright}\\
 &
\textit{moxod} {\textquoteleft}drop s.o./sth.{\textquoteright} &
\\
\end{supertabular}
\end{flushleft}
An explanation of the behaviour of the verb (i.e., whether it has a prefix) based on verb semantics is likely to fail. Verbs with similar semantics can vary, such as the verb {\textquoteleft}to cradle{\textquoteright} in (), in contrast to the verb {\textquoteleft}to hold{\textquoteright} in ():

Teiwa (Response to video clip P15\_hold\_person\_24, SP3)

\begin{flushleft}
\tablehead{}
\begin{supertabular}{m{0.43165985in}m{0.7379598in}m{0.39555985in}m{0.62615985in}m{1.1011599in}m{0.90805984in}}
\label{bkm:Ref353452785}() &
\itshape Kri &
\itshape nuk &
\itshape g-oqai &
\itshape g-u{\textquoteright}an-an &
\itshape tas-an.\\
 &
old\_man &
one &
3-child &
3-cradle-\textsc{real} &
stand-\textsc{real}\\
 &
\multicolumn{5}{m{4.08386in}}{{\textquoteleft}An old man is standing cradling his child.{\textquoteright}}\\
\end{supertabular}
\end{flushleft}
Teiwa (Klamer 2010a: 425)

\begin{flushleft}
\tablehead{}
\begin{supertabular}{m{0.43165985in}m{0.48725984in}m{0.40385985in}m{0.48725984in}m{0.40315986in}m{0.7566598in}m{0.62685984in}m{0.47545984in}m{0.7344598in}m{0.32955986in}}
\label{bkm:Ref353452809}() &
\itshape Qau &
\itshape ba &
\itshape iman &
\itshape ta\ \  &
\itshape mauqubar &
\itshape g-oqai &
\itshape pin &
\itshape bir-an &
\itshape gi.\\
 &
good &
\scshape seq &
\scshape 3pl &
\scshape top &
frog &
3-child &
hold &
run-\textsc{real} &
go\\
 &
\multicolumn{5}{m{2.85316in}}{{\textquoteleft}So they hold the baby frog and go, [{\dots}].{\textquoteright}} &
 &
 &
 &
\\
\end{supertabular}
\end{flushleft}
Some verbs which typically occur with inanimates, e.g., \textit{pin} {\textquoteleft}hold{\textquoteright}, could well occur with animates, as in (). It is very difficult to identify certain verb semantics which would be associated with the verb taking a prefix. Generally, when looking at verbs of similar semantics, some verbs will have a prefix while others do not, for example, \textit{u{\textquoteright}an} {\textquoteleft}hold s.o.{\textquoteright} and \textit{{}-sar }{\textquoteleft}notice s.o./sth.{\textquoteright} take a prefix while \textit{pin }{\textquoteleft}hold s.o./sth.{\textquoteright} and \textit{bali }{\textquoteleft}see s.o./sth.{\textquoteright} do not.

As already mentioned above, it is not the case that prefixation in Teiwa is purely a matter of sensitivity to the animacy property of the argument, but rather a manifestation of the class to which a verb belongs. We do, however, find a few cases where one and the same verb alternates between prefixation and non-prefixation or between two different sets of prefixes, depending on the animacy of the object the verb happened to be taking. Such verbs make up the classes 3 and 4, respectively.

Transitive verbs of class 3 either have a prefix and an animate object or no prefix and an inanimate object. This class is small and consists of five verbs, given in ():

\begin{flushleft}
\tablehead{}
\begin{supertabular}{m{0.43165985in}m{4.21086in}}
\label{bkm:Ref306281469}() &
Transitive verbs with or without prefix (class 3)\\
 &
\textit{{}-sii} {\textquoteleft}bite s.o.{\textquoteright} and \textit{sii} {\textquoteleft}bite (into) sth.{\textquoteright}\\
 &
{}-\textit{dee }{\textquoteleft}burn s.o.{\textquoteright} and \textit{dee }{\textquoteleft}burn sth.{\textquoteright}\\
 &
\textit{{}-mai }{\textquoteleft}keep for s.o.{\textquoteright} and \textit{mai} {\textquoteleft}save sth.{\textquoteright} \\
 &
\textit{{}-mar }{\textquoteleft}follow s.o.{\textquoteright} and \textit{mar }{\textquoteleft}take/get sth.{\textquoteright}\\
 &
\textit{{}-mian} {\textquoteleft}give to s.o.{\textquoteright}, \textit{mian} {\textquoteleft}put at sth.{\textquoteright}\\
\end{supertabular}
\end{flushleft}
Two examples are given in () and () which illustrate this type of prefixation where the presence or absence of the prefix is actually dependent on the animacy value of the P. In (), the P of the verb \textit{mar}, the second person, is animate; in (), the P \textit{met }{\textquoteleft}betel vine{\textquoteright} is inanimate:

Teiwa (Klamer, Teiwa corpus TAS:0166)

\begin{flushleft}
\tablehead{}
\begin{supertabular}{m{0.43165985in}m{0.40665984in}m{0.89765984in}m{0.5344598in}m{0.5358598in}m{0.5184598in}}
\label{bkm:Ref353453098}() &
\itshape Na &
\itshape ha-mar. &
 &
 &
\\
 &
1\textsc{sg} &
\textsc{2sg-}follow &
 &
 &
\\
 &
\multicolumn{5}{m{3.2080598in}}{{\textquoteleft}I follow you.{\textquoteright}}\\
\end{supertabular}
\end{flushleft}
Teiwa (Klamer, Teiwa corpus, TAS:0394)

\begin{flushleft}
\tablehead{}
\begin{supertabular}{m{0.43165985in}m{0.40665984in}m{0.7656598in}m{0.8316598in}m{0.5365598in}m{0.7531598in}}
\label{bkm:Ref353453104}() &
\itshape Na &
\itshape met &
\itshape mar-an &
\itshape ma   &
\itshape ga-mian.\\
 &
1\textsc{sg} &
betelvine &
take-\textsc{real} &
come &
3-give\\
 &
\multicolumn{5}{m{3.6086597in}}{{\textquoteleft}I take some betel vine and give it to him.{\textquoteright}}\\
\end{supertabular}
\end{flushleft}
For these verbs, the animate-inanimate distinction constitutes an agreement feature realized by the presence or the absence of the prefix.

Transitive verbs of class 4 select one prefix set with animate objects and another prefix set with inanimate objects. This class comprises only four items, listed in ():

\begin{flushleft}
\tablehead{}
\begin{supertabular}{m{0.5490598in}m{3.57406in}}
\label{bkm:Ref306281514}() &
Transitive verbs taking different prefixes (class 4) \\
 &
\textit{{}-kiid} {\textquoteleft}cry for s.o., cry about sth.{\textquoteright}\\
 &
{}-\textit{tad }{\textquoteleft}strike s.o., strike at sth.{\textquoteright}\\
 &
{}-\textit{wultag }{\textquoteleft}talk to s.o., talk about sth.{\textquoteright}\\
 &
{}-\textit{wulul }{\textquoteleft}tell s.o., tell sth.{\textquoteright}\\
\end{supertabular}
\end{flushleft}
Class 4 shows alternation between two different prefixes in the 3\textsuperscript{rd} person. Inanimate objects are indexed with the normal \textit{ga-} prefix whereas animate objects take an augmented form (with a glottal stop). Compare () and (): 

Teiwa (Klamer 2010a: 92)

\begin{flushleft}
\tablehead{}
\begin{supertabular}{m{0.43165985in}m{0.40665984in}m{0.32125986in}m{0.84555984in}}
\label{bkm:Ref353453181}() &
\itshape Ha &
\itshape gi &
\itshape ga{\textquoteright}-wulul.\\
 &
2\textsc{sg} &
go &
3\textsc{an}{}-talk\\
 &
\multicolumn{3}{m{1.7309599in}}{{\textquoteleft}You go tell him.{\textquoteright}}\\
\end{supertabular}
\end{flushleft}
Teiwa (Klamer 2010a: 92)

\begin{flushleft}
\tablehead{}
\begin{supertabular}{m{0.5038598in}m{0.47545984in}m{0.37755984in}m{0.9115598in}m{0.6226598in}m{0.37405986in}}
\label{bkm:Ref353453185}() &
\itshape Ha &
\itshape gi &
\itshape ga-wulul. &
 &
\\
 &
2\textsc{sg} &
go &
3-talk &
 &
\\
 &
\multicolumn{5}{m{3.0762599in}}{{\textquoteleft}You go tell it (i.e., some proposition)!{\textquoteright}}\\
\end{supertabular}
\end{flushleft}
This contrast exists in the third person only. Although the first and second persons are always animate, they nonetheless take the unaugmented prefix forms with the class 4 verbs, e.g., \textit{ha gi na-wulul/*na{\textquoteright}-wulul} {\textquoteleft}You go tell me{\textquoteright}.

There is a potential issue in these examples because the semantic roles of the non-subject arguments in (), a human recipient, and (), a proposition or messages expressed as the object, are different but this need not concern us because Teiwa (as indeed all Alor-Pantar languages) has secundative alignment (Klamer 2010b: 449, 454).\footnote{On the notion of secundative alignment, see Dryer (1986).} This means that the language generally treats recipients (and goals, including those of ballistic motion and comitatives) like patients, both of which are indexed with a prefix, e.g., \textit{an} {\textquoteleft}give to s.o.{\textquoteright}, \textit{honan} {\textquoteleft}come to s.o.{\textquoteright}, \textit{{}-ayas} {\textquoteleft}throw at s.o.{\textquoteright}, and \textit{{}-yix} {\textquoteleft}descend with s.o.{\textquoteright}. Therefore, it is fully expected that the non-subject arguments in () and () -- despite their difference in semantic role -- are both indexed with a prefix. For the verbs in class 4,
 we can see the development of a small inflectional paradigm in which the animate-inanimate distinction constitutes an agreement feature realized by different prefix types. Although more evidence is required to confirm this, a reasonable hypothesis is that this has arisen as a second stage of grammaticalization in Teiwa, where a free pronoun (e.g., \textit{ga{\textquoteright}an}) with its glottal stop has attached to the verb. Importantly, it also contrasts with class 3, which in essence realizes the same animate-inanimate distinction, but uses prefixation vs. lack of prefixation to do it rather than different prefix forms. These are, therefore, examples of arbitrary inflection classes, as the same animate-inanimate distinction (in classes 3 and 4) has different reflexes depending on the verb. So there is strong evidence for Teiwa contrasting with Abui and Kamang, and this appears to be associated with a move from semantic related factors to a greater role for animacy and verb classes. 

\section[Discussion and conclusion]{Discussion and conclusion}
The Alor-Pantar languages are of significant macro-typological interest for pronominal indexing because they show contrasting behaviours in terms of the degree to which purely lexical information is involved. For Abui prefixation is determined to a greater extent by the semantics of the event, rather than the semantics associated directly with the lexical item. Volitionality and affectedness are interpreted at the level of the event itself, rather than a constant and indefeasible part of a verb{\textquoteright}s semantics. For Kamang, which has what would still be broadly defined as a semantic alignment system, affectedness also plays a role, but there appears to be greater scope for arbitrary association between a prefix-class and a particular verb, so that verbs are more restricted in terms of the choice of prefix with which they may occur. We noted the marginal role that animacy played in Kamang with verbs such as the one meaning {\textquoteleft}lift up{\textquoteright} or {\textquoteleft}cradle{\
textquoteright} where the former occurred without a prefix and the latter with. It is reasonable to infer that the restriction of a given verb to one prefix series, as happens in Kamang, results from the strengthening of associations between particular verbs and the prefix series on the basis of those verbs{\textquoteright} frequent occurrences in constructions related to the original event-related semantics. These prefixes then become conventionally associated with subsets of verbs, as is the case in Kamang, and are restricted to those verbs. In contrast with Kamang, for Teiwa animacy plays an important role in effecting this conventional association. While we cannot be entirely sure about the diachronic scenario, the most entrenched conventionalization is associated with the prefix series which is the oldest, namely the \textsc{pat} series.

For the micro-typological level, we have taken Abui, Kamang and Teiwa as representing three important types found within the family. Lexical stipulation is lowest for Abui, higher for Kamang and very high for Teiwa. The event-related semantic factors involved are affectedness and volitionality for Abui and affectedness for Kamang. Kamang has a greater degree of lexical stipulation, which we assume has arisen from conventionalization of the event-related semantics to particular verbs, leading to classes based on the lexical semantics of the verb, or arbitrary associations. As we noted in {\S}4 the conventional association of animate objects with certain verbs in Teiwa leads to a stronger role for lexical stipulation there. 

The three Alor-Pantar languages considered in this chapter provide important typological insights into the relationship between referential properties and lexical stipulation as evinced in a language{\textquoteright}s patterns of pronominal indexing. In all of the languages we have discussed here, properties of the verb play some role. In the semantically aligned languages, this emerges from the lexical semantics of verbs with regard to affectedness or volitionality. But we can observe a change in orientation from properties expressing a relationship between participants and events, as in Abui and Kamang, to properties involving lexical features of the verb itself. Semantic factors in events are reinterpreted as constraints on individual verbs. The role of animacy is increasingly important in Teiwa. The language has a very small set of verbs (classes 3 and 4) in which animacy figures as an agreement feature. Thus, in Teiwa a conventionalization has taken place where verb classes become associated with the 
animacy value of the objects with which the verbs in a given class typically occur.

Across the three languages, the nature of the semantic restrictions on pronominal indexing differs, and animacy is a property which actually allows for arbitrary classes to emerge, much more so than affectedness and volitionality. This is because it classifies the argument of the verb according to animacy but also involves an expectation based on the verb{\textquoteright}s own semantics (about the properties of the objects it selects for), while at the same time not directly classifying the relationship between the participant and the event. Given this dual nature of animacy, there is therefore a strong potential for properties based on what is expected to clash with what actually occurs, and there is greater potential for arbitrary classes to emerge. A reasonable hypothesis is that the Teiwa system represents one possible trajectory within Alor-Pantar from a system which is highly dependent on the event semantics to one where the restrictions on prefixes lead to a much smaller number of verbs being prefixed.
 

{\bfseries
References}

Aissen, Judith. 2003. Differential object marking: Iconicity and economy. \textit{Natural Language and Linguistic Theory} 21(3): 435-483.

Arkadiev, Peter. 2008. Thematic roles, event structure, and argument encoding in semantically aligned languages. In Mark Donohue and S{\o}ren Wichmann, eds., \textit{The typology of semantic alignment}: 101-117. Oxford: Oxford University Press.

Baayen, R. Harald. 1992. Quantitative aspects of morphological productivity. In Geert Booij and Jaap van Marle, eds., \textit{Yearbook of Morphology 1991}: 109{}-149. Dordrecht: Kluwer Academic Publishers.

Baird, Louise. 2008. \textit{A grammar of Klon: A non-Austronesian language of Alor}, Indonesia. Canberra: Pacific Linguistics. 

Beavers, John. 2011. On affectedness. \textit{Natural Language and Linguistic Theory} 29(2): 335-370.

Bickel, Balthasar. 2008. On the scope of the referential hierarchy in the typology of grammatical relations. In Greville G. Corbett and Michael Noonan, eds., \textit{Case and grammatical relations. Papers in honour of Bernard Comrie}: 191{}-210. Amsterdam: John Benjamins.

Bossong, Georg. 1991. Differential object marking in Romance and beyond. In Dieter Wanner and Douglas A. Kibbee, eds., \textit{New analyses in Romance linguistics, selected papers from the XVIII linguistic symposium on Romance languages 1988}: 143-170. Amsterdam: Benjamins.

Brown, Dunstan and Sebastian Fedden. In preparation. Canonical agreement in the Alor-Pantar languages.

Croft, William. 1988. Agreement vs. case marking and direct objects. In Michael Barlow and Charles A. Fergusson, eds., \textit{Agreement in natural language: Approaches, theories, descriptions}: 159-180. Stanford: Center for the Study of Language and Information.

Dalrymple, Mary, and Irina Nikolaeva. 2011. \textit{Objects and information structure}. Cambridge: Cambridge University Press.

DeLancey, Scott. 1985. On active typology and the nature of agentivity. In Frans Plank, ed., \textit{Relational typology}: 47{}-60. Berlin: Mouton.

Donohue, Mark. 2008. Bound pronominals in the West Papuan languages. In Claire Bowern, Bethwyn Evans and Luisa Miceli, eds., \textit{Morphology and language history: In honour of Harold Koch}: 43{}-58. Amsterdam: John Benjamins.

Donohue, Mark, and S{\o}ren Wichmann, eds., 2008. \textit{The typology of semantic alignment}. Oxford: Oxford University Press.

Dowty, David. 1991. Thematic proto-roles and argument selection. \textit{Language }67(3): 547{}-619.

Dryer, Matthew S. 1986. Primary objects, secondary objects, and antidative. \textit{Language} 62(4): 808{}-845.

Fedden, Sebastian, Dunstan Brown and Greville G. Corbett. 2010. Conditions on pronominal marking: A set of 42 video stimuli for field elicitation. Surrey Morphology Group, University of Surrey [Available at: http://www.alor-pantar.surrey.ac.uk/index.php/field-materials/].

Fedden, Sebastian, Dunstan Brown, Greville G. Corbett, Marian Klamer, Gary Holton, Laura C. Robinson, and Antoinette Schapper. 2013. Conditions on pronominal marking in the Alor-Pantar languages. \textit{Linguistics} 51(1): 33-74.

Giv\'on, Talmy. 1976. Subject, topic, and grammatical agreement. In Charles Li, ed., \textit{Subject and topic}: 57-98. New York: Academic Press.

Haan, Johnson W. 2001. The grammar of Adang: A Papuan language spoken on the island of Alor East Nusa Tenggara -- Indonesia\textit{. }PhD thesis, University of Sydney.

Heusinger, Klaus von, and Georg Kaiser. 2011. Affectedness and differential object marking in Spanish. \textit{Morphology} 21(1): 1-25.

Holton, Gary. 2010. Person-marking, verb classes, and the notion of grammatical alignment in Western Pantar (Lamma). In Michael Ewing and Marian Klamer, eds., \textit{Typological and areal analyses: Contributions from East Nusantara}: 97-117. Canberra: Pacific Linguistics.

Holton, Gary, Marian Klamer, Franti\v{s}ek Kratochv\'il, Laura C. Robinson, and Antoinette Schapper. 2012. The historical relations of the Papuan languages of Alor and Pantar. \textit{Oceanic Linguistics} 51(1): 86-122.

Hopper, Paul, and Sandra Thompson. 1980. Transitivity in grammar and discourse. \textit{Language }56(2): 251-299.

Hurford, James R. 2007. \textit{The origins of meaning: Language in the light of evolution}. Oxford: Oxford University Press. (Studies in the Evolution of Language 8).

Klamer, Marian. 2008. The semantics of semantic alignment in eastern Indonesia. In Mark Donohue and S{\o}ren Wichmann, eds., \textit{The typology of semantic alignment}: 221-251. Oxford: Oxford University Press.

Klamer, Marian. 2010a. \textit{A grammar of Teiwa}. (Mouton Grammar Library 49.) Berlin: Mouton De Gruyter.

Klamer, Marian. 2010b. Ditransitive constructions in Teiwa. In Andrej Malchukov, Martin Haspelmath and Bernard Comrie. \textit{Studies in ditransitive constructions: A comparative handbook}: 427-455 Berlin: Mouton de Gruyter.

Kratochv\'il, Franti\v{s}ek. 2007. \textit{A grammar of Abui}. Utrecht: LOT.

Kratochv\'il, Franti\v{s}ek. 2011. Transitivity in Abui. \textit{Studies in Language }35(3): 589{}-636.

Mithun, Marianne. 1991. Active/agentive case marking and its motivations. \textit{Language} 67(3): 510-546.

Mohanan, Tara. 1990. Arguments in Hindi. PhD thesis. Stanford University.

N{\ae}ss, {\AA}shild. 2004. What markedness marks: The markedness problem with direct objects. \textit{Lingua} 114(9-10): 1186-1212.

N{\ae}ss, {\AA}shild. 2006. Case semantics and the agent-patient opposition. In Leonid Kulikov, Andrej Malchukov and Peter de Swart, eds., \textit{Case, valency and transitivity}: 309-327. Amsterdam. John Benjamins.

N{\ae}ss, {\AA}shild. 2007. \textit{Prototypical transitivity}. Amsterdam: John Benjamins.

Schapper, Antoinette. To appear. Kamang. In Antoinette Schapper, ed., \textit{Papuan languages of Timor-Alor-Pantar: Sketch grammars}.

Schapper, Antoinette, and Marten Manimau. 2011. Kamus Pengantar bahasa Kamang-Indonesia-Inggris. (Introductory Kamang-Indonesian-English dictionary.) UBB Language \& Culture Series, A-7 (Charles E. Grimes, series editor). Kupang: Unit Bahasa dan Budaya. 

Siewierska, Anna. 2004. \textit{Person}. Cambridge: Cambridge University Press.

Siewierska, Anna. 2011. Verbal person marking. In Matthew S. Dryer and Martin Haspelmath, eds., \textit{The World Atlas of Language Structures Online}. Munich: Max Planck Digital Library, chapter 102. Available online at http://wals.info/chapter/1. Accessed on 2011-09-19.

Tsunoda, Tasaku. 1981. Split case-marking in verb-types and tense/aspect/mood. \textit{Linguistics} 19(5-6): 389-438.

Tsunoda, Tasaku. 1985. Remarks on transitivity. \textit{Journal of Linguistics} 21(2): 385-396.

Please cite the video clips as:



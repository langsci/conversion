
[Warning: Draw object ignored]\clearpage\setcounter{page}{1}\pagestyle{Standard}
{\centering
\textit{Chapter 3}
\par}

{\centering
\textbf{The relatedness of Timor-Kisar and Alor-Pantar languages: A preliminary demonstration}
\par}

{\centering
\textit{Antoinette Schapper, Juliette Huber and Aone van Engelenhoven}
\par}

\setcounter{tocdepth}{3}
\renewcommand\contentsname{}
\tableofcontents
\clearpage{\centering
\textit{Chapter 3}
\par}

{\centering
\textbf{The relatedness of Timor-Kisar and Alor-Pantar languages: A preliminary demonstration}
\par}

{\centering
\textit{Antoinette Schapper, Juliette Huber and Aone van Engelenhoven}
\par}

Abstract: The Papuan languages of Timor, Alor, Pantar and Kisar have long been thought to be members of a single family. However, their relatedness has not yet been established through the rigorous application of the comparative method. Recent historical work has shown the relatedness of the languages of Alor and Pantar on the one hand (Holton et al. 2012), and those of Timor and Kisar on the other (Schapper et al. 2012). In this chapter, we present a preliminary demonstration of the relatedness of the Timor-Alor-Pantar family based on a comparison of these two reconstructions. We identify a number of regular consonant correspondences across cognate vocabulary between the two groups and reconstruct a list of 89 proto-TAP roots. 

\section[Introduction]{Introduction}
\hypertarget{RefHeading65403871885726}{}This chapter looks at the historical relationship between the Papuan languages of Alor-Pantar (AP) and those of Timor-Kisar (TK). The TK group of Papuan languages consists of Bunaq, spoken in central Timor, Makasae, Makalero and Fataluku, three languages spoken in a contiguous region of far eastern Timor, and Oirata, spoken on the southern side of Kisar Island to the north of Timor (Map 1). Due to their geographical proximity, AP and TK languages have typically been assumed to be related to one another (e.g., Stokhof 1975; Capell 1975). Together they have been referred to as the Timor-Alor-Pantar (TAP) family. However, there has been no substantive data-driven investigation of the claim.

In this chapter, we test the hypothesis that AP and TK languages are related to one another through the application of the comparative method. Specifically, we compare the results of two recent reconstructions, the one of AP (Holton et al. 2012) and the other of TK (Schapper et al. 2012). The sources of the lexical data used are listed in the Appendix. We establish that the AP and TK languages are indeed related by demonstrating that there are regular sound correspondences across cognate vocabulary between the two groups. 

In comparing Holton et al. (2012) and Schapper et al. (2012) in this chapter, we assume the existence of two nodes in the TAP tree, namely Proto-AP (pAP) and Proto-TIM (pTIM). Whilst pAP appears to be a robust node, the existence of pTIM is less secure. As Schapper et al. (2012: 227-228) point out, it is possible that Bunaq and the Eastern Timor languages (reconstructed as Proto-ET in Schapper et al. 2012) both form their own separate primary subgroups within TAP. Our aim here is not to make claims about the high-level subgrouping of the AP and TK languages, and we do not presume to definitively determine the constituency of the TK-AP tree at this stage, but merely seek to show that TK and AP languages are related. Conclusive evidence of innovations shared by Bunaq and ET languages to the exclusion of AP languages is the subject of ongoing research. 

{\centering
Map 1: The Papuan languages of Timor and Kisar {\dag}
\par}

{\dag} Hatching marks areas where Papuan languages are found. Only Timor-Kisar languages are marked by name.

Section 2 presents the sound correspondences we find in cognate vocabulary between pAP and pTIM. Section 3 summarises our preliminary findings and discusses issues arising out of them. An Appendix is included with supporting language data for any reconstructions that do not appear in Holton et al. (2012) or Schapper et al. (2012), as well as a list of pTAP forms that can be reconstructed on the basis of the sound correspondences identified in this chapter. New, additional reconstructions have in some cases been necessary since the two articles each reconstruct only a small number of lexemes with only partial overlap between them. We also throw out several cognate sets from the AP reconstruction as they reflect borrowing from Austronesian languages. 

\section[Sound correspondences]{Sound correspondences}
\hypertarget{RefHeading65405871885726}{}In this section, we describe the consonant correspondences that we have identified between AP and TK languages. We do draw on vowel correspondences where they condition particular sound changes in consonants, but otherwise do not deal with vowels in this preliminary demonstration of relatedness. We chiefly draw attention to the correspondences in cognate vocabulary between pAP and pTIM. However, we provide the reader also with the forms of the lexemes in the TK languages as they are not available elsewhere in this volume. The argumentation and underpinning data for pAP is given in Holton and Robinson (this volume) and is based on Holton et al. (2012).

In the subsections that follow, transcription of language data adheres to IPA conventions. Long vowels are indicated with a length mark {\textquoteleft}{\textlengthmark}{\textquoteright}. Bracketed segments {\textquoteleft}( ){\textquoteright} are those deemed to be non-etymological, that is, typically reflecting some morpheme which has fossilised on a root. The sign {\textquoteleft}\emph{\textup{\~{}{\textquoteright} joins morphological variants of the same lexeme. }}In the correspondence tables, square brackets {\textquoteleft}[]{\textquoteright} are used where an item is cognate but doesn{\textquoteright}t reflect the segment in question. The inverted question mark {\textquoteleft}?`{\textquoteright} is used where a cognate shows an unexpected reflex of the segment in question. Grammatical items are glossed in small caps. Reconstructions marked with {\textquoteleft}!!{\textquoteright} are new reconstructions not found in Holton et al. (2012) or Schapper et al. (2012), or are revised from Holton et al. (
2012). The symbol {\textquoteleft}!!{\textquoteright} signals that the full data set on which the reconstruction in question is based is given in the appendices. AP data supporting the additional pAP reconstructions is given in Appendix I and TK data in Appendix II. In the text of the chapter itself, for reasons of compactness, we only give simple one word glosses which reflect the presumed meaning of the proto-lexeme. Should the reader need more information, he can refer to the appendix. We also do not provide information on irregular changes, such as metathesis or apocope, in the correspondence tables, except where directly relevant to the reconstruction of the segment in question. The appendix provides the reader with fuller information on any irregularities in form or meaning in individual languages. 

\subsection[Reconstruction of bilabial stops]{\textbf{Reconstruction of bilabial stops}}
\hypertarget{RefHeading65407871885726}{}We identify two robust correspondent sets for bilabial plosives, reconstructing to pTAP *p and *b. Note that in Schapper et al. (2012), we reconstruct a three-way distinction (*p, *b, and *f) for bilabial obstruents in pTIM, despite the fact that it is not maintained in any of the modern TK languages: Bunaq, Makasae and Fataluku have merged reflexes of pTIM *p and *f, whereas in Fataluku and Oirata, *f and *b are merged. We find no evidence to support a three-way split in pTAP; instead, it looks like pTIM underwent a conditioned phoneme split, with distinct reflexes of pTAP *b in initial and non-initial positions, respectively. 

Table 1 and Table 2 present the forms for these two correspondence sets respectively. In the first, pAP *p corresponds to pTIM *f in all positions. In the second, pTAP *b was retained as *b in pAP, but split to pTIM *b initially and pTIM *p non-initially. In these sets, there are three notable irregularities: (i) pAP *tiara {\textquoteleft}expel{\textquoteright} lost the medial bilabial that is retained in pTIM *tifar {\textquoteleft}run{\textquoteright}; (ii) pAP *siba(r) {\textquoteleft}new{\textquoteright} and pTIM *(t,s)ifa(r) {\textquoteleft}new{\textquoteright} show an irregular correspondence of pAP *b with pTIM *f; and (iii) pAP *karab {\textquoteleft}scratch{\textquoteright} and pTIM *gabar {\textquoteleft}scratch{\textquoteright}, which show an irregular correspondence of pAP *b with pTIM *b. 

{\centering
Table 1: Correspondence sets for pTAP *p
\par}

\begin{center}
\tablehead{}
\begin{supertabular}{m{0.9094598in}m{0.8580598in}m{0.8934598in}m{0.7684598in}m{0.7691598in}m{0.7691598in}m{0.7691598in}m{0.77335984in}}
\hline
 &
pAP &
pTIM &
Bunaq &
Makasae &
Makalero &
Fataluku &
Oirata\\\hline
\textbf{initial *p} &
\textbf{*p} &
\textbf{*f} &
\textbf{p, w} &
\textbf{f} &
\textbf{f} &
\textbf{f} &
\textbf{p}\\\hline
spit &
*purVn !! &
*fulu(k, n) !! &
puluk &
{}-{}- &
fulun &
fulu &
{}-{}-\\
taboo &
*palol !! &
*falu(n) &
por &
falun &
falun &
falu &
{}-{}-\\
1PI &
*pi- &
*fi &
{}-{}- &
fi &
fi &
afa &
ap-\\
\textsc{low} \textsuperscript{1} &
*po !! &
*ufe !! &
{}-{}- &
he- ?` &
ufe- &
[ua] &
[ua]\\
girl &
*pon !! &
*fana \textsuperscript{2} &
pana &
fana(rae) &
fana(r) &
fana(r) &
pana(rai)\\
scorpion &
*pVr &
*fe(r, R)e !! &
wele &
{}-{}- &
{}-{}- &
{}-{}- &
{}-{}-\\\hline
\textbf{medial *p} &
\textbf{*p} &
\textbf{*f} &
\textbf{w, }\textbf{{\O}} &
\textbf{f} &
\textbf{f} &
\textbf{f} &
\textbf{p}\\\hline
face &
*-pona !! &
*-fanu !! &
{}-ewen &
fanu &
fanu &
fanu &
panu\\
dream &
*hipar &
*ufar(ana) !! &
waen  &
ufarena &
ofarana &
ufarana &
upar(a)\\
run &
[*tiara] &
*tifar &
t{\textesh}iwal &
[ditar] &
[titar] &
tifar(e) &
tipar(e)\\
pound &
*tapai  &
*tafa &
tao \textsuperscript{3} &
{}-{}- &
tafa &
tafa &
tapa\\\hline
\end{supertabular}
\end{center}
\textsuperscript{1} This item is a deictic marker for items at lower elevation than the deictic center. See Schapper (this volume) for more information on this deictic distinction.

\textsuperscript{2} The bracketed \textit{rae/r/rai} element appears to be an innovation in the Eastern Timor languages, presumably a lexical doublet or a derivational morpheme related to the nominalising \textit{{}-r} formative found in Makalero. We have no evidence for reconstructing this element higher than Proto-Eastern Timor.

\textsuperscript{3} This would have originally been *tawo in pre-Bunaq, but in the modern language medial /w/ is not preserved before back vowels. 

{\centering
Table 2: Correspondence sets for pTAP *b
\par}

\begin{center}
\tablehead{}
\begin{supertabular}{m{0.9094598in}m{0.8580598in}m{0.8927598in}m{0.7691598in}m{0.7698598in}m{0.7691598in}m{0.8844598in}m{0.77195984in}}
\hline
 &
pAP &
pTIM &
Bunaq &
Makasae &
Makalero &
Fataluku &
Oirata\\\hline
\textbf{inital *b} &
\textbf{*b} &
\textbf{*b} &
\textbf{b} &
\textbf{b} &
\textbf{p} &
\textbf{p} &
\textbf{h}\\\hline
pig &
*baj &
*baj &
{}-{}- &
baj &
paj &
paj &
haj\\
price &
*bol !! &
*bura &
bol &
bura &
pura &
pura &
hura\\
mat &
*bis &
*biti !! &
{}-{}- &
{}-{}- &
piti &
pet(u) &
het(e)\\
leg &
*-bat !! &
*-buta !! &
{}-but  &
{}-{}- &
{}-{}- &
{}-{}- &
{}-{}-\\
mountain &
*buku !! &
*bugu !! &
{}-{}- &
bu{\textglotstop}u &
pu{\textglotstop}u &
{}-{}- &
{}-{}-\\\hline
\textbf{non-initial *b} &
\textbf{*b} &
\textbf{*p} &
\textbf{p, w} &
\textbf{f} &
\textbf{f} &
\textbf{p} &
\textbf{h}\\\hline
fish &
*habi !! &
*hapi !! &
{}-{}- &
afi &
afi &
api &
ahi\\
star &
*jibV \textsuperscript{1} &
*ipi(-bere) &
[bi] \textsuperscript{2} &
ifi-bere &
ifi &
ipi(naka) &
ihi\\
shark &
* sib(a,i)r !! \textsuperscript{3} &
*supor !! &
{}-{}- &
{}-{}- &
[su] \textsuperscript{4} &
hopor(u)  &
{}-{}-\\
sugarcane &
*hu{\textlengthmark}ba !! &
*upa &
up &
ufa &
ufa &
upa &
uha\\
tongue &
*-lebur !! &
*-ipul &
{}-up &
ifi &
ifil &
epul(u) &
uhul(u)\\
dog &
*jibar !! \textsuperscript{5} &
*Depar &
zap &
defa &
sefar &
ipar(u) &
ihar(a)\\
other &
*aben(VC) !! &
*epi !! &
ewi &
{}-{}- &
{}-{}- &
{}-{}- &
{}-{}-\\
scratch &
*karab !! &
*gabar ?` !! \textsuperscript{6 } &
{}-{}- &
{}-{}- &
kapar  &
kafur(e) &
{}-{}-\\
new &
*siba(r) !! &
*(t,s)ipa(r) ?` !! &
tip  &
sufa &
hofar &
{}-{}-  &
{}-{}-\\\hline
\end{supertabular}
\end{center}
\textsuperscript{1} Several AP languages have a compound for {\textquoteleft}star{\textquoteright}, although the second element does not appear to be cognate to that reconstructed for pTIM. Note also that Holton et al. (2012) gave this item as *jibC; this error has been corrected for Holton and Robinson (this volume).

\textsuperscript{2 }The Bunaq form reflects the second half of the pTIM doublet that is not found in AP languages.

\textsuperscript{3} The cognate set for this item is given in Holton et al. (2012), but no pAP reconstruction is given.

\textsuperscript{4 }The reflex of the relevant bilabial has been lost in Makalero due to apocope.

\textsuperscript{5} The cognate set for this item is given in Holton et al. (2012), but no pAP reconstruction is given.

\textsuperscript{6} This form shows liquid-stop metathesis. There is no evidence of *b occurring word-finally in pTIM.

At this stage, we have no evidence for the reconstruction of a third bilabial obstruent to pTAP, as is found in pTIM (*p, *b and *f), but not pAP (*p and *b). Based on the current correspondence sets, the three-way distinction appears to have arisen due to pTAP medial *b changing to pTIM *p, while pTAP *b stayed *b initially in pTIM. We are yet to find any AP cognates for words reconstructing with initial *p in pTIM.

\subsection[Reconstruction of coronal stops]{\textbf{Reconstruction of coronal stops}}
\hypertarget{RefHeading65409871885726}{}There are two coronal stops, *t and *d, reconstructed to pAP, and four, *t, *d, *T and *D to pTIM. Schapper et al. (2012) note the uncertainty of pTIM *d, which is supported by three cognate sets only, all of which are in initial position. This is played out also when comparing coronals between AP and TK languages. We can reconstruct the pTAP coronal stops *t with relative certainty, and *d, albeit with less security. The latter segment split in pTIM to *T and *D. At present, we cannot reconstruct pTIM *d to pTAP. There are, however, a substantial number of coronal correspondences which remain unexplained. 

Our most consistent correspondence is pTIM *t to pAP *t and *s (Table 3). Initially, we find a steady and unchanging correspondence of pAP *t and pTIM *t, supported by a sizeable number of cognates. Only Bunaq shows a change of *t to /t{\textesh}/ before a high front vowel. Non-initially, we find fewer cognates, but nevertheless a steady and unchanging correspondence. In two cognate sets ({\textquoteleft}sit{\textquoteright} and {\textquoteleft}mat{\textquoteright}), pAP final *s preceded by *i corresponds to pTIM *t.

{\centering
Table 3: Correspondence sets for pTAP *t
\par}

\begin{center}
\tablehead{}
\begin{supertabular}{m{1.0358598in}m{0.7663598in}m{0.8115598in}m{0.6302598in}m{0.7268598in}m{0.6844598in}m{0.77265984in}m{0.6400598in}}
\hline
 &
pAP &
pTIM &
Bunaq &
Makasae &
Makalero &
Fataluku &
Oirata\\\hline
\textbf{initial *t} &
\textbf{*t} &
\textbf{*t } &
\textbf{t, t}\textbf{{\textesh}} &
\textbf{t} &
\textbf{t} &
\textbf{t} &
\textbf{t}\\\hline
hand &
*-tan &
*-tana &
{}-ton &
tana &
tana &
tana &
tana\\
sea &
*tam &
*mata &
[mo] &
{}-{}- &
{}-{}- &
mata &
mata\\
six &
*talam &
*tamal !! \textsuperscript{1} &
tomol &
{}-{}- &
{}-{}- &
{}-{}- &
{}-{}-\\
pound &
*tapai  &
*tafa &
tao &
{}-{}- &
tafa &
tafa &
tapa\\
run &
*tiara &
*tifar &
t{\textesh}iwal &
ditar ?` &
titar &
tifar(e) &
tipar(e)\\
sleep &
*tia &
*tia(r) &
t{\textesh}ier &
ta{\textglotstop}e &
tia &
taia &
taja\\\hline
\textbf{non-initial *t} &
\textbf{*t, *s} &
\textbf{*t } &
\textbf{t} &
\textbf{t} &
\textbf{t} &
\textbf{t} &
\textbf{t}\\\hline
tree &
*tei &
*hate !! &
hotel &
ate &
ate &
ete &
ete\\
stand &
*nate(r) !! &
*nat &
net &
[na] ?` &
nat &
(a)nat(e) &
nat(e)\\
clew &
*maita !! &
*matar &
mot &
{}-{}- &
{}-{}- &
matar(u) &
matar(a)\\
flat &
*tatok !! &
*tetok !! &
toi{\textglotstop} &
{}-{}- &
tetu{\textglotstop} &
{}-{}- &
{}-{}-\\
leg &
*-bat !! &
*-buta !! &
{}-but &
{}-{}- &
{}-{}- &
{}-{}- &
{}-{}-\\
sit &
*mis &
*mit &
mit &
mit\~{}[mi] &
mit &
[(i)mir(e)] ?` &
[mir(e)] ?`\\
mat &
*bis &
*biti !! &
{}-{}- &
{}-{}- &
piti &
pet(u) &
het(e)\\\hline
\end{supertabular}
\end{center}
\textsuperscript{1} Bunaq /o/ is a regular reflex of pTIM *a, as seen, for instance, from the {\textquoteleft}hand{\textquoteright}, {\textquoteleft}sea{\textquoteright}, {\textquoteleft}wake{\textquoteright}, {\textquoteleft}tree{\textquoteright} and {\textquoteleft}clew{\textquoteright} sets. 

The reconstruction of pTAP *d is supported by only a small number of cognate sets (Table 4) and therefore still needs confirmation. In these sets, initial pAP *d corresponds to pTIM *D, while non-initial pAP *d corresponds with pTIM *T. This is consistent with what we observed with the bilabial stops, where a medial voiced stop in pAP corresponds to a voiceless stop in pTIM. Note that the cognate set for {\textquoteleft}bird{\textquoteright} is listed under the heading of initial *d, even though its pTIM and (arguably) pAP reflexes are in medial position. We place it there due to the fact that the sound correspondence is parallel to that for {\textquoteleft}rat{\textquoteright}. However, more sets supporting this reconstruction are clearly needed before we can be certain of it. 

{\centering
Table 4: Correspondence sets for pTAP *d
\par}

\begin{center}
\tablehead{}
\begin{supertabular}{m{0.93585986in}m{0.9219598in}m{0.9219598in}m{0.64695984in}m{0.7754598in}m{0.6851598in}m{0.7580598in}m{0.78025985in}}
\hline
 &
pAP &
pTIM &
Bunaq &
Makasae &
Makalero &
Fataluku &
Oirata\\\hline
\textbf{initial *d} &
\textbf{*d} &
\textbf{*D} &
\textbf{z, s} &
\textbf{d, s} &
\textbf{s} &
\textbf{c} &
\textbf{{\textrtailt}, s}\\\hline
rat &
*dur &
*Dura &
zul &
dura &
sura &
cura &
{\textrtailt}ura\\
dog &
*jibar ?` !! \textsuperscript{1} &
*Depar &
zap &
defa &
sefar  &
[ipar(u)] ?` &
[ihar(a)] ?`\\
bird &
*(a)dVl !! &
*haDa &
hos &
asa &
asa &
aca &
asa\\\hline
\textbf{medial *d} &
\textbf{*d} &
\textbf{*T} &
\textbf{t } &
\textbf{t} &
\textbf{t} &
\textbf{c  } &
\textbf{{\textrtailt}}\\\hline
bat &
*madel &
*maTa !! \textsuperscript{2} &
{}-{}- &
{}-{}- &
{}-{}- &
maca &
ma{\textrtailt}a\\
fire &
*hada !! &
*haTa &
hoto &
ata &
ata &
aca &
a{\textrtailt}a\\
far &
*lete !! &
*eTar !! &
ate &
{}-{}- &
{}-{}- &
icar &
{}-{}-\\
sun &
*wadi !! &
*waTu &
hot &
watu &
watu &
wacu &
wa{\textrtailt}u\\
garden &
*magad(a)  &
[*(u, a)mar] \textsuperscript{3} &
mar  &
ama &
ama &
{}-{}- &
uma\\\hline
\end{supertabular}
\end{center}
\textsuperscript{1} We note the irregularity of pAP *jibar {\textquoteleft}dog{\textquoteright} where we would expect pAP *dipar {\textquoteleft}dog{\textquoteright}. This is likely the result of a change pre-pAP *d {\textgreater} *j.

\textsuperscript{2} The cognate set for this item is given in Schapper et al. (2012), but no pTIM reconstruction is given.

\textsuperscript{3} This form shows metathesis with associated loss of the syllable with pTAP *g, thus: pTAP *magad {\textgreater} *madag {\textgreater} *amar.

Furthermore, there are a range of cognate sets which show as yet unexplained correspondences (Table 5). In these, we find coronal correspondences between pAP and pTIM and between TK languages (especially in Bunaq and Fataluku) that don{\textquoteright}t fit well in the above given sets. More work is needed to clarify the history of the coronals in TAP.Table 5: Problematic coronal cognate sets

\begin{center}
\tablehead{}
\begin{supertabular}{m{0.8726598in}m{0.8705598in}m{0.86635983in}m{0.64765984in}m{0.8650598in}m{0.6844598in}m{0.7587598in}m{0.64835984in}}
\hline
 &
pAP &
pTIM &
Bunaq &
Makasae &
Makalero &
Fataluku &
Oirata\\\hline
grandparent \textsuperscript{1} &
*tam(a, u) !! &
*moTo  &
mata(s) &
mata &
mata &
moco &
mo{\textrtailt}o\\
far &
*lete !! &
*eTar !! &
ate &
{}-{}- &
{}-{}- &
icar &
{}-{}-\\
wake &
*-ten &
*Tani &
otin &
tane &
tane &
tani \~{} cani &
{}-{}-\\
coconut &
*wata !!  &
*wa(t, D)a &
hoza &
wata &
wata &
$\beta $ata &
wata\\
\textit{P. indicus} &
*matar !! &
*ma(t, D)ar &
mazo{\textglotstop} &
mater &
mater &
matar(ia) &
{}-{}-\\
excrement &
*has &
*a(t, D)u !! &
ozo &
atu(-gu{\textglotstop}u) &
atu &
atu &
atu\\\hline
\end{supertabular}
\end{center}
\textsuperscript{1} This is a reciprocal kinship term, denoting either {\textquoteleft}grandparent{\textquoteright} or {\textquoteleft}grandchild{\textquoteright}. PTIM *moTo means {\textquoteleft}child{\textquoteright}.

\subsection[Reconstruction of velar stops]{\textbf{Reconstruction of velar stops}}
\hypertarget{RefHeading65411871885726}{}We reconstruct two velar stops for pTAP, *k and *g. We find insufficient evidence, however, for the uvular stop reconstructed for pAP in Holton et al. (2012) and Holton and Robinson (this volume). 

PTAP *k and *g are retained as *k and *g in pAP, but merged to *g in pTIM. Note that, based on the comparative TAP evidence and the additional pTAP reconstructions in this chapter, we have to substantially revise Schapper et al.{\textquoteright}s (2012) pTIM reconstructions with regard to velar stops. Concretely, we can trace only one pTIM velar back to pTAP. We find no pAP reflexes for any of the small sets of roots reconstructed for pTIM with initial *k and medial *g; those for pTIM medial *g, in particular, are rather tenuous, as noted in Schapper et al. (2012: 212). The cognate sets that we can trace back to pTAP involve Schapper et al.{\textquoteright}s initial *g and medial *k, and the comparative evidence is consistent with these being differential realisations of a single pTIM segment *g: initially, pTIM *g is reflected as /g/ in Bunaq and Makasae, and as /k/ in Makalero and Fataluku. We currently only have no evidence for Oirata. In non-initial position, *g is reflected in Bunaq as /g/ medially and 
as /k/ finally, consistent with Bunaq phonotactic rules, which prohibit voiced stops from codas; in Makasae, Makalero and Fataluku, *g is reflected in non-initial position as /{\textglotstop}/, and variably as /{\textglotstop}/ and {\O} in Oirata. 

The cognate sets that support the reconstruction of pTAP *k are given in Table 6. 

{\centering
Table 6: Correspondence set for pTAP *k
\par}

\begin{center}
\tablehead{}
\begin{supertabular}{m{0.8018598in}m{0.7691598in}m{0.7698598in}m{0.7386598in}m{0.7386598in}m{0.7393598in}m{0.7386598in}m{0.7393598in}}
\hline
 &
pAP &
pTIM &
Bunaq &
Makasae &
Makalero &
Fataluku &
Oirata\\\hline
 &
\textbf{*k} &
\textbf{*g} &
\textbf{g (k)} &
\textbf{(g) {\textglotstop}} &
\textbf{k, }\textbf{{\textglotstop}}\textbf{ } &
\textbf{k, }\textbf{{\textglotstop}}\textbf{ }\textbf{ }\textsuperscript{1} &
\textbf{({\textglotstop}) }\textbf{{\O} }\textsuperscript{1}\\\hline
scratch &
*karab !! &
*gabar !! \textsuperscript{2} &
{}-{}- &
{}-{}- &
kapar  &
kafur(e) &
{}-{}-\\
bite &
*(ta)ki !! \textsuperscript{3} &
*(ga)gel !! \textsuperscript{3} &
gagil &
ga{\textglotstop}el &
ka{\textglotstop}el &
(ki)ki{\textglotstop}(e) \textsuperscript{4} &
{}-{}-\\
dirty &
*karok !! &
*gari !! &
gar  &
ra{\textglotstop}i &
ra{\textglotstop}i &
ra{\textglotstop}e(ne) &
{}-{}-\\
walk 1 &
*laka !! &
*lagar !! &
lagor &
la{\textglotstop}a &
la{\textglotstop}a &
la{\textglotstop}a &
[lare] ?`\\
itchy &
*(i)ruk !!\textsuperscript{ 5} &
*ilag !! &
{}-{}- &
ila{\textglotstop} &
ile{\textglotstop} &
{}-{}- &
{}-{}-\\
mountain &
*buku !! &
*bugu !! &
{}-{}- &
bu{\textglotstop}u &
pu{\textglotstop}u &
{}-{}- &
{}-{}-\\\hline
\end{supertabular}
\end{center}
\textsuperscript{1} See Schapper et al. (2012: 211-212) for more Fataluku and Oirata correspondences.

\textsuperscript{2} This form shows liquid-stop metathesis.

\textsuperscript{3} The bracketed initial segments in these forms reflect different inflectional prefixes which have fossilized on these verbs.

\textsuperscript{4 }The initial bracketed syllable is a fossilized reduplicated CV. This item also has the variant pronunciation \textit{ciki[241?]e}.

\textsuperscript{5} This form represents a different root from the {\textquoteleft}itchy{\textquoteright} root given in Holton and Robinson (this volume). See Appendix I for supporting AP forms.

As in both pTIM (Schapper et al. 2012: 213-214) and pAP (Holton et al. 2012: 98), the reconstruction of initial *g in pTAP hinges on third person markers. Two forms are reconstructable (Table 7): a prefix *g(a,i)- {\textquoteleft}3INLN{\textquoteright} occurring on verbs and inalienably possessed nouns, and a free form *gie {\textquoteleft}3ALN{\textquoteright} encoding 3\textsuperscript{rd} person alienable possessors. Number marking was lost in TK languages, so the correspondence we observe is between pAP third person singular forms and pTIM third person forms which are unmarked for number (i.e., can be used in singular and plural contexts). The zero correspondence that we observe in Fataluku and Oirata is the result of the stripping off of the *g marking 3\textsuperscript{rd} person (as set out in Schapper et al. 2012: 214). In the case of the alienable possessive marker, this means we are left with the possessive root pTIM *-ie {\textquoteleft}ALN{\textquoteright} alone.

{\centering
Table 7: Correspondence set for pTAP 3\textsuperscript{rd} person prefixes
\par}

\begin{center}
\tablehead{}
\begin{supertabular}{m{0.9406598in}m{0.6650598in}m{0.5170598in}m{0.6955598in}m{0.6962598in}m{0.6955598in}m{0.6962598in}m{0.6969598in}}
\hline
 &
pAP &
pTIM &
Bunaq &
Makasae &
Makalero &
Fataluku &
Oirata\\\hline
 &
\textbf{*g} &
\textbf{*g} &
\textbf{g} &
\textbf{g} &
\textbf{k} &
\textbf{{\O}} &
\textbf{{\O}}\\\hline
3INLN &
*ga- &
*g- &
g- &
g- &
k- &
{}-{}- &
{}-{}-\\
3ALN &
*ge \textsuperscript{1} &
*gie &
gie &
gi &
ki &
i &
ue\\\hline
\end{supertabular}
\end{center}
\textsuperscript{1} We reconstruct this as a free form on account of the existence of free reflexes in at least two AP languages (Blagar and Adang); morphologisation must thus post-date the break-up of pAP. 

In non-initial positions, we find numerous cognates reflecting pTAP *g, corresponding to pAP *g and pTIM *g as set out in Table 8. 

{\centering
Table 8: Correspondence set for pTAP *g
\par}

\begin{center}
\tablehead{}
\begin{supertabular}{m{0.64905983in}m{0.97755986in}m{1.0101599in}m{0.7226598in}m{0.7226598in}m{0.7226598in}m{0.7226598in}m{0.7240598in}}
\hline
 &
pAP &
pTIM &
Bunaq &
Makasae &
Makalero &
Fataluku &
Oirata\\\hline
 &
\textbf{*g} &
\textbf{*g} &
\textbf{g, k} &
\textbf{g, {\textglotstop}} &
\textbf{(k) {\textglotstop}} &
\textbf{(k) {\textglotstop}} &
\textbf{{\textglotstop}, {\O}}\\\hline
yellow &
*bagori !! \textsuperscript{1} &
*gabar !! \textsuperscript{2} &
{}-{}- &
gabar &
{}-{}- &
{}-{}- &
{}-{}-\\
green &
*(wa)logar !! &
*ugar  &
ugar &
(h)u{\textglotstop}ur &
(h)u{\textglotstop}ur &
u{\textglotstop}ur(eke) &
u{\textglotstop}ul(e)\\
laugh &
*jagir !! &
*jiger !! &
higal &
hi{\textglotstop}a &
hi{\textglotstop}e &
he{\textglotstop}e &
{}-{}-\\
path &
*jega !! &
*jiga !! &
hik &
hi{\textglotstop}a &
hi{\textglotstop}a &
i{\textglotstop}a &
ia(ra)\\
banana &
*mogol &
*mugu !! &
mok &
mu{\textglotstop}u &
mu{\textglotstop}u &
mu{\textglotstop}u &
mu{\textlengthmark}\\
hear &
*magi !! \textsuperscript{3} &
*mage(n) !! &
mak &
ma{\textglotstop}en &
ma{\textglotstop}en &
{}-{}- &
{}-{}-\\
garden &
*magad(a)  &
[*(u, a)mar] !! \textsuperscript{4} &
mar &
ama &
ama &
{}-{}- &
uma\\\hline
\end{supertabular}
\end{center}
\textsuperscript{1 }The cognate set for this item is given in Holton et al. (2012), but no pAP reconstruction is given.

\textsuperscript{2} This form is apparently metathesised from pTAP *bagori {\textquoteleft}yellow{\textquoteright}.

\textsuperscript{3} The cognate set for this item is given in Holton et al. (2012), but no pAP reconstruction is given.

\textsuperscript{4} This form shows metathesis with associated loss of the syllable with pTAP *g, thus: pTAP *magad {\textgreater} *madag {\textgreater} *amar. Loss of *g is found occasionally in AP languages (e.g. {\textquoteleft}laugh{\textquoteright}, see Appendix 1), suggesting a certain degree of instability for this segment.

Finally, there is as yet an insufficient number of reconstructions of pAP *q with cognates in TK languages to allow for a higher-level pTAP reconstruction. Currently, we have only Bunaq \textit{{}-ol} {\textquoteleft}child{\textquoteright} (presumably reflecting pTIM *-al) as cognate with pAP *-uaqal {\textquoteleft}child{\textquoteright}. We await further reconstructions with TK cognates for the determination of the pTAP form.

\subsection[Reconstruction of fricatives]{\textbf{Reconstruction of fricatives}}
\hypertarget{RefHeading65413871885726}{}Two fricatives *s and *h can be reconstructed to pTAP. The number of cognates is still small for both phonemes, but the correspondences are relatively well-behaved. 

Table 9 sets out the cognate sets for pTAP *s. Initial pTAP *s is supported by several cognate sets and has been maintained without change in pAP and pTIM. Non-initial cognates of pAP *s are difficult to find in TK languages, as many instances of reconstructed word-final *s in pAP correspond to pTIM *t (e.g., pAP *mis {\textquoteleft}sit{\textquoteright}, *bis {\textquoteleft}mat{\textquoteright} and *has {\textquoteleft}excrement{\textquoteright}  [336?]  see also our discussion of these in section 4).

{\centering
Table 9: Correspondences of pTAP *s
\par}

\begin{center}
\tablehead{}
\begin{supertabular}{m{0.9677598in}m{0.9406598in}m{0.9004598in}m{0.63935983in}m{0.66775984in}m{0.7052598in}m{0.8316598in}m{0.65735984in}}
\hline
 &
pAP &
pTIM &
Bunaq &
Makasae &
Makalero &
Fataluku &
Oirata \textsuperscript{1}\\\hline
\textbf{initial *s} &
\textbf{*s} &
\textbf{*s} &
\textbf{s} &
\textbf{s} &
\textbf{h, s }\textsuperscript{2} &
\textbf{h} &
\textbf{s }\\\hline
bone &
*ser !! &
*(se)sa(r, R) !! &
sesal &
{}-{}- &
{}-{}- &
{}-{}- &
{}-{}-\\
shark &
* sib(a ,i)r

!! &
*supor !! &
{}-{}- &
{}-{}- &
su- &
hopor(u) &
{}-{}-\\
spoon &
*surV !! &
*sula &
sulu  &
sulu  &
hulu  &
hula  &
sulu \\
weave &
*sine(N) !! &
*sina &
sien  &
sina  &
hina  &
hina  &
hina(na) ?`\\
new  &
*siba(r) !! \textsuperscript{3} &
*(t, s)ipa(r) !! \textsuperscript{3} &
tip ?` &
sufa &
hofar &
{}-{}- &
{}-{}-\\\hline
\textbf{non-initial *s} &
\textbf{*s} &
\textbf{*s} &
\textbf{s} &
\textbf{s} &
\textbf{s} &
\textbf{h} &
\textbf{{\O}}\\\hline
meat &
*iser !! \textsuperscript{4} &
*seor &
sael &
seu &
seur &
[leura] ?` &
[leura] ?`\\
tooth\ \  &
*-uasin !!  &
*-wasin !! &
[(-e)we] ?` &
wasi &
wasi &
{\ss}ahin(u) &
wain(i)\\\hline
\end{supertabular}
\end{center}
\textsuperscript{1 }See Schapper et al. (2012: 209) for more instances of Oirata cognates.

\textsuperscript{2}\textbf{ }Makalero seems to be part-way through a sound change s {\textgreater} h. See Schapper et al. (2012: 209-211) for more cognates showing the variable s \~{} h reflexes in Makalero. 

\textsuperscript{3} Cognates for these reconstructions show a relatively high degree of irregularity in both AP and TK indicating that there may have been variable realisations in not only pAP and pTIM, but also pTAP.

\textsuperscript{4} Denotes {\textquoteleft}meat{\textquoteright} or {\textquoteleft}game{\textquoteright}.

PTAP *h can be reconstructed as a word-initial segment, but not in other positions. The segment corresponds to pTIM *h and pAP *h except before back vowels (Table 10). Based on the cognate sets available, pAP *h did not occur before back vowels. In this environment, pTAP *h changed either to *w (as in pAP *wur {\textquoteleft}moon{\textquoteright}) or was lost (as in pAP *tei {\textquoteleft}tree{\textquoteright}) in pAP (cf. Table 11 for the items and vocalic environments in which pAP *w is attested). The reconstruction pTIM *h hinges on Bunaq, which retains it as /h/, while the eastern Timor languages have all lost pTIM *h (which, in turn, reflects pTAP *h). This means that where we have no Bunaq reflex (as in the {\textquoteleft}fish{\textquoteright}, {\textquoteleft}breast{\textquoteright} and {\textquoteleft}dream{\textquoteright} sets) we have no modern language attesting pTIM *h, and the presence of the phoneme can only be inferred from the fact that *h is reconstructed for the pAP cognate. 

{\centering
Table 10: Correspondence set for *h
\par}

\begin{center}
\tablehead{}
\begin{supertabular}{m{0.5872598in}m{0.84555984in}m{0.9886598in}m{0.76295984in}m{0.7684598in}m{0.7698598in}m{1.0101599in}m{0.8045598in}}
\hline
 &
pAP &
pTIM &
Bunaq &
Makasae &
Makalero &
Fataluku &
Oirata\\\hline
 &
\textbf{*h (*w/{\O})} &
\textbf{*h } &
\textbf{h} &
\textbf{{\O}} &
\textbf{{\O}} &
\textbf{{\O}} &
\textbf{{\O}}\\\hline
fire &
*hada !! &
*haTa &
hoto &
ata &
ata &
aca &
a{\textrtailt}a\\
fish &
*habi !! &
*hapi !! &
{}-{}- &
afi &
afi &
api &
ahi\\
breast &
*hami &
*hami !! &
{}-{}- &
ami &
{}-{}- &
ami(-tapunu) &
{}-{}-\\
moon &
*wur &
*huru &
hul &
uru &
uru &
uru &
uru\\
tree &
*tei \textsuperscript{1} &
*hate !! &
hotel &
ate &
ate &
ete &
ete\\\hline
\end{supertabular}
\end{center}
\textsuperscript{1} The loss of initial syllable may have to do with the fact that stress was apparently based on syllable weight. See also {\textquoteleft}dog{\textquoteright} in Appendix 1 and Holton and Robinson (this volume). 

\subsection[Reconstruction of glides]{\textbf{Reconstruction of glides}}
\hypertarget{RefHeading65415871885726}{}Two glides can be reconstructed to pTAP, *w and *j. Both appear to have only occurred in initial position. It is unclear whether the reconstructed glides could occur before all vowel qualities. Nevertheless, the cognate sets supporting these proto-phonemes are robust and show little irregularity.

The pTAP glide *w shows a stable and unchanging correspondence of *w in pAP and pTIM for the most part (Table 11). The major change is that pTAP *w is vocalised in pAP to *u root-initially on inalienably possessed nouns. In TK languages, Bunaq shows conditioned reflexes of pTAP *w, maintaining it as /w/ before front vowels, but changing it to /h/ before non-back vowels. Fataluku shows the change of *w to /{\ss}/, though we note that this is an allophone of /w/ in many languages.

{\centering
Table 11: Correspondence set for pTAP *w
\par}

\begin{center}
\tablehead{}
\begin{supertabular}{m{0.8927598in}m{0.8573598in}m{0.7788598in}m{0.7788598in}m{0.7788598in}m{0.7788598in}m{0.7781598in}m{0.78305984in}}
\hline
 &
pAP &
pTIM &
Bunaq &
Makasae &
Makalero &
Fataluku &
Oirata\\\hline
 &
\textbf{*w, *u} &
\textbf{*w} &
\textbf{h, w} &
\textbf{w} &
\textbf{w} &
\textbf{{\ss}} &
\textbf{w}\\\hline
blood &
*wai &
*waj &
ho &
waj &
wej &
{\ss}ehe &
we\\
coconut &
*wata !! &
*wa(t, D)a &
hoza &
wata &
wata &
{\ss}aca &
wata\\
stone &
*war &
*war &
hol &
{}-{}- &
war &
{}-{}- &
war(aha)\\
sun &
*wadi !! &
*waTu &
hot &
watu &
watu &
{\ss}acu &
wa{\textrtailt}u\\
bathe &
*weli  &
*weru &
wer &
waru{\textglotstop} &
waro{\textglotstop} &
{\ss}ahu &
wau\\
ear &
*-uari !! &
*-wali &
{}-{}- &
wala(ku{\textlengthmark}) &
wali &
{\ss}ali &
wali\\
tooth &
*-uasin !! &
*-wasin !! &
{}-(e)we &
wasi &
wasi &
{\ss}ahin(u) &
wain(i)\\\hline
\end{supertabular}
\end{center}
Table 12 gives the four clear cognate sets that we have across TAP languages for pTAP *j. We see that pTAP *j is maintained as *j in pAP, but is variably lost or maintained as *j in pTIM. It may be that differing vocalic environments in pTAP conditioned the different reflexes in pTIM, but we don{\textquoteright}t have enough understanding of the history of vowels yet to determine this. There is no direct evidence for pTIM *j, that is, no TK language still reflects the proto-phoneme as /j/, but the sound correspondences between TK languages make it differentiable form sets reflecting pTIM *h (see Table 10).

{\centering
Table 12: Correspondence set for pTAP *j
\par}

\begin{center}
\tablehead{}
\begin{supertabular}{m{0.9219598in}m{0.8393598in}m{0.78025985in}m{0.78025985in}m{0.7795598in}m{0.78025985in}m{0.78025985in}m{0.78095984in}}
\hline
 &
pAP &
pTIM &
Bunaq &
Makasae &
Makalero &
Fataluku &
Oirata\\\hline
 &
\textbf{*j} &
\textbf{j, {\O}} &
\textbf{h, {\O}} &
\textbf{h, {\O}} &
\textbf{h, {\O}} &
\textbf{{\O}} &
\textbf{{\O}}\\\hline
star &
*jibV &
*ipi(-bere) &
[bi] &
ifi-bere &
ifi &
ipi(-naka) &
ihi\\
water &
*jira &
*ira &
il &
ira &
ira &
ira &
ira\\
laugh &
*jagir !! &
*jiger !!  &
higal &
hi{\textglotstop}a &
hi{\textglotstop}e &
he{\textglotstop}e ?` &
{}-{}-\\
path &
*jega !! &
*jiga !! &
hik  &
hi{\textglotstop}a &
hi{\textglotstop}a &
i{\textglotstop}a &
ia(ra)\\\hline
\end{supertabular}
\end{center}
\subsection[Reconstruction of liquids]{\textbf{Reconstruction of liquids}}
\hypertarget{RefHeading65417871885726}{}We identify three robust liquid correspondence sets between pAP and pTIM and as such reconstruct three pTAP liquids: *r, *R, and *l.

The most robust set is that for pTAP *r, which is reflected as *r in both pAP and pTIM (Table 13). PTAP *r is only found in non-initial positions, as are its reflexes in the daughter languages pAP and pTK. Word-finally in polysyllabic words pTAP *r is particularly susceptible to sporadic loss, as is attested by the various irregular forms in Table 13. In one instance (pTAP *sibar {\textquoteleft}new{\textquoteright}), the occurrence of a reflex of final *r is so erratic in both primary subgroups that we perhaps must consider it already partly lost in pTAP{\textquoteright}s daughter languages.

{\centering
Table 13: Correspondence set for pTAP *r
\par}

\begin{center}
\tablehead{}
\begin{supertabular}{m{0.79205984in}m{0.65665984in}m{-0.045440156in}m{0.89625984in}m{-0.053040158in}m{0.72955984in}m{-0.060640156in}m{0.75185984in}m{0.7538598in}m{0.7781598in}m{0.75255984in}}
\hline
 &
\multicolumn{2}{m{0.6899598in}}{pAP} &
\multicolumn{2}{m{0.9219598in}}{pTIM} &
\multicolumn{2}{m{0.74765986in}}{Bunaq} &
Makasae &
Makalero &
Fataluku &
Oirata\\\hline
 &
\multicolumn{2}{m{0.6899598in}}{\textbf{*r}} &
\multicolumn{2}{m{0.9219598in}}{\textbf{*r}} &
\multicolumn{2}{m{0.74765986in}}{\textbf{l}} &
\textbf{r} &
\textbf{r} &
\textbf{r} &
\textbf{r}\\\hline
run &
\multicolumn{2}{m{0.6899598in}}{*tiara} &
\multicolumn{2}{m{0.9219598in}}{*tifar} &
\multicolumn{2}{m{0.74765986in}}{t{\textesh}iwal} &
ditar &
titar &
tifar(e) &
tipar(e)\\
moon &
\multicolumn{2}{m{0.6899598in}}{*wur} &
\multicolumn{2}{m{0.9219598in}}{*huru} &
\multicolumn{2}{m{0.74765986in}}{hul} &
uru &
uru &
uru &
uru\\
rat &
\multicolumn{2}{m{0.6899598in}}{*dur} &
\multicolumn{2}{m{0.9219598in}}{*Dura} &
\multicolumn{2}{m{0.74765986in}}{zul} &
dura &
sura &
cura &
{\textrtailt}ura\\
stone &
\multicolumn{2}{m{0.6899598in}}{*war} &
\multicolumn{2}{m{0.9219598in}}{*war} &
\multicolumn{2}{m{0.74765986in}}{hol} &
{}-{}- &
war &
{}-{}- &
war(aha)\\
vagina &
\multicolumn{2}{m{0.6899598in}}{*-ar} &
\multicolumn{2}{m{0.9219598in}}{*-aru} &
\multicolumn{2}{m{0.74765986in}}{{}-{}-} &
aru &
aru &
aru &
aru\\
water &
\multicolumn{2}{m{0.6899598in}}{*jira} &
\multicolumn{2}{m{0.9219598in}}{*ira} &
\multicolumn{2}{m{0.74765986in}}{il} &
ira &
ira &
ira &
ira\\
crawl &
\multicolumn{2}{m{0.6899598in}}{*er !!} &
\multicolumn{2}{m{0.9219598in}}{*er !!} &
\multicolumn{2}{m{0.74765986in}}{el} &
{}-{}- &
{}-{}- &
er(eke) &
{}-{}-\\
dream &
\multicolumn{2}{m{0.6899598in}}{*hipar} &
\multicolumn{2}{m{0.9219598in}}{*ufar(ana) !!} &
\multicolumn{2}{m{0.74765986in}}{[waen] ?`} &
ufarena &
ofarana &
ufar(e) &
upar(a)\\
meat &
*iser !! &
\multicolumn{2}{m{0.9295599in}}{*seor} &
\multicolumn{2}{m{0.7552598in}}{sael} &
\multicolumn{2}{m{0.76995987in}}{[seu] ?`} &
seur &
leura &
leura\\
dog &
\multicolumn{2}{m{0.6899598in}}{*jibar !!} &
\multicolumn{2}{m{0.9219598in}}{*Depar} &
\multicolumn{2}{m{0.74765986in}}{[zap] ?`} &
[defa] ?` &
sefar  &
ipar(u) &
ihar(a)\\
bamboo &
\multicolumn{2}{m{0.6899598in}}{*mari} &
\multicolumn{2}{m{0.9219598in}}{*mari} &
\multicolumn{2}{m{0.74765986in}}{[ma] ?`} &
maeri &
mar &
{}-{}- &
{}-{}-\\
\textit{P. indicus} &
\multicolumn{2}{m{0.6899598in}}{*matar !!} &
\multicolumn{2}{m{0.9219598in}}{*ma(t, D)er} &
\multicolumn{2}{m{0.74765986in}}{[mazo{\textglotstop}] ?`} &
mater &
mater &
matar(ia) &
{}-{}-\\
shark &
\multicolumn{2}{m{0.6899598in}}{*sib(a, i)r !! } &
\multicolumn{2}{m{0.9219598in}}{*supor !!} &
\multicolumn{2}{m{0.74765986in}}{{}-{}-} &
{}-{}- &
[su] ?` &
hopor(u) &
{}-{}-\\
new &
\multicolumn{2}{m{0.6899598in}}{*siba(r) !! } &
\multicolumn{2}{m{0.9219598in}}{*(t, s)ipa(r) !!} &
\multicolumn{2}{m{0.74765986in}}{[tip] ?`} &
[sufa] ?` &
hofar &
{}-{}- &
{}-{}-\\\hline
\end{supertabular}
\end{center}
PTAP *R is reflected in pAP as *r and in pTIM as *l. Like pTAP *r, *R does not appear in word-initial positions and is sporadically lost word-finally in polysyllabic words. The sets supporting the reconstruction of *R (Table 14) are also fewer and less robust than for pTAP *r. 

{\centering
Table 14: Correspondence set for pTAP *R
\par}

\begin{center}
\tablehead{}
\begin{supertabular}{m{0.7650598in}m{0.7393598in}m{0.88725984in}m{0.64835984in}m{0.77055985in}m{0.77055985in}m{0.77055985in}m{0.77265984in}}
\hline
 &
pAP &
pTIM &
Bunaq &
Makasae &
Makalero &
Fataluku &
Oirata\\\hline
 &
\textbf{*r} &
\textbf{*l} &
\textbf{l} &
\textbf{l} &
\textbf{l} &
\textbf{l} &
\textbf{l}\\\hline
spoon &
*surV !! &
*sula &
sulu &
sulu &
hulu &
hula &
{}-{}-\\
tail &
*-ora !! &
*-ula({\textglotstop}) &
{}-ulo({\textglotstop}) &
ula &
ula &
ula(fuka) &
ula(pua)\\
tongue &
*-lebur !! &
*-ipul &
[-up] ?` &
[ifi] ?` &
ifil &
epul(u) &
uhul(u)\\
laugh &
*jagir !! &
*jiger &
higal &
[hi{\textglotstop}a] ?` &
[hi{\textglotstop}a] ?` &
[he{\textglotstop}e] ?` &
{}-{}-\\
spit &
*purVn !! &
*fulu(k, n) !! &
puluk &
{}-{}- &
fulun &
fulun &
{}-{}-\\
ear &
*-uari !! &
*-wali &
{}-{}- &
wala(ku{\textlengthmark}) &
wali &
{\ss}ali &
wali\\\hline
\end{supertabular}
\end{center}
The three pTIM cognates listed in Table 15 are based on Bunaq only, in which pTIM *r and *R are merged. We have thus no means of determining whether these forms are to be reconstructed to pTAP with *r or with *R.

{\centering
Table 15: Cognate sets reconstructable to either pTAP *r or *R
\par}

\begin{center}
\tablehead{}
\begin{supertabular}{m{0.76435983in}m{0.6615598in}m{0.88725984in}m{0.6497598in}m{0.7698598in}m{0.7698598in}m{0.7698598in}m{0.77335984in}}
\hline
 &
pAP &
PTIM &
Bunaq &
Makasae &
Makalero &
Fataluku &
Oirata\\\hline
 &
\textbf{*r} &
\textbf{*(r, R)} &
\textbf{l} &
\textbf{{}-{}-} &
\textbf{{}-{}-} &
\textbf{{}-{}-} &
\textbf{{}-{}-}\\\hline
bone &
*ser !! &
*(se)sa(r, R) !! &
sesal &
{}-{}- &
{}-{}- &
{}-{}- &
{}-{}-\\
scorpion &
*pVr &
*fe(r, R)e !! &
wele &
{}-{}- &
{}-{}- &
{}-{}- &
{}-{}-\\
rain &
*anur !! &
*ine(r, R) !! &
inel &
{}-{}- &
{}-{}- &
{}-{}- &
{}-{}-\\\hline
\end{supertabular}
\end{center}
Cognate sets for pTAP *l are relatively infrequent in both pAP and pTIM (Table 16).\footnote{ Holton and Robinson (this volume) remark that, even though correspondences appear relatively regular for initial and medial *l in pAP, they can identify only a few cognates that are widely distributed across the AP subgroup. Similarly, Schapper et al. (2012) caution that their reconstruction for pTIM *l cannot yet be called secure due to the small number of cognate sets identified.} Cognates reflecting initial pTAP *l with pAP *l and pTIM *l (i.e., {\textquoteleft}bark{\textquoteright}, {\textquoteleft}new place{\textquoteright} and {\textquoteleft}crouch{\textquoteright}) have only a low degree of certainty. Based on the data available, there also appears to be a tendency to lose pTAP initial *l in pTIM, as in {\textquoteleft}far{\textquoteright}, {\textquoteleft}tongue{\textquoteright} and {\textquoteleft}green{\textquoteright}, but a clear conditioning environment for this is not yet obvious. Word-finally in 
polysyllabic words, pTAP *l is regularly lost in pTIM, as in {\textquoteleft}banana{\textquoteright}, {\textquoteleft}bat{\textquoteright}, {\textquoteleft}bird{\textquoteright} and {\textquoteleft}taboo{\textquoteright}, However, it is retained in {\textquoteleft}walk 2{\textquoteright} and {\textquoteleft}six{\textquoteright}, apparently due to nasal-liquid metathesis, and in {\textquoteleft}child{\textquoteright} due to the loss of the item{\textquoteright}s medial syllable with *q prior to the application of the final polysyllabic deletion rule in pAP.

{\centering
Table 16: Correspondence set for pTAP *l
\par}

\begin{center}
\tablehead{}
\begin{supertabular}{m{0.7768598in}m{0.97815984in}m{0.88585985in}m{0.71775985in}m{0.7268598in}m{0.72815984in}m{0.72885984in}m{0.7240598in}}
\hline
 &
pAP &
pTIM &
Bunaq &
Makasae &
Makalero &
Fataluku &
Oirata\\\hline
 &
\textbf{*l} &
\textbf{*l, {\O}} &
\textbf{l ({\O})} &
\textbf{l ({\O})} &
\textbf{l ({\O})} &
\textbf{l ({\O})} &
\textbf{l ({\O})}\\\hline
bark &
*lVu &
*le(k)u(l) !! &
{}-{}- &
leu &
leu &
le{\textglotstop}ul(e) &
leul(e)\\
new place &
*lan !! &
*lan !! &
lon &
{}-{}- &
{}-{}- &
{}-{}- &
{}-{}-\\
crouch &
*luk(V) &
*luk !! &
lu{\textglotstop}(-lu{\textglotstop}) &
{}-{}- &
{}-{}- &
{}-{}- &
{}-{}-\\
far &
*lete !! &
[*eTar] !! &
ate &
{}-{}- &
{}-{}- &
icar &
{}-{}-\\
tongue &
*-lebur !! &
[*-ipul] &
{}-up &
ifi &
ifil &
epul(u) &
uhul(u)\\
green &
*(wa)logar !! &
[*ugar] &
ugar &
hu{\textglotstop}ur &
(h)u{\textglotstop}ur &
u{\textglotstop}ur(eke) &
u{\textglotstop}ul(e)\\
banana &
*mogol &
[*mugu] !! &
mok &
mu{\textglotstop}u &
mu{\textglotstop}u &
mu{\textglotstop}u &
mu{\textlengthmark}\\
bat &
*madel &
[*maTa] !! &
{}-{}- &
{}-{}- &
{}-{}- &
maca &
ma{\textrtailt}a\\
bird &
*(a)dVl !! &
[*haDa] &
hos &
asa &
asa &
aca &
asa\\
taboo &
*palol !! &
[*falu(n)] &
por &
falun &
falun &
falu &
{}-{}-\\
walk 2 &
*lam(ar) !! &
*male !! &
mele &
{}-{}- &
{}-{}- &
{}-{}- &
{}-{}-\\
six &
*talam &
*tamal !! &
tomol &
{}-{}- &
{}-{}- &
{}-{}- &
{}-{}-\\
child &
*-uaqal &
*-al !! &
{}-ol &
{}-{}- &
{}-{}- &
{}-{}- &
{}-{}-\\\hline
\end{supertabular}
\end{center}
Finally, there are several cases in which the appearance of liquids in AP and TK languages can be reconciled with none of the three sets we have identified here. Table 17 lists these problematic instances (the relevant segments are bolded). These sets pointedly express that we are still a long way away from a complete understanding of liquids in pTAP. 

{\centering
Table 17: Problematic liquid cognate sets
\par}

\begin{center}
\tablehead{}
\begin{supertabular}{m{0.6295598in}m{0.97815984in}m{0.7698598in}m{0.72055984in}m{0.71985984in}m{0.71985984in}m{0.72055984in}m{0.7226598in}}
\hline
 &
pAP &
pTIM &
Bunaq &
Makasae &
Makalero &
Fataluku &
Oirata\\\hline
price &
*bo\textbf{l} !! &
*bu\textbf{r}a &
bo\textbf{l} &
bu\textbf{r}a &
pu\textbf{r}a &
pu\textbf{r}a &
hu\textbf{r}a\\
bathe &
*we\textbf{l}i &
*we\textbf{r}u &
we\textbf{r}  &
wa\textbf{r}u{\textglotstop} &
wa\textbf{r}o{\textglotstop} &
vahu ?` &
wau ?`\\
garden &
*maga\textbf{d}(a) &
*(u, a)ma\textbf{r} &
ma\textbf{r} &
[ama] &
[ama] &
{}-{}- &
[uma]\\
green &
*(wa)loga\textbf{r} !! &
*uga\textbf{r} &
uga\textbf{r} &
(h)u{\textglotstop}u\textbf{r} &
(h)u{\textglotstop}u\textbf{r} &
u{\textglotstop}u\textbf{r}(eke) &
u{\textglotstop}u\textbf{l}(e) ?`\\
taboo &
*pa\textbf{l}ol !! &
*fa\textbf{l}u(n) &
po\textbf{r} &
fa\textbf{l}un &
fa\textbf{l}un &
fa\textbf{l}u &
{}-{}-\\\hline
\end{supertabular}
\end{center}
\subsection[Reconstruction of nasals]{\textbf{Reconstruction of nasals}}
\hypertarget{RefHeading65419871885726}{}Two nasals can be reconstructed to pTAP, *m and *n. For the most part, they are relatively stable and unchanging in both pAP and pTIM.

Table 18 presents a selection of the many cognate sets for pTAP *m. In word-initial position, pTAP *m corresponds unproblematically to pAP *m and pTIM *m. Identifying non-initial instances of pTAP *m is somewhat more difficult, with *hami {\textquoteleft}breast{\textquoteright} being the only straightforward case. Word-final *m in pAP has only non-final reflexes in pTIM, apparently because, as in the modern TK languages, word-final *m was not permitted. This issue is resolved in pTIM through metathesis of the nasal out of the final position, as in {\textquoteleft}sea{\textquoteright} and {\textquoteleft}six{\textquoteright}. Other instances of medial pTIM *m correspond to root-initial *m in pAP (as in {\textquoteleft}garden{\textquoteright} and {\textquoteleft}die{\textquoteright}). 

{\centering
Table 18: Correspondence sets for pTAP *m
\par}

\begin{center}
\tablehead{}
\begin{supertabular}{m{1.0518599in}m{0.88865983in}m{0.77055985in}m{0.77055985in}m{0.7698598in}m{0.77055985in}m{1.0129598in}m{0.8087598in}}
\hline
 &
pAP &
pTIM &
Bunaq &
Makasae &
Makalero &
Fataluku &
Oirata\\\hline
\textbf{initial *m} &
\textbf{*m} &
\textbf{*m} &
\textbf{m} &
\textbf{m} &
\textbf{m} &
\textbf{m} &
\textbf{m}\\\hline
bamboo &
*mari &
*mari &
ma &
maeri &
mar &
{}-{}- &
{}-{}-\\
banana &
*mogol &
*mugu !! &
mok &
mu{\textglotstop}u &
mu{\textglotstop}u &
mu{\textglotstop}u &
mu{\textlengthmark}\\
sit &
*mis &
*mit &
mit &
mit \~{} [mi] &
mit &
(i)mir(e) &
mir(e)\\
bat &
*madel &
*maTa !! &
{}-{}- &
{}-{}- &
{}-{}- &
maca &
ma{\textrtailt}a\\
inside &
*mi &
*mi &
mi(l) &
mu(tu) &
mu(tu-) &
mu(cu) &
mu({\textrtailt}u)\\
hear &
*magi !! &
*mage(n) !!  &
mak &
ma{\textglotstop}en &
ma{\textglotstop}en &
{}-{}- &
{}-{}-\\\hline
\textbf{non-initial *m} &
\textbf{*m} &
\textbf{*m} &
\textbf{m} &
\textbf{m} &
\textbf{m} &
\textbf{m} &
\textbf{m}\\\hline
breast &
*hami &
*hami !! &
{}-{}- &
ami &
{}-{}- &
ami(-tapunu) &
{}-{}-\\
sea &
*tam &
*mata &
mo &
{}-{}- &
{}-{}- &
mata &
mata\\
six &
*talam &
*tamal !! &
tomol &
{}-{}- &
{}-{}- &
{}-{}- &
{}-{}-\\
garden &
*magad(a) &
*(u, a)mar !! &
mar &
ama &
ama &
{}-{}- &
uma\\
die &
*min(a) &
*-umV &
{}-ume &
umu &
(k)umu &
umu &
umu\\
nose &
*-mim &
*-muni !! &
[-inup] ?` &
muni(kai) &
mini &
mini(ku) &
{}-{}-\\\hline
\end{supertabular}
\end{center}
Table 19 presents the many cognate sets for pTAP *n. Initial and medial correspondences are abundant, but final correspondences are difficult to identify. pTIM *n did not appear to occur in final position; all instances of pAP final *n are either followed by a vowel or are lost in pTIM.

{\centering
Table 19: Correspondence sets for *n
\par}

\begin{center}
\tablehead{}
\begin{supertabular}{m{1.0538598in}m{0.8608598in}m{0.7788598in}m{0.7788598in}m{0.7781598in}m{0.7788598in}m{0.7788598in}m{0.7594598in}}
\hline
 &
pAP &
pTIM &
Bunaq &
Makasae &
Makalero &
Fataluku &
Oirata\\\hline
\textbf{initial *n} &
\textbf{*n} &
\textbf{*n} &
\textbf{n} &
\textbf{n} &
\textbf{n} &
\textbf{n} &
\textbf{n}\\\hline
stand &
*nate(r) !! &
*nat &
net &
nat \~{} na &
nat &
(a)nat(e) &
nat(e)\\
\textsc{1sg} &
*na- &
*n-~!! &
n- &
{}-{}- &
{}-{}- &
{}-{}- &
{}-{}-\\
eat &
*nai &
*nua !! &
[a \~{} -ia] &
nawa &
nua &
una, na$\beta $a &
una, nawa\\
one &
*nuk &
*uneki~!! &
uen, en &
[u] &
[u] \~{} un &
ukani &
a{\textglotstop}uni\\\hline
\textbf{non-initial *n} &
\textbf{*n} &
\textbf{*n} &
\textbf{n} &
\textbf{n} &
\textbf{n} &
\textbf{n} &
\textbf{n}\\\hline
face &
*-pona !! &
*-fanu !! &
{}-(e)wen &
fanu &
fanu &
fanu &
panu\\
ripe &
*tena !! &
*tena !! &
ten &
tina &
tina &
{}-{}- &
{}-{}-\\
name &
*-en(i, u) !! &
*-nej &
{}-ini(l) &
naj &
nej &
ne &
ne{\textlengthmark}(ne)\\
give &
*-ena &
*-inV &
{}-ini &
(g)ini &
(k-)ini &
ina &
ina\\
wake &
*-ten &
*Tani &
otin &
tane &
tane &
tani \~{} cani &
{}-{}-\\
girl &
*pon !! &
*fana &
pana &
fana(rae) &
fana(r) &
fana(r) &
pana(rai)\\
person &
*anin !! &
*anu~!! &
en &
anu  &
anu  &
{}-{}- &
{}-{}-\\
other &
*aben(VC) !! &
*epi !! &
[ewi] &
{}-{}- &
{}-{}- &
{}-{}- &
{}-{}-\\\hline
\end{supertabular}
\end{center}
\section[Summary of correspondences and reconstructed \ \ phonemes]{Summary of correspondences and reconstructed \ \ phonemes}
\hypertarget{RefHeading65421871885726}{}For the first time since the start of TAP studies some sixty years ago (see Schapper and Huber forthcoming for a historical perspective on TAP studies), we have rigorously shown in this chapter that the TAP languages form a family: the regularity of sound correspondences in cognate vocabulary demonstrates that the AP and TK Papuan languages are indeed genetically related to one another.

In Table 20, we provide an overview of the consonant correspondences we observed in cognate vocabulary between pAP and pTIM and their reconstruction in their ancestral language pTAP. In this table, we indicate whether the correspondence applies in initial (\#\_ ), medial (V\_V), or final ( \_\#) position. An empty slot means that there is no particular conditioning environment for the correspondence. The symbol {\textquoteleft}{\O}{\textquoteright} in a column indicates that a pTAP sound is lost in the daughter language in question. 

{\centering
Table 20: Summary of sound correspondences from pTAP to pAP and pTIM
\par}

\begin{center}
\tablehead{}
\begin{supertabular}{m{0.9240598in}m{0.9247598in}m{0.9247598in}m{0.9240598in}}
\hline
pTAP &
environment &
pAP &
pTIM\\\hline
*p &
 &
*p &
*f\\
*b &
\#\_ &
*b &
*b\\
 &
V\_V &
*b &
*p\\
*t &
\#\_ &
*t &
*t\\
 &
V\_V, \_\# &
*t, *s &
*t\\
*d &
\#\_ &
*d &
*D\\
 &
V\_V &
*d &
*T\\
*k &
 &
*k &
*k\\
*g &
 &
*g &
*g\\
*s &
 &
*s &
*s\\
*h &
 &
*h (*w/{\O}) &
*h\\
*w &
 &
*w, *u &
*w\\
*j &
 &
*j &
*j, {\O}\\
*r &
 &
*r &
*r\\
*R &
 &
*r &
*l\\
*l &
 &
*l &
*l, {\O}\\
*m &
 &
*m &
*m\\
*n &
 &
*n &
*n\\\hline
\end{supertabular}
\end{center}
\section[Discussion]{Discussion}
\hypertarget{RefHeading65423871885726}{}Whilst we have been able to show clearly that AP and TK languages are related to one another, the comparative data presented here draws into question a number of aspects of the existing reconstructions of pAP and pTIM and necessitates revisions to these. In this final section, we will draw attention to the issues, provide a general discussion of them and suggest some possible solutions to them. 

A major issue for the current pAP reconstruction is the apparent invalidity of many word-final consonant reconstructions. It is argued in Holton et al. (2012: 95) that the gemination of medial stops in modern Western Pantar can be used as a diagnostic for determining whether a given pAP root was consonant-final or vowel-final. Specifically, the authors claim that geminate medial stops in modern Western Pantar reflect pAP medial stops, whereas non-geminate medial stops in Western Pantar reflect an original consonant-final form, or perhaps a borrowing from another AP language. However, this argument cannot be sustained on closer inspection of the comparative evidence. Consider the items in Table 21 that are reconstructed as basically consonant final in pAP because of the lack of stop gemination in WP. In each case, we have between three and nine reflexes in modern AP languages with a V(C) following the supposed historically final consonant. We must ask ourselves where so many additional final segments came 
from in so many of these languages. Holton et al. (2012) seek to explain these appearances with vowel epenthesis. Yet, under this scenario, we would expect to be able to predict the type of the epenthetic vowel from the shape of the root, but this is not the case; instead, the epenthetic vowels are of all different values from one item to the next and bear no apparent relationship to the vowel of the root (as defined by Holton et al. 2012). What is more, the final V(C) elements we observe in AP languages are not erratic, rather they in general adhere to correspondences observed elsewhere. This suggests that these final V(C) elements were not epenthetic to the items after the break-up of pAP, but have been inherited from pAP. This is further supported by the fact that we find clearly corresponding V(C) segments on cognate vocabulary in TK languages, meaning that the segments reconstruct to pTAP and that they were inherited into pAP. The alternative leaves us without explanation for the cognacy of the final 
segments in these (and other items) across the family. 

{\centering
Table 21: Dubious consonant-final reconstructions in AP and beyond
\par}

\begin{center}
\tablehead{}
\begin{supertabular}{m{0.93375987in}m{0.77335984in}m{0.77405983in}m{0.7747598in}m{0.77335984in}m{0.77335984in}m{0.7768598in}}
\hline
 &
{\textquoteleft}fish{\textquoteright} &
{\textquoteleft}sun{\textquoteright} &
{\textquoteleft}fire{\textquoteright} &
\centering {\textquoteleft}coconut{\textquoteright} &
\centering {\textquoteleft}tongue{\textquoteright} &
{\textquoteleft}ripe{\textquoteright}\\\hline
pTAP &
\textbf{*habi} &
\textbf{*wad(u, i)} &
\textbf{*hadi} &
\textbf{*wata} &
\textbf{*(l)ebur} &
\textbf{*tena}\\
pTIM &
*hapi &
*waTu &
*haTa &
*wa(t, D)a &
*-ipul &
*tena\\
pAP original &
*hab(i) &
*wad(i) &
*had(a) &
*wat(a) &
*-leb(ur) &
*ten\\\hline
Teiwa &
[127?]a{\textphi} &
war &
[127?]ar &
wat &
{}-livi &
tanan\\
Nedebang &
a{\textlengthmark}fi &
(get) &
ar &
wata &
{}-lefu &
{}-{}-\\
Kaera &
ab &
wer &
ad &
wat &
{}-leb &
ten \\
WPantar &
hap &
wer &
a{\textlengthmark}d &
wata &
{}-lebu &
ta{\ng}\\
Blagar &
a{\textlengthmark}b &
war &
{}-{}- &
vet &
{}-lebul &
tena\\
Adang &
a{\textlengthmark}b &
ved &
 &
fa{\textglotstop} &
{}-lib(u{\ng}) &
tene\\
Klon &
{\textschwa}bi &
f{\textepsilon}d &
{\textschwa}da &
{}-{}- &
{}-l{\textepsilon}b &
{\textschwa}t{\textepsilon}n\\
Kui &
eb &
{}-{}- &
ar &
bat &
{}-liber &
tain\\
Abui &
afu &
wari &
ara &
wata &
{}-lifi &
{}-{}-\\
Kamang &
api &
wati &
ati &
wate &
{}-opui {\textonesuperior} &
iten \~{} iton\\
Sawila &
api &
wadi &
ada &
wata &
{}-li(m)puru &
iti{\textlengthmark}na\\
Wersing &
api &
widi &
ada &
wata &
{}-jebur &
{}-{}-\\\hline
\end{supertabular}
\end{center}
{\textonesuperior} Holton et al. (2012) state that these and other Kamang forms missing pAP *l medially are irregular. However, pAP *l is regularly lost in Kamang between non-front vowels, e.g., pAP *talam {\textquoteleft}six{\textquoteright} {\textgreater} Kamang \textit{ta{\textlengthmark}m}, pAP *palol {\textquoteleft}taboo{\textquoteright}{\textgreater} Kamang \textit{fo{\textlengthmark}i}, etc. The vowel of the inalienable possessive prefix is /a/, thus providing the right environment for the loss in \textit{{}-opui} {\textquoteleft}tongue{\textquoteright} of the root-initial /l/.  

The problem then is how to explain medial geminate and non-geminate stops in WP. One answer would to be maintain that the difference in stop gemination was still due to a final- non-final distinction. For example, it could be said that the loss of the final vowel occurred after the breakup of pAP but prior to the application of the gemination rule. This cannot, however, be fully sustained as WP has in some cases final vowels which clearly reflect pTAP and pAP (e.g., {\textquoteleft}tongue{\textquoteright}). A more attractive explanation is presented by stress-induced gemination. Although little is known about the historical prosody of TAP, it seems a good possibility that WP gemination may have been a result of final stress. That is, we suggest roots of the shape /(C)V{\textprimstress}CV(C)/ surfaced as [(C)V{\textprimstress}C{\textlengthmark}V(C)], while roots of the shape /{\textprimstress}(C)VCV(C)/ surfaced as [{\textprimstress}(C)VCV(C)]. While this scenario remains to be confirmed by a more detailed 
study, discarding the final/non-final explanation for geminates in WP allows for a more satisfactory account of final segments in TAP.

A second issue for the pAP reconstruction is the presence of many unexplained phonemes in a range of environments in different languages. Velars, post-velar and laryngeal consonants are a case in point. Most of the complexity in this domain is found in the languages of Pantar and the Pantar Straits, whose phoneme inventories generally include not only velar and glottal stops, but also uvular ones, as well as a velar or pharyngeal fricative next to the glottal fricative /h/. This contrasts with the situation as found in most of Alor and the TK languages, which tend to be rather simpler. Table 22 exemplifies the velar and post-velar plosives and fricatives in a language of Pantar (Teiwa), Alor (Kamang), and Timor (Bunaq).

{\centering
Table 22: Velar and post-velar phonemes in TAP languages
\par}

\begin{flushleft}
\tablehead{}
\begin{supertabular}{m{0.6087598in}m{0.07685984in}m{0.49765983in}m{0.49625984in}m{0.8177598in}m{0.51225984in}m{0.07815985in}m{0.6823598in}m{0.51225984in}m{0.08445985in}m{0.6240598in}}
\hline
 &
 &
\multicolumn{4}{m{2.56016in}}{\centering Teiwa} &
 &
\multicolumn{2}{m{1.2733599in}}{\centering Kamang} &
 &
\centering\arraybslash Bunaq\\\hhline{-~----~--~-}
 &
 &
\centering velar &
\centering uvular &
\centering pharyngeal &
\centering glottal &
 &
\centering velar &
\centering glottal &
 &
\centering\arraybslash velar\\
\centering plosive &
 &
\centering k g &
\centering q &
 &
\centering {\textglotstop} &
 &
\centering k g &
\centering ({\textglotstop}) &
 &
\centering\arraybslash k g\\
\centering fricative &
 &
 &
 &
\centering [127?] &
\centering h &
 &
 &
\centering (h) &
 &
\\\hline
\end{supertabular}
\end{flushleft}
The existing pAP reconstruction leaves a significant part of the complexity in the (post)velar domain in the Pantar languages unexplained; for instance, it does not account for /g/ in Blagar and the relation between the various (post-)velar phonemes such as /q/ and /x/ found in different dialects of Blagar.\footnote{ See Steinhauer (1995).} It also does not explain the origin of /{\textglotstop}/ in languages other than Blagar and Adang, and does not give reflexes for pAP medial *k in Teiwa and pAP final *k in Sawila, leaving the field in question blank in the table summarising the correspondences (Holton and Robinson this volume). Finally, note a variety of irregularities in the reconstructions involving velars in Appendix 1, especially in the Pantar languages. In short, the frequency of irregularities and unexplained occurrences of (post-)velar phonemes shows how limited our understanding of this domain in AP still is, and serves as a reminder that much more extensive reconstruction work needs to be 
undertaken.

A similar issue is presented by the phonemic velar nasal /{\ng}/ in many AP languages. This phoneme is not reconstructed for pAP, and is also absent in all of the TK languages. According to Holton and Robinson (this volume), pAP *n became /{\ng}/ in word-final position in all AP languages except Teiwa, where it was retained as /n/. This historical scenario does work well for some languages, for instance, Wersing, where [{\ng}] is synchronically a word-final allophone of /n/. However, in other languages, questions remain. For instance, Kamang has an unexplained contrast between /{\ng}/ and /n/ in codas (e.g., \textit{ee{\ng}} {\textquoteleft}\textsc{2sg.poss{\textquoteright}} versus \textit{een} {\textquoteleft}\textsc{2sg.foc{\textquoteright}}). Similarly, the existence of /{\ng}/ in coda and medial position in Teiwa is unexplained, as well as the occurrence of /{\ng}/ in other positions than the final one in various languages (e.g. Sar \textit{la{\ng}ja} {\textquoteleft}digging stick{\textquoteright} and 
Kula \textit{{\ng}apa} {\textquoteleft}father{\textquoteright}). 

Vowels also present a major challenge to the reconstruction of the ancestral TAP language. The various vowel systems as illustrated in Table 23 are yet to be historically reconciled with one another. Most AP languages have a length distinction in their vowels: the most common system is 5 short and 5 long cardinal vowels (Kaera, Blagar, Abui and Kamang), though matching long vowels may be missing in the mid-vowel range (Teiwa and Klon). Blagar has a marginal length distinction with only a small number of items occurring with long vowels (Steinhauer to appear), while it is Klon{\textquoteright}s short mid-vowels that are marginal. A length distinction is entirely absent from WP{\textquoteright}s and Wersing{\textquoteright}s five vowel system and Adang{\textquoteright}s seven vowel system. A relationship, if any, between the mid-vowels in Adang and length distinctions in other languages remains to be established. Non-cardinal vowels are found in Sawila /y, y{\textlengthmark}/ and in Klon /{\textschwa}/. TK 
languages all have simple five cardinal vowels and there is a marginal length distinction in only one language, Makalero. Stress in conjunction with length appears to have played an important role in vowel histories. For instance, Klon /{\textschwa}/ seems to originate in a short, unstressed pAP *a (e.g., Klon \textit{{\textschwa}bi} appears to go back to pAP *ha{\textprimstress}bi {\textquoteleft}fish{\textquoteright}). In Wersing, historically short unstressed vowels are lost in words with long vowels, which in turn become short stressed vowels (e.g., Wersing \textit{tlam} appears to go back to pAP *tala{\textlengthmark}m {\textquoteleft}six{\textquoteright}, cf. Abui \textit{tala{\textlengthmark}ma}). In short, much careful bottom-up reconstructive work needs to be done in order to reconcile these different systems to a single ancestral system.

{\centering
Table 23: TAP vowel systems\footnote{ The data in these tables are from Holton (to appear) for Western Pantar, Klamer (2010) for Teiwa, Klamer (to appear) for Kaera, Steinhauer (to appear) for Blagar, Haan (2001) for Adang, Baird (2008) for Klon, Kratochv\'il (2007) for Abui, Schapper (n.d. b.) for Kamang, Kratochv\'il (to appear) for Sawila, Schapper and Hendery (to appear) for Wersing, Schapper (2010) for Bunaq, and Huber (2011) for Makalero.}
\par}

\begin{center}
\tablehead{}
\begin{supertabular}{m{0.40185985in}m{0.39275986in}m{0.39135984in}m{0.08585984in}m{0.39835986in}m{0.39065984in}m{0.39275986in}m{0.085159846in}m{0.43375984in}m{0.39065984in}m{0.39345986in}m{0.08585984in}m{0.43235984in}m{0.39065984in}}
\hline
\multicolumn{3}{m{1.3434598in}}{\centering WP} &
 &
\multicolumn{3}{m{1.3392599in}}{\centering Teiwa} &
 &
\multicolumn{3}{m{1.3753599in}}{\centering Kaera} &
 &
\multicolumn{2}{m{0.9017598in}}{\centering Blagar}\\\hhline{---~---~---~--}
\centering i &
 &
\centering u &
 &
\centering i, i{\textlengthmark} &
 &
\centering u, u{\textlengthmark} &
 &
\centering i, i{\textlengthmark} &
 &
\centering u, u{\textlengthmark} &
 &
\centering i, i{\textlengthmark} &
\\
\centering e &
 &
\centering o &
 &
\centering e &
 &
\centering o &
 &
\centering e, e{\textlengthmark} &
 &
\centering o, o{\textlengthmark} &
 &
\centering e, e{\textlengthmark} &
\\
 &
 &
 &
 &
 &
 &
 &
 &
 &
 &
 &
 &
 &
\\
 &
\centering a &
 &
 &
 &
\centering a, a{\textlengthmark} &
 &
 &
 &
\centering a, a{\textlengthmark} &
 &
 &
 &
\centering\arraybslash a, a{\textlengthmark}\\\hline
\multicolumn{3}{m{1.3434598in}}{\centering Adang} &
 &
\multicolumn{3}{m{1.3392599in}}{\centering Klon} &
 &
\multicolumn{3}{m{1.3753599in}}{\centering Abui} &
 &
\multicolumn{2}{m{0.9017598in}}{\centering Kamang}\\\hhline{---~---~---~--}
\centering i &
 &
\centering u &
 &
\centering i, i{\textlengthmark} &
 &
\centering u, u{\textlengthmark} &
 &
\centering i, i{\textlengthmark} &
 &
\centering u, u{\textlengthmark} &
 &
\centering i, i{\textlengthmark} &
\\
\centering e &
 &
\centering o &
 &
\centering e &
 &
\centering o, o{\textlengthmark} &
 &
\centering e, e{\textlengthmark} &
 &
\centering o, o{\textlengthmark} &
 &
\centering e, e{\textlengthmark} &
\\
\centering {\textepsilon} &
 &
\centering {\textopeno} &
 &
\centering {\textepsilon}, {\textepsilon}{\textlengthmark} &
\centering {\textschwa} &
\centering {\textopeno} &
 &
 &
 &
 &
 &
 &
\\
 &
\centering a &
 &
 &
 &
\centering a, a{\textlengthmark} &
 &
 &
 &
\centering a, a{\textlengthmark} &
 &
 &
 &
\centering\arraybslash a, a{\textlengthmark}\\\hline
\multicolumn{3}{m{1.3434598in}}{\centering Sawila} &
 &
\multicolumn{3}{m{1.3392599in}}{\centering Wersing} &
 &
\multicolumn{3}{m{1.3753599in}}{\centering Bunaq} &
 &
\multicolumn{2}{m{0.9017598in}}{\centering Makalero}\\\hhline{---~---~---~--}
\multicolumn{2}{m{0.87335986in}}{ i, i{\textlengthmark} y, y{\textlengthmark}} &
\centering u, u{\textlengthmark} &
 &
\centering i &
 &
\centering u &
 &
\centering i &
 &
\centering u &
 &
\centering i &
\\
\centering e, e{\textlengthmark} &
 &
\centering o, o{\textlengthmark} &
 &
\centering e &
 &
\centering o &
 &
\centering e &
 &
\centering o &
 &
\centering e &
\\
 &
 &
 &
 &
 &
 &
 &
 &
 &
 &
 &
 &
 &
\\
 &
\centering a, a{\textlengthmark} &
 &
 &
 &
\centering a &
 &
 &
 &
\centering a &
 &
 &
 &
\centering\arraybslash a\\\hline
\end{supertabular}
\end{center}
In sum, with the positive establishment of the relatedness of the Papuan languages scattered across the islands of Timor, Kisar, Alor, Pantar and the Pantar Straits, a start has been made towards a history of the TAP languages. However, we are still a long way off a complete and nuanced understanding of the family and its development (cf. Schapper and Huber{\textquoteright}s (forthcoming) statement of prospective research questions). It will be the task of future reconstructive historical work to definitively solve remaining issues in the comparative data. 

\section[References]{References}
\hypertarget{RefHeading65425871885726}{}Baird, Louise. 2008. \textit{A grammar of Klon: A non-Austronesian language of Alor, Indonesia}. Canberra: Pacific Linguistics.

Brotherson, Anna. 2003. A spatial odyssey: Referring to space in Makasai. BA thesis, Australian National University, Canberra.

Capell, Arthur. 1975. The West Papuan Phylum: General, and Timor and areas further west. In Stephen A. Wurm, ed., \textit{New Guinea area languages and language study}, 667-716. Canberra: Pacific Linguistics.

Carr, Felicita M. 2004. Pupuuk, Masu Moru and Data: Three genres of Makasai. MA thesis, The Australian National University, Canberra.

Haan, Johnson. 2001. The grammar of Adang, a Papuan language spoken on the island of Alor, East Nusa Tenggara, Indonesia. PhD thesis, University of Sydney.

Holton, Gary. To appear. Western Pantar. In Antoinette Schapper, ed., \textit{Papuan languages of Timor-Alor-Pantar: Sketch grammars Volume I}. Berlin/New York: Mouton de Gruyter.

Holton, Gary, Marian Klamer, Franti\v{s}ek Kratochv\'il, Laura C. Robinson, and Antoinette Schapper. 2012. The historical relations of the Papuan languages of Alor and Pantar. \textit{Oceanic Linguistics} 51 (1): 86-122.

Holton, Gary, and Mahalalel Lamma Koly. 2008. \textit{Kamus pengantar bahasa Pantar Barat}. Kupang, Indonesia: Unit Bahasa dan Budaya (UBB).

Holton, Gary, and Laura C. Robinson. This volume. The internal history of the Alor-Pantar language family. In Marian Klamer, ed., \textit{Alor-Pantar languages: History and typology}. Leiden: Brill.

Huber, Juliette. 2008. \textit{First steps towards a grammar of Makasae}. M\"unchen: Lincom.

Huber, Juliette. 2011. \textit{A grammar of Makalero: A Papuan language of East Timor}\textit{.} Utrecht: LOT.

Josselin de Jong, J.P.B. de. 1937. \textit{Studies in Indonesian culture I. Oirata, a Timorese settlement on Kisar}. Amsterdam: N.V. Noord-Hollandsche Uitgevers-Maatschappij.

Klamer, Marian. 2010. \textit{A grammar of Teiwa}. Berlin/New York: Mouton de Gruyter.

Klamer, Marian. To appear. Kaera. In Antoinette Schapper, ed., \textit{Papuan languages of Timor-Alor-Pantar: Sketch grammars Volume I}. Berlin/New York: Mouton de Gruyter. 

Klamer, Marian, and Amos Sir. 2011. \textit{Kosakata bahasa Teiwa-Indonesia-Inggris} [Teiwa-Indonesian-English word list]. Kupang, Indonesia: Unit Bahasa dan Budaya (UBB).

Klamer, Marian, and Antoinette Schapper. 2012. The history of {\textquoteleft}give{\textquoteright} constructions in the Papuan languages of Timor-Alor-Pantar. \textit{Linguistic Discovery} 10 (3): 174-207. 

Kratochv\'il, Franti\v{s}ek. 2007. \textit{A grammar of Abui: A Papuan language of Alor}. Utrecht: LOT.

Kratochv\'il, Franti\v{s}ek. To appear. Sawila. In Antoinette Schapper, ed., \textit{Papuan languages of Timor-Alor-Pantar: Sketch grammars Volume II}. Berlin/New York: Mouton de Gruyter.

Kratochv\'il, Franti\v{s}ek. n.d. Dictionary of Sawila. [Manuscript, Nanyang Technological University.]

Kratochv\'il, Franti\v{s}ek, and Benediktus Delpada. 2008. \textit{Abui stories from Takalelang}. Kupang, Indonesia: Unit Bahasa dan Budaya (UBB).

Rodemeier, Susanne. 1992. \textit{Lego-lego Platz und naga-Darstellung. Jenseitige Kr\"afte im Zentrum einer Quellenstudie \"uber die ostindonesische Insel Alor}. MA thesis, Ludwig-Maximilians-Universit\"at M\"unchen.

Schapper, Antoinette. This volume. Elevational systems in AP languages. In Marian Klamer, ed., \textit{Alor-Pantar languages: History and typology}. Leiden: Brill

Schapper, Antoinette. 2010. Bunaq, a Papuan language of central Timor. PhD thesis, The Australian National University, Canberra.

Schapper, Antoinette. n.d. a. Dictionary of Bunaq. [Manuscript, Australian National University.]

Schapper, Antoinette. n.d. b. Grammar of Kamang. [Manuscript, Leiden University.]

Schapper, Antoinette, and Rachel Hendery. To appear. Wersing. In Antoinette Schapper, ed., \textit{Papuan languages of Timor-Alor-Pantar: Sketch grammars Volume I}. Berlin/New York: Mouton de Gruyter.

Schapper, Antoinette, and Juliette Huber. Forthcoming. State-of-the-art in the documentation of the Papuan languages of Timor, Alor, Pantar and Kisar. \textit{Wacana, Journal of the Humanities of Indonesia}. 

Schapper, Antoinette, Juliette Huber, and Aone van Engelenhoven. 2012. The historical relations of the Papuan languages of Timor and Kisar. \textit{Language and Linguistics in Melanesia: Special Issue on the History, Contact \& Classification of Papuan languages}: 194-242.

Schapper, Antoinette, and Marten Manimau 2011. \textit{Dictionary of Kamang: Kamang -- Indonesian -- English}. Kupang, Indonesia: Kupang, Indonesia: Unit Bahasa dan Budaya (UBB) 

Steinhauer, Hein. 1995. Two varieties of the Blagar language (Alor, Indonesia). In Conny Baak, Mary Bakker, and Dick van der Meij, eds., \textit{Tales from a concave world. Liber amicorum Bert Voorhoeve}: 265-296. Leiden: Department of languages and cultures of South-East Asia and Oceania, Leiden University.

Steinhauer, Hein. To appear. Blagar. In Antoinette Schapper, ed., \textit{Papuan languages of Timor-Alor-Pantar: Sketch grammars Volume I}. Asia-Pacific Linguistics, Mouton de Gruyter.

Stokhof, W.A.L. 1975. \textit{Preliminary notes on the Alor and Pantar languages (East Indonesia)}. Canberra: Pacific Linguistics Series B-43.

\clearpage\section[Notes to the Appendices]{Notes to the Appendices}
\hypertarget{RefHeading65427871885726}{}Appendix I presents the AP data supporting the additional and revised pAP reconstructions; Appendix II gives the TK data supporting the additional and revised pTIM reconstructions; and finally, Appendix III gives a list of the 89 pAP and pTIM cognates, together with reconstructed pTAP forms, on which the sound correspondences identified in this chapter are based. 

The AP data presented in Appendix I derive from the comparative lexical database compiled by the authors of Holton et al. (2012), as well as from Schapper{\textquoteright}s field notes on Kamang and Wersing. Other than Holton et al. (2012), which bases its reconstruction on a subset of 12 AP languages, we give a full list of the AP languages found in the AP lexical database. This allows us to draw on cognate forms found only in some of the smaller languages. Language names are abbreviated according to the conventions listed in the Introduction to this volume; the table with Sources below also explains the abbreviations.

The TK data presented in Appendix II rely on Schapper{\textquoteright}s and Huber{\textquoteright}s field notes on Bunaq and Makalero, respectively. The Makasae data are drawn from Huber{\textquoteright}s field notes as well as a number of online resources. The sources of both AP and TK language data are detailed below.

\textbf{Sources}

\begin{flushleft}
\tablehead{}
\begin{supertabular}{m{1.3525599in}m{1.7087599in}m{1.0622599in}m{1.7108599in}}
Abui (\textsc{Ab}) &
Kratochv\'il 2007, Kratochv\'il and Delpada 2008, Schapper fieldnotes 2010 &
Kui (\textsc{Ki}) &
Holton fieldnotes 2010\\
Adang (\textsc{Ad}) &
Robinson fieldnotes 2010 &
Kula (\textsc{Ku}) &
Holton fieldnotes 2010, Nicholas Williams p.c. 2011\\
Blagar (\textsc{Bl}) &
Robinson fieldnotes 2010 &
Makalero &
Huber 2011, Huber fieldnotes 2007-2013\\
Bunaq (Lamaknen) &
Schapper n.d. a, Schapper 2010 &
Makasae &
Brotherson 2003, Carr 2004, Huber 2008, Huber fieldnotes 2005, 2012-2013, Language Documentation Training Center of the University of Hawaii\footnotemark{}\\
Deing (\textsc{De}) &
Robinson fieldnotes 2010 &
Nedebang (N\textsc{d}) &
Robinson fieldnotes 2010\\
Fataluku &
Fataluku online dictionary\footnotemark{}, van Engelenhoven fieldnotes &
Oirata &
de Josselin de Jong 1937, van Engelenhoven fieldnotes\\
Hamap (H\textsc{m}) &
Robinson fieldnotes 2010 &
Reta (\textsc{Rt}) &
 Robinson fieldnotes 2010\\
Kamang (\textsc{Km}) &
Schapper n.d. b, Schapper and Manimau 2011 &
Sar (S\textsc{r}) &
Robinson fieldnotes 2010\\
Kabola (K\textsc{b}) &
Robinson fieldnotes 2010 &
Sawila (S\textsc{w}) &
Kratochv\'il n.d.\\
Kaera (K\textsc{e}) &
Klamer Kaera corpus 2005-2007 &
Teiwa (\textsc{Tw}) &
Klamer Teiwa corpus, Klamer and Sir 2011, Robinson fieldnotes 2010\\
Kafoa (\textsc{Kf}) &
Baird fieldnotes 2003 &
Wersing (\textsc{We}) &
Schapper and Hendery fieldnotes 2012, Holton fieldnotes 2010\\
Klon (\textsc{Kl}) &
Baird fieldnotes 2003 &
Western Pantar (WP) &
Holton and Lamma Koly 2008, Holton fieldnotes 2010\\
\end{supertabular}
\end{flushleft}
\addtocounter{footnote}{-2}
\stepcounter{footnote}\footnotetext{ Online at http://www.ling.hawaii.edu/ldtc/languages/makasae\_fatum/ and http://www.ling.hawaii.edu/ldtc/languages/makasae\_osor/.}
\stepcounter{footnote}\footnotetext{ Online at www.fataluku.com.}
The orthographic conventions used in the Appendices are the following: {\textquoteleft}\emph{\textup{\~{}{\textquoteright} joins morphological variants of the same lexeme. In Appendix I and Appendix II, material given in }}\emph{\textup{round brackets {\textquoteleft}( ){\textquoteright} represents fossilized morphology or other unetymological material. In Appendix III, round brackets indicate that a given phoneme cannot be reconstructed with absolute certainty. Furthermore, {\textquoteleft}N{\textquoteright} is used to represent an unspecified nasal; {\textquoteleft}L{\textquoteright} an unspecified liquid, and {\textquoteleft}Q{\textquoteright} a putative postvelar stop for which we have only very weak evidence. An empty slot in the pTAP column means that the reconstructed pAP and pTIM forms, although clearly cognate, are too different to allow for a secure pTAP reconstruction. }}

\clearpage\setcounter{page}{1}\pagestyle{Convertedvi}
\section[Appendix I: data supporting the additional pAP reconstructions]{Appendix I: data supporting the additional pAP reconstructions}
\hypertarget{RefHeading65429871885726}{}\begin{flushleft}
\tablehead{}
\begin{supertabular}{m{0.7170598in}m{1.4011599in}m{0.81295985in}m{0.81365985in}m{0.81295985in}m{0.81365985in}m{0.81365985in}m{0.81295985in}m{0.81365985in}m{0.81295985in}m{0.8205598in}}
\hline
\textbf{gloss} &
bark &
bird &
bite &
bone &
clew, stone circle \textsuperscript{3} &
coconut &
crawl &
die &
dirty &
dog\\\hline
\textbf{pAP original} &
{}-{}- &
*dVl &
{}-{}- &
{}-{}- &
{}-{}- &
*wat(a) &
{}-{}- &
*minV &
{}-{}- &
{}-{}-\\\hline
\textbf{pAP new} &
*lVu &
*(a)dVl &
*(ta)ki &
*ser &
*maita  &
*wata &
*er &
*min(a) &
*karok \textsuperscript{5} &
*jibar \textsuperscript{6}\\\hline
\textbf{\textsc{Sr}} &
{}-{}- &
dal &
{}-{}- &
{}-{}- &
{}-{}- &
wat &
{}-{}- &
min &
{}-{}- &
jifar\\
\textbf{\textsc{De}} &
{}-{}- &
dal &
{}-{}- &
{}-{}- &
{}-{}- &
wat &
{}-{}- &
mi{\ng} &
{}-{}- &
jewar\\
\textbf{\textsc{Tw}} &
{}-{}- &
dai &
{}-{}- &
{}-{}- &
{}-{}- &
wat &
{}-{}- &
min &
{}-{}- &
jifar\\
\textbf{\textsc{Nd}} &
{}-{}- &
daya &
{}-{}- &
{}-{}- &
{}-{}- &
wata &
{}-{}- &
min{\textlengthmark}a &
{}-{}- &
bar\\
\textbf{\textsc{Ke}} &
{}-{}- &
{}-{}- &
{}-{}- &
{}-{}- &
{}-{}- &
wat &
{}-{}- &
min &
{}-{}- &
ibar\\
\textbf{\textsc{WP}} &
lau &
{}-{}- &
{}-{}- &
{}-{}- &
{}-{}- &
hatua &
{}-{}- &
{}-{}- &
{}-{}- &
jab{\textlengthmark}e\\
\textbf{\textsc{Bl}} &
olovi &
{}-{}- &
(ga)ki &
{}-{}- &
{}-{}- &
vet &
{}-{}- &
(i)mina &
{}-{}- &
jabar\\
\textbf{\textsc{Rt}} &
lu &
{}-{}- &
ki(-ki) &
{}-{}- &
{}-{}- &
vat &
{}-{}- &
(a)mina &
{}-{}- &
jobal\\
\textbf{\textsc{Ad}} &
lowo{\textglotstop} &
{}-{}- &
{}-{}- &
{}-{}- &
{}-{}- &
fa{\textglotstop} &
{}-{}- &
mini{\textglotstop} &
karo{\textglotstop}o  &
bel\\
\textbf{\textsc{Hm}} &
{}-{}- &
{}-{}- &
{}-{}- &
{}-{}- &
{}-{}- &
{}-{}- &
{}-{}- &
min &
{}-{}- &
b{\o}l\\
\textbf{\textsc{Kb}} &
olowo &
{}-{}- &
{}-{}- &
{}-{}- &
{}-{}- &
wa{\textglotstop} &
{}-{}- &
mini &
(na)karo{\textglotstop}o &
bel\\
\textbf{\textsc{Ki}} &
{}-{}- &
adol &
{}-{}- &
{}-{}- &
{}-{}- &
bat &
{}-{}- &
min &
{}-{}- &
{}-{}-\\
\textbf{\textsc{Kf}} &
{}-{}- &
{}-{}- &
{}-{}- &
{}-{}- &
{}-{}- &
{}-{}- &
{}-{}- &
(i)mon &
{}-{}- &
{}-{}-\\
\textbf{\textsc{Kl}} &
{}-{}- &
{}-{}- &
{}-{}- &
{}-{}- &
{}-{}- &
{}-{}- &
{}-{}- &
{}-{}- &
{}-{}- &
{}-{}-\\
\textbf{\textsc{Ab}} &
lou &
{}-{}- &
(ta)kai &
{}-{}- &
masa{\ng} ?` \textsuperscript{4} &
wata &
{}-{}- &
mo{\ng} &
{}-{}- &
{}-{}-\\
\textbf{\textsc{Km}} &
{}-{}- &
atul &
ka(te) \textsuperscript{1} &
s{\textepsilon}l ?` \textsuperscript{2} &
maita &
{}-{}- &
\textstyleHeadingiiChar{\textmd{eei \~{} eel}} &
{}-{}- &
{}-{}- &
{}-{}-\\
\textbf{\textsc{Ku}} &
leloja &
{}-{}- &
{}-{}- &
(gi)saja &
{}-{}- &
g\textsuperscript{w}ata &
{}-{}- &
{}-{}- &
{}-{}- &
{}-{}-\\
\textbf{\textsc{Sw}} &
{}-{}- &
adala &
{}-{}- &
sara &
{}-{}- &
wata &
{}-{}- &
{}-{}- &
{}-{}- &
{}-{}-\\
\textbf{\textsc{We}} &
aloi &
adol &
(mi)kik &
(ge)seri &
{}-{}- &
wata &
er &
{}-{}- &
{}-{}- &
{}-{}-\\\hline
\end{supertabular}
\end{flushleft}
\textsuperscript{1} Metathesised form; denotes {\textquoteleft}eat{\textquoteright}.  \textsuperscript{2} Kamang normally\textsuperscript{ }reflects pAP *ras i in final position.  \textsuperscript{3} See Rodemeier (1992) on clews in Alor.  \textsuperscript{4 }Abui normally reflects pAP *t as t.  \textsuperscript{5 }This reconstruction must be viewed as tentative, since Kabola does not make part of the existing pAP reconstruction.  \textsuperscript{6} Note the loss of the initial syllable in several of the daughter languages. According to Holton et al. (2012) and Holton and Robinson (this volume), this has to do with stress being based on syllable weight. The heavy *bar syllable attracts stress, which leads to the loss of the initial syllable. A similar case is, possibly, pAP *tei {\textquoteleft}tree{\textquoteright}.

\begin{flushleft}
\tablehead{}
\begin{supertabular}{m{0.80385983in}m{0.7969598in}m{0.7976598in}m{0.7948598in}m{0.80315983in}m{0.8018598in}m{0.7990598in}m{0.7962598in}m{0.9455598in}m{0.8108598in}m{0.80315983in}}
\hline
\textbf{gloss} &
ear &
face &
far &
fire &
fish &
flat &
girl &
grandparent, grandchild &
green &
hear\\\hline
\textbf{pAP original} &
*-uar(i) &
{}-{}- &
{}-{}- &
*had(a) &
*hab(i) &
{}-{}- &
{}-{}- &
{}-{}- &
{}-{}- &
{}-{}-\\\hline
\textbf{pAP new} &
*-uari &
*-pona &
*lete &
*hada &
*habi &
*tatok &
*pon &
*tam(a, u) \textsuperscript{1} &
*(wa)logar \textsuperscript{5} &
*magi\\\hline
\textbf{\textsc{Sr}} &
{}-{}- &
{}-{}- &
{}-{}- &
{}-{}- &
{}-{}- &
{}-{}- &
{}-{}- &
{}-{}- &
logar &
{}-{}-\\
\textbf{\textsc{De}} &
{}-war &
{}-{}- &
{}-{}- &
{}-{}- &
{}-{}- &
{}-{}- &
{}-{}- &
{}-{}- &
alogur &
{}-{}-\\
\textbf{\textsc{Tw}} &
{}-uar &
{}-{}- &
{}-{}- &
[127?]ar &
[127?]af &
{}-{}- &
{}-{}- &
{}-{}- &
ajogar ?` &
{}-{}-\\
\textbf{\textsc{Nd}} &
{}-ow &
{}-{}- &
{}-{}- &
ar &
a{\textlengthmark}fi &
{}-{}- &
{}-{}- &
{}-{}- &
aejaga ?` &
{}-{}-\\
\textbf{\textsc{Ke}} &
{}-uar &
{}-{}- &
{}-{}- &
ad &
ab &
{}-{}- &
{}-{}- &
{}-{}- &
ojogi ?` &
{}-{}-\\
\textbf{\textsc{WP}} &
{}-ue &
{}-{}- &
{}-{}- &
{}-{}- &
hap &
{}-{}- &
{}-{}- &
{}-{}- &
haluaga &
{}-{}-\\
\textbf{\textsc{Bl}} &
{}-veli &
{}-{}- &
{}-{}- &
a{\textlengthmark}d &
a{\textlengthmark}b &
{}-{}- &
{}-{}- &
{}-{}- &
{}-{}- &
m{\textepsilon}{\textglotstop}{\textepsilon}\\
\textbf{\textsc{Rt}} &
{}-{}- &
{}-{}- &
{}-{}- &
{}-{}- &
{}-{}- &
{}-{}- &
{}-{}- &
{}-{}- &
{}-{}- &
{}-{}-\\
\textbf{\textsc{Ad}} &
{}-{}- &
{}-{}- &
{}-{}- &
{}-{}- &
a{\textlengthmark}b &
{}-{}- &
{}-{}- &
{}-{}- &
{}-{}- &
ma{\textglotstop}eh\\
\textbf{\textsc{Hm}} &
{}-{}- &
{}-{}- &
{}-{}- &
{}-{}- &
{}-{}- &
{}-{}- &
{}-{}- &
{}-{}- &
{}-{}- &
{}-{}-\\
\textbf{\textsc{Kb}} &
{}-{}- &
{}-{}- &
{}-{}- &
{}-{}- &
{}-{}- &
{}-{}- &
{}-{}- &
{}-{}- &
{}-{}- &
me{\textglotstop}ehe\\
\textbf{\textsc{Ki}} &
{}-uel &
{}-{}- &
{}-{}- &
ar &
eb &
{}-{}- &
{}-{}- &
{}-{}- &
{}-{}- &
magi\\
\textbf{\textsc{Kf}} &
{}-{}- &
{}-{}- &
{}-{}- &
{}-{}- &
{}-{}- &
{}-{}- &
{}-{}- &
{}-{}- &
{}-{}- &
{}-{}-\\
\textbf{\textsc{Kl}} &
{}-u{\textepsilon}r &
{}-{}- &
{}-{}- &
{\textschwa}d{\textscripta} &
{\textschwa}bi &
{}-{}- &
{}-{}- &
{}-{}- &
w{\textschwa}w{\textepsilon}l{\textepsilon}{\ng} ?` &
m{\textschwa}gih\\
\textbf{\textsc{Ab}} &
{}-uei &
{}-po{\ng} &
{}-{}- &
ara &
afu &
{}-{}- &
{}-{}- &
{}-{}- &
wala{\ng}aj &
mahi\\
\textbf{\textsc{Km}} &
{}-uai &
{}-funa{\textlengthmark} &
letei &
ati &
api &
tatok &
fon &
dum \textsuperscript{2} &
{}-{}- &
{}-mai\\
\textbf{\textsc{Ku}} &
{}-{}- &
{}-{}- &
{}-{}- &
{}-{}- &
{}-{}- &
{}-{}- &
{}-{}- &
atamu \textsuperscript{3} &
wala{\ng}ka &
magin\\
\textbf{\textsc{Sw}} &
uari &
{}-{}- &
{}-{}- &
ada &
api &
{}-{}- &
{}-{}- &
(ga)ta{\textlengthmark}mu \textsuperscript{3} &
wala{\ng}ara ?` &
maji{\textlengthmark}{\ng}\\
\textbf{\textsc{We}} &
{}-ueri &
{}-{}- &
{}-{}- &
ada &
api &
{}-{}- &
{}-{}- &
(ne)tamu \textsuperscript{4} &
walar &
{}-{}-\\\hline
\end{supertabular}
\end{flushleft}
\textsuperscript{1 }This is a reciprocal term. The reflexes in the modern languages denote either {\textquoteleft}grandparent{\textquoteright} or {\textquoteleft}grandchild{\textquoteright}.  \textsuperscript{2} Semantic shift to {\textquoteleft}child{\textquoteright}.  \textsuperscript{3} Denotes {\textquoteleft}grandchild{\textquoteright}.  \textsuperscript{4} Denotes {\textquoteleft}grandparent{\textquoteright}.  \textsuperscript{5 }While clearly cognate, the forms in this set show a variety of unexpected or irregular sound changes: Teiwa, Nedebang and Kaera normally reflect pAP *l as l in initial and medial position, rather than j; Teiwa and Nedebang normally reflect pAP *g as [127?] and x, respectively, in medial position, rather than g; pAP *g is normally reflected as g in Klon and j in Sawila; and finally, initial h in Western Pantar is usually a reflex of pAP *h, rather than *w. The pAP reconstruction must thus be seen as somewhat tentative.

\begin{flushleft}
\tablehead{}
\begin{supertabular}{m{0.81365985in}m{0.81365985in}m{0.81435984in}m{0.81365985in}m{0.8122598in}m{0.81435984in}m{0.81365985in}m{0.81295985in}m{0.81435984in}m{0.81365985in}m{0.8170598in}}
\hline
\textbf{gloss} &
itchy &
laugh &
leg &
LOW &
meat &
mountain &
name &
new &
new place &
other\\\hline
\textbf{pAP original} &
{}-{}- &
*jari &
{}-{}- &
{}-{}- &
{}-{}- &
{}-{}- &
*-ain(i, u) &
*siba &
{}-{}- &
{}-{}-\\\hline
\textbf{pAP new} &
*(i)ruk &
*jagir \textsuperscript{1} &
*-bat &
*po \textsuperscript{2} &
*iser \textsuperscript{3} &
*buku &
*-en(i, u) &
*siba(r) &
*lan &
*abenVC\\\hline
\textbf{\textsc{Sr}} &
{}-{}- &
jehar &
{}-fat &
{}-{}- &
{}-{}- &
{}-{}- &
{}-{}- &
{}-{}- &
{}-{}- &
{}-{}-\\
\textbf{\textsc{De}} &
{}-{}- &
jaxar &
{}-wat &
{}-{}- &
{}-{}- &
{}-{}- &
{}-{}- &
sib &
{}-{}- &
{}-{}-\\
\textbf{\textsc{Tw}} &
{}-{}- &
j{\textschwa}[127?]ar &
{}-fat &
{}-{}- &
{}-{}- &
{}-{}- &
{}-{}- &
sib &
{}-{}- &
{}-{}-\\
\textbf{\textsc{Nd}} &
{}-{}- &
gela &
{}-{}- &
{}-{}- &
{}-{}- &
{}-{}- &
{}-einu &
sava({\textglotstop}a) &
{}-{}- &
{}-{}-\\
\textbf{\textsc{Ke}} &
{}-{}- &
agar &
at &
{}-{}- &
{}-{}- &
buku{\textlengthmark} &
{}-en &
sib- &
{}-{}- &
bani{\ng}\\
\textbf{\textsc{WP}} &
{}-{}- &
jali ?` &
{}-{}- &
{}-{}- &
{}-{}- &
{}-{}- &
{}-in{\textlengthmark}u &
sab{\textlengthmark}a &
{}-{}- &
{}-{}-\\
\textbf{\textsc{Bl}} &
{}-{}- &
iriga &
{}-{}- &
po &
{}-{}- &
buku &
{}-ene &
hiba &
{}-{}- &
\emph{\textup{abeu}}{\ng}\emph{\textup{ \~{} ebeu}}{\ng}\\
\textbf{\textsc{Rt}} &
{}-{}- &
agala &
{}-{}- &
{}-{}- &
{}-{}- &
{}-{}- &
{}-{}- &
haba &
{}-{}- &
{}-{}-\\
\textbf{\textsc{Ad}} &
{}-{}- &
{}-{}- &
{}-{}- &
p{\textopeno} &
hiri ?` &
{}-{}- &
{}-ani{\ng} &
habar &
{}-{}- &
{}-{}-\\
\textbf{\textsc{Hm}} &
{}-{}- &
{}-{}- &
{}-{}- &
{}-{}- &
(ma)hil &
{}-{}- &
an{\textepsilon} &
habar &
{}-{}- &
{}-{}-\\
\textbf{\textsc{Kb}} &
{}-{}- &
ja{\textlengthmark}la &
{}-{}- &
{}-{}- &
{}-{}- &
{}-{}- &
{}-{}- &
{}-{}- &
{}-{}- &
{}-{}-\\
\textbf{\textsc{Ki}} &
rok &
jeri ?` &
{}-{}- &
{}-{}- &
Is &
{}-{}- &
{}-enei &
saba &
{}-{}- &
aba{\ng}an\\
\textbf{\textsc{Kf}} &
{}-{}- &
{}-{}- &
{}-{}- &
{}-{}- &
(ma)he:l &
{}-{}- &
{}-n{\textepsilon}i &
hifa &
{}-{}- &
afenaj\\
\textbf{\textsc{Kl}} &
{}-{}- &
{\textschwa}gar &
{}-{}- &
{}-{}- &
(m{\textschwa})h{\textepsilon}l &
{}-{}- &
{}-{\textschwa}n{\textepsilon}{\textglotstop} &
h{\textschwa}b{\textscripta}{\textlengthmark} &
{}-{}- &
ebe{\ng}\\
\textbf{\textsc{Ab}} &
joku{\ng} &
{}-{}- &
{}-{}- &
pa &
mahiti{\ng} &
buku ?` \textsuperscript{4} &
{}-ane &
t{\i}f{\textscripta} &
{}-{}- &
{}-{}-\\
\textbf{\textsc{Km}} &
joku{\ng} &
{}-{}- &
{}-{}- &
fe &
isei &
buk ?` \textsuperscript{4} &
{}-nei &
supa(ka) &
la{\ng} &
{}-{}-\\
\textbf{\textsc{Ku}} &
joka &
geja &
{}-{}- &
{}-{}- &
{}-{}- &
{}-{}- &
{}-{}- &
tupa &
{}-{}- &
{}-{}-\\
\textbf{\textsc{Sw}} &
{}-{}- &
jara ?` &
{}-{}- &
{}-{}- &
isi ?` &
{}-{}- &
{}-ani &
tipea &
la{\textlengthmark}{\ng} \textsuperscript{5} &
{}-{}-\\
\textbf{\textsc{We}} &
iruk &
jer ?` &
{}-{}- &
{}-{}- &
(ge)is ?` &
{}-{}- &
{}-{}- &
t{\textschwa}pa &
la{\ng} \textsuperscript{5} &
{}-{}-\\\hline
\end{supertabular}
\end{flushleft}
\textsuperscript{1} Holton et al. (2012) reconstruct *jari for {\textquoteleft}laugh{\textquoteright}. We revise this form on the basis of the clear presence of a medial velar in the reflexes of many AP languages. Note, however, the irregular loss of reflexes of pAP *g in Western Pantar, Kui, Sawila and Wersing.  \textsuperscript{2} See Schapper (this volume) for details on this reconstruction.  \textsuperscript{3} The reflexes of this form denote {\textquoteleft}game{\textquoteright} or {\textquoteleft}meat{\textquoteright}. Note that there are several irregularities in this set: Adang normally reflects pAP *r as l, rather than r; and Sawila and Wersing normally reflect *s as t, rather than s.  \textsuperscript{4} Abui normally reflects pAP *b as f, rather than b, and pAP *b is usually reflected in Kamang as p, rather than b.  \textsuperscript{5 }Denotes {\textquoteleft}coast{\textquoteright}. The relationship between the two senses is explained by the typical settlement patterns in the region: older 
settlements are located in high places, often on top of knolls or ridges, whilst newer settlements are downhill towards the coast.

\begin{flushleft}
\tablehead{}
\begin{supertabular}{m{0.81365985in}m{0.81365985in}m{0.81435984in}m{0.81365985in}m{0.8122598in}m{0.81435984in}m{0.81365985in}m{0.81295985in}m{0.81435984in}m{0.81365985in}m{0.8170598in}}
\hline
\textbf{gloss} &
path &
person &
price &
\textit{P. }

\textit{indicus }\textsuperscript{4} &
rain &
ripe &
scratch &
shark &
spit &
spoon\\\hline
\textbf{pAP original} &
{}-{}- &
{}-{}- &
{}-{}- &
{}-{}- &
{}-{}- &
*ten &
{}-{}- &
{}-{}- &
*purVN &
{}-{}-\\\hline
\textbf{pAP new} &
*jega \textsuperscript{1} &
*anin &
*bol \textsuperscript{2} &
*matar &
*anur &
*tena &
*karab &
*sib(a, i)r &
*purVn &
*surV \textsuperscript{9}\\\hline
\textbf{\textsc{Sr}} &
{}-{}- &
{}-{}- &
{}-{}- &
{}-{}- &
{}-{}- &
{}-{}- &
k{\textschwa}ra{\textlengthmark}b &
sifir &
{}-{}- &
{}-{}-\\
\textbf{\textsc{De}} &
{}-{}- &
{}-{}- &
{}-{}- &
{}-{}- &
{}-{}- &
ten{\textlengthmark}a{\ng} &
krab &
sib{\textlengthmark}ir &
{}-{}- &
{}-{}-\\
\textbf{\textsc{Tw}} &
{}-{}- &
{}-{}- &
{}-{}- &
{}-{}- &
{}-{}- &
{}-{}- &
{}-{}- &
sifar &
puran &
{}-{}-\\
\textbf{\textsc{Nd}} &
ji{\textlengthmark}ja ?` &
{}-{}- &
{}-{}- &
{}-{}- &
{}-{}- &
tanan &
(ki)kar ?` \textsuperscript{5} &
{}-{}- &
{}-{}- &
{}-{}-\\
\textbf{\textsc{Ke}} &
{}-{}- &
{}-{}- &
{}-{}- &
{}-{}- &
{}-{}- &
ten- &
krabis ?` \textsuperscript{6} &
sibar &
pura{\ng} &
{}-{}-\\
\textbf{\textsc{WP}} &
ja ?` &
{}-{}- &
{}-{}- &
mat{\textlengthmark}e &
{}-{}- &
ta{\ng} &
karasi ?` \textsuperscript{7} &
sib{\textlengthmark}u &
{}-{}- &
{}-{}-\\
\textbf{\textsc{Bl}} &
iga ?` &
{}-{}- &
{}-{}- &
{}-{}- &
onor &
tena &
{}-{}- &
sibir \textsuperscript{8} &
puru{\ng} &
{}-{}-\\
\textbf{\textsc{Rt}} &
viag &
{}-{}- &
(ta){\texthtb}eli \textsuperscript{3} &
{}-{}- &
{}-{}- &
{}-{}- &
{}-{}- &
hibil &
puru{\ng} &
{}-{}-\\
\textbf{\textsc{Ad}} &
se{\textglotstop} &
{}-{}- &
{}-{}- &
{}-{}- &
nui &
tene &
{}-{}- &
{}-{}- &
{}-{}- &
hur \\
\textbf{\textsc{Hm}} &
se{\textglotstop} &
{}-{}- &
{}-{}- &
{}-{}- &
{}-{}- &
t{\textepsilon}n &
{}-{}- &
{}-{}- &
{}-{}- &
{}-{}-\\
\textbf{\textsc{Kb}} &
je{\textglotstop} &
{}-{}- &
({\textglotstop}o)wol \textsuperscript{3} &
{}-{}- &
nui &
tena{\ng} &
{}-{}- &
{}-{}- &
para{\ng} &
{}-{}-\\
\textbf{\textsc{Ki}} &
{}-{}- &
anin(ou) &
{}-{}- &
{}-{}- &
anor &
tain &
ukuberi &
sobor &
puri{\ng} &
{}-{}-\\
\textbf{\textsc{Kf}} &
{\textglotstop}ij{\textepsilon} &
{}-{}- &
{}-{}- &
{}-{}- &
{}-{}- &
{}-{}- &
ukafi &
{}-{}- &
{}-{}- &
{}-{}-\\
\textbf{\textsc{Kl}} &
{\textepsilon}g{\textepsilon}{\textglotstop} &
{\textscripta}n{\textsci}n(ok) &
{}-{}- &
mtar &
{}-{}- &
{\textschwa}t{\textepsilon}n &
k{\textschwa}r{\textopeno}b &
{}-{}- &
p{\textschwa}r{\textupsilon}in &
{}-{}-\\
\textbf{\textsc{Ab}} &
{}-{}- &
{}-{}- &
(he)bel \textsuperscript{3} &
mitai &
anui &
{}-{}- &
kafi &
{}-{}- &
puina &
tur \\
\textbf{\textsc{Km}} &
{}-{}- &
{}-{}- &
bol \textsuperscript{3} &
{}-{}- &
{}-{}- &
iten \~{} iton &
{}-{}- &
{}-{}- &
{}-{}- &
su{\textlengthmark}t \\
\textbf{\textsc{Ku}} &
{}-{}- &
ani{\ng}(na) &
{}-{}- &
{}-{}- &
{}-{}- &
{}-{}- &
kapi &
{}-{}- &
{}-{}- &
{}-{}-\\
\textbf{\textsc{Sw}} &
{}-{}- &
ani{\ng}(ka{\textlengthmark}) &
{}-{}- &
mata{\textlengthmark}ri &
{}-{}- &
iti{\textlengthmark}na &
kapari &
{}-{}- &
{}-{}- &
{}-{}-\\
\textbf{\textsc{We}} &
{}-{}- &
ani{\ng} &
{}-{}- &
{}-{}- &
{}-{}- &
{}-{}- &
k{\textschwa}pir &
{}-{}- &
{}-{}- &
sire \\\hline
\end{supertabular}
\end{flushleft}
\textsuperscript{1 }There are a number of irregularities in this set: Nedebang normally reflects medial *g as x, Western Pantar as g{\textlengthmark}, and Blagar as either {\O} or [241?].\textsuperscript{  2} This root is likely an Austronesian loan: PMP *b{\textschwa}li {\textquoteleft}price{\textquoteright}, bride price{\textquoteright}.  \textsuperscript{3} Denotes {\textquoteleft}bride price{\textquoteright}.  \textsuperscript{4} New Guinea rosewood (Petrocapus indicus), typically referred to in eastern Malay as \textit{kayu merah}.  \textsuperscript{5} Note the irregular loss of the final syllable.  \textsuperscript{6} Semantic shift to {\textquoteleft}claw{\textquoteright}. Also, note the unetymological s, present in both Kaera and Western Pantar.  \textsuperscript{7} While this form is very likely related, it includes several irregularities: the expected reflex of pAP *r in medial position is l in Western Pantar; there is no reflex of pAP *b, which is normally reflected as b; and there is an 
unetymological s.  \textsuperscript{8} Blagar normally reflects pAP *s as h in word-initial position.  \textsuperscript{9} This set shows a variety of irregularities: Adang normally reflects pAP *r as l or I, rather than r; pAP *r is normally reflected as i in final position in both Abui and Kamang; and Wersing normally reflects pAP *s as t, rather than s.

\begin{flushleft}
\tablehead{}
\begin{supertabular}{m{0.8115598in}m{0.81295985in}m{0.81435984in}m{0.81295985in}m{0.8122598in}m{0.81295985in}m{0.81295985in}m{0.8115598in}m{0.81365985in}m{0.8205598in}m{0.8177598in}}
\hline
\textbf{gloss} &
stand &
sugarcane &
sun &
taboo &
tail &
tongue &
tooth &
walk 1 &
walk 2 &
weave\\\hline
\textbf{pAP original} &
{}-{}- &
*u{\textlengthmark}b &
*wad(i) &
{}-{}- &
*-or &
*-leb(ur) &
*-uas &
{}-{}- &
{}-{}- &
{}-{}-\\\hline
\textbf{pAP new} &
*nate(r) \textsuperscript{1} &
*hu:ba &
*wadi &
*palol &
*-ora &
*-lebur &
*-uasin &
*laka \textsuperscript{3} &
*lam(ar) &
*sine(N)\\\hline
\textbf{\textsc{Sr}} &
{}-{}- &
{}-{}- &
war &
{}-{}- &
{}-or &
{}-{}- &
{}-{}- &
{}-{}- &
{}-{}- &
{}-{}-\\
\textbf{\textsc{De}} &
{}-{}- &
{}-{}- &
{}-{}- &
{}-{}- &
{}-or &
{}-{}- &
{}-{}- &
{}-{}- &
{}-{}- &
{}-{}-\\
\textbf{\textsc{Tw}} &
{}-{}- &
{}-{}- &
war (get) &
{}-{}- &
{}-or &
{}-livi &
{}-usan &
{}-{}- &
lam \textsuperscript{5} &
{}-{}-\\
\textbf{\textsc{Nd}} &
{}-{}- &
u{\textlengthmark}fa &
weri &
{}-{}- &
{}-ola &
{}-lefu &
{}-usi{\ng} &
{}-{}- &
{}-{}- &
{}-{}-\\
\textbf{\textsc{Ke}} &
{}-{}- &
u{\textlengthmark}b &
wer &
{}-{}- &
{}-or &
{}-le{\textlengthmark}b &
{}-uasi{\ng} &
{}-{}- &
amar ?` \textsuperscript{6} &
{}-{}-\\
\textbf{\textsc{WP}} &
natar ?` \textsuperscript{2} &
{}-{}- &
war &
{}-{}- &
{}-{}- &
{}-lebu &
{}-wasi{\ng} &
{}-{}- &
lama &
sin{\textlengthmark}a{\ng}\\
\textbf{\textsc{Bl}} &
{}-{}- &
ub &
ved &
{}-{}- &
ora &
{}-d{\textyogh}ebur &
{}-ve{\ng} &
{}-{}- &
lamar &
{}-{}-\\
\textbf{\textsc{Rt}} &
{}-{}- &
juwab &
vid &
{}-{}- &
 &
{}-lebul &
{}-{}- &
{}-{}- &
lamal &
{}-{}-\\
\textbf{\textsc{Ad}} &
{}-{}- &
so{\textlengthmark}b &
f{\textepsilon}d &
{}-{}- &
olo{\textglotstop} &
{}-l{\textepsilon}b &
{}-w{\textepsilon}h{\textepsilon}{\ng} &
{}-{}- &
lami &
{}-{}-\\
\textbf{\textsc{Hm}} &
{}-{}- &
{}-{}- &
f{\o}d &
{}-{}- &
ol &
{}-{}- &
{}-fi{\textglotstop}i{\ng} &
{}-{}- &
lam{\textepsilon} &
{}-{}-\\
\textbf{\textsc{Kb}} &
{}-{}- &
job &
wer &
{}-{}- &
{\textglotstop}ol &
{}-leb &
{}-{}- &
la{\textglotstop}aw &
{}-{}- &
{}-{}-\\
\textbf{\textsc{Ki}} &
{}-{}- &
u{\textlengthmark}b &
ber &
{}-{}- &
{}-or &
{}-liber &
{}-wes &
lak &
{}-{}- &
{}-{}-\\
\textbf{\textsc{Kf}} &
natei &
{}-{}- &
uru &
{}-{}- &
{}-{}- &
{}-lip &
{}-wehe{\ng} &
la{\textlengthmark}ka &
{}-{}- &
{}-{}-\\
\textbf{\textsc{Kl}} &
{}-{}- &
{}-{}- &
{}-{}- &
{}-{}- &
{}-or &
{}-l{\textepsilon}b &
{}-w{\textepsilon}h &
{}-{}- &
(g{\textepsilon}pun)lam &
hnan\\
\textbf{\textsc{Ab}} &
nate &
fa &
wari &
palol &
{}-{}- &
{}-lifi &
{}-weiti &
la{\textlengthmark}k &
{}-{}- &
tinei\\
\textbf{\textsc{Km}} &
{}-{}- &
{}-{}- &
wati &
fo{\textlengthmark}i &
{}-(w)ui &
{}-opei &
{}-weh &
lo{\textlengthmark} ?` \textsuperscript{4} &
{}-{}- &
sine\\
\textbf{\textsc{Ku}} &
{}-{}- &
p\textsuperscript{w}a &
wad &
{}-{}- &
{}-{}- &
il{\i}p &
{}-{}- &
{}-{}- &
{}-{}- &
{}-{}-\\
\textbf{\textsc{Sw}} &
{}-{}- &
{}-{}- &
wadi &
{}-{}- &
{}-(w)o{\textlengthmark}ra &
{}-{}- &
{}-wa &
{}-{}- &
{}-{}- &
{}-{}-\\
\textbf{\textsc{We}} &
{}-{}- &
upa &
widi &
{}-{}- &
wori &
{}-jebur &
{}-wesi &
{}-{}- &
{}-{}- &
{}-{}-\\\hline
\end{supertabular}
\end{flushleft}
\textsuperscript{1} There is a competing and morphologically unrelated form *tas {\textquoteleft}stand{\textquoteright}, which is more widely distributed across modern AP languages (see Holton and Robinson, this volume).  \textsuperscript{2} Western Pantar normally reflects pAP *r as {\O} in word-final position.  \textsuperscript{3} This root is possibly an Austronesian loan: PMP *lakaj {\textquoteleft}stride, take a step{\textquoteright}.  \textsuperscript{4} Kamang normally reflects pAP *k as k.  \textsuperscript{5} Semantic shift to {\textquoteleft}follow{\textquoteright}.  \textsuperscript{6} Kaera normally reflects pAP *l as l in word-initial position.

\begin{flushleft}
\tablehead{}
\begin{supertabular}{m{0.81365985in}m{0.81365985in}}
\hline
\textbf{gloss} &
yellow\\\hline
\textbf{pAP original} &
{}-{}-\\\hline
\textbf{pAP new} &
*bagori\\\hline
\textbf{\textsc{Sr}} &
bahar\\
\textbf{\textsc{De}} &
bug\\
\textbf{\textsc{Tw}} &
ba[127?]ari\\
\textbf{\textsc{Nd}} &
baxori\\
\textbf{\textsc{Ke}} &
bagari\\
\textbf{\textsc{WP}} &
bug{\textlengthmark}a\\
\textbf{\textsc{Bl}} &
bagori ?` \textsuperscript{1}\\
\textbf{\textsc{Rt}} &
bagori\\
\textbf{\textsc{Ad}} &
ba{\textglotstop}oi\\
\textbf{\textsc{Hm}} &
ba{\textglotstop}oil\\
\textbf{\textsc{Kb}} &
ba{\textglotstop}oil\\
\textbf{\textsc{Ki}} &
bagura\\
\textbf{\textsc{Kf}} &
fij{\textupsilon}i\\
\textbf{\textsc{Kl}} &
b{\textupsilon}b{\textupsilon}g{\textopeno}r\\
\textbf{\textsc{Ab}} &
{}-{}-\\
\textbf{\textsc{Km}} &
{}-{}-\\
\textbf{\textsc{Ku}} &
{}-{}-\\
\textbf{\textsc{Sw}} &
{}-{}-\\
\textbf{\textsc{We}} &
{}-{}-\\\hline
\end{supertabular}
\end{flushleft}
\textsuperscript{1} Blagar normally reflects pAP *g as {\O} or [241?] in medial position.

\clearpage\section[Appendix II: data supporting the additional pTIM reconstructions]{Appendix II: data supporting the additional pTIM reconstructions}
\hypertarget{RefHeading65431871885726}{}\begin{flushleft}
\tablehead{}
\begin{supertabular}{m{0.89625984in}m{0.8115598in}m{0.8101598in}m{0.8094598in}m{0.9754599in}m{0.8094598in}m{0.8101598in}m{0.91565984in}m{0.81365985in}m{0.9219598in}m{0.81365985in}}
\hline
\textbf{gloss} &
banana &
bark &
bat &
bite &
bone &
breast &
child &
crawl &
crouch &
dirty\\\hline
\textbf{pTIM original} &
*muku &
{}-{}- &
{}-{}- &
*gakel &
{}-{}- &
{}-{}- &
{}-{}- &
*er(ek) &
{}-{}- &
{}-{}-\\\hline
\textbf{pTIM new} &
*mugu &
*le(k)u(l) &
*maTa &
*(ga)gel &
*(se)sa(r, R) &
*hami &
*-al &
*er &
*luk &
*gari\\\hline
\textbf{\textsc{Bunaq}} &
mok &
{}-{}- &
{}-{}- &
gagil &
sesal &
{}-{}- &
{}-ol &
el &
lu{\textglotstop}(-lu{\textglotstop}) \textsuperscript{4} &
gar\\
\textbf{\textsc{Makasae}} &
mu{\textglotstop}u &
leu \textsuperscript{1} &
{}-{}- &
ga{\textglotstop}el &
{}-{}- &
ami &
{}-{}- &
{}-{}- &
{}-{}- &
ra{\textglotstop}i \textsuperscript{5}\\
\textbf{\textsc{Makalero}} &
mu{\textglotstop}u &
leu \textsuperscript{1} &
{}-{}- &
ka{\textglotstop}el &
{}-{}- &
{}-{}- &
{}-{}- &
{}-{}- &
{}-{}- &
ra{\textglotstop}i \textsuperscript{5}\\
\textbf{\textsc{Fataluku}} &
mu{\textglotstop}u &
le{\textglotstop}ul(e) \textsuperscript{2} &
maca &
(ki)ki{\textglotstop}(e) &
{}-{}- &
ami(-tapunu) \textsuperscript{3} &
{}-{}- &
er(eke) &
{}-{}- &
ra{\textglotstop}e(ne) \textsuperscript{5, 6}\\
\textbf{\textsc{Oirata}} &
mu{\textlengthmark} &
leule \textsuperscript{2} &
ma{\textrtailt}a &
{}-{}- &
{}-{}- &
{}-{}- &
{}-{}- &
{}-{}- &
{}-{}- &
{}-{}-\\\hline
\end{supertabular}
\end{flushleft}
\textsuperscript{1 }Semantic shift to {\textquoteleft}call{\textquoteright}.  \textsuperscript{2 }Semantic shift to {\textquoteleft}sing{\textquoteright}.  \textsuperscript{3} This lexeme is a lexical doublet, i.e. originally a compound or a lexicalized parallel expression (see Schapper et al. 2012: 224).  \textsuperscript{4 }Semantic shift to {\textquoteleft}bent over (as with age){\textquoteright}.  \textsuperscript{5 }This form shows metathesis in Proto-Eastern Timor: *kari {\textgreater} *raki {\textgreater} ra{\textglotstop}i / ra{\textglotstop}e(ne).  \textsuperscript{6} Semantic shift to {\textquoteleft}littered with stones{\textquoteright}.

\begin{flushleft}
\tablehead{}
\begin{supertabular}{m{0.81365985in}m{0.8122598in}m{0.81365985in}m{0.81435984in}m{0.81435984in}m{0.81365985in}m{0.81435984in}m{0.81365985in}m{0.81435984in}m{0.81435984in}m{0.8156598in}}
\hline
\textbf{gloss} &
dream &
eat &
excrement &
face &
far &
fish &
flat &
garden &
hear &
itchy\\\hline
\textbf{pTIM original} &
{}-{}- &
{}-{}- &
{}-{}- &
*fenu &
{}-{}- &
*api &
{}-{}- &
*(u)mar &
*make(n) &
{}-{}-\\\hline
\textbf{pTIM new} &
*ufar(ana) &
*nua &
*a(t, D)u &
*-fanu &
*eTar &
*hapi &
*tetok &
*(u, a)mar &
*mage(n) &
*ilag\\\hline
\textbf{Bunaq} &
waen \textsuperscript{1} &
a \~{} -ia &
ozo &
{}-ewen &
ate &
{}-{}- &
toi{\textglotstop} \textsuperscript{4} &
mar &
mak &
{}-{}-\\
\textbf{Makasae} &
ufarena &
nawa &
atu[-gu{\textglotstop}u] \textsuperscript{2} &
fanu &
{}-{}- &
afi &
{}-{}- &
ama &
ma{\textglotstop}en &
ila{\textglotstop}\\
\textbf{Makalero} &
ofarana &
nua &
atu &
fanu &
{}-{}- &
afi &
tetu{\textglotstop} &
ama &
ma{\textglotstop}en &
ile{\textglotstop}\\
\textbf{Fataluku} &
ufarana &
una \~{} na$\beta $a &
atu \textsuperscript{3} &
fanu &
icar &
api &
{}-{}- &
{}-{}- &
{}-{}- &
{}-{}-\\
\textbf{Oirata} &
upar(a) &
una \~{} nawa &
atu \textsuperscript{3} &
panu &
{}-{}- &
ahi &
{}-{}- &
uma &
{}-{}- &
{}-{}-\\\hline
\end{supertabular}
\end{flushleft}
\textsuperscript{1} This item shows metathesis: waen {\textless} *awen following on fusion from the two halves of the reconstructed doublet.  \textsuperscript{2} The Bunaq cognate for the second half of this lexical doublet is \textit{g-io} {\textquoteleft}3AN-faeces{\textquoteright}, but it doesn{\textquoteright}t appear in a doublet with \textit{ozo} {\textquoteleft}faeces{\textquoteright}.  \textsuperscript{3} Semantic shift to {\textquoteleft}belly{\textquoteright}.  \textsuperscript{4} The final glottal stop in Bunaq is likely a reflex of final *k in pTIM. However, more evidence is needed to substantiate this claim.

\begin{flushleft}
\tablehead{}
\begin{supertabular}{m{0.81365985in}m{0.8122598in}m{0.81365985in}m{0.81435984in}m{0.81435984in}m{0.81365985in}m{0.81435984in}m{0.81365985in}m{0.81435984in}m{0.81435984in}m{0.8156598in}}
\hline
\textbf{gloss} &
laugh &
leg &
LOW &
mat &
mountain &
new &
new place &
nose &
one &
other\\\hline
\textbf{pTIM original} &
*hika &
{}-{}- &
{}-{}- &
{}-{}- &
{}-{}- &
*(t, s)ifa &
{}-{}- &
{}-{}- &
{}-{}- &
{}-{}-\\\hline
\textbf{pTIM new} &
*jiger &
*-buta &
*ufe &
*biti &
*bugu &
*(t, s)ipa(r) &
*lan &
*-muni &
*uneki &
*epi\\\hline
\textbf{Bunaq} &
higal &
{}-but \textsuperscript{1} &
{}-{}- &
{}-{}- &
{}-{}- &
tip &
lon &
{}-inup ?` \textsuperscript{4} &
uen \~{} en &
ewi \textsuperscript{6}\\
\textbf{Makasae} &
hi{\textglotstop}a &
{}-{}- &
he- ?` \textsuperscript{2} &
{}-{}- &
bu{\textglotstop}u &
sufa &
{}-{}- &
muni(kai) \textsuperscript{5} &
u &
{}-{}-\\
\textbf{Makalero} &
hi{\textglotstop}e &
{}-{}- &
ufe- &
piti &
pu{\textglotstop}u \textsuperscript{3} &
hofar &
{}-{}- &
mini &
u \~{} un &
{}-{}-\\
\textbf{Fataluku} &
he{\textglotstop}e &
{}-{}- &
ua- ?` \textsuperscript{2} &
pet(u) &
{}-{}- &
{}-{}- &
{}-{}- &
mini(ku) &
ukani &
{}-{}-\\
\textbf{Oirata} &
{}-{}- &
{}-{}- &
ua ?` \textsuperscript{2} &
het(e) &
{}-{}- &
{}-{}- &
{}-{}- &
{}-{}- &
a{\textglotstop}uni &
{}-{}-\\\hline
\end{supertabular}
\end{flushleft}
\textsuperscript{1} Semantic shift to mean {\textquoteleft}knee{\textquoteright}.  \textsuperscript{2} The reflex of pTIM *f as /h/ in Makasae and {\O} in Fataluku and Oirata is irregular; /f/ is expected for Makasae and Fataluku, and /p/ for Oirata.  \textsuperscript{3} Semantic shift to {\textquoteleft}gable, top of house{\textquoteright}.  \textsuperscript{4} This item appears to show metathesis in the following stages: pTIM *-muni {\textgreater} *-minu {\textgreater} *-imun {\textgreater} *-inum {\textgreater} Bunaq \textit{{}-inup} {\textquoteleft}nose{\textquoteright}. The change of *m to Bunaq p is explainable as the result of m being prohibited from codas in Bunaq.  \textsuperscript{5} The suffix --\textit{kai} is frequently found in body part terms in Makasae.  \textsuperscript{6} It seems likely that medial *p changes to /w/ in Bunaq. However, we currently lack sufficient data to support this conclusion. There has also been a semantic shift to {\textquoteleft}foreigner{\textquoteright}.

\begin{flushleft}
\tablehead{}
\begin{supertabular}{m{0.81365985in}m{0.81435984in}m{0.81365985in}m{0.81435984in}m{0.81365985in}m{0.81435984in}m{0.81365985in}m{0.81435984in}m{0.81435984in}m{0.81365985in}m{0.81435984in}}
\hline
\textbf{gloss} &
path &
person &
rain &
ripe &
scorpion &
scratch &
shark &
six &
spit &
tooth\\\hline
\textbf{pTIM original} &
*hika &
{}-{}- &
{}-{}- &
*tina(k) &
{}-{}- &
{}-{}- &
{}-{}- &
{}-{}- &
{}-{}- &
*wasi\\\hline
\textbf{pTIM new} &
*jiga &
*anu &
*ine(r, R) &
*tena &
*fe(r, R)e &
*gabar &
*supor &
*tamal &
*fulu(k, n) &
*-wasin\\\hline
\textbf{Bunaq} &
hik &
en &
inel &
ten \textsuperscript{1} &
wele \textsuperscript{4} &
{}-{}- &
{}-{}- &
tomol &
puluk &
(-e)we\\
\textbf{Makasae} &
hi{\textglotstop}a &
anu &
{}-{}- &
tina \textsuperscript{2} &
{}-{}- &
{}-{}- &
{}-{}- &
{}-{}- &
{}-{}- &
wasi\\
\textbf{Makalero} &
hi{\textglotstop}a &
anu &
{}-{}- &
tina \~{} 

dina \textsuperscript{2} &
{}-{}- &
kapar &
su(-amulafu) \textsuperscript{5} &
{}-{}- &
fulun &
wasi\\
\textbf{Fataluku} &
i{\textglotstop}a &
{}-{}- &
{}-{}- &
tina \textsuperscript{3} &
{}-{}- &
kafur(e) &
hopor(u) \textsuperscript{6} &
{}-{}- &
fulu &
{\ss}ahin(u)\\
\textbf{Oirata} &
ia(ra) &
{}-{}- &
{}-{}- &
{}-{}- &
{}-{}- &
{}-{}- &
{}-{}- &
{}-{}- &
{}-{}- &
wain(i)\\\hline
\end{supertabular}
\end{flushleft}
\textsuperscript{1 }Semantic shift to {\textquoteleft}be cooked, ready{\textquoteright}.  \textsuperscript{2 }Semantic shift to {\textquoteleft}cook{\textquoteright}.  \textsuperscript{3} Semantic shift to {\textquoteleft}set alight{\textquoteright}.  \textsuperscript{4} It seems likely that initially before front vowels *f changes to /w/ in Bunaq. However, we currently lack sufficient data to support this conclusion.  \textsuperscript{5} The meaning of the compound \textit{su-amulafu} is not quite clear. It seems to refer to a large sea creature, possibly a dolphin or a dugong. The second element, \textit{amulafu}, translates as {\textquoteleft}human being, person{\textquoteright}.  \textsuperscript{6} This form is glossed in various ways in the different Fataluku sources either as {\textquoteleft}shark{\textquoteright} or {\textquoteleft}dugong{\textquoteright}.

\begin{flushleft}
\tablehead{}
\begin{supertabular}{m{0.81365985in}m{0.81435984in}m{0.81295985in}m{0.81365985in}m{0.81435984in}m{0.81365985in}m{0.8156598in}}
\hline
\textbf{gloss} &
tree &
walk 1 &
walk 2 &
yellow &
1SG &
1PI\\\hline
\textbf{pTIM original} &
*hote &
*lakor &
{}-{}- &
{}-{}- &
{}-{}- &
{}-{}-\\\hline
\textbf{pTIM new} &
*hate &
*lagar \textsuperscript{1} &
*male &
*gabar &
*n- &
*fi\\\hline
\textbf{Bunaq} &
hotel &
lagor &
mele &
{}-{}- &
n- &
{}-{}-\\
\textbf{Makasae} &
ate &
la{\textglotstop}a &
{}-{}- &
gabar &
{}-{}- &
fi\\
\textbf{Makalero} &
ate &
la{\textglotstop}a &
{}-{}- &
{}-{}- &
{}-{}- &
fi\\
\textbf{Fataluku} &
ete &
la{\textglotstop}a &
{}-{}- &
{}-{}- &
{}-{}- &
afi\\
\textbf{Oirata} &
ete &
lare &
{}-{}- &
{}-{}- &
{}-{}- &
ap-\\\hline
\end{supertabular}
\end{flushleft}
\textsuperscript{1} This root is possibly an Austronesian loan: PMP *lakaj {\textquoteleft}stride, take a step{\textquoteright}.

\clearpage\setcounter{page}{1}\pagestyle{Convertedvii}
\section[Appendix III: list of cognates and pTAP reconstruction]{Appendix III: list of cognates and pTAP reconstruction}
\hypertarget{RefHeading65433871885726}{}\begin{center}
\tablehead{}
\begin{supertabular}{m{0.9011598in}m{0.81295985in}m{0.81295985in}m{0.83515984in}}
\hline
\textbf{gloss} &
\textbf{pTAP} &
\textbf{pAP} &
\textbf{pTIM}\\\hline
bamboo &
*mari &
*mari &
*mari\\
banana &
*mugul &
*mogol &
*mugu\\
bark, call &
 &
*lVu &
*le(k)u(l)\\
bat &
*madel &
*madel &
*maTa\\
bathe &
*weLi &
*weli &
*weru\\
bird &
*(h)adul &
*(a)dVl &
*haDa\\
bite &
*ki(l) &
*(ta)ki &
*(ga)gel\\
blood &
*waj &
*wai &
*waj\\
bone &
*se(r, R) &
*ser &
*(se)sa(r, R)\\
breast &
*hami &
*hami &
*hami\\
grandparent &
*(t, d)ama &
*tam(a, u) &
*moTo\\
child &
*-uaQal &
*-uaqal &
*-al\\
clew &
*ma(i)ta(r) &
*maita &
*matar\\
coconut &
*wata &
*wata &
*wa(t, D)a\\
crawl &
*er &
*er &
*er\\
crouch &
*luk(V) &
*luk(V) &
*luk\\
die &
*mV(n) &
*min(a) &
*-umV\\
dirty &
*karV(k) &
*karok &
*gari\\
dog &
*dibar &
*jibar &
*Depar\\
dream &
*(h)ipar &
*hipar &
*ufar(ana)\\
ear &
*-waRi &
*-uari &
*-wali\\
eat &
*nVa &
*nai &
*nua\\
excrement &
*(h)at(V) &
*has &
*a(t, D)u\\
face &
*panu &
*-pona &
*-fanu\\
far &
*le(t, d)e &
*lete &
*eTar\\
fire &
*hada &
*hada &
*haTa\\
fish &
*habi &
*habi &
*hapi\\
flat &
*tatok &
*tatok &
*tetok\\
garden &
*magad &
*magad(a) &
*(u, a)mar\\
girl &
*pan(a) &
*pon &
*fana\\
give &
*-(e, i)na &
*-ena &
*-inV\\
green &
*lugar &
*(wa)logar &
*ugar\\
hand &
*-tan(a) &
*-tan &
*-tana\\
hear &
*ma(g, k)e(n) &
*magi &
*mage(n)\\
inside &
*mi &
*mi &
*mi\\
itchy &
*iRak &
*(i)ruk &
*ilag\\
laugh &
*jagir &
*jagir &
*jiger\\
leg &
*buta &
*-bat &
*-buta\\
LOW &
*po &
*po &
*ufe\\
mat &
*bi(s, t) &
*bis &
*biti\\
meat &
*isor &
*iser &
*seor\\
moon &
*hur(u) &
*wur &
*huru\\
mountain &
*buku &
*buku &
*bugu\\
name &
 &
*-en(i, u) &
*-nej\\
new &
*(t, s)iba(r) &
*siba(r) &
*(t, s)ipa(r)\\
new place &
*lan &
*lan &
*lan\\
nose &
*-mVN &
*-mim &
*-muni\\
one &
*nukV &
*nuk &
*uneki\\
other &
*abe(nVC) &
*aben(VC) &
*epi\\
\textit{P. indicus} &
*matar &
*matar &
*ma(t, D)ar\\
path &
*jega &
*jega &
*jiga\\
person &
*anV(N) &
*anin &
*anu\\
pig &
*baj &
*baj &
*baj\\
pound &
*tapa(i) &
*tapai &
*tafa\\
price &
*boL &
*bol &
*bura\\
rain &
*anu(r, R) &
*anur &
*ine(r, R)\\
rat &
*dur(a) &
*dur &
*Dura\\
ripe &
*tena &
*tena &
*tena\\
run &
*tipar &
*tiara &
*tifar\\
scorpion &
*pV(r, R) &
*pVr &
*fe(r, R)e\\
scratch &
*karab &
*karab &
*gabar\\
sea &
*tam(a) &
*tam &
*mata\\
shark &
*sibar &
*sib(a ,i)r &
*supor\\
sit &
*mit &
*mis &
*mit\\
six &
*talam &
*talam &
*tamal\\
sleep &
*tia(r) &
*tia &
*tia(r)\\
spit &
*puRV(n) &
*purVn &
*fulu(k, n)\\
spoon &
*suRa &
*surV &
*sula\\
stand &
*nat(er) &
*nate(r) &
*nat\\
star &
*jibV &
*jibV &
*ipi(-bere)\\
stone &
*war &
*war &
*war\\
sugarcane &
*ub(a) &
*hu{\textlengthmark}ba &
*upa\\
sun &
*wad(i, u) &
*wadi &
*waTu\\
taboo &
*palu(l, n) &
*palol &
*falu(n)\\
tail &
*-oRa &
*-ora &
*-ula({\textglotstop})\\
tongue &
*-lebuR &
*-lebur &
*-ipul\\
tooth &
*-wasin &
*-uasin &
*-wasin\\
tree &
*hate &
*tei &
*hate\\
vagina &
*-ar(u) &
*-ar &
*-aru\\
wake &
*tan(i) &
*-ten &
*Tani\\
walk 1 &
*lak(Vr) &
*laka &
*lagar\\
walk 2 &
*lamV &
*lam(ar) &
*male\\
water &
*jira &
*jira &
*ira\\
weave &
*sine(N) &
*sine(N) &
*sina\\
yellow &
*bagur(V) &
*bagori &
*gabar\\
1PI &
*pi &
*pi- &
*fi\\
1SG &
*na- &
*na- &
*n-\\
3ALN &
*gie &
*ge &
*gie\\
3INLN &
*g(a, i)- &
*ga- &
*g-\\\hline
\end{supertabular}
\end{center}

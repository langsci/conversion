\clearpage\setcounter{page}{1}\pagestyle{Standard}
{\centering
\textit{Chapter 4}
\par}

\begin{center}
 [Warning: Image ignored] % Unhandled or unsupported graphics:
%\includegraphics[width=5.6803in,height=3.9862in,width=\textwidth]{Ch4Holton-img1}

\end{center}
{\centering
\textbf{The Linguistic Position of }
\par}

{\centering
\textbf{the Timor-Alor-Pantar Languages}
\par}

{\centering
\textit{Gary Holton and Laura C. Robinson}
\par}

\setcounter{tocdepth}{3}
\renewcommand\contentsname{}
\tableofcontents
\clearpage{\centering
\textit{Chapter 4}
\par}

{\centering
\textbf{T}\textbf{he Linguistic Position of }
\par}

{\centering
\textbf{the Timor-Alor-Pantar }\textbf{Languages}
\par}

{\centering
\textit{Gary Holton and Laura C. Robinson}
\par}

Abstract: The wider genealogical affiliations of the Timor-Alor-Pantar languages have been the subject of much speculation. These languages are surrounded by unrelated Austronesian languages, and attempts to locate related languages have focused on Papuan languages 800 km or more distant. This chapter draws on typological, pronominal, and especially lexical evidence to examine three hypotheses regarding the higher-level affiliations of the Timor-Alor-Pantar languages: (1) the languages are related to the North Halmaheran (West Papuan) languages; (2) the languages are part of the Trans-New Guinea family; and (3) the languages are related to the West Bomberai family, with no link to Trans-NewGuinea more broadly. We rely in particular on recent reconstructions of proto-Timor-Alor-Pantar vocabulary (chapter 3). Of the hypotheses evaluated here, we find the most striking similarities between TAP and the West Bomberai family. However, we conclude that the evidence currently available is insufficient to confirm a 
genealogical relationship with West Bomberai or any other family, and hence, TAP must be considered a family-level isolate. 

\section[Introduction]{Introduction}
\hypertarget{RefHeading72062871885726}{}\hypertarget{Toc376958109}{}The non-Austronesian languages of the Alor and Pantar islands in eastern Indonesia have been shown to form a genealogical unit (see chapter 2) and these, in turn, have been shown to be part of a larger family which includes the non-Austronesian languages of Timor (see chapter 3). Here we examine the wider genealogical affiliations of the Timor-Alor-Pantar family, following (Robinson and Holton 2012)\footnote{ This chapter differs from Robinson and Holton (2012) in that it includes a discussion of the typological profiles of the TAP family and putative relatives, and has also been updated to reflect new reconstructions, especially the proto-Timor-Alor-Pantar reconstructions in chapter 3. In the absence of reconstructions for proto-Timor (now available in Schapper et al. 2012) and proto-Timor-Alor-Pantar (see chapter 3), Robinson and Holton (2012) relied exclusively on proto-Alor-Pantar reconstructions, with Timor look-alikes included where 
available. }. Prior to this work most authors assumed a connection to Trans-New Guinea languages, based primarily on evidence from pronominal paradigms (Ross 2005). However, several other plausible hypotheses have been proposed, which we shall examine in this chapter.The Timor-Alor-Pantar (TAP) languages are surrounded on all sides by Austronesian languages, with the nearest Papuan (non-Austronesian) language located some 800 km distant.\footnote{ The extinct language of Tambora, known only from nineteenth century wordlists, was spoken some 650 km west of Pantar, and it is presumed to have been non-Austronesian (Donohue 2007a).} Some putative relatives of the TAP family are shown in Figure 1 below.

{\centering
Figure 1: Location of Timor-Alor-Pantar languages (lower left) and putative related families discussed in this chapter
\par}

{\centering   [Warning: Image ignored] % Unhandled or unsupported graphics:
%\includegraphics[width=5.6807in,height=3.9862in,width=\textwidth]{Ch4Holton-img2}
 \par}

In this chapter, we will consider three hypotheses about the wider relationships of the TAP family: (1) the TAP languages are related to the North Halmaheran (NH) languages; (2) the TAP languages belong to the Trans-New Guinea (TNG) family (broadly defined); and (3) the TAP languages are related to certain Papuan languages within the putative TNG family, even though the evidence linking them with TNG as a whole is indeterminate and these languages may not in fact be TNG. In order to examine the first two hypotheses we compare TAP reconstructed forms with proposed reconstructions for North Halmahera and Trans-New Guinea, respectively. In order to evaluate the third hypothesis we compare TAP reconstructions with languages from four smaller families: South Bird{\textquoteright}s Head; Wissel Lakes; Dani; and West Bomberai. Although each of these families has been claimed to be a part of some version of the larger Trans-New Guinea group, the composition of these smaller families is uncontroversial and thus 
allows us to evaluate potential wider affiliations while remaining agnostic as to the status of Trans-New Guinea itself. Ideally, we would compare TAP to reconstructed proto-languages for each of these four families; however, given the limited historical work done on those families, we instead choose individual languages from each family for comparison with TAP. We examine each of the three hypotheses in light of recently collected data on the TAP languages, considering pronominal, typological, and lexical evidence. Finally, we conclude with a discussion of the null hypothesis that the TAP languages form a family-level isolate.

The first hypothesis was suggested (and quickly discarded) by Capell (1944), who noted similarities between the Papuan languages of Timor and those of North Halmahera but initially refrained from asserting a genealogical relationship. By that time, the non-Austronesian character of the NH languages had long since been recognized, having been mentioned by Robide van der Aa (1872) and later rigorously demonstrated by van der Veen (1915). Anceaux (1973), commenting on a field work report from the Pantar language Teiwa (Watuseke 1973), proposed including Teiwa and several Alor languages (Abui, Wersing, Kui) with Cowan{\textquoteright}s (1957) West Papuan group, which included NH.\footnote{ Watuseke (1973) does not identify the language as Teiwa but merely refers to it as {\textquotedblleft}a language of Pantar{\textquotedblright}. However, inspection of the data leave no doubt that this is Teiwa.} As later formulated, Capell{\textquoteright}s (1975) West Papuan Phylum included the {\textquotedblleft}Alor-Timor{\
textquotedblright} languages. In fact, only one Alor language, Abui, was included in Capell{\textquoteright}s grouping, as Capell only belatedly became aware of the other extant Alor sources. Even with these additional data, Capell was quite conscious of the tenuous nature of the putative relationship between TAP (actually Alor-Timor) and North Halmahera, particularly the lack of identifiable lexical correspondences. He thus proposed a major split between Alor-Timor (and some Bird{\textquoteright}s Head languages) on the one hand, and the rest of the West Papuan Phylum on the other. Stokhof suggested connecting TAP with several languages of the Western Bird{\textquoteright}s Head of New Guinea, concluding that {\textquotedblleft}the Alor-Pantar languages form a closely related group with Cowan{\textquoteright}s West Papuan Phylum{\textquotedblright} (1975: 26). However, the putative West Papuan languages with which Stokhof compared Alor-Pantar were later reclassified as Trans-New Guinea, rendering this 
lexical evidence moot. More recently Donohue (2008) has revived the NH hypothesis, based largely on pronominal evidence. 

With the exception of this recent work by Donohue, the second hypothesis connecting TAP with TNG has largely supplanted the NH hypothesis in the literature. Capell{\textquoteright}s (1975) paper arguing for the NH hypothesis was published with an editorial preface noting that the TAP languages should instead be included within TNG (Wurm 1975: 667). However, the accompanying paper on the TNG hypothesis in the same volume provides no data to back up this classification and instead remains skeptical as to whether TAP should be classified as Trans-New Guinea or West Papuan. In particular, the authors assert that {\textquotedblleft}whichever way they [the TAP languages] are classified, they contain strong substratum elements of the other {\dots} phyla involved{\textquotedblright} (Wurm et al. 1975: 318). Only recently have additional data been provided to support the TNG hypothesis. Pawley (2001) cites lexical evidence from TAP languages in support of proto Trans-New Guinea (pTNG) reconstructions. Ross (2005) 
connects TAP to TNG more broadly based on pronominal evidence. Although the evidence for the TNG hypothesis is far from overwhelming, it is today the most widely received classification, appearing for example in the most recent edition of the Ethnologue (Lewis et al. 2013).

One of the challenges to finding support for the TNG hypothesis is the sheer size and diversity which exists within the family. Rather than only considering TNG as a whole, it is also useful to consider smaller families within TNG. Two proposals stand out. Reesink (1996) suggests connections between TAP and the South Bird{\textquoteright}s Head family (specifically the Inanwatan language). Cowan (1953) also made this connection, though he went further to group both TAP and South Bird{\textquoteright}s Head within his West Papuan Phylum. A second proposal is made by Ross (2005), who considers TAP {\textquotedblleft}possibly part of a western TNG linkage{\textquotedblright} including West Bomberai, Wissel Lakes, and Dani. As Ross suggests, this more circumscribed linkage is a group of languages descended from a dialect chain and therefore characterized by overlapping innovations. In particular, Ross notes that these languages (including the Timor languages, but excluding the Alor and Pantar languages) all show 
an innovative metathesis of CV to VC in the first person singular pronoun and that the TAP languages share an innovative first person plural pronoun with the West Bomberai languages (2005: 36). We are not aware of any serious proposals connecting TAP to Papuan languages outside NH (and the West Papuan Phylum) and TNG.

The possibility that the TAP languages form a family-level isolate not demonstrably related to other Papuan languages was actually suggested by Capell, who concluded:

{\itshape
{\textquotedblleft}Neither are the {\textquoteleft}Papuan{\textquoteright} languages outside New Guinea, in the Solomons, New Britain, Halmahera or Timor related to each other or to those of New Guinea. At least it cannot be assumed that any two are related{\dots}.{\textquotedblright} \textup{(1944: 313)}}

However, this null hypothesis has not, to our knowledge, been given serious consideration in the literature. We return to this point in our conclusion (section 6). In the meantime we evaluate the first two hypotheses in light of the typological evidence (section 2), pronominal evidence (section 3), and lexical evidence (section 4). Evidence for the third hypothesis linking the TAP family with individual languages in Papua is considered in section 5.

\section[Typological evidence]{Typological evidence}
\hypertarget{RefHeading72064871885726}{}Given that typological features can easily cross genealogical boundaries, typological evidence for genealogical relationships should be approached with caution. Klamer et al. (2008) argue that the region under consideration here---spanning from TAP to NH to New Guinea---is part of the East Nusantara linguistic area which shares a number of typological features in spite of genealogical differences among languages. Moreover, these features are not particularly unique and hence do not provide any special proof of genealogical connection in the sense of Meillet (1967). On the other hand, we feel that a volume on the Alor-Pantar languages would not be complete without a discussion of how the typological profile of the family relates to those of the surrounding Papuan languages. Nonetheless, we find little evidence for shared typological features between TAP and either the NH or TNG families. In this section we provide examples contrasting the typological profiles of these 
families, considering phonology (section 2.1), morphology (section 2.2), and syntax (section 2.3).

\subsection[Phonology]{Phonology}
\hypertarget{RefHeading72066871885726}{}Foley (1998) suggests two typically Papuan phonological features: the presence of a single liquid phoneme and the presence of pre-nasalized stops. Neither of these putative Papuan phonological features is found in proto-Timor-Alor-Pantar (pTAP), which had at least two liquids and lacks pre-nasalized stops. The pTAP consonant inventory (based on chapter 3), is shown in Table 1.

{\centering
\label{bkm:Ref196728383}Table 1: pTAP consonants (based on chapter 3)
\par}

\begin{center}
\tablehead{}
\begin{supertabular}{m{1.2108599in}m{0.5545598in}m{0.78235984in}m{0.6726598in}m{0.51085985in}m{0.6802598in}}
\hline
 &
\textsc{labial} &
\textsc{alveolar} &
\textsc{palatal} &
\textsc{velar} &
\textsc{glottal}\\\hline
\textsc{voiceless stops} &
\centering p &
\centering t &
 &
\centering k &
\\
\textsc{voiced stops} &
\centering b &
\centering d &
 &
\centering g &
\\
\textsc{nasals} &
\centering m &
\centering n &
 &
 &
\\
\textsc{fricatives} &
 &
\centering s &
 &
 &
\centering\arraybslash h\\
\textsc{glides} &
\centering w &
 &
\centering j &
 &
\\
\textsc{liquids} &
 &
\centering l r\footnotemark{} &
 &
 &
\\\hline
\end{supertabular}
\end{center}
\footnotetext{ Schapper et al. (chapter 3) note that there are three correspondence sets between AP on the one hand, and Timor-Kisar, on the other, and so they reconstruct a third liquid *R, but they do not speculate about the phonetic value of *R. Since none of the modern TAP languages has more than two liquids, we believe that the proto-language had just two liquids, and that the third correspondence set should be attributed to either *r or *l, with some as yet to be identified conditioning. }
Nor are these features present in proto-North Halmahera (pNH), shown in Table 2.

{\centering
Table 2: pNH consonants (after Wada 1980)\footnote{ \textstyleFootnoteTextChar{Wada (1980) actually reconstructs two *p series, two *l series, and two *r series. Closer inspection of the data reveals that these distinctions in correspondence sets are conditioned by position. Wada{\textquoteright}s *P1 and *L2 sets actually represent medial correspondences; *R1 represents a non-initial correspondence.}}
\par}

\begin{center}
\tablehead{}
\begin{supertabular}{m{1.2108599in}m{0.5545598in}m{0.78235984in}m{0.83375984in}m{0.51085985in}m{0.6802598in}}
\hline
 &
\textsc{labial} &
\textsc{alveolar} &
\textsc{retroflex} &
\textsc{velar} &
\textsc{glottal}\\\hline
\textsc{voiceless stops} &
\centering p &
\centering t &
 &
\centering k &
\\
\textsc{voiced stops} &
\centering b &
\centering d &
\centering {\textrtaild} &
\centering g &
\\
\textsc{nasals} &
\centering m &
\centering n &
 &
\centering {\ng} &
\\
\textsc{fricatives} &
 &
\centering s &
 &
 &
\centering\arraybslash h\\
\textsc{glides} &
\centering w &
 &
 &
 &
\\
\textsc{liquids} &
 &
\centering l &
\centering r &
 &
\\\hline
\end{supertabular}
\end{center}
On the other hand, both pre-nasalized stops and a single liquid phoneme are found in pTNG. Additionally, in contrast to either pTAP or pNH, pTNG contains only a single fricative.

{\centering
Table 3: pTNG consonants (Pawley 1995, 2001)
\par}

\begin{center}
\tablehead{}
\begin{supertabular}{m{1.4830599in}m{0.5545598in}m{0.5483598in}m{0.51155984in}}
\hline
 &
\textsc{labia}\textsc{l} &
\textsc{apica}\textsc{l} &
\textsc{vela}\textsc{r}\\\hline
\textsc{oral stops} &
\centering p &
\centering t\footnotemark{} &
\centering\arraybslash k\\
\textsc{prenasalized stops} &
\centering mb &
\centering nd &
\centering\arraybslash {\ng}g\\
\textsc{nasals} &
\centering m &
\centering n &
\centering\arraybslash {\ng}\\
\textsc{fricatives} &
 &
\centering s &
\\
\textsc{glides} &
\centering w &
\centering y &
\\
\textsc{liquid} &
 &
\centering l &
\\\hline
\end{supertabular}
\end{center}
\footnotetext{ Note that the pTNG apical stop *t may have had a flap or trill allophone (Pawley 2001: 273).}
In many respects, these three consonant inventories are similar. Each contains two sets of stops. In pTAP and pNH, the distinction between the two sets is voicing, with one voiced set and one voiceless set. In pTNG the distinction is between oral and pre-nasalized. It is plausible that the pTNG pre-nasalized stops developed into the pTAP voiced stops. Nevertheless, considering just the four phonological features discussed above we find greater similarity between TAP and NH than between TAP and TNG, as summarized in Table 4.

{\centering
Table 4: Summary of TAP, TNG and NH phonological features
\par}

\begin{center}
\tablehead{}
\begin{supertabular}{m{1.2372599in}m{0.36015984in}m{0.38585985in}m{0.29205984in}}
 &
TAP &
TNG &
NH\\
pre-nasalized stops &
\centering {}- &
\centering [F0FC?] &
\centering\arraybslash {}-\\
single liquid &
\centering {}- &
\centering [F0FC?] &
\centering\arraybslash {}-\\
uvular consonant &
\centering [F0FC?] &
\centering {}- &
\centering\arraybslash {}-\\
single fricative &
\centering {}- &
\centering [F0FC?] &
\centering\arraybslash {}-\\
\end{supertabular}
\end{center}
\subsection[Morphology]{Morphology}
\hypertarget{RefHeading72068871885726}{}Among the few typologically distinctive morphological features of the TAP languages is the presence of pronominal indexing of the patient-like argument of a transitive verb (P) via a pronominal prefix (see chapter 10). Reflexes of a P prefix are widely distributed across the family and can be reconstructed to pTAP. These prefixes generally have the same form as those which index possessors on nouns, as in the Teiwa example in (1), where the third singular prefix on the verb indexes the third singular P argument, while the first singular prefix on the noun {\textquoteleft}child{\textquoteright} indexes the possessor.

Teiwa (AP, Klamer 2010: 159)

\begin{flushleft}
\tablehead{}
\begin{supertabular}{m{0.39205986in}m{0.39275986in}m{0.51155984in}m{0.8073598in}m{2.18656in}}
(1) &
Name, &
ha[241?]an &
n-oqai &
g-unba{\textglotstop}.\\
 &
Sir &
\textsc{2sg} &
\textsc{1sg}{}-child &
\textsc{3sg}{}-meet\\
\end{supertabular}
\end{flushleft}
{\textquoteleft}Sir, did you see (lit. meet) my child?{\textquoteright}

However, P prefixes are in general not obligatory in TAP, and the conditions on pronominal alignment vary considerably among the individual languages of the family (Fedden et al. 2013, chapter 10). For example, Bunaq (Timor) does not use pronominal prefixes to index inanimate P arguments. In example (2), there is no prefix on the verb because the P argument \textit{zo }{\textquoteleft}mango{\textquoteright} is inanimate. In example (3), in contrast, the verb takes a third person prefix which indexes \textit{zap} {\textquoteleft}dog{\textquoteright}.

Bunaq (Timor, Schapper 2009: 122)

\begin{flushleft}
\tablehead{}
\begin{supertabular}{m{0.36155984in}m{0.5462598in}m{0.5288598in}m{0.6712598in}}
(2) &
Markus &
zo &
poi\\
 &
Markus &
mango  &
choose\\
\end{supertabular}
\end{flushleft}
{\textquoteleft}Markus chose a mango.{\textquoteright} 

\begin{flushleft}
\tablehead{}
\begin{supertabular}{m{0.36155984in}m{0.5469598in}m{0.29625985in}m{0.6712598in}}
(3) &
Markus &
zap &
go-poi\\
 &
Markus &
dog &
\textsc{3}{}-choose\\
\end{supertabular}
\end{flushleft}
{\textquoteleft}Markus chose a dog.{\textquoteright} 

In the AP language Abui, alignment is semantic, and most non-volitional arguments are marked with pronominal prefixes, including non-volitional S arguments (Fedden et al. 2013, chapter 10). In (4) the sole argument is volitional, so there is no marking on the verb. In (5) the first person undergoer is non-volitional and is indexed on the verb with the prefix \textit{no-.} Likewise, in (6) the verb \textit{wel }{\textquoteleft}pour{\textquoteright} takes the third person prefix \textit{ha-} because the undergoer \textit{Simon} is non-volitional. Finally, we see in (7) that even the sole argument of the verb can be indexed with a prefix if it is non-volitional.

Abui (AP, Kratochv\'il 2007: 80, 171)

\begin{flushleft}
\tablehead{}
\begin{supertabular}{m{0.39005986in}m{0.30735984in}m{0.78095984in}}
(4) &
Na &
sei.\\
 &
\textsc{1sg} &
come.down\\
\end{supertabular}
\end{flushleft}
{\textquoteleft}I come down.{\textquoteright} 

\begin{flushleft}
\tablehead{}
\begin{supertabular}{m{0.38935986in}m{0.47055987in}m{0.69905984in}}
(5) &
Simon &
no-dik.\\
 &
Simon &
\textsc{1sg}{}-tickle\\
\end{supertabular}
\end{flushleft}
{\textquoteleft}Simon is tickling me.{\textquoteright} 

\begin{flushleft}
\tablehead{}
\begin{supertabular}{m{0.38935986in}m{0.30805984in}m{0.47055987in}m{0.5254598in}}
(6) &
Na &
Simon &
ha-wel.\\
 &
\textsc{1sg} &
Simon &
3-pour\\
\end{supertabular}
\end{flushleft}
{\textquoteleft}I washed Simon.{\textquoteright} 

\begin{flushleft}
\tablehead{}
\begin{supertabular}{m{0.49625984in}m{0.7358598in}}
(7) &
No-lila.\\
 &
\textsc{1sg}{}-be.hot\\
 &
{\textquoteleft}I am hot.{\textquoteright} \\
\end{supertabular}
\end{flushleft}
A few TAP languages also permit indexing of both A and P arguments via pronominal prefixes. In such cases, the prefix paradigms for each argument are identical.

Western Pantar (AP, Holton 2010)

\begin{flushleft}
\tablehead{}
\begin{supertabular}{m{0.41365984in}m{0.48025987in}m{2.5184598in}}
(8) &
Ke{\textquoteright}e &
pi-ga-ussar.\\
 &
fish &
\textsc{1pl-3sg}{}-catch\\
\end{supertabular}
\end{flushleft}
{\textquoteleft}We{\textquoteright}re catching fish.{\textquoteright} 

The North Halmaheran languages also index P arguments on the verb, and as in TAP, the conditions on pronominal indexing vary considerably across different languages in the family (Holton 2008). However, pronominal indexing in NH languages differs in several respects from that found in TAP. First, not just P but also A is referenced on the verb in NH. Second, for most NH languages pronominal indexing is obligatory. Third, unlike TAP languages, the forms of A and P pronominal prefixes differ from each other in NH. That is, A and P arguments are marked by distinct paradigms, and this holds for both pronominal prefixes as well as independent pronouns. The Tobelo example in (9) illustrates these properties.

Tobelo (NH, Holton 2003)

\begin{flushleft}
\tablehead{}
\begin{supertabular}{m{0.48795983in}m{0.70875984in}m{2.06016in}}
(9) &
(Ngohi) &
t-i-ngoriki.\\
 &
(\textsc{1sg}) &
\textsc{1sg-3sg.m}{}-see\\
\end{supertabular}
\end{flushleft}
\ \ {\textquoteleft}I see him.{\textquoteright}  

Moreover, in NH languages the order of verbal referents is fixed as actor-undergoer, while for TAP languages which permit two pronominal prefixes, the order may in some cases be reversed as undergoer-actor, as in (10).

Western Pantar (AP, Holton fieldnotes)

\begin{flushleft}
\tablehead{}
\begin{supertabular}{m{0.39005986in}m{0.46985987in}m{0.36775985in}m{0.33515984in}m{0.9094598in}}
(10) &
gai &
ya &
me &
ga-na-asang\\
 &
\textsc{3poss} &
road &
\textsc{loc} &
\textsc{3sg-1sg}{}-say\\
\end{supertabular}
\end{flushleft}
\ \ {\textquoteleft}I will tell him the way.{\textquoteright} (lit., {\textquoteleft}I will him about his road.{\textquoteright})

Indexing of P arguments is also a prominent feature of verbs in Trans-New Guinea languages. Verbs with P arguments indexed via prefixes are found for example in the Finisterre-Huon family, and P-marking prefixes can be reconstructed at the level of pTNG (Suter 2012). Indexing of P arguments is illustrated in (11) with data from Fore, where the first person singular object is indicated with a verbal prefix.

Fore (TNG, Scott 1978: 107)

\begin{flushleft}
\tablehead{}
\begin{supertabular}{m{0.32545984in}m{0.31775984in}m{1.8156599in}}
(11) &
N\'ae &
na-ka-y-e.\\
 &
1\textsc{sg} &
\textsc{1sg.und}{}-see-\textsc{3sg.act-decl}\\
\end{supertabular}
\end{flushleft}
{\textquoteleft}He sees me.{\textquoteright}

In contrast to both TAP and NH languages, pTNG indexed subjects (both A and S) via suffixes, not prefixes (Foley 2000). However, subject prefixes are not unknown in TNG languages. Foley cites Marind as an example of a Papuan language with both subject and object prefixes, noting that {\textquotedblleft}Marind is the only Papuan language I know which consistently exhibits A-U-V{\textquotedblright} order (1986: 138). 

Marind (TNG, Drabbe 1955, cited in Foley 1986: 138)

\begin{flushleft}
\tablehead{}
\begin{supertabular}{m{0.48795983in}m{1.5948598in}}
(12) &
A-na-kipraud.\\
 &
3\textsc{sg.subj-1sg.obj-}tie\\
\end{supertabular}
\end{flushleft}
{\textquoteleft}He ties me.{\textquoteright}  

While the Marind example in (12) may not be typical for TNG languages, it certainly shows much affinity with pronominal indexing patterns in both TAP and NH languages. 

The TAP languages exhibit preposed possessor constructions, a typically Papuan feature, at least for East Nusantara (Klamer et al. 2008). The possessor precedes the possessum, whether the possessor is expressed as a full noun phrase (13) or just with a pronoun (14).

Western Pantar (AP, Holton fieldnotes)

\begin{flushleft}
\tablehead{}
\begin{supertabular}{m{0.48795983in}m{0.51225984in}m{0.41295984in}m{0.8073598in}m{0.51295984in}}
(13) &
yabbe &
si &
gai &
bla\\
 &
dog &
that &
\textsc{3sg.poss} &
house\\
\end{supertabular}
\end{flushleft}
{\textquoteleft}the dog{\textquoteright}s house{\textquoteright} 

\begin{flushleft}
\tablehead{}
\begin{supertabular}{m{0.48795983in}m{0.8073598in}m{0.6108598in}}
 (14) &
nai &
bla\\
 &
\textsc{1sg.poss} &
house\\
\end{supertabular}
\end{flushleft}
 {\textquoteleft}my house{\textquoteright} 

NH languages exhibit a similar pattern of possessor-possessum order, as in the Tobelo examples below.

Tobelo (NH, Holton 2003)

\begin{flushleft}
\tablehead{}
\begin{supertabular}{m{0.32545984in}m{0.55315983in}m{0.6809598in}}
(15) &
o-kaho &
ma-tau\\
 &
\textsc{nm}{}-dog &
\textsc{nm}{}-house\\
\end{supertabular}
\end{flushleft}
{\textquoteleft}the dog{\textquoteright}s house{\textquoteright}  

\begin{flushleft}
\tablehead{}
\begin{supertabular}{m{0.32545984in}m{1.0518599in}}
(16) &
ahi-tau\\
 &
\textsc{1sg.poss}{}-house\\
\end{supertabular}
\end{flushleft}
{\textquoteleft}my house{\textquoteright}  

The order possessor-possessum is also found widely among TNG languages, as illustrated by the Enga and Mian examples below. 

Enga (TNG, Foley 1986: 264)

\begin{flushleft}
\tablehead{}
\begin{supertabular}{m{0.39005986in}m{1.1427599in}m{0.43305984in}}
(17) &
namba-ny\'a &
men\'a\\
 &
\textsc{1sg-poss} &
pig\\
\end{supertabular}
\end{flushleft}
{\textquoteleft}my pig{\textquoteright} 

Mian (TNG, Fedden 2011: 217)

\begin{flushleft}
\tablehead{}
\begin{supertabular}{m{0.38935986in}m{0.75255984in}m{1.0045599in}}
(18) &
\=ob &
imak\\
 &
\textsc{2sg.f.pos}\textsc{s} &
husband\\
\end{supertabular}
\end{flushleft}
{\textquoteleft}your husband{\textquoteright} 

The order possessum-possessor is also found in many TNG languages, particularly with inalienable nouns, as illustrated by the following examples from Fore and Barai. 

Fore (TNG, Scott 1978: 31)

\begin{flushleft}
\tablehead{}
\begin{supertabular}{m{0.32545984in}m{0.8906598in}}
(19) &
yaga-nene\\
 &
pig-\textsc{1sg.poss}\\
\end{supertabular}
\end{flushleft}
{\textquoteleft}my pig{\textquoteright}  

Barai (TNG, Olson 1981, cited in Foley 1986)

\begin{flushleft}
\tablehead{}
\begin{supertabular}{m{0.32545984in}m{0.47885987in}m{0.65805984in}}
(20) &
e &
n-one\\
 &
person &
\textsc{1sg-poss}\\
\end{supertabular}
\end{flushleft}
{\textquoteleft}my people{\textquoteright} 

A distinction between alienable and inalienable possession is considered a typical Papuan feature, and TAP languages share this feature. While TAP languages vary in exactly how they realize this distinction, Western Pantar is typical in realizing this distinction in the possessive pronouns. In Western Pantar the third person singular inalienable form is \textit{ga-} rather than \textit{gai-}, as in (21).

Western Pantar (AP, Holton fieldnotes)

\begin{flushleft}
\tablehead{}
\begin{supertabular}{m{0.32545984in}m{1.8962599in}}
(21) &
ga-uta  (*gai-)\\
 &
\textsc{3sg.inal}{}-foot  (3\textsc{sg.alien-)}\\
\end{supertabular}
\end{flushleft}
{\textquoteleft}his/her/its foot{\textquoteright}

Many of the TNG languages also share this distinction. In Inanwatan, alienably possessed nouns take independent pronouns, like \textit{tig\'aeso} in (22), while inalienably possessed nouns take pronominal prefixes, like \textit{na- }in (23). 

Inanwatan (South Bird{\textquoteright}s Head, de Vries 2004: 29, 30)\footnote{ The acute accent indicates lexical stress, which is distinctive in Inanwatan.}

(22)\ \ tig\'ae-so\ \ suq\'ere

\textsc{\ \ 3sg.f}{}-\textsc{m\ \ }sago.\textsc{m}

{\textquoteleft}her sago{\textquoteright} 

(23)\ \ n\'a-wiri

\textsc{\ \ 1sg}{}-belly.\textsc{m}

{\textquoteleft}my belly{\textquoteright} 

While NH languages also have obligatorily possessed nouns, these languages lack a distinct inalienable possession construction. In particular, in NH languages the same possessive construction is used regardless of whether the noun is obligatorily possessed or not. In Tobelo obligatorily possessed nouns such as \textit{lako }{\textquoteleft}eye{\textquoteright} (24) use the same possessive strategy as non-obligatorily possessed nouns such as \textit{tau }{\textquoteleft}house{\textquoteright} (16). 

Tobelo (NH, Holton fieldnotes)

\begin{flushleft}
\tablehead{}
\begin{supertabular}{m{0.48795983in}m{0.31505984in}m{0.70945984in}}
(24) &
a. &
ma-lako\\
 &
 &
\textsc{nm}{}-eye\\
\end{supertabular}
\end{flushleft}
\ \ \ \ {\textquoteleft}eye{\textquoteright}

\begin{flushleft}
\tablehead{}
\begin{supertabular}{m{0.48795983in}m{0.31505984in}m{0.70875984in}m{0.84625983in}}
 &
b. &
o-kaho &
ma-lako\\
 &
 &
\textsc{nm}{}-dog &
\textsc{nm}{}-eye\\
\end{supertabular}
\end{flushleft}
\ \ \ \ {\textquoteleft}the dog{\textquoteright}s eye{\textquoteright}

\begin{flushleft}
\tablehead{}
\begin{supertabular}{m{0.48795983in}m{0.31505984in}m{1.0045599in}}
 &
c. &
ahi-lako\\
 &
 &
\textsc{1sg.poss}{}-eye\\
\end{supertabular}
\end{flushleft}
\ \ \ \ {\textquoteleft}my eye{\textquoteright}

The morphological features for TAP, TNG, and NH are summarized in Table 5. 

{\centering
Table 5: Summary of TAP, TNG, and NH morphological features
\par}

\begin{center}
\tablehead{}
\begin{supertabular}{m{2.07756in}m{0.36015984in}m{0.38585985in}m{0.29205984in}}
 &
\centering TAP &
\centering TNG &
\centering NH\par

\\
pronominal object prefixes (P) &
\centering [F0FC?] &
\centering [F0FC?] &
\centering\arraybslash [F0FC?]\\
pronominal subject affixes (A/S) &
\centering ([F0FC?]) &
\centering [F0FC?] &
\centering\arraybslash [F0FC?]\\
preposed possessors &
\centering [F0FC?] &
\centering ([F0FC?]) &
\centering\arraybslash [F0FC?]\\
alienable/inalienable distinction &
\centering [F0FC?] &
\centering [F0FC?] &
\centering\arraybslash {}-\\
\end{supertabular}
\end{center}
\subsection[Syntax]{Syntax}
\hypertarget{RefHeading72070871885726}{}The TAP languages, like most NH and TNG (Foley 2000) languages, are right-headed and verb-final. 

Adang (AP, Haan 2001: 121)

\begin{flushleft}
\tablehead{}
\begin{supertabular}{m{0.48795983in}m{0.48445985in}m{0.32125986in}m{0.39135984in}m{0.31365985in}m{0.32125986in}m{0.69905984in}}
(25) &
Pen &
ti &
mat{\textepsilon} &
s{\textepsilon}l &
al{\textopeno} &
[241?]a-b{\textopeno}[241?]{\textopeno}i.\\
 &
John &
tree &
big &
\textsc{clf} &
two &
3\textsc{obv}{}-cut\\
\end{supertabular}
\end{flushleft}
{\textquoteleft}John cut the two big trees.{\textquoteright} 

Tobelo (NH) (NH, Holton 2003)

\begin{flushleft}
\tablehead{}
\begin{supertabular}{m{0.48795983in}m{0.58655983in}m{0.5893598in}m{0.93655986in}}
(26) &
Ngohi &
o-pine &
t-a-ija.\\
 &
\textsc{1sg} &
\textsc{nm-}rice &
\textsc{1sg-3sg}{}-buy\\
\end{supertabular}
\end{flushleft}
{\textquoteleft}I bought the rice.{\textquoteright} 

Mian (TNG, Fedden 2011: 344)

\begin{flushleft}
\tablehead{}
\begin{supertabular}{m{0.39005986in}m{0.32265985in}m{0.7650598in}m{1.7031599in}}
(27) &
N\'e &
imen-o &
wen-b-i=be.\\
 &
\textsc{1sg} &
taro-\textsc{nc.pl} &
eat-\textsc{ipfv-1sg.subj=decl}\\
\end{supertabular}
\end{flushleft}
{\textquoteleft}I am eating taro.{\textquoteright} 

Also like the NH languages and the TNG languages, the TAP languages have postpositions, as in the Bunaq example (28), where the locative postposition \textit{gene} follows its nominal complement \textit{reu} {\textquoteleft}house{\textquoteright}. 

Bunaq (Timor, Schapper 2009: 104)

\begin{flushleft}
\tablehead{}
\begin{supertabular}{m{0.36155984in}m{0.5462598in}m{0.47955987in}m{0.5663598in}m{0.6726598in}}
(28) &
neto &
reu &
gene &
mit\\
 &
\textsc{1sg} &
house &
\textsc{loc} &
sit\\
\end{supertabular}
\end{flushleft}
{\textquoteleft}I sit at home.{\textquoteright} 

In many TAP languages, however, the postpositions display verbal properties, as in (29), where the postposition/verb \textit{mi} {\textquoteleft}(be) in{\textquoteright} is modified by an aspectual marker.

Adang (AP, Robinson fieldnotes)

\begin{flushleft}
\tablehead{}
\begin{supertabular}{m{0.32545984in}m{0.39345986in}m{0.29205984in}m{0.36845985in}m{0.44485983in}m{0.37265986in}m{0.40595984in}}
(29) &
{\textglotstop}am{\textopeno} &
nu &
meja &
far &
mi &
eh.\\
 &
cat &
one &
table &
below &
be.in &
\textsc{prog}\\
\end{supertabular}
\end{flushleft}
{\textquoteleft}A cat is beneath a table.{\textquoteright} 

Another typically Papuan feature in East Nusantara languages is the presence of clause-final negation (Klamer et al. 2008). This feature is indeed found in TAP languages (30), though in NH languages the negator morpheme just follows the verb root rather than occurring in absolute final position (31).

Western Pantar (AP, Holton fieldnotes)

\begin{flushleft}
\tablehead{}
\begin{supertabular}{m{0.32545984in}m{0.59625983in}m{0.36845985in}m{0.24975985in}m{0.40185985in}m{0.5469598in}m{0.5087598in}}
(30) &
Gang &
ke{\textquoteright}e &
na &
wang &
yawang &
kauwa.\\
 &
\textsc{3sg.act} &
meat &
eat &
exist &
agree &
\textsc{neg}\\
\end{supertabular}
\end{flushleft}
{\textquoteleft}He doesn{\textquoteright}t like to eat meat.{\textquoteright} 

Tobelo (NH, Holton 2003)

\begin{flushleft}
\tablehead{}
\begin{supertabular}{m{0.32545984in}m{1.4629599in}}
(31) &
Wo-honenge-ua-ahi.\\
 &
\textsc{3sg.act}{}-die-\textsc{neg-ipfv}\\
\end{supertabular}
\end{flushleft}
{\textquoteleft}He is not yet dead.{\textquoteright} 

One notable syntactic feature absent from TAP is clause-chaining, which is one of the most distinctive features of Papuan languages in general and is particularly associated with TNG languages (Foley 1986: 175; Roberts 1997). Clause-chaining is also absent from NH languages. However, while clause-chaining may be one of the key distinguishing features of Papuan languages, it is important to note that this feature is completely absent from some TNG languages, such as Marind. 

{\centering
Table 6: Summary of TAP, TNG, and NH syntactic features
\par}

\begin{center}
\tablehead{}
\begin{supertabular}{m{1.3351599in}m{0.36015984in}m{0.38585985in}m{0.29205984in}}
 &
\centering TAP &
\centering TNG &
\centering\arraybslash NH\\
verb-final &
\centering [F0FC?] &
\centering [F0FC?] &
\centering\arraybslash [F0FC?]\\
postpositions &
\centering [F0FC?] &
\centering [F0FC?] &
\centering\arraybslash [F0FC?]\\
clause final negation &
\centering [F0FC?] &
\centering [F0FC?] &
\centering\arraybslash [F0FC?]\\
clause chaining &
\centering {}- &
\centering ([F0FC?]) &
\centering\arraybslash {}-\\
\end{supertabular}
\end{center}
In general, syntactic features do not distinguish the TAP languages from TNG or NH. While the TAP languages share a number of morphological and syntactic features with TNG and NH languages, these features are typologically common, may be interrelated (such as verb-final syntax and postpositions), and they may be indicative of a linguistic area (Klamer et al. 2008). We therefore do not find the typological evidence convincing of genealogical relationship. 

\section[Pronominal evidence]{Pronominal evidence}
\hypertarget{RefHeading72072871885726}{}When combined with other lines of evidence, homologous pronominal paradigms can provide strong support for proposals of genealogical relatedness. However, the use of pronominal paradigms as the \textit{sole }evidence for genealogical relatedness has been repeatedly questioned in the literature (cf. Campbell and Poser 2008). Pronominal paradigms were an important basis for the development of the Trans-New Guinea hypothesis (Wurm et al. 1975), and pronouns have continued to play a starring role in attempts to subgroup the TNG languages (Ross 2005, 2006).\footnote{ As originally formulated, the Trans New Guinea hypothesis linked Central and South New Guinea languages with the Finisterre-Huon languages based not on pronominal evidence but on lexical similarities (McElhanon and Voorhoeve 1970).} In this section we consider the strength of the pronominal evidence in evaluating the Trans-New Guinea and North Halmaheran hypotheses.

Since the full pronominal paradigm has not been reconstructed for pTAP, we consider the reconstructed pAP pronouns here. They are shown in Table 7, together with the pTNG (Ross 2005) and pNH (Wada 1980) pronouns. Note that North Halmaheran pronouns are reconstructed in two forms corresponding to actor ({\textquotedblleft}subject{\textquotedblright}) and undergoer ({\textquotedblleft}object{\textquotedblright}).

{\centering
Table 7: pAP, pTNG, and pNH pronouns
\par}

\begin{center}
\tablehead{}
\begin{supertabular}{|m{0.6941598in}|m{0.47885987in}|m{0.7372598in}|m{0.8525598in}|m{0.021259844in}|}
\hline
 &
\centering pAP &
\centering pTNG &
\multicolumn{2}{m{0.95255977in}|}{\centering pNH}\\\hline
 &
 &
 &
\centering \textsc{act} &
\centering\arraybslash \textsc{und}\\\hline
\textsc{1sg} &
\centering *na- &
\centering *na &
\centering *to- &
\centering\arraybslash *si-\\\hline
\textsc{2sg} &
\centering *(h)a- &
\centering *{\ng}ga &
\centering *no- &
\centering\arraybslash ni-\\\hline
\textsc{3sg} &
\centering *ga- &
\centering *ua, *(j)a &
\centering *mo- (\textsc{fem})\par

\centering *wo- (\textsc{mas})\par

\centering *i- (\textsc{neu}) &
\centering *mi- (\textsc{fem})\par

\centering *wi- (\textsc{mas})\par

\centering\arraybslash *ya- (\textsc{neu})\\\hline
\textsc{1pl.incl} &
\centering *pi- &
\centering *nu, *ni &
\centering *po- &
\centering\arraybslash *na-\\\hline
\textsc{1pl.excl} &
\centering *ni- &
 &
\centering *mi- &
\centering\arraybslash *mi-\\\hline
\textsc{1distr} &
\centering *ta- &
\centering {}- &
\centering {}- &
\centering\arraybslash {}-\\\hline
\textsc{2pl} &
\centering *(h)i- &
\centering *nja, *{\ng}gi &
\centering *ni- &
\centering\arraybslash *ni-\\\hline
\textsc{3pl} &
\centering *gi- &
\centering *i &
\centering *jo- &
\centering\arraybslash *ja-\\\hline
\end{supertabular}
\end{center}
Several structural differences are noticeable between these pronoun sets. First, AP and NH show an inclusive/exclusive distinction in first person plural which is not found in TNG. This has been argued to be an areal feature resulting from Austronesian influence (Klamer et al. 2008). Second, NH but not AP or TNG distinguish gender in third person pronouns. Third, a distributive pronoun is found only in AP. 

We consider first the TNG pronouns. The pTNG pronominal reconstructions provide what some consider to be the strongest support for the genealogical connection between AP and TNG (Ross 2005). Both pTNG and pAP show a paradigmatic distinction between \textit{a }in the singular and \textit{i }in the plural. However, the correspondence is problematic due to the mismatch between the second and third person pronouns. Proto-TNG shows velar consonants in the second person forms, while pAP shows velar consonants in the third person forms. It has been suggested that the pTNG second person pronouns could have developed into the pAP second person pronouns by lenition of pTNG *{\ng}g {\textgreater} *g {\textgreater} *k {\textgreater} h. While this is possible, we find stronger evidence that the pTNG prenasalized obstruents should correspond to the pAP voiced stops (see section 4.2), if indeed the two are related at all. 

Another possible scenario connecting these two paradigms is to posit a flip-flop between the second and third person pronouns, as in (32). As far as we are aware, such an inversion scenario was first proposed by Donohue and Schapper (2007).

(32)\ \ Putative flip-flop between second and third person pronouns

pTNG *{\ng}ga {\textquoteleft}\textsc{2sg}{\textquoteright} {\textgreater} pAP *ga- {\textquoteleft}\textsc{3sg}{\textquoteright}, pTNG *{\ng}gi {\textquoteleft}\textsc{2pl}{\textquoteright} {\textgreater} pAP *gi- {\textquoteleft}\textsc{3pl}{\textquoteright}

pTNG *(y)a {\textquoteleft}\textsc{3sg}{\textquoteright} {\textgreater} pAP *(h)a- {\textquoteleft}\textsc{2sg}{\textquoteright}, pTNG *i {\textquoteleft}\textsc{3pl}{\textquoteright} {\textgreater} pAP *(h)i- {\textquoteleft}\textsc{2pl}{\textquoteright}

This leaves only the fricative in the pAP second person forms unexplained, but external evidence from the Timor languages suggests that perhaps the pAP second person forms should be vowel initial (i.e., pAP *a {\textquoteleft}\textsc{2sg{\textquoteright}} and *i {\textquoteleft}\textsc{2pl{\textquoteright}}). While it is not impossible that the pAP pronouns descend from the pTNG pronouns in this way, connecting the two requires us to posit a flip which makes the correspondence much less striking.  

The putative correspondence between the pAP and pTNG pronouns leaves at least one AP form unexplained: the AP distributive *ta- has no correspondent form in TNG. Donohue (2008) posits a connection between the AP distributive and the pNH first-singular active form *to-. According to this hypothesis the resemblance between the AP distributive and the pNH first-singular active is evidence not of a genealogical relationship but rather a borrowing relationship within a contact area encompassing the Bomberai Peninsula and South Bird{\textquoteright}s Head region. The semantic plausibility of this connection is based on an analysis of *ta- as the minimal 1/2-person pronoun in a minimal-augmented system (Donohue 2007b). However, the augmented counterpart is filled anomalously by *pi-, rather than the expected *ti-, though pAP *pi- does show striking semantic and structural similarity with pNH first person inclusive *po-. Yet in the modern Alor-Pantar languages, reflexes of *ta-, where they exist, have a clear 
distributive function. For example, compare the Adang first person plural inclusive (33a) with the distributive (33b).

Adang distributive (AP, Haan 2001)

\begin{flushleft}
\tablehead{}
\begin{supertabular}{m{0.48795983in}m{0.41365984in}m{0.41365984in}m{1.0073599in}m{0.35455984in}}
(33) &
a. &
Sa &
pi-ri &
b{\textepsilon}h.\\
 &
 &
\textsc{3sg} &
\textsc{1pl.incl-acc} &
hit\\
\end{supertabular}
\end{flushleft}
\ \ {\textquoteleft}She hit (all of) us.{\textquoteright} 

\begin{flushleft}
\tablehead{}
\begin{supertabular}{m{0.5872598in}m{0.31435984in}m{0.41365984in}m{0.8066598in}m{0.35385984in}}
 &
b. &
Sa &
ta-ri &
b{\textepsilon}h.\\
 &
 &
\textsc{3sg} &
\textsc{distr-acc} &
hit\\
\end{supertabular}
\end{flushleft}
\ \ {\textquoteleft}She hit each one of us.{\textquoteright} 

The distributive function is expressed quite differently in NH languages. In Tobelo the distributive is expressed with the verb prefix \textit{koki-} (34) rather than with a pronoun.

Tobelo distributive (NH, Holton 2003)

\begin{flushleft}
\tablehead{}
\begin{supertabular}{m{0.32545984in}m{0.7247598in}m{1.3198599in}}
(34) &
ma-homoa &
yo-koki-honeng-oka \\
 &
\textsc{nm}{}-other &
\textsc{3pl-distr}{}-die-\textsc{perf}\\
\end{supertabular}
\end{flushleft}
{\textquoteleft}Each of the others died.{\textquoteright} 

The AP distributive prefix is extra-paradigmatic: it does not show the vowel grading found in the other prefixes; and  related independent pronouns are either absent or of limited distribution. This suggests that the pAP distributive has a distinct history from that of the other pAP pronominal forms, and that the resemblance between pNH *to {\textquoteleft}\textsc{1sg}{\textquoteright} and pAP *ta {\textquoteleft}\textsc{1pl.dist}{\textquoteright} is coincidental. 

The structural features of the pronominal systems are compared in Table 8. It is apparent that the AP pronominal system as a whole has relatively little in common with TNG and NH.

{\centering
Table 8: Summary of AP, TNG, and NH pronominal features
\par}

\begin{center}
\tablehead{}
\begin{supertabular}{m{1.9254599in}m{0.26705986in}m{0.38585985in}m{0.29205984in}}
 &
\centering AP &
\centering TNG\par

 &
\centering\arraybslash NH\\
[a] singular, [i] plural &
\centering [F0FC?] &
\centering [F0FC?] &
\centering\arraybslash {}-\\
distributive pronoun &
\centering [F0FC?] &
\centering {}- &
\centering\arraybslash {}-\\
inclusive/exclusive distinction &
\centering [F0FC?] &
\centering {}- &
\centering\arraybslash [F0FC?]\\
gender distinction &
\centering {}- &
\centering {}- &
\centering\arraybslash [F0FC?]\\
\end{supertabular}
\end{center}
Given the rather speculative nature of the second-third person inversion hypothesis, the pronominal evidence does not provide very strong support for either the TNG or NH hypothesis. Nevertheless, the formal correspondence in first-person forms between AP and TNG provide tentative support for a connection between TAP and TNG.

\section[Lexicon]{Lexicon}
\hypertarget{RefHeading72074871885726}{}When combined with evidence from morphological paradigms, such as pronouns, lexical evidence based on regular sound correspondences is usually considered to be compelling evidence for positing genealogical relationships between languages. Unfortunately, very little in the way of lexical evidence had been previously considered in assessing the wider genealogical relationships of the TAP languages before Robinson and Holton 2012. We consider first the lexical evidence for the NH hypothesis and then the lexical evidence for the TNG hypothesis.

\subsection[Lexical evidence for the NH hypothesis]{Lexical evidence for the NH hypothesis}
\hypertarget{RefHeading72076871885726}{}The lexical evidence for a connection between TAP and NH languages is not particularly convincing. In a list of 92 basic vocabulary terms, Capell identifies 11 which seem to show {\textquotedblleft}common roots{\textquotedblright} with AP languages (1975: 685). Capell did not include data from Pantar languages and hence refers to this family as Alor-Timor. In many cases Capell{\textquoteright}s proposed Alor-Timor forms differ from the pTAP reconstructions in chapter 3. This may be due in some cases to excessive reliance on Timor forms. In Table 9 we list Capell{\textquoteright}s Alor-Timor alongside updated pTAP forms. Where available, we use pTAP reconstructions (chapter 3), but if no pTAP reconstruction exists, then we show lower-level reconstructions or forms from individual languages. In two cases Capell{\textquoteright}s {\textquoteleft}Alor-Timor{\textquoteright} form is quite different from the updated TAP form. Capell{\textquoteright}s \textit{hele} {\
textquoteleft}stone{\textquoteright} differs from pTAP *war but compares to Bunaq (Timor) \textit{hol. }We have no reconstruction for {\textquoteleft}cut{\textquoteright} in pTAP, but Capell{\textquoteright}s form \textit{uti }compares with Makalero (Timor) \textit{teri. }Three of Capell{\textquoteright}s NH reconstructions are also problematic; we have noted these problems in the last column in Table 9. Capell{\textquoteright}s NH *utu\textit{ }{\textquoteleft}fire{\textquoteright} should clearly be *uku, perhaps a typographical error. Capell{\textquoteright}s *helewo {\textquoteleft}stone{\textquoteright} is found in Tobelo but does not reconstruct to NH. We are not able to identify Capell{\textquoteright}s *hate {\textquoteleft}tree{\textquoteright}; the form *gota reconstructs for the family.

{\centering
Table 9: Comparison of Capell{\textquoteright}s TAP and NH, with modern TAP and NH reassessments\footnote{ Capell was not originally aware of the Pantar languages and so referred to TAP as {\textquotedblleft}Alor-Timor{\textquotedblright}.}
\par}

\begin{center}
\tablehead{}
\begin{supertabular}{m{0.6011598in}m{1.0608599in}m{1.4108598in}m{0.8309598in}m{1.1712599in}}
 &
TAP (Capell) &
TAP (revised) &
NH (Capell) &
NH (revised)\\
{\textquoteleft}bitter{\textquoteright} &
malara &
proto-Alor (but not pAP or pTAP) *makal &
*mali &
\\
{\textquoteleft}cold{\textquoteright} &
palata &
Abui, Kui \textit{palata} &
*malata &
\\
{\textquoteleft}cry out{\textquoteright} &
(k)ole &
Nedebang \textit{uwara}, Sawila \textit{kawa}, Makasae \textit{kaul }{\textquoteleft}sing{\textquoteright} &
*orehe &
\\
{\textquoteleft}cut{\textquoteright} &
uti &
Makalero \textit{teri} &
*{\ng}uki &
\\
{\textquoteleft}fall{\textquoteright} &
tapa &
Western Pantar \textit{tasing}, Sawila \textit{taani} &
*tiwa &
\\
{\textquoteleft}fire{\textquoteright} &
ata &
pTAP *hada &
*utu  &
*uku\\
{\textquoteleft}flower{\textquoteright} &
buk &
Blagar \textit{buma}, Klon \textit{b}\textit{{\textupsilon}}\textit{:m}, Kui \textit{bungan}, Makasae \textit{puhu}, Makakero, Bunaq \textit{buk}  &
*hohoko &
\\
{\textquoteleft}fly (n.){\textquoteright} &
uhur(u) &
Kaera \textit{ubar}, Makalero \textit{uful}, Makasae \textit{ufulae}, Fataluku \textit{upuru}, Oirata \textit{uhur} &
*guhuru &
\\
{\textquoteleft}smell{\textquoteright} &
{\textglotstop}amuhu &
Teiwa \textit{min}, Kaera \textit{mim-, }Nedebang \textit{mini}, Blagar \textit{miming}, Adang \textit{muning}, Klon \textit{moin}, Kui \textit{mun}, Wersing \textit{muing}, Makasae \textit{amuh}, Makalero \textit{kamuhata}, also pTAP *-mVN {\textquoteleft}nose{\textquoteright} &
*ami &
\\
{\textquoteleft}stone{\textquoteright} &
hele &
pTAP *war &
*helewo &
Galela \textit{teto}, Tabaru \textit{madi}\\
{\textquoteleft}tree{\textquoteright} &
ate &
pTAP *hate &
*hate &
*gota\\
\end{supertabular}
\end{center}
Even allowing for problematic forms in Table 9, it is difficult to infer much about regular sound correspondences from this list, since few of the correspondences repeat. A correspondence *m:*m is found in {\textquoteleft}bitter{\textquoteright} and {\textquoteleft}smell{\textquoteright}; however, the forms for {\textquoteleft}cold{\textquoteright} reflect a different correspondence *p:*m. Careful inspection of Capell{\textquoteright}s proposed correspondence reveals little or no evidence for a relationship between TAP and NH languages. 

Donohue (2008) lists two proposed lexical correspondences between pTAP and pNH. One of these, {\textquoteleft}tree{\textquoteright}, is also found in Capell{\textquoteright}s list, though Donohue reconstructs pTAP *aDa. The other, pTAP *jar, pNH *aker {\textquoteleft}water{\textquoteright} supports a correspondence between pTAP *r and pNH *r.\footnote{ \textstyleFootnoteTextChar{Donohue actually cites the form *gala as the reconstruction for pNH {\textquoteleft}water{\textquoteright}, rather than Wada{\textquoteright}s *aker. Moreover, the updated pTAP reconstruction for {\textquoteleft}water{\textquoteright} is *jira (see chapter 3), not *jar.} }\textsuperscript{ } As with Capell{\textquoteright}s similar forms, it is difficult to infer anything about sound correspondences from these two forms. Chance resemblance remains the most economical explanation, though some similarities may also be due to loans from a common source.

The lack of lexical correspondences in the data cited by Capell and by Donohue may be due in part to the unavailability of extensive lexical data for TAP. Thanks to recent work, we now have available a number of pTAP and lower-level reconstructions (see chapters 2 and 3, and Schapper et al. 2012). Examining the pTAP reconstructions (excluding pronouns), and drawing on pAP forms where no pTAP form is found, 63 have glosses which can also be found in Wada{\textquoteright}s (1980) pNH reconstructions or can be easily reconstructed based on existing NH data. These 63 forms are compared in Table 10. 

{\centering
Table 10: pNH forms (after Wada 1980) with TAP equivalents (after Schapper et al., this volume). A double dagger {\ddag} indicates a pNH form which is not in Wada or a pAP form which is not reconstructed at the level of pTAP.\footnote{ In the pTAP / pAP reconstructions, V stands for an unidentified vowel, and N stands for an unidentified nasal. The other reconstructed consonants have their values as laid out in Table 1. The vowels, while very tentative, are assumed to have their IPA values. }
\par}

\begin{center}
\tablehead{}
\begin{supertabular}{m{3.12756in}m{3.12756in}}
\textbf{pNH}

\textbf{pTAP}

take, hold

*aho

*p(i,u)nV {\ddag}

water

*aker

*jira

blood

*aun

*waj

tail

*bikin

*-o(l,r)a\footnotemark{}

come

*bola

*mai {\ddag}

banana

*bole{\ddag}

*mugul

six

*buta{\ng}a

*talam

smoke

*[1E0B?]opo

*bunaq {\ddag}

louse/flea

*gani

*kVt {\ddag}

salt/saltwater

*gasi

*tam(a)

hand

*giam

*-tan(a)

nail

*gitipir

*kusin {\ddag}

sit

*goger

*mit

bite

*goli

*ki(l)

tree

*gota

*hate

give

*hike

*-(e,i)na

laugh

*hijete

*jagir 

village

*hoana{\ddag}

*haban {\ddag}

spit

*hobir

*pu(l,r)V(n)

coconut

*igono{\ddag}

*wata

tooth

*i{\ng}ir

*-wasin

spear

*kamanu

*qaba(k) {\ddag}

thick

*kipirin

*dumV{\ddag}

tongue

*akir

*-lebu(l,r)

bat

*mano

{\ddag}

*madel

moon

*mede

*hur(u)

ten 

*mogiowok

*qar- {\ddag}

one

*moi

*nukV

betel nut

*mokoro{\ddag}

*bui {\ddag}

five

*motoha

*jiwesin {\ddag}

bird

*namo

*(h)adul

 &
\begin{flushleft}
\begin{tabular}{m{0.69835985in}m{-0.0134401545in}m{0.76435983in}m{-0.031540155in}m{0.7170598in}}
\multicolumn{2}{m{0.76365983in}}{} &
\textbf{pNH} &
\multicolumn{2}{m{0.7642598in}}{\textbf{pTAP}}\\
dream &
\multicolumn{3}{m{0.87685984in}}{*naner{\ddag}} &
*(h)ipar\\
fish &
\multicolumn{3}{m{0.87685984in}}{*nawok} &
*habi\\
ear &
\multicolumn{3}{m{0.87685984in}}{*{\ng}auk} &
*-wa(l,r)i\\
sea &
\multicolumn{3}{m{0.87685984in}}{*{\ng}olot} &
*tam(a)\\
star &
\multicolumn{3}{m{0.87685984in}}{*{\ng}oma} &
*jib(V)\\
child &
\multicolumn{3}{m{0.87685984in}}{*{\ng}opak} &
*-uaqal\footnotemark{}\\
nose &
\multicolumn{3}{m{0.87685984in}}{*{\ng}unu{\ng}} &
*-mVN\\
eat &
\multicolumn{3}{m{0.87685984in}}{*o[1E0B?]om} &
*nVa\\
bathe &
\multicolumn{3}{m{0.87685984in}}{*ohik{\ddag}} &
*we(l,r)i\\
stand &
\multicolumn{3}{m{0.87685984in}}{*oko} &
*nat(er)\\
they &
\multicolumn{3}{m{0.87685984in}}{*ona, yo} &
*gi- {\ddag}\\
belly &
\multicolumn{3}{m{0.87685984in}}{*pokor} &
*-tok {\ddag}\\
knee &
\multicolumn{3}{m{0.87685984in}}{*puku} &
*uku {\ddag}\\
name &
\multicolumn{3}{m{0.87685984in}}{*ro{\ng}a} &
*-en(i,u) {\ddag}\\
fat, grease &
\multicolumn{3}{m{0.87685984in}}{*saki} &
*tama {\ddag}\\
throw &
\multicolumn{3}{m{0.87685984in}}{*sariwi} &
*od {\ddag}\\
two &
\multicolumn{3}{m{0.87685984in}}{*sinoto} &
*araqu {\ddag}\\
die  &
\multicolumn{3}{m{0.87685984in}}{*sone{\ng}} &
*mV(n)\\
fruit &
\multicolumn{3}{m{0.87685984in}}{*sopok} &
*is(i) {\ddag}\\
burn &
\multicolumn{3}{m{0.87685984in}}{*sora, so{\ng}ara} &
*ede {\ddag}\\
fly (v.) &
\multicolumn{3}{m{0.87685984in}}{*sosor} &
*jira(n) {\ddag}\\
black &
\multicolumn{3}{m{0.87685984in}}{*tarom} &
*aqana {\ddag}\\
stone &
\multicolumn{3}{m{0.87685984in}}{*teto} &
*war\\
short &
\multicolumn{3}{m{0.87685984in}}{*timisi} &
*tukV {\ddag}\\
pierce &
\multicolumn{3}{m{0.87685984in}}{*topok} &
*tapa(i)\\
bad &
\multicolumn{3}{m{0.87685984in}}{*torou} &
*jasi {\ddag}\\
drink &
\multicolumn{3}{m{0.87685984in}}{*u[1E0B?]om} &
*nVa\\
fire &
\multicolumn{3}{m{0.87685984in}}{*uku} &
*hada\\
he &
\multicolumn{3}{m{0.87685984in}}{*una, wo} &
*ga- {\ddag}\\
sun &
\multicolumn{3}{m{0.87685984in}}{*wa{\ng}e} &
*wad(i,u)\\
\end{tabular}
\end{flushleft}
\\
\end{supertabular}
\end{center}
\addtocounter{footnote}{-2}
\stepcounter{footnote}\footnotetext{ As mentioned in Footnote 4, Schapper et al. (chapter 3), reconstruct three liquids: *l, *r, and *R based on three correspondence sets. Since none of the modern TAP languages has three liquids, we assume that *R was actually *l or *r, with some as yet to be identified conditioning, and we have therefore modified the relevant reconstructions to reflect this.}
\stepcounter{footnote}\footnotetext{ Schapper et al. (chapter 3) reconstruct pTAP *uaQal, where *Q is {\textquotedblleft}\emph{\textmd{\textup{a putative postvelar stop for which we have only very weak evidence{\textquotedblright}. We prefer to render this as *uaqal, showing more transparently the value we believe this consonant would have had. }}}}
Of these 61 forms, only 5 items (highlighted grey in the table) show some kind of plausible correspondence: *b:*m, *t:*t, and *k:*q. Again, with so few items it is impossible to infer anything about regular sound correspondences. And with only 8\% of these basic vocabulary items showing potential cognacy, there is no clear lexical evidence for a genealogical connection between TAP and NH languages.

\subsection[Lexical evidence for the TNG hypothesis]{Lexical evidence for the TNG hypothesis}
\hypertarget{RefHeading72078871885726}{}In this section we consider the lexical evidence for the TNG hypothesis as reflected in regular sound correspondences. For this purpose we use the rather broad formulation of TNG in Pawley (2005) and Ross (2005), which includes both TAP and South Bird{\textquoteright}s Head. While no bottom-up reconstruction of proto-TNG has been completed, a set of top-down lexical reconstructions with extensive reflexes has been widely circulated as Pawley (n.d.). Some of these forms were included as support for the reconstruction of pTNG obstruents (Pawley 2001) and in other discussions of pTNG (Pawley 1998, 2012). We are not in a position here to assess the validity or quality of Pawley{\textquoteright}s reconstructions. Rather, our intent is to assess the lexical evidence for a connection between TAP and TNG based on the available data. In contrast to the NH data, the pTNG lexicon shows more striking correspondences with TAP languages. Pawley (n.d.) proposes 21 pTNG 
reconstructions with putative TAP reflexes, out of approximately 180 pTNG reconstructions. Of those, thirteen (shown in (33)-(47) below) appear to exhibit regular sound correspondences. Examples (33) through (38) are reconstructed to pTAP. In (33), the reconstructed pTNG form encompasses the meanings {\textquoteleft}tree{\textquoteright}, {\textquoteleft}wood{\textquoteright}, and {\textquoteleft}fire{\textquoteright}, but in the TAP languages, only the latter two meanings are found. There is a separate reconstruction for {\textquoteleft}tree{\textquoteright} in pTAP. 

(35)\ \ pTNG *inda {\textquoteleft}tree, wood, fire{\textquoteright}, pTAP *hada {\textquoteleft}fire, wood{\textquoteright}

(36)\ \ pTNG *panV {\textquoteleft}woman{\textquoteright}, pTAP *pan(a) {\textquoteleft}girl{\textquoteright}

(37)\ \ pTNG *amu, pTAP *hami {\textquoteleft}breast{\textquoteright}

(38)\ \ pTNG *na-, pTAP *nVa {\textquoteleft}eat, drink{\textquoteright}

(39)\ \ pTNG *kumV, pTAP *mV(n) {\textquoteleft}die{\textquoteright} (cf., pTim *-umV )

(40)\ \ pTNG *ata, pTAP *(h)at(V) {\textquoteleft}excrement{\textquoteright}

Examples (39) through (43) are found in a number of languages in both AP and Timor but have not yet been reconstructed to pTAP. Note that pTNG *L is probably a laterally released velar stop, so pharyngeal and velar fricatives would not be strange reflexes. 

(41)\ \ pTNG *maL[a], Teiwa (AP) \textit{mo[127?]o{\textglotstop}}, Kaera (AP) \textit{maxa}, Klon (AP) \textit{m[1DD?]k{\textepsilon}{\textglotstop}}, pTim *muka {\textquoteleft}ground, earth{\textquoteright}\footnote{ This pTim form is from Schapper et al. (2012). It does not appear in chapter 3. }

(42)\ \ pTNG *gatata , Blagar (AP) \textit{tata}, Adang (AP) \textit{ta}\textit{[241?]}\textit{ata}, Klon (AP) \textit{t}\textit{[1DD?]}\textit{kat}, Kui (AP) \textit{takata}, Abui (AP) \textit{takata} Fataluku (Tim), Oirata (Tim) \textit{tata} {\textquoteleft}dry{\textquoteright}

(43)\ \ pTNG *ini, Blagar (AP), Adang (AP) \textit{e{\ng}}, Klon (AP), Kui (AP) \textit{{}-en}, Abui (AP) \textit{{}-ei{\ng}}, Kamang (AP) \textit{{\ng}}, Fataluku (Tim) \textit{ina}, Makalero (Tim) \textit{ina}, Oirata (Tim) \textit{ina} {\textquoteleft}eye{\textquoteright}

Examples (42) through (45) are found in just one of the two main branches of TAP.

(44)\ \ pTNG *tukumba(C), pAP *tukV {\textquoteleft}short{\textquoteright}

(45)\ \ pTNG *mundu {\textquoteleft}internal organ{\textquoteright}, Oirata (Tim) \textit{mu{\textrtailt}u }{\textquoteleft}inside{\textquoteright}, Makalero \textit{mutu} {\textquoteleft}inside{\textquoteright}, Fataluku \textit{mucu }{\textquoteleft}inside{\textquoteright},\textit{ }Makasae (Tim) \textit{mutu }{\textquoteleft}in{\textquoteright}

(46)\ \ pTNG *sasak, Oriata (Tim) \textit{asah(a)}, Makasae (Tim), Fataluku (Tim) \textit{asa}, Makalero (Tim) \textit{hasa }{\textquoteleft}leaf{\textquoteright} 

(47)\ \ pTNG *kitu {\textquoteleft}leg (possibly {\textquoteleft}calf{\textquoteright}), Bunaq (Tim) \textit{{}-iri}, Makasae (Tim) \textit{{}-iti} {\textquoteleft}leg{\textquoteright}

The correspondences which emerge from this set are not striking, but they are regular. Most interesting is the correspondence between the pTNG prenasalized stop and the pTAP voiced stop. Note that a correspondence between a prenasalized stop in pTNG and a voiced stop in pTAP (also a voiced stop in pAP) supports a hypothesis that pAP reflects a flip of the pTNG second person pronouns *{\ng}ga {\textquoteleft}\textsc{2sg}{\textquoteright}, *{\ng}gi {\textquoteleft}\textsc{2pl}{\textquoteright} to pAP third person pronouns *ga {\textquoteleft}\textsc{3sg}{\textquoteright}, *gi {\textquoteleft}\textsc{3pl}{\textquoteright}, respectively, although the correspondence here is velar rather than the expected alveolar, as in Table 11.

{\centering
Table 11: pTNG and pTAP sound correspondences
\par}

\begin{center}
\tablehead{}
\begin{supertabular}{m{0.46155986in}m{0.46985987in}m{2.13866in}}
pTNG &
pTAP &
examples\\
*t &
*t &
dry, short, leg, excrement\\
*k &
*k &
die, leg, short, leaf\\
*nd &
*d &
internal organ, fire\\
*n &
*n &
eat, eye, woman, \textsc{1sg, 1pl}\\
*m &
*m &
die, ground, internal organ, breast\\
{\O} &
*h &
fire, breast, excrement\\
\end{supertabular}
\end{center}
Two more forms might be included in the thirteen above, but they are somewhat problematic. The correspondence of {\textquoteleft}neck{\textquoteright} is based on two nasal phonemes and reflexes in just three of the nearly thirty TAP languages. 

(48)\ \ pTNG *kuma(n,{\ng})[V] (first syllable lost in some cases), Sawila (AP) \textit{{}-ma{\ng}}, Oirata (Tim), Fataluku (Tim) \textit{mani} {\textquoteleft}neck{\textquoteright}  

The form for {\textquoteleft}lightning{\textquoteright} likewise has a very limited distribution, with similar-looking forms occurring in just three closely related AP languages. Moreover, the vowels in the pTNG reconstruction were determined in part on the basis of the Blagar, possibly making the pTNG artificially more similar to the AP languages than otherwise warranted.

(49)\ \ pTNG *(mb, m)elak, Blagar (AP) \textit{merax}, Retta (AP) \textit{melak}, Kabola (AP) \textit{mere}\textit{[241?]}, {\textquoteleft}lightning{\textquoteright} 

The pTNG form for {\textquoteleft}older sibling{\textquoteright} shows a striking correspondence with TAP languages, but this is a nursery form, and should be excluded from determinations of genealogical similarity. 

(50)\ \ pTNG *nan(a,i), pAP *nan(a), Bunaq (Tim) \textit{nana} {\textquoteleft}older sibling{\textquoteright}

The pTNG form for {\textquoteleft}to come{\textquoteright} is also strikingly similar to the pAP, but the pAP form may have its origins in Proto-Malayo Polynesian *maRi, which is irregularly reflected as \textit{ma }or \textit{mai }in many Austronesian languages in the region, for example Mambai (Timor) \textit{ma}, Manggarai (Flores) \textit{mai}. 

(51)\ \ pTNG *me-, pAP *mai {\textquoteleft}to come{\textquoteright}

A further four forms were excluded because their correspondences were not regular. The form for {\textquoteleft}nose{\textquoteright} looks promising, but pTNG *nd should correspond with pTAP *d, not a nasal.

(52)\ \ pTNG *mundu, pTAP *-mVN {\textquoteleft}nose{\textquoteright} 

The pTNG reconstruction *wani {\textquoteleft}who{\textquoteright} looks similar to the Abui form \textit{hanin} that was cited in Pawley (n.d.), but more recent research on Abui shows that {\textquoteleft}who{\textquoteright} is \textit{maa}, and we know of  no word \textit{hanin} in Abui. The AP languages Adang, Hamap, and Kabola, all quite closely related, show somewhat similar forms, but the lack of correspondence in the initial consonants, combined with the limited geographic distribution, make these unlikely cognates. 

(53)\ \ pTNG *wani, Adang (AP) \textit{ano}, Hamap (AP) \textit{hano}, Kabola (AP) \textit{hanado} {\textquoteleft}who{\textquoteright}

A further two proposed cognates are simply not very similar in form to their putative TAP reflexes. The pTNG form *pululu {\textquoteleft}fly, flutter{\textquoteright} was originally considered cognate with Blagar (AP) \textit{iriri}, \textit{alili}, but our data show Blagar \textit{liri}, and other cognates point to proto-Alor *liri. The competing form pAP *jira(n) has a wider distribution and is therefore reconstructed to pAP. Proto-Timor *lore suggests that Alor-Pantar *liri is older than previously assumed, but at any rate, the initial consonant from pTNG is only found in one TAP language (Fataluku (Tim) \textit{ipile}). It seems much more likely that the resemblance between pTNG and the TAP languages is due to onomatopoeia. 

(54)\ \ pTNG *pululu {\textquoteleft}fly, flutter{\textquoteright}, Blagar (AP) \textit{liri}, Adang (AP) \textit{lili}\textit{[241?]}, Klon (AP) \textit{liir}, Kui (AP) \textit{lir}, Abui (AP) \textit{li}\textit{[241?]}, Kamang (AP) \textit{lila}, pTim *lore {\textquoteleft}to fly{\textquoteright}\footnote{ Though note Makalero \textit{uful}, Makasae \textit{ufulae}, Fataluku \textit{upuru}, and Oirata \textit{uhur} {\textquoteleft}fly (n.){\textquoteright}. }

Likewise, further data on pTNG reconstructions for {\textquoteleft}urine{\textquoteright} cast doubt on the purported cognacy with TAP languages. The pTNG *[si]si, *siti, *pisi {\textquoteleft}urine{\textquoteright} was originally considered cognate with Oirata (Tim) \textit{iri} {\textquoteleft}urine, excrement{\textquoteright}. The forms in the AP languages seem to be doublets with {\textquoteleft}water{\textquoteright}, which is reconstructed as pTAP *jira. Although we have not established TAP correspondences for pTNG *s, there is insufficient formal similarity between the two reconstructions to retain them as cognate sets. 

(55)\ \ pTNG *[si]si, *siti, *pisi {\textquoteleft}urine{\textquoteright}, Western Pantar (AP) \textit{jir}, Blagar (AP) \textit{ir}, Klon (AP) \textit{wri}, Retta (AP) \textit{vil}, Sawila (AP) \textit{iripi{\ng}} {\textquoteleft}urine{\textquoteright}, Makalero \textit{irih }{\textquoteleft}urinate{\textquoteright}, Makasae \textit{iri }{\textquoteleft}urine{\textquoteright}, Oirata (Tim) \textit{iri} {\textquoteleft}urine, excrement{\textquoteright}

In terms of lexicon, then, we are left with thirteen potential pTNG - TAP cognates and a few tentative sound correspondences (Table 11). 

\section[Comparison with individual languages]{Comparison with individual languages}
\hypertarget{RefHeading72080871885726}{}In the preceding section we examined evidence for a connection between TAP and TNG drawing on data from a top-down reconstruction of pTNG. Given that Pawley{\textquoteright}s putative TNG contains some five hundred languages, and that little historical reconstruction work has been done for lower level subgroups, pTNG reconstructions must be considered tentative (though some reconstructed forms are more secure than others). Hence, it is useful also to examine potential relationships of TAP directly with lower level subgroups. We focus here on four such families. The first, South Bird{\textquoteright}s Head (SBH), is not actually included in Pawley{\textquoteright}s TNG but was included in Wurm{\textquoteright}s (1982) previous formulation of TNG. This classification is detailed in Voorhoeve (1975), who along with Stokhof (1975) argues for a somewhat distant ({\textquotedblleft}subphylic{\textquotedblright}) connection between TAP and SBH.  

The other three families considered here are all classified within Pawley{\textquoteright}s TNG. The Dani and Wissel Lakes families were part of the original core group of TNG languages proposed by Wurm et al. (1975). Their membership in TNG is likely quite secure. The other TNG family considered here is West Bomberai. Like SBH, West Bomberai was originally classified by Cowan (1957) as part of the West Papuan Phylum, but it was later reclassified as TNG and included as such by Pawley. Ross (2005) also includes West Bomberai within TNG based on pronominal evidence. In fact, Ross proposes a {\textquotedblleft}West Trans-New Guinea linkage{\textquotedblright} within TNG consisting of West Bomberai, Dani, Wissel Lakes, and TAP. All of these languages, including the Timor languages (but notably excluding Alor-Pantar) share an innovation whereby the pTNG first singular pronoun *na is replaced by \textit{ani}. Ross (2005: 37) also notes that the TAP languages share with West Bomberai an innovative first-person 
plural form *bi (though this is an inclusive pronoun in TAP but an exclusive pronoun in West Bomberai). 

In the following sub-sections we compare TAP languages to each of these four families in turn, while remaining agnostic as to the status of TAP vis-\`a-vis TNG. Since we lack robust reconstructions at the level of any of these families, we instead compare pTAP reconstructions (see chapter 3) to selected individual languages from each of these families.

\subsection[South Bird{\textquoteright}s Head]{South Bird{\textquoteright}s Head}
\hypertarget{RefHeading72082871885726}{}The South Bird{\textquoteright}s Head family is here represented by Inanwatan (ISO 639-3 szp) and Kokoda (ISO 639-3 xod). The Inanwatan pronouns are given in Table 12 (with pAP for comparison). Like the pAP and pTNG pronoun sets, these show /a/ in the singulars and /i/ in the plurals, although the Inanwatan third person singular does not follow this pattern. These are similar to the pAP pronouns in reflecting *na {\textquoteleft}\textsc{1sg}{\textquoteright} instead of *an. As in the TAP languages, the pTNG first person plural pronoun *ni (if indeed Inanwatan is a TNG language) has been assigned to the exclusive, and a new form has been innovated for the inclusive. The inclusive form in Inanwatan, however, is not cognate with the inclusive in pAP. Inanwatan is also different from TAP languages in distinguishing between masculine and feminine in the third person singular. 

{\centering
Table 12: Inanwatan pronouns (de Vries 2004: 27-29)
\par}

\begin{center}
\tablehead{}
\begin{supertabular}{|m{0.6941598in}|m{0.6733598in}|m{0.7754598in}|m{0.47885987in}|}
\hline
 &
subject &
possessive

prefix &
pAP\\\hline
\textsc{1sg} &
n\'aiti/n\'ari &
na- &
*na-\\\hline
\textsc{2sg} &
\'aiti/\'ari &
a- &
*(h)a-\\\hline
\textsc{3sg} &
\'itigi (\textsc{m})

\'itigo (\textsc{f}) &
{\O} &
*ga-\\\hline
\textsc{1pl.incl} &
d\'aiti &
da- &
*pi-\\\hline
\textsc{1pl.excl} &
n\'iiti &
ni- &
*ni-\\\hline
\textsc{2pl} &
\'iiti &
i(da)- &
*(h)i-\\\hline
\textsc{3pl} &
\'itiga &
{\O} &
*gi-\\\hline
\end{supertabular}
\end{center}
In the Inanwatan vocabulary, five forms stand out as potentially cognate with TAP. 

(56)\ \ Comparison of TAP with Inanwatan (de Vries 2004)

a.\ \ Inanwatan \textit{mo-}, pAP *mai {\textquoteleft}to come{\textquoteright}

b.\ \ Inanwatan \textit{ni- }{\textquoteleft}eat, drink, smoke{\textquoteright}, pTAP *nVa {\textquoteleft}eat, drink{\textquoteright}

c.\ \ Inanwatan \textit{[241?]}\textit{ero}, pTAP *-wa(l,r)i {\textquoteleft}ear{\textquoteright}

d.\ \ Inanwatan \textit{oro}, pTAP *-ar(u) {\textquoteleft}vagina{\textquoteright}

e.\ \ Inanwatan \textit{durewo} {\textquoteleft}wing, bird{\textquoteright}, pTAP *(h)adul {\textquoteleft}bird{\textquoteright}

The form for {\textquoteleft}to come{\textquoteright} is likely a loan from an Austronesian language (and it is not found in Timor languages). The other correspondences look promising, although we see an r:r correspondence in (d), an r:l correspondence in (e), and a correspondence between \textit{r} and an unidentified liquid in (c). 

The South Bird{\textquoteright}s Head language Kokoda also shows several promising lexical similarities with TAP, although both {\textquoteleft}pig{\textquoteright} and {\textquoteleft}come{\textquoteright} may be Austronesian loans, and the remaining items do not reconstruct to the level of pTAP. Curiously, only one of these has the same meaning as those we identified from Inanwatan even though Inanwatan and Kokoda share 20\% possible lexical correspondences (de Vries 2004: 133). 

(57)\ \ Comparison of TAP with Kokoda (de Vries 2004)

\begin{enumerate}
\item Kokoda \textit{ta{\textprimstress}bai}, pTAP *baj {\textquoteleft}pig{\textquoteright}\footnote{ Robinson (forthcoming) provides evidence that words for {\textquoteleft}pig{\textquoteright} were borrowed separately into pAP and proto-Timor after the breakup of pTAP.}
\item Kokoda \textit{k{\textopeno}{\textprimstress}tena}, pAP *-tok {\textquoteleft}belly, stomach{\textquoteright}
\item Kokoda \textit{{\textprimstress}{\textbardotlessj}{\textepsilon}ria}, pAP *jira(n) {\textquoteleft}to fly{\textquoteright}
\item Kokoda \textit{m{\textopeno}e}, pAP *mai {\textquoteleft}to come{\textquoteright}
\end{enumerate}
If the suspected Austronesian loans are omitted from the list above, the number of lexical similarities between TAP and Kokoda is reduced by half to only two items.

\subsection[Dani]{Dani}
\hypertarget{RefHeading72084871885726}{}The Dani family is here represented by Lower Grand Valley Dani (ISO 639-3 dni) for the pronouns and Western Dani (ISO 693-3 dnw) for the vocabulary. The Dani pronouns are given in Table 13 (with pAP for comparison since pTAP reconstructions are not yet available). Like the pAP and pTNG pronouns, they have the paradigmatic vowels /a/ for singulars and /i/ for plurals, plus the use of /n/ for first person, which is why Ross (2005) suggested they might be related to the TAP languages. The Dani pronouns more closely match the reconstructed pAP pronouns than either match the pTNG pronouns, in that Dani also lacks a velar consonant in the second person forms (cf. Table 7). As with pAP, the Dani pronouns could be explained by positing a flip between the second and third person pronouns. If AP were indeed TNG, then this flip could constitute evidence of shared innovation in the AP and Dani group. 

{\centering
Table 13: Lower Grand Valley Dani pronouns (van der Stap 1966: 145-6)
\par}

\begin{center}
\tablehead{}
\begin{supertabular}{|m{0.32265985in}|m{0.6920598in}|m{0.7754598in}|m{0.6920598in}|}
\hline
 &
\centering personal\par

\centering pronouns &
\centering possessive\par

\centering prefixes &
\centering\arraybslash pAP\\\hline
\textsc{1sg} &
\centering an &
\centering n(a)- &
\centering\arraybslash *na-\\\hline
\textsc{2sg} &
\centering hat &
\centering h(a)- &
\centering\arraybslash *(h)a-\\\hline
\textsc{3sg} &
\centering at &
\centering {\O}- &
\centering\arraybslash *ga-\\\hline
\textsc{1pl} &
\centering nit &
\centering nin- &
\centering\arraybslash *pi-, *ni-\\\hline
\textsc{2pl} &
\centering hit &
\centering hin- &
\centering\arraybslash *(h)i-\\\hline
\textsc{3pl} &
\centering it &
\centering in- &
\centering\arraybslash *gi-\\\hline
\end{supertabular}
\end{center}
Curiously, Dani shows \textit{an} for the independent pronoun and \textit{n(a)-} for the pronominal prefix. The pAP \textsc{1sg} pronouns (both the reconstructed prefix, and the various derived independent pronouns found in individual AP languages) reflect *na-, like the pTNG *na. The Timor languages, in contrast, reflect *an in the \textsc{1sg}. Donohue (p.c.) suggests that perhaps the pTNG reconstruction should instead be *an, and that many TNG languages have independently leveled the pronominal paradigm so that all the singulars are of the shape Ca. Donohue suggests that this is a simpler explanation for the pronominal distributions than claiming independent changes of *na {\textgreater} *an. On the other hand, the fact that the bound 1\textsc{sg TNG }pronoun reconstructs as *na- suggests that the CV is older.

In the vocabulary, Western Dani shares a handful of look-alikes with the TAP languages. These are given below. 

(58)\ \ Comparison of TAP with Western Dani (Purba et al. 1993)

a.\ \ Western Dani \textit{ji}, pTAP *jira {\textquoteleft}water{\textquoteright}

b.\ \ Western Dani \textit{mugak} {\textquoteleft}ko banana{\textquoteright}, pTAP *mugul {\textquoteleft}banana{\textquoteright}

c.\ \ Western Dani \textit{maluk}, proto-Alor (but not pAP or pTAP) *makal {\textquoteleft}bitter{\textquoteright}

d.\ \ Western Dani \textit{nono} {\textquoteleft}what{\textquoteright},  Adang (AP) \textit{ano}, Hamap (AP) \textit{hano}, Kabola (AP) \textit{hanado} {\textquoteleft}who{\textquoteright}

e.\ \ Western Dani \textit{o} {\textquoteleft}house{\textquoteright}, Kui (AP) \textit{ow}, Klon (AP) \textit{[1DD?]}\textit{wi}

Terms for {\textquoteleft}water{\textquoteright} and {\textquoteleft}banana{\textquoteright} are reconstructable to pTAP, but the other look-alikes occur only in restricted geographic subset of the TAP languages, significantly increasing the probability of chance resemblance due to researcher bias. That is, with some 30 languages, there are bound to be chance resemblances with individual languages, so methodologically, we should restrict ourselves to comparing proto-language with proto-language, rather than comparing to individual daughter languages within TAP.

\subsection[Wissel Lakes]{Wissel Lakes}
\hypertarget{RefHeading72086871885726}{}The Wissel Lakes family is here represented by Ekari (ISO 639-3 ekg). The Ekari pronouns are listed in Table 14 (with pAP for comparison). As in pAP and pTNG, Ekari pronouns have the paradigmatic vowels /a/ for singulars and /i/ for plurals, plus the use of /n/ for first person. Like the Dani pronouns and the Timor pronouns, the Ekari pronouns show \textit{ani} in the independent pronouns and \textit{na-} in the prefixes. Unlike TAP and Dani, however, the Ekari pronouns show velar consonants in the second person, suggesting a straightforward inheritance from the prenasalized velars of pTNG.

{\centering
Table 14: Ekari pronouns (Drabbe 1952) 
\par}

\begin{center}
\tablehead{}
\begin{supertabular}{m{0.32755986in}m{0.40255985in}m{0.8469598in}m{0.6406598in}}
 &
free &
object prefix &
\centering\arraybslash pAP\\
\textsc{1sg} &
\centering ani &
\centering na- &
\centering\arraybslash *na-\\
\textsc{2sg} &
\centering aki &
\centering ka- &
\centering\arraybslash *(h)a-\\
\textsc{3sg} &
\centering okai[32F?] &
\centering e- &
\centering\arraybslash *ga-\\
\textsc{1du} &
\centering inai[32F?] &
\centering {}- &
\\
\textsc{2du} &
\centering ikai[32F?] &
 &
\\
\textsc{3du} &
\centering okeai[32F?] &
 &
\\
\textsc{1pl} &
\centering inii &
\centering ni- &
\centering\arraybslash *pi-, *ni-\\
\textsc{2pl} &
\centering ikii &
\centering ki- &
\centering\arraybslash *(h)i-\\
\textsc{3pl} &
\centering okei[32F?] &
\centering e- &
\centering\arraybslash *gi-\\
\end{supertabular}
\end{center}
We identified five potential cognates in the vocabulary; these are listed in (59) below.

(59)\ \ Comparison of TAP with Ekari (Steltenpool 1969)

\begin{enumerate}
\item Ekari \textit{nai} {\textquoteleft}eat, drink{\textquoteright}, pTAP *nVa {\textquoteleft}eat, drink{\textquoteright}
\item Ekari \textit{menii} {\textquoteleft}give to him/her/them (irregular){\textquoteright}, pTAP *-(e,i)na {\textquoteleft}to give{\textquoteright}
\item Ekari \textit{mei} {\textquoteleft}come{\textquoteright}, pAP *mai {\textquoteleft}come{\textquoteright}
\item Ekari \textit{maki} {\textquoteleft}land{\textquoteright}, Teiwa (AP) \textit{mo[127?]o}\textit{{\textglotstop}}, Kaera (AP) \textit{maxa}, Klon (AP) \textit{m}\textit{[1DD?]}\textit{k}\textit{{\textepsilon}{\textglotstop}}, pTim *muka 
\item Ekari \textit{owaa} {\textquoteleft}house{\textquoteright}, Kui (AP) \textit{ow}, Klon (AP) \textit{[1DD?]}\textit{wi}
\end{enumerate}
Of these potential cognates, only {\textquoteleft}eat{\textquoteright} and {\textquoteleft}give{\textquoteright} are reconstructed to pTAP, though {\textquoteleft}give{\textquoteright} only matches in a subset of phonemes. As mentioned before, it is likely that both Ekari and AP borrowed {\textquoteleft}come{\textquoteright} from Austronesian sources (see discussion in Section 4). The forms for {\textquoteleft}house{\textquoteright} are only found in a geographical subset of the TAP languages, leaving only {\textquoteleft}eat, drink{\textquoteright} and {\textquoteleft}land{\textquoteright} as solid-looking potential cognates. 

\subsection[West Bomberai ]{West Bomberai }
\hypertarget{RefHeading72088871885726}{}In the West Bomberai languages, stronger lexical similarities to TAP languages emerge, and we can posit tentative sound correspondences. The West Bomberai family is composed of three languages: Iha (ISO 639-3 ihp), Baham (bdw) and Karas (kgv), with the latter of these thought to be more distantly related to the other two.

The Iha pronouns are given in Table 15 (with pAP for comparison). Iha shows /o/ in the first and second person singular and /i/ in the other pronouns, paralleling the /a/ - /i/ paradigms of pTNG and pAP. Like Dani, Ekari, and the Timor languages, the Iha first person singular pronoun is VC as opposed to the CV pronouns of Inanwatan, pTNG, and pAP. Iha also shows a similar metathesis in the first person inclusive \textit{in} from pTNG *ni. Like pTNG, Iha shows velar consonants in the second person, as opposed to the velar third person seen in pAP, suggesting that Iha did not share the proposed innovative flip of second and third person pronouns. On the other hand, one of the sound correspondences outlined below (Iha k : pAP {\O}) suggests that perhaps Iha \textit{ko} {\textquoteleft}\textsc{2sg}{\textquoteright} and \textit{ki} {\textquoteleft}\textsc{2pl}{\textquoteright} correspond to pAP *(h)a- {\textquoteleft}\textsc{2sg}{\textquoteright} pAP *(h)i- {\textquoteleft}\textsc{2pl}{\textquoteright}, 
respectively. The reconstruction of *h in the second person pAP pronouns is based on only two languages (Teiwa and Western Pantar), and the other AP languages have vowel-initial second person pronouns, which matches with the Iha k : pAP {\O} correspondence. 

{\centering
Table 15: Iha personal pronouns (Donohue, p.c.)\footnote{ Flassy and Animung (1992) list \textit{bi} for first-plural exclusive and \textit{in} for inclusive, an apparent reversal of the forms found in Donohue{\textquoteright}s word list. They also list \textit{wat} rather than \textit{mi} for third-person plural, while Donohue gives \textit{wat} {\textquoteleft}friends{\textquoteright}.}
\par}

\begin{center}
\tablehead{}
\begin{supertabular}{|m{0.6941598in}|m{0.33165985in}|m{0.47885987in}|}
\hline
 &
Iha &
pAP\\\hline
\textsc{1sg} &
on &
*na-\\\hline
\textsc{2sg} &
ko &
*(h)a-\\\hline
\textsc{3sg} &
mi &
*ga-\\\hline
\textsc{1pl.incl} &
mbi &
*pi-\\\hline
\textsc{1pl.exc}\textsc{l} &
in &
*ni-\\\hline
\textsc{2pl} &
ki &
*(h)i-\\\hline
\textsc{3pl} &
mi &
*gi-\\\hline
\end{supertabular}
\end{center}
We identified thirteen potential TAP cognates in the Iha vocabulary (Donohue, p.c.), although some do not reconstruct to the level of pTAP and instead show similarities with the reconstructed pAP or forms in individual languages. The form {\textquoteleft}eat, drink{\textquoteright} has been reconstructed as pTNG *na- {\textquoteleft}eat, drink{\textquoteright}. As mentioned in Section 3, the term for older sibling has been reconstructed as pTNG *nan(a,i), although this could be a nursery form. 

(60)\ \ Potential cognates between Iha and TAP

\begin{enumerate}
\item Iha \textit{nwV} {\textquoteleft}eat{\textquoteright}, pTAP *nVa {\textquoteleft}eat, drink{\textquoteright}
\item Iha \textit{tan}, pTAP *-tan(a) {\textquoteleft}arm/hand{\textquoteright}
\item Iha \textit{wor}, pAP *-o(l,r)a {\textquoteleft}tail{\textquoteright}\footnote{ As mentioned above,  Schapper et al. (chapter 3) reconstruct a liquid third liquid (in addition to *l and *r), but we believe that third correspondence set should be assigned to either *l or *r with an as yet to be identified conditioning. }
\item Iha \textit{kar, }pTAP *-ar(u) {\textquoteleft}vagina{\textquoteright}
\item Iha \textit{wek}, pTAP *waj {\textquoteleft}blood{\textquoteright}
\item Iha \textit{ne}, pAP *-en(i,u), pTim *-nej {\textquoteleft}name{\textquoteright}
\item Iha \textit{jet}, pTAP *jagir {\textquoteleft}laugh{\textquoteright}
\item Iha \textit{mbjar}, pTAP *dibar {\textquoteleft}dog{\textquoteright} 
\item Iha \textit{m[127?]en}, pTAP *mit {\textquoteleft}sit{\textquoteright}
\item Iha \textit{i[127?]}, pAP *is(i) {\textquoteleft}fruit{\textquoteright}
\item Iha \textit{nen }{\textquoteleft}older brother{\textquoteright}, Iha \textit{nan} {\textquoteleft}older sister{\textquoteright}, pAP *nan(a) {\textquoteleft}elder sibling{\textquoteright}
\item Iha \textit{nemehar}, Teiwa (AP) \textit{masar }{\textquoteleft}man, male{\textquoteright}
\item Iha \textit{ja, }Blagar (AP) \textit{{\textdyoghlig}}\textit{e} {\textquoteleft}boat{\textquoteright} 
\end{enumerate}
{\centering
Based on these thirteen potential cognates in the lexicon, plus the potential cognates in the pronouns, we can suggest possible sound correspondences. But some of these correspondences conflict with each other. Note, for example that the h:s correspondence of {\textquoteleft}man{\textquoteright} and the [127?]:s correspondence of {\textquoteleft}fruit{\textquoteright} conflict with [127?]:t correspondence of {\textquoteleft}sit{\textquoteright}. Without more examples, it is difficult to determine whether these conflicts are due to conditioned sound change or false cognates. We posit only one conditioned correspondence, that of w:{\O} before a back rounded vowel and w:w elsewhere. Table 16: Possible Iha: pTAP sound correspondences
\par}

\begin{center}
\tablehead{}
\begin{supertabular}{m{0.32055986in}m{0.8997598in}m{2.31016in}}
Iha &
pTAP &
examples\\
r &
r &
vagina, man, dog, tail\\
n &
n &
eat, name, arm, older sibling, \textsc{1sg}\\
m &
m &
sit, man\\
w &
{\O} before /o/

w elsewhere &
tail 

blood\\
k &
{\O} &
vagina, blood\\
k &
h &
\textsc{2sg, 2pl}\\
h, [127?] &
s &
man, fruit\footnotemark{}\\
[127?] &
t &
sit\\
mb &
b &
dog\\
mb &
p &
\textsc{1pl.incl}\\
j &
j &
laugh, boat\footnotemark{}\\
t &
t &
arm\\
t &
r &
laugh\\
{\O} &
g &
laugh\\
\end{supertabular}
\end{center}
\addtocounter{footnote}{-2}
\stepcounter{footnote}\footnotetext{ Note that Teiwa [s] is the regular reflex of pAP *s, which is, in turn, the regular reflex of pTAP *s.  }
\stepcounter{footnote}\footnotetext{ Note that Blagar [{\textdyoghlig}] is the regular reflex of pAP *j, which, in turn, is the regular reflex of pTAP *j.}
The West Bomberai language Baham also shows striking similarities to TAP languages. The Baham pronouns are given in Table 17, with the pAP pronouns for comparison. In the possessives, these pronouns show a first singular \textit{ne}, a third singular \textit{ka}, and a first plural \textit{ni} that appear cognate to the corresponding pAP pronouns. The third person plural may be cognate in the first segment. Other pronouns appear innovative. 

{\centering
Table 17: Baham pronouns (Flassy et al. 1987)
\par}

\begin{center}
\tablehead{}
\begin{supertabular}{|m{0.32265985in}|m{0.63655984in}|m{0.7754598in}|m{0.6920598in}|}
\hline
 &
personal &
possessive &
pAP\\\hline
\textsc{1sg} &
anduu &
ne &
*na-\\\hline
\textsc{2sg} &
tow &
te &
*(h)a-\\\hline
\textsc{3sg} &
kpwaw &
ka &
*ga-\\\hline
\textsc{1pl} &
unduu &
ni &
*pi-, *ni-\\\hline
\textsc{2pl} &
kujuu &
kuju &
*(h)i-\\\hline
\textsc{3pl} &
kinewat &
kinewaat &
*gi-\\\hline
\end{supertabular}
\end{center}
The Baham vocabulary reveals thirteen potential TAP cognates. Six of these terms are also found in Iha, and three have been reconstructed for pTNG: pTNG *na- {\textquoteleft}eat, drink{\textquoteright}, pTNG *inda {\textquoteleft}tree{\textquoteright}, and pTNG *tukumba(C) {\textquoteleft}short{\textquoteright}. 

(61)\ \ Potential cognates between TAP and Baham (Flassy et al. 1987)

\begin{enumerate}
\item Baham \textit{nowa} {\textquoteleft}eat{\textquoteright}, pTAP *nVa {\textquoteleft}eat, drink{\textquoteright}
\item Baham \textit{adoq} {\textquoteleft}tree{\textquoteright}, pTAP *hada {\textquoteleft}fire, wood{\textquoteright}
\item Baham \textit{toqoop}, pAP *tukV {\textquoteleft}short{\textquoteright}
\item Baham \textit{pkwujer}, pTAP *wa(l,r)i {\textquoteleft}ear{\textquoteright}
\item Baham \textit{kaar}, pAP *-ar(u) {\textquoteleft}vagina{\textquoteright}
\item Baham \textit{wijek}, pTAP *waj {\textquoteleft}blood{\textquoteright}
\item Baham \textit{mungguo}, pTAP *mugul {\textquoteleft}banana{\textquoteright}
\item Baham \textit{wuor tare}, pTAP *o(l,r)a {\textquoteleft}tail{\textquoteright}
\item Baham \textit{waar}, pTAP *war {\textquoteleft}stone{\textquoteright}
\item Baham \textit{{\textltailn}ie}, pAP *-en(i,u), pTim *-nej {\textquoteleft}name{\textquoteright}
\item Baham \textit{meheen}, pTAP *mit {\textquoteleft}sit{\textquoteright}
\item Baham \textit{jambar}, pTAP *dibar {\textquoteleft}dog{\textquoteright}
\item Baham \textit{wawa}, cf., Teiwa (AP) \textit{wow}, Nedebang (AP) \textit{wowa}, Kaera (AP) \textit{wow} {\textquoteleft}mango{\textquoteright}
\end{enumerate}
Once again, based on these thirteen potential cognates and the pronouns we can suggest potential sound correspondences. Unsurprisingly, these correspondences are similar to the ones we propose for Iha, including a correspondence of pre-nasalized stops in Baham to voiced stops in pTAP, although the Baham form for {\textquoteleft}tree{\textquoteright} (cf. TAP {\textquoteleft}fire, wood{\textquoteright}) does not fit that trend. 

{\centering
Table 18: Possible Baham : pTAP sound correspondences
\par}

\begin{center}
\tablehead{}
\begin{supertabular}{m{0.54275984in}m{0.89905983in}m{1.8677598in}}
Baham &
pTAP &
examples\\
r &
r &
ear, vagina, tail, stone, dog\\
k &
{\O} &
ear, vagina, blood\\
k &
h &
\textsc{3sg}\\
q &
k &
short\\
q &
{\O} &
fire\\
p &
{\O} &
short, ear\\
w &
{\O} before /o/

w elsewhere &
tail

blood, mango, stone, ear\\
n, {\textltailn} &
n &
eat, name, \textsc{1sg, 1pl}\\
m &
m &
banana, sit\\
mb &
b &
dog\\
{\ng}g &
g/k &
banana\\
d &
d &
fire\\
j &
d &
dog\\
t &
t &
short\\
h &
t &
sit\\
{\O} &
h &
fire\\
{\O} &
l &
banana\\
\end{supertabular}
\end{center}
The West Bomberai language Karas also shows several potential cognates with TAP languages, although information on Karas is more sparse than for Iha or Baham. In the vocabulary (Donohue, p.c.), nine potential cognates were identified, six of which are also found in both Iha and Baham. Three of these are reconstructed for pTNG: *na- {\textquoteleft}eat, drink{\textquoteright}, pTNG *me-{\textquoteleft}to come{\textquoteright}, and pTNG *amu {\textquoteleft}breast{\textquoteright}. 

(62)\ \ Potential cognates between TAP/AP and Karas

\begin{enumerate}
\item Karas \textit{n}\textit{{\textsci}}\textit{n} {\textquoteleft}eat{\textquoteright}, pTAP *nVa {\textquoteleft}eat, drink{\textquoteright}
\item Karas \textit{tan}, pTAP *-tan(a) {\textquoteleft}arm, hand{\textquoteright}
\item Karas \textit{{\textopeno}}\textit{r}\textit{{\textupsilon}}\textit{n}, pTAP *o(l,r)a {\textquoteleft}tail{\textquoteright}
\item Karas \textit{bal}, pTAP *dibar {\textquoteleft}dog{\textquoteright}
\item Karas \textit{wat}, pTAP *wata {\textquoteleft}coconut{\textquoteright}
\item Karas \textit{am}, pTAP *hami {\textquoteleft}breast{\textquoteright}
\item Karas \textit{i:n}, pAP *-en(i,u), pTim *-nej {\textquoteleft}name{\textquoteright}
\item Karas \textit{mej}, pAP *mai {\textquoteleft}to come{\textquoteright}
\end{enumerate}
We can establish tentative correspondences from these forms, although most correspondences occur only once in these data, and the final /n/ in Karas {\textquoteleft}tail{\textquoteright} is unexplained. 

{\centering
Table 19: Possible Karas : pAP sound correspondences
\par}

\begin{center}
\tablehead{}
\begin{supertabular}{m{0.46015987in}m{0.36705986in}m{1.0531598in}}
Karas &
pAP &
examples\\
m &
m &
come, breast\\
n &
n &
eat, arm, name \\
n &
{\O} &
tail, eat\\
t &
t &
arm, coconut\\
r &
L &
tail\\
b &
b &
dog\\
l &
r &
dog\\
w &
w &
coconut\\
{\O} &
h &
breast\\
\end{supertabular}
\end{center}
In the lexicon, then, the strongest correspondences are with West Bomberai languages, allowing us to posit some (very tentative) sound correspondences. In the pronouns, Iha shows an inclusive/exclusive distinction, with an exclusive pronoun that looks superficially similar to the reconstructed pAP inclusive pronoun *pi-. However, the sound correspondences suggest Iha mb : pTAP p, so perhaps both forms are independently innovated, with the similarity in vowels due to analogy with other pronouns in the paradigm (i.e., plurals have the vowel /i/) and the similarity in consonants due to chance. An alternative explanation would rely on borrowing, which we return to in the following section. 

\section[Discussion]{Discussion}
\hypertarget{RefHeading72090871885726}{}We have considered three hypotheses regarding the wider genealogical affiliations of the TAP languages. We now return to the null hypothesis proposed in section 1 (that the TAP languages are a family-level isolate) and consider the strength of the evidence with regard to each of the proposals.

The pronominal evidence points much more clearly toward a link with TNG as opposed to NH. The TAP pronouns share with TNG a vowel grading /a/ vs. /i/ in the singular vs. plural, respectively. In addition, TNG second person pronouns correspond well with TAP third person pronouns, although this correspondence requires us to posit a semantic flip between second and third person forms. This flip renders the pronominal evidence much weaker than it otherwise might be. The primary trace of similarity between the TAP and NH pronouns lies in the TAP first person distributive form, which resembles the NH first person singular. It is of course possible that the TAP pronoun system has been influenced by both TNG and NH languages, as suggested by Donohue (2008). 

In the lexicon, there is no evidence supporting a genealogical connection between TAP and NH languages. The lexical evidence for a link with TNG is more promising, and a few regular sound correspondences emerge, but a critical eye limits the number to thirteen, so we cannot establish a robust connection. However, if we focus our attention just on the West Bomberai languages, the pronominal and lexical evidence looks more promising and warrants further investigation. It is possible that the TAP and Bomberai languages are related either via a deep genealogical connection or via a more casual contact relationship. If it is a genealogical relationship, it is not yet clear whether they are both part of TNG or whether they share a relationship independent of that family. 

The spread of TNG is conventionally linked to the development of agriculture in the New Guinea highlands about 10,000 years ago (Bellwood 2001), with a westward spread somewhat later, perhaps around 6,000 BP (Pawley 1998). This would place any putative TAP-TNG genealogical connection at the upper limits of what is possible using the comparative method. Another possibility is that the weak signal linking TAP with Bomberai is the result not of an ancient genealogical connection, but rather of more recent contact. The West Bomberai groups, for example, have a history of slaving (Klamer et al. 2008: 109). It is possible that they took Timor-Alor-Pantar peoples as slaves at some point, and that this is the source of the connection between the two groups. More investigation of the social history of pre-Austronesian contact in East Nusantara is greatly needed.

In conclusion, the existing evidence provides only weak support for a connection between TAP and Papuan languages spoken to the east. The most promising hypothesis would connect TAP with the West Bomberai languages, but even here the evidence is thin and does not support a definitive conclusion. We hope that new field research on the Bomberai languages, combined with reconstruction of proto-Bomberai, will eventually help clarify this question. 

{\bfseries
Acknowledgements}

Field work on the Alor-Pantar languages was supported by grants from the Netherlands Organization for Scientific Research, the UK Arts and Humanities Council, and the US National Science Foundation (NSF-SBE 0936887), under the aegis of the European Science Foundation EuroBABEL programme. The authors are indebted to their colleagues in the EuroBABEL Alor-Pantar project for generously sharing their data and analyses, and for providing feedback on early versions of this paper. The authors also wish to thank numerous colleagues in Alor and Pantar who assisted with data collection. 

{\bfseries
References}

Anceaux, J.C. 1973. Naschrift. \textit{Bijdragen to de Taal-, Land- en Volkenkunde} 129: 345-6.

Bellwood, Peter. 2001. Early agriculturist population diasporas? Farming, language, and genes. \textit{Annual Review of Anthropology} 30: 181-207.

Campbell, Lyle and William J. Poser. 2008. \textit{Language Classification: History and Method}. Cambridge: Cambridge University Press.

Capell, A. 1944. Peoples and languages of Timor. \textit{Oceania} 15(3): 19-48.

Capell, A. 1975. The West Papuan phylum: General, and Timor and areas further west. In S.A. Wurm, ed., \textit{New Guinea Area Languages and Language Study, vol. I, Papuan Languages and the New Guinea Linguistic Scene}. (Pacific Linguistics C-38): 667-716. Canberra: Australian National University.

Cowan, H. K. J. 1957. A large Papuan language phylum in West New Guinea. \textit{Oceania} 28: 159-66.

Cowan, H.K.J. 1953. Voorlopige Resultaten van een Ambtelijk Taalonderzoek in Nieuw-Guinea. {\textquotesingle}s-Gravenhage: Marinus Nijhoff.

Donohue, Mark. 2007a. The Papuan language of Tambora. \textit{Oceanic Linguistics} 46(2): 520-37.

Donohue, Mark. 2007b. The phonological history of the languages of the non-Austronesian languages of Southern Indonesia. Paper presented at the Fifth East Nusantara Conference. Kupang, Indonesia, August 1-3.

Donohue, Mark. 2008. Bound pronominals in the West Papuan languages. In Claire Bowern, Bethwyn Evans and Luisa Miceli, eds., \textit{Morphology and Language History: In honor of Harold Koch}: 43-58. Amsterdam: John Benjamins.

Donohue, Mark and Antoinette Schapper. 2007. Towards a morphological history of the languages of Timor, Alor and Pantar. Paper presented at the Fifth East Nusantara Conference. Kupang, Indonesia, August 1-3.

Drabbe, Peter. 1952. Spraakkunst van het Ekagi, Wisselmeren, Nederlands Nieuw Guinea. Den Haag: Marinus Nijhoff.

Drabbe, Peter. 1955. \textit{Spraakkunst van het Marind}, (Studia Instituti Anthropos Vol. 11). Vienna.

Fedden, Sebastian. 2011. \textit{A Grammar of Mian}. (Mouton Grammar Library, vol. 55.) Berlin: Walter de Gruyter.

Fedden, Sebastian, Dunstan Brown, Greville Corbett, Gary Holton, Marian Klamer, Laura C. Robinson and Antoinette Schapper. 2013. Conditions on pronominal marking in the Alor-Pantar languages. \textit{Linguistics} 51(1): 33-74.

Flassy, Don A.L. and Lisidius Animung. 1992. \textit{Struktur Bahasa Iha}. Jakarta: Pusat Bahasa dan Pengembangan Bahasa, Departemen Pendidikan Nasional.

Flassy, Don A.L., Constantinoepel Ruhukael and Frans Rumbrawe. 1987. \textit{Fonologi bahasa Bahaam}. Jakarta: Departemen Pendidikan dan Kebudayaan.

Foley, William A. 1986. \textit{The Papuan Languages of New Guinea}. Cambridge: Cambridge University Press.

Foley, William A. 1998. Toward understanding Papuan languages. In Jelle Miedema, Cecilia  Od\'e and Rien A.C. Dam, eds., \textit{Perspectives on the Bird{\textquotesingle}s Head of Irian Jaya, Indonesia}: 503-18. Amsterdam: Rodopi.

Foley, William A. 2000. The languages of New Guinea. \textit{Annual Review of Anthropology}: 357-404.

Haan, Johnson Welem. 2001. A grammar of Adang: a Papuan language spoken on the Island of Alor, East Nusa Tenggara, Indonesia. Ph.D. dissertation, University of Sydney.

Holton, Gary. 2003. \textit{Tobelo}, (Languages of the World/Materials 328). Munich: LINCOM Europa.

Holton, Gary. 2008. The rise and fall of semantic alignment in North Halmahera, Indonesia. In Mark Donohue and S{\o}ren Wichmann, eds., \textit{The Typology of Semantic Alignment}: 252-76. Oxford: Oxford University Press.

Holton, Gary. 2010. Person-marking, verb classes, and the notion of grammatical alignment in Western Pantar (Lamma). In Michael Ewing and Marian Klamer, eds., \textit{Typological and Areal Analyses: Contributions from East Nusantara}: 101-21. Canberra: Pacific Linguistics.

Klamer, Marian. 2010. \textit{A Grammar of Teiwa}. Berlin: Mouton.

Klamer, Marian, Ger P. Reesink and Miriam van Staden. 2008. East Nusantara as a linguistic area. In Pieter Muysken, ed., \textit{From Linguistic Areas to Areal Linguistics}: 95-150. Amsterdam: John Benjamins.

Kratochv\'il, Franti\v{s}ek. 2007. A Grammar of Abui. Ph.D. dissertation, Leiden University.

Lewis, M. Paul, Gary Simons and Charles D. Fennig. 2013. \textit{Ethnologue: Languages of the World, 17th edition}. Dallas: SIL International.

McElhanon, Kenneth A. and C.L. Voorhoeve. 1970. \textit{The Trans-New Guinea Phylum: Explorations in deep-level genetic relationships}, (Pacific Linguistics C-15). Canberra: Australian National University.

Meillet, Antoine. 1967. The Comparative Method in Historical Linguistics. Paris: Champion.

Olson, M. 1981. Barai clause junctures: Toward a functional theory of interclausal relations. Ph.D. dissertation, Australian National University.

Pawley, Andrew. 1995. C.L. Voorhoeve and the Trans New Guinea phylum hypothesis. In Connie Baak, Mary Bakker and Dick van der Meij, eds., \textit{Tales from a Concave World: Liber Amicorum Bert Voorhoeve}: 83-123. Leiden: Department of Languages and Cultures of Southeast Asia, Leiden University.

Pawley, Andrew. 1998. The Trans New Guinea Phylum hypothesis: A reassessment. In Jelle Miedema, Cecilia  Od\'e and Rien A.C. Dam, eds., \textit{Perspectives on the Bird{\textquotesingle}s Head of Irian Jaya, Indonesia}: 655-90. Amsterdam: Rodopi.

Pawley, Andrew. 2001. The Proto Trans New Guinea obstruents: arguments from top--down reconstruction. In Andrew Pawley, Malcolm Ross and Darrell Tryon, eds., \textit{The Boy from Bundaberg: Studies in Melanesian Linguistics in Honour of Tom Dutton}: 261-300. Canberra: Pacific Linguistics.

Pawley, Andrew. 2012. How reconstructable is proto Trans New Guinea? Problems, progress, prospects. In Harald Hammarstr\"om and Wilco van der Heuvel, eds., \textit{History, Contact and Classification of Papuan Languages}. (Language and Linguistics in Melanesia, Special Issue 2012 Part I): 88-164. Port Moresby: Linguistic Society of New Guinea.

Pawley, Andrew. n.d. \textit{Some Trans New Guinea Phylum cognate sets.} Canberra: Department of Linguistics, Research School of Pacific and Asian Studies, Australian National University.

Purba, Theodorus T., Onesimus Warwer and Reimundus Fatubun. 1993. \textit{Fonologi Bahasa Dani Barat}. Jakarta: Departemen Pendidikan dan Kebudayaan.

Reesink, Ger P. 1996. Morpho-syntactic features of the Bird{\textquotesingle}s Head languages. \textit{NUSA, Studies in Irian Languages} 40: 1-20.

Roberts, John R. 1997. Switch-reference in Papua New Guinea: a preliminary survey. In Andrew Pawley, ed., \textit{Papers in Papuan Linguistics 3}: 101-241. Canberra: Pacific Linguistics.

Robide van der Aa, Pieter Jan Batist Carel. 1872. Een tweetal bijdragen tot de kennis van Halmahera. \textit{Bijdragen tot de Taal, Land en Volkenkunde} 19: 233-9.

Robinson, Laura C. forthcoming. The Alor-Pantar (Papuan) languages and Austronesian contact in East Nusantara. In Malcom Ross ed. \textit{Language Contact and Austronesian Historical Linguistics}. Canberra: Pacific Linguistics.

Robinson, Laura C. and Gary Holton. 2012. Reassessing the wider genetic affiliations of the Timor-Alor-Pantar languages. In Harald Hammarstr\"om and Wilco van der Heuvel, eds., \textit{History, Contact and Classification of Papuan Languages}. (Language and Linguistics in Melanesia, Special Issue 2012 Part I): 59-87. Port Moresby: Linguistic Society of New Guinea.

Ross, Malcolm. 2005. Pronouns as a preliminary diagnostic for grouping Papuan languages. In Andrew Pawley, Robert Attenborough, Jack Golson and Robin Hide, eds., \textit{Papuan Pasts: Cultural, linguistic and biological histories of Papuan-speaking peoples }(Pacific Linguistics PL 572): 15-66. Canberra: Pacific Linguistics.

Ross, Malcolm. 2006. Pronouns as markers of genetic stocks in non-Austronesian languages of New Guinea, Island Melanesia and eastern Indonesia. In Andrew Pawley, Malcolm Ross and Meredith Osmond, eds., \textit{Papuan Languages and the Trans New Guinea Family}. Canberra: Pacific Linguistics.

Schapper, Antoinette. 2009. Bunaq: A Papuan language of central Timor. Ph.D. dissertation, Australian National University.

Schapper, Antoinette, Juliette Huber and Aone van Engelenhoven. 2012. The historical relation of the Papuan languages of Timor and Kisar. In Harald Hammarstr\"om and Wilco van der Heuvel, eds., \textit{History, Contact and Classification of Papuan Languages}. (Language and Linguistics in Melanesia, Special Issue 2012 Part I): 194-242. Port Moresby: Linguistic Society of New Guinea.

Scott, Graham. 1978. \textit{The Fore Language of Papua New Guinea}, (Pacific Linguistics B-47). Canberra: Pacific Linguistics.

Sheldon, Deidre. 1986. Topical and non-topical participants in Galela narrative discourse. \textit{Papers in New Guinea Linguistics no. 25.} (Pacific Linguistics A-74): 233-48. Canberra: Australian National University.

van der Stap, P.A.M. 1966. \textit{Outline of Dani Morphology}, (Verhandelingin KITLV 48). {\textquotesingle}s-Gravenhage.

Steltenpool, J. 1969. \textit{Ekagi-Dutch-English-Indonesian Dictionary}, (VKI 56). The Hague: Martinus Nijhoff.

Stokhof, W. A. L. 1975. \textit{Preliminary notes on the Alor and Pantar languages (East Indonesia)}, (Pacific Linguistics B-43). Canberra: Australian National University.

Suter, Edgar. 2012. Verbs with pronominal object prefixes in Finisterre-Huon languages. In Harald Hammarstr\"om and Wilco van der Heuvel, eds., \textit{History, Contact and Classification of Papuan Languages}. (Language and Linguistics in Melanesia, Special Issue 2012 Part I): 23-59. Port Moresby: Linguistic Society of New Guinea.

van der Veen, Hendrik. 1915. De Noord-Halmahera{\textquotesingle}se Taalgroep tegenover de Austronesiese talen. Leiden: Van Nifterik.

Voorhoeve, C. L. 1975. Central and western Trans-New Guinea Phylum languages. In S.A. Wurm, ed., \textit{New Guinea Area Languages and Language Study, vol. 1: Papuan languages and the New Guinea linguistics scene}: 345-460. Canberra: Australian National University.

de Vries, Lourens. 2004. \textit{A Short Grammar of Inanwatan: An endangered language of the Bird{\textquotesingle}s Head of Papua, Indonesia}, (Pacific Linguistics 560). Canberra: Australian National University.

Wada, Yuiti. 1980. Correspondence of consonants in North Halmahera languages and the conservation of archaic sounds in Galela. In N. Ishige, ed., \textit{The Galela of Halmahera: A Preliminary Survey}: 33 p. Osaka: Museum of Ethnology.

Watuseke, F.S. 1973. Gegevens over de taal van Pantar: een Irian taal. \textit{Bijdragen tot de Koninklijk Instituut voor Taal, Land en Volkenkunde} 129: 340-6.

Wurm, Stephen A. 1982. \textit{The Papuan Languages of Oceania}. T\"ubingen: Narr.

Wurm, Stephen A., ed. 1975. New Guinea Area Languages and Language Study, vol. 1: Papuan languages and the New Guinea linguistics scene, Pacific Linguistics C-38. Canberra: Australian National University.

Wurm, Stephen A., C.L. Voorhoeve and Kenneth A. McElhanon. 1975. The Trans-New Guinea Phylum in general. In S.A. Wurm, ed., \textit{New Guinea Area Languages and Language Study, vol. I, Papuan Languages and the New Guinea Linguistic Scene}. (Pacific Linguistics C-38): 299-322. Canberra: Australian National University.


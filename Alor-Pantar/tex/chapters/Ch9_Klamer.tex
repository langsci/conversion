
\clearpage\setcounter{page}{1}\pagestyle{Standard}
{\centering\itshape
Chapter 9
\par}

{\centering\bfseries
Plural number words in the 
\par}

{\centering\bfseries
Alor-Pantar languages
\par}

{\centering
\textit{Marian Klamer}\textit{, }\textit{Antoinette Schapper and Greville Corbett}
\par}

\clearpage{\centering\bfseries\itshape
Chapter 9
\par}

{\centering\bfseries
Plural number words in the 
\par}

{\centering\bfseries
Alor-Pantar languages
\par}

{\centering
\textit{Marian Klamer}\textit{, }\textit{Antoinette Schapper and Greville Corbett}
\par}

Abstract: In this chapter, we investigate the variation in form, syntax and semantics of the plural words found across the Alor-Pantar languages. We study five AP languages: Western Pantar, Teiwa, Abui, Kamang and Wersing. We show that plural words in Alor-Pantar family are diachronically instable: although proto-Alor-Pantar had a plural number word *non, many AP languages have innovated new plural words. Plural words in these languages exhibit not only a wide variety of different syntactic properties but also variable semantics, thus likening them more to the range exhibited by affixal plural number than previously recognised.

\section[1\ \ Introduction]{1\ \ Introduction\footnotemark{}}
\hypertarget{Toc376962648}{}\footnotetext{We are grateful to the following colleagues for answering our questions and gracefully sharing their data: Gary Holton for Western Pantar, Franti\v{s}ek Kratochv\'il and Benny Delpada for Abui, and Louise Baird for Klon. We are also very grateful to Mary Darlymple and Martin Haspelmath for their comments on an earlier version of this paper. }
The majority of the world{\textquoteright}s languages express nominal plurality by affixation. After affixation, the use of independent plural words is the most widespread strategy: it is used in 16\% of Dryer{\textquoteright}s (2011) sample of 1066 languages. Yet, {\textquoteleft}plural words{\textquoteright} have received remarkably little attention since their preliminary treatment in Dryer (1989). In this chapter, we build on Schapper and Klamer (2011) in furthering the investigation of plural words using data from the Alor-Pantar (AP) languages, which are of great typological interest. 

A plural word is {\textquotedblleft}a morpheme whose meaning and function is similar to that of plural affixes in other languages, but which is a separate word{\textquotedblright} (Dryer 1989: 865, 2007: 166). Plural words are the most common example of a more general category, that of grammatical number words - a number of languages employ singular or dual words as well as plural words. For Dryer, to be a plural word a lexeme must be the prime indicator of plurality: {\textquotedblleft}I do not treat a word as a plural word if it co-occurs with an inflectional indication of plural on the noun{\textquotedblright} (1989: 867). Dryer further makes a distinction between {\textquoteleft}pure{\textquoteright} number words and other kinds of number expressions: {\textquotedblleft}We can [...] distinguish {\textquoteleft}pure{\textquoteright} plural words, which only code plurality, from articles that code number in addition to other semantic or grammatical features of the noun phrase, in which these articles are 
the sole indicator of number in noun phrases{\textquotedblright}. Thus the bar is set quite high: plural words are the prime indicator of plurality, and in the pure case they have this as their unique function.

 Plural words in Alor-Pantar languages carry also a range of additional semantic connotations beyond simple plurality, including completeness, abundance, individuation, and partitivity. These are interrelated to the other options the individual languages have for marking plurality. This means that our discussion of plural words in Alor-Pantar languages necessarily also touches on other plurality expressing strategies available in the languages. We will see that the form, syntax and semantics of plural words across the Alor-Pantar languages display a high degree of diversity.

This paper is structured as follows. Section 2 introduces the lexical forms of the plural words of the languages and the sources of the data discussed in this paper. Section 3 discusses their syntax, while section 4 looks in detail at the semantics of the plural words. Section 5 places AP plural words in a wider typological context, and section 6 presents our conclusions.

\section[2\ \ Plural number words across Alor{}-Pantar ]{2\ \ Plural number words across Alor-Pantar }
Plural words are found across the Alor-Pantar languages, as shown in Table 1. Cognate forms attested in Teiwa (West Pantar), Klon (West Alor) and Kamang (Central-East Alor) indicate that a plural word \textit{*non} can be reconstructed for proto-Alor-Pantar (pAP). Western Pantar, Abui, Wersing, Kula and Sawila do not reflect this item, and instead appear to have innovated new lexemes for plural words. Several AP languages in our sample (Klon, Abui, Wersing, Kula and Sawila) have two plural words encoding different kinds of plurality, though the other languages do have a range of plural-marking strategies in addition to their plural word. There are also Alor-Pantar languages for which no plural word has been attested; an example is Kaera (North-East Pantar; Klamer, to appear). 

{\centering
Table 1: Cognate and non-cognate plural words in Alor-Pantar languages
\par}

\begin{flushleft}
\tablehead{}
\begin{supertabular}{|m{0.8816598in}|m{0.90595984in}|m{1.1545599in}|m{3.35666in}|}
\hline
\bfseries Language  &
{\bfseries Reflecting }

\bfseries\itshape *non &
{\bfseries Not reflecting }

\bfseries\itshape *non  &
\bfseries Source\\\hline
Western Pantar &
 &
\itshape maru(ng) &
Holton and Lamma Koly 2008;  

Holton 2012, to appear, p.c.\\\hline
Teiwa &
\itshape non &
 &
Klamer 2010, Teiwa corpus;  

Schapper and Klamer 2011\\\hline
Adang &
\itshape nun &
 &
Robinson and Haan, to appear \\\hline
Klon &
\itshape (o)non &
\itshape maang &
Baird 2008, Klon corpus, p.c. \\\hline
Abui &
 &
{\itshape loku,}

\itshape we &
Schapper fieldnotes;  

Kratochv\'il 2007, Abui corpus\\\hline
Kamang &
\itshape nung &
 &
Schapper Kamang corpus, , to appear, fieldnotes; Schapper and Klamer 2011;  Stokhof 1978, 1982\\\hline
Wersing &
 &
{\itshape deing,}

\itshape naing &
Schapper and Hendery to appear, fieldnotes,  Wersing corpus; Malikosa n.d.\\\hline
Kula &
 &
{\itshape du(a),}

\itshape araman &
Nicholas Williams p.c.\\\hline
Sawila &
 &
{\itshape do,}

\itshape maarang &
Franti\v{s}ek Kratochv\'il p.c.\\\hline
\end{supertabular}
\end{flushleft}
In all Alor-Pantar languages, nouns are uninflected for number, and a noun phrase without a plural word can refer to any number of individuals. For instance, Teiwa \textit{qavif }{\textquoteleft}goat{\textquoteright} in (a) can be interpreted as either singular or plural, depending on the context. Those Alor-Pantar languages that have a plural word use it to express plurality: {\textquoteleft}more than one{\textquoteright}. Illustrations are Teiwa \textit{non} in (b), and Klon \textit{onon }in (b). The plural word pluralises the preceding nominal expression. In none of the AP languages we investigated is the plural word obligatory when plural reference is intended. 

Teiwa (Klamer, Teiwa corpus) 

\begin{flushleft}
\tablehead{}
\begin{supertabular}{m{0.39135984in}m{0.39135984in}m{0.5254598in}m{0.5615598in}m{0.41845986in}m{0.67815983in}m{0.5483598in}m{0.42195985in}m{0.35255986in}m{0.35315984in}m{0.35665986in}}
\label{bkm:Ref334184518}()   &
a. &
\itshape Qavif &
\textit{ita}\textit{[241?]}\textit{a } &
\itshape ma &
\itshape gi? &
 &
 &
 &
 &
\\
 &
 &
goat &
where &
\scshape obl &
go &
 &
 &
 &
 &
\\
 &
 &
\multicolumn{9}{m{4.84626in}}{{\textquoteleft}Where did the goat(s) go?{\textquoteright}}\\
 &
 &
\multicolumn{9}{m{4.84626in}}{}\\
\end{supertabular}
\end{flushleft}
\begin{flushleft}
\tablehead{}
\begin{supertabular}{m{0.39275986in}m{0.39275986in}m{0.5261598in}m{0.40595984in}m{0.56365985in}m{0.6802598in}m{0.5497598in}m{0.42265984in}m{0.35385984in}m{0.35315984in}m{0.35805985in}}
 &
b. &
\itshape Qavif &
\itshape non &
\textit{ita}\textit{[241?]}\textit{a } &
\itshape ma &
\itshape gi? &
 &
 &
 &
\\
 &
 &
goat &
\scshape pl &
where &
\scshape obl &
Go &
 &
 &
 &
\\
 &
 &
\multicolumn{9}{m{4.8434596in}}{{\textquoteleft}Where did the (several) goats go?{\textquoteright};  *{\textquoteleft}Where did the goat go?{\textquoteright} ~}\\
\end{supertabular}
\end{flushleft}
Klon (Baird, Klon corpus, p.c.)

\begin{flushleft}
\tablehead{}
\begin{supertabular}{m{0.39135984in}m{0.39205986in}m{1.0754598in}m{0.76365983in}m{0.95255977in}m{0.7893598in}m{0.5476598in}m{0.42195985in}m{0.35255986in}m{0.35255986in}m{0.35735986in}}
\label{bkm:Ref354060976}()   &
a. &
\itshape Ge-ebeng &
\itshape go-thook. &
 &
 &
 &
 &
 &
 &
\\
 &
 &
3.\textsc{gen}{}-friend &
3-meet &
 &
 &
 &
 &
 &
 &
\\
 &
 &
\multicolumn{9}{m{6.24306in}}{{\textquoteleft}(He) met his friend(s).{\textquoteright}\footnotemark{}}\\
 &
 &
\multicolumn{9}{m{6.24306in}}{}\\
 &
b. &
\itshape Ge-ebeng &
\itshape onon &
\itshape go-thook. &
 &
 &
 &
 &
 &
\\
 &
 &
3.\textsc{gen}{}-friend &
\scshape pl &
3-meet &
 &
 &
 &
 &
 &
\\
 &
 &
\multicolumn{9}{m{6.24306in}}{{\textquoteleft}His friends met him{\textquoteright}/{\textquoteleft}(He) met his friends.{\textquoteright} 

(*{\textquoteleft}(He) met his friend.{\textquoteright}; *{\textquoteleft}(They) met their friend.{\textquoteright})}\\
 &
 &
\multicolumn{9}{m{6.24306in}}{}\\
 &
c. &
\itshape Ininok  &
\itshape onon &
\itshape ge-ebeng &
\itshape go-thook. &
 &
 &
 &
 &
\\
 &
 &
person  &
\scshape pl &
3.\textsc{gen}{}-friend &
3-meet &
 &
 &
 &
 &
\\
 &
 &
\multicolumn{9}{m{6.24306in}}{{\textquoteleft}The people met their friend.{\textquoteright}}\\
\end{supertabular}
\end{flushleft}
\footnotetext{Compare \textit{Iniq ge-ebeng go-thook} {\textquoteleft}They met his friend(s){\textquoteright}, where the non-singular pronoun \textit{iniq} encodes the subject (Baird, p.c. ).}
While plural words only occur with third person referents, none of the languages seems to have semantic restrictions on which referents can be marked plural. For instance, in all the languages we examined, both animate and inanimate entities can be pluralised. There does not seem to be a preference to use a plural word more often with animate than with animate nouns, or vice versa. In Wersing, for example, the plural word can be used to signal the plurality of a human (3), animal (4) or inanimate referent (5). There is similarly no difference in the plural marking of large versus small referents, as illustrated for Western Pantar \textit{raya }{\textquoteleft}chief{\textquoteright} (6) and \textit{bal }{\textquoteleft}ball{\textquoteright} (7). \textit{Bal marung }{\textquoteleft}ball \textsc{pl{\textquoteright} }in (7) refers to an unspecified number of balls. This can be a small number of balls, say two or three; it does not have to be a large number of balls.

\clearpage
Wersing (Schapper and Hendery, Wersing corpus)

\begin{flushleft}
\tablehead{}
\begin{supertabular}{m{0.30665985in}m{0.27885985in}m{0.40455985in}m{0.44275984in}m{0.55385983in}m{0.40455985in}m{0.7851598in}m{0.43305984in}m{0.58375984in}m{0.5705598in}m{1.3684598in}}
 (3) &
{\dots}, &
\itshape saku &
\itshape deing &
\itshape bias &
\itshape ol &
\itshape tamu &
\itshape poko &
\itshape dein=a &
\itshape ge-pai &
\itshape ge-tai...\\
 &
 &
adult &
\scshape pl &
usually &
child &
grandchild &
small &
\scshape pl=art &
\textsc{3-}make &
3-sleep\\
 &
 &
\multicolumn{9}{m{6.1766596in}}{{\textquoteleft}{\dots}, the adults would usually [do it] to make the children and grandchildren sleep{\dots}{\textquoteright}}\\
\end{supertabular}
\end{flushleft}
Wersing (Schapper and Hendery, Wersing corpus)

\begin{flushleft}
\tablehead{}
\begin{supertabular}{m{0.26505986in}m{0.8601598in}m{0.37755984in}m{0.66565984in}m{0.44205984in}m{0.32815984in}m{0.6219598in}m{0.8281598in}m{0.075459845in}m{0.08795985in}}
(4) &
\itshape Ne-karbau &
\itshape wari &
\itshape ne-wai &
\itshape deing &
\itshape na &
\itshape yeta &
\textit{le-gadar.}\footnotemark{} &
 &
\\
 &
\textsc{1sg}\textsc{{}-}buffalo &
and &
\textsc{1sg-}goat &
\scshape pl &
\scshape foc &
\scshape 2pl.agt &
\textsc{appl-}guard &
 &
\\
 &
\multicolumn{7}{m{4.59616in}}{{\textquoteleft}You watch out for my buffalos and my goats.{\textquoteright}} &
 &
\\
\end{supertabular}
\end{flushleft}
\footnotetext{Here the plural word must have scope over both nouns, such that this example cannot be read to mean {\textquotedblleft}my buffalo and my goats{\textquotedblright}.}
Wersing (Schapper and Hendery, Wersing corpus)

\begin{flushleft}
\tablehead{}
\begin{supertabular}{m{0.26505986in}m{0.5066598in}m{0.44205984in}m{0.34005985in}m{0.6087598in}m{0.46015987in}m{0.075459845in}m{0.075459845in}m{0.07475985in}m{0.08725984in}}
(5) &
\itshape Kiki &
\itshape deing &
\itshape aso &
\itshape ge-mira &
\itshape susa. &
 &
 &
 &
\\
 &
flower &
\scshape pl &
also &
3-inside &
suffer &
 &
 &
 &
\\
 &
\multicolumn{5}{m{2.6726599in}}{{\textquoteleft}The flowers were also suffering.{\textquoteright}} &
 &
 &
 &
\\
\end{supertabular}
\end{flushleft}
Western Pantar (Holton 2012)

\begin{flushleft}
\tablehead{}
\begin{supertabular}{m{0.39135984in}m{1.0754598in}m{0.76365983in}m{0.7893598in}m{0.67815983in}m{0.5483598in}m{0.42195985in}m{0.35255986in}m{0.35315984in}m{0.35595986in}}
(6) &
\itshape Raya &
\itshape marung &
\itshape wang &
\itshape hundar. &
 &
 &
 &
 &
\\
 &
chief &
\scshape pl &
exist &
amazed &
 &
 &
 &
 &
\\
 &
\multicolumn{9}{m{5.96856in}}{{\textquoteleft}The chiefs were amazed.{\textquoteright} (*{\textquoteleft}The chief is amazed.{\textquoteright})}\\
\end{supertabular}
\end{flushleft}
Western Pantar (Holton 2012)

\begin{flushleft}
\tablehead{}
\begin{supertabular}{m{0.39135984in}m{0.6622598in}m{0.76295984in}m{0.5094598in}m{0.67885983in}m{0.5476598in}m{0.42125985in}m{0.35315984in}m{0.35255986in}m{0.35665986in}}
(7) &
\itshape Bal  &
\itshape marung &
\itshape mea &
\itshape tang &
\itshape pering. &
 &
 &
 &
\\
 &
ball &
\scshape pl &
table &
on &
pour &
 &
 &
 &
\\
 &
\multicolumn{9}{m{5.2747602in}}{{\textquoteleft}A bunch of balls are spread out on the table.{\textquoteright}}\\
\end{supertabular}
\end{flushleft}
Where the plural words do differ from plural affixes in other languages is in their shape and distribution: they are for the most part free word forms, and they need not occur next to the noun they pluralise. This is illustrated in (8), where Teiwa \textit{non }occurs next to the adjective \textit{sib }{\textquoteleft}clean{\textquoteright} while pluralising \textit{gakon }{\textquoteleft}his shirt{\textquoteright}. Similarly in (9) we see Adang \textit{nun} follows the verb \textit{mat}\textit{{\textepsilon}} {\textquoteleft}large{\textquoteright} modifying the head noun \textit{ti} {\textquoteleft}tree{\textquoteright}.

Teiwa (Klamer, Teiwa corpus) 

\begin{flushleft}
\tablehead{}
\begin{supertabular}{m{0.26505986in}m{0.5150598in}m{0.48865983in}m{1.0323598in}m{0.42335984in}m{0.32055986in}m{0.5275598in}m{0.43165985in}m{0.5150598in}m{0.07885984in}}
(8) &
\itshape Uy &
\itshape masar &
\itshape ga-kon  &
\itshape sib &
\itshape non &
\textit{ga}\textit{[241?]}\textit{an,} &
\itshape ma &
\textit{tona}\textit{[241?]}\textit{.} &
\\
 &
person &
male &
\textsc{3sg.poss}{}-shirt &
clean &
\scshape pl &
\scshape dem &
come &
collect &
\\
 &
\multicolumn{9}{m{4.96306in}}{{\textquoteleft}Those clean shirts of that man, collect them.{\textquoteright} }\\
\end{supertabular}
\end{flushleft}
Adang (Robinson and Haan to appear)\textit{\ \ \ \ }

\begin{flushleft}
\tablehead{}
\begin{supertabular}{m{0.26505986in}m{0.33095986in}m{0.32125986in}m{0.40455985in}m{0.32125986in}m{0.7344598in}m{0.075459845in}m{0.075459845in}m{0.075459845in}m{0.08655984in}}
(9) &
\itshape Pen &
\itshape ti &
\textit{mat}\textit{{\textepsilon}} &
\itshape nun &
\textit{[241?]a-b}\textit{{\textopeno}}\textit{[241?]}\textit{{\textopeno}}\textit{i.} &
 &
 &
 &
\\
 &
Pen &
tree &
large &
\scshape pl &
\textsc{3i.obj}{}-cut &
 &
 &
 &
\\
 &
\multicolumn{5}{m{2.4274597in}}{{\textquoteleft}Pen cut some large trees.{\textquoteright}} &
 &
 &
 &
\\
\end{supertabular}
\end{flushleft}
Plural words in AP languages cannot co-occur with a numeral in a single NP. For instance, in Teiwa, a noun can be pluralised with either a plural word or with a numeral (plus optional classifier) (10a-b), but not with both at the same time (10c). Adang shows the same restriction; the plural word \textit{nun} cannotco-occur with a numeral, compare (11a-b).\footnote{A combination of a mass noun and a numeral is also ungrammatical: *\textit{s}\textit{{\textepsilon}}\textit{i} \textit{ut }{\textquoteleft}water four{\textquoteright} (Haan 2001: 296).}

Teiwa (Klamer, Teiwa corpus) 

\begin{flushleft}
\tablehead{}
\begin{supertabular}{m{0.34905985in}m{0.31155986in}m{0.36775985in}m{0.43165985in}m{0.46985987in}m{0.32055986in}m{0.07615984in}m{0.19555986in}m{0.36775985in}m{0.43235984in}m{0.47195986in}}
\label{bkm:Ref354063073}(10) &
a.  &
\itshape war &
\itshape non &
 &
 &
 &
b.  &
\itshape war &
\itshape (bag) &
\itshape haraq\\
 &
 &
rock &
\scshape pl &
 &
 &
 &
 &
rock &
\scshape clf &
two\\
 &
 &
\multicolumn{4}{m{1.8260599in}}{{\textquoteleft}(several/many) rocks{\textquoteright}} &
 &
 &
\multicolumn{3}{m{1.42956in}}{{\textquoteleft}two rocks{\textquoteright}}\\
 &
 &
\multicolumn{9}{m{3.76356in}}{}\\
 &
c. * &
\itshape war &
\itshape (bag) &
\itshape haraq  &
\itshape non &
 &
 &
 &
 &
\\
 &
 &
rock &
\scshape clf &
two &
\scshape pl &
 &
 &
 &
 &
\\
 &
 &
\multicolumn{9}{m{3.76356in}}{Intended: {\textquoteleft}two rocks{\textquoteright}}\\
\end{supertabular}
\end{flushleft}
Adang (Robinson and Haan to appear)

\begin{flushleft}
\tablehead{}
\begin{supertabular}{m{0.34835985in}m{0.35595986in}m{0.5247598in}m{0.34975985in}m{0.31985986in}m{0.48725984in}m{0.31225985in}m{0.41435984in}m{1.5108598in}m{0.35315984in}}
(11) &
a. &
\textit{s}\textit{{\textepsilon}}\textit{i} &
\itshape nun &
\itshape ho &
\textit{{\textglotstop}}\textit{uhu}\textit{{\textltailn}} &
\itshape {\textepsilon} &
\textit{b}\textit{{\textepsilon}}\textit{{\ng}} &
\itshape tanib &
\\
 &
 &
water &
\scshape pl &
\scshape def &
pour &
and &
other &
draw.water.from.well &
\\
 &
 &
\multicolumn{8}{m{4.8234596in}}{{\textquoteleft}Pour out that little bit of water and get some more from the well.{\textquoteright}}\\
\end{supertabular}
\end{flushleft}
\begin{flushleft}
\tablehead{}
\begin{supertabular}{m{0.39005986in}m{0.31435984in}m{0.5247598in}m{0.34975985in}m{0.32055986in}m{0.32055986in}m{0.48795983in}m{0.31155986in}m{0.41365984in}m{1.5108598in}m{0.46705982in}}
 &
b. &
*\textit{s}\textit{{\textepsilon}}\textit{i} &
\itshape nun &
\textit{al}\textit{{\textopeno}} &
\itshape ho &
\textit{{\textglotstop}}\textit{uhu}\textit{{\textltailn}} &
\itshape {\textepsilon} &
\textit{b}\textit{{\textepsilon}}\textit{{\ng}} &
\itshape tanib &
\\
 &
 &
water &
\scshape pl &
two &
\scshape def &
pour &
and &
other &
draw.water.from.well &
\\
 &
 &
\multicolumn{9}{m{5.3366594in}}{Intended: {\textquoteleft}Pour out the two bits of water and get some more from the well.{\textquoteright}}\\
\end{supertabular}
\end{flushleft}
In sum, proto-Alor-Pantar had a plural word of the shape *non. Some Alor-Pantar languages inherited both form and function, others innovated a plural word. The languages under investigation do not show restrictions on which referents can be marked plural, and in none of the languages does the plural word co-occur with a numeral in an NP.

\section[3\ \ Syntax of plural words in Alor{}-Pantar ]{3\ \ Syntax of plural words in Alor-Pantar }
The plural words investigated in Dryer (1989) are very heterogeneous in their categorial properties. They belong to one of the following classes: (i) articles; (ii) numerals; (iii) grammatical number words like singular, dual, trial; (iv) closed class of noun modifiers; and (v) a class of their own. Dryer concludes that {\textquotedblleft}there is little basis for using the term [plural word] as a syntactic category{\textquotedblright} (1989: 879). 

In this section, we investigate the syntax of plural words in Western Pantar (section 3.1), Teiwa (section 3.2), Kamang (section 3.3), Abui (section 3.4) and Wersing (3.5). For each language, we describe the template of the NP as well as the position and combinatorial properties of the plural word. We confirm Dryer{\textquoteright}s observation that there is little syntactic unity in plural words across languages. Our description focuses on the following issues: 

(i)\ \ Does the plural word occur in the NP? 

(ii)  \ \ How does the plural word behave in respect to quantifiers in the NP? 

(iii) \ \ Can the plural word alone form an NP? 

The languages under discussion differentiate the plural word from other syntactic classes. We will see that significant variation exists in terms of which syntactic class the plural word class resembles most. In Wersing, the plural word shares many properties with nouns, while in Kamang the plural word is most similar to pronouns. In Western Pantar and Teiwa, the plural words are comparable with numerals and quantifiers. 

\subsection[3.1\ \ Western Pantar ]{3.1\ \ Western Pantar }
The template of the Western Pantar NP is presented in (12) (Holton, to appear). The NP is maximally composed of a head noun (N) followed by an adjective in the attribute slot (\textsc{Attr),} followed by numeral phrases with an optional classifier (\textsc{(Clf) Num) }or a plural word (\textsc{Pl), }a demonstrative \textsc{(Dem)} and an article \textsc{(Art)}. Western Pantar has no dedicated slot for (non-numeral) quantifiers, as these behave like adjectives or like nouns: adjectival quantifiers go in the A\textsc{ttr} slot (13), while nominal quantifiers occur in apposition to the NP, to the right of the article (14).

~(12)\ \ Template of the Western Pantar NP\footnote{Western Pantar does not have relative clauses.~}

\ \ [\textsc{N  Attr\{(Clf) Num / Pl\}Dem  Art]}\textsc{\textsubscript{NP}}

Western Pantar (Holton to appear)

\begin{flushleft}
\tablehead{}
\begin{supertabular}{m{0.38445985in}m{1.0205599in}m{0.7268598in}m{0.49695984in}m{0.64835984in}m{0.54275984in}m{0.6962598in}m{0.30805984in}m{0.30875984in}m{0.31225985in}}
(13) &
\itshape Wakke-wakke &
\itshape haweri &
\itshape wang &
\itshape Tubbe &
\itshape birang &
\itshape kalalang. &
 &
 &
\\
 &
child\~{}\textsc{rdp} &
many &
exist &
T. &
speak &
know &
 &
 &
\\
 &
\multicolumn{9}{m{5.6907597in}}{ {\textquoteleft}Most/many children can speak the Tubbe language.{\textquoteright}}\\
\end{supertabular}
\end{flushleft}
Western Pantar (Holton to appear)

\begin{flushleft}
\tablehead{}
\begin{supertabular}{m{0.38515985in}m{1.0163599in}m{0.7393598in}m{0.5462598in}m{0.6441598in}m{0.7344598in}m{0.6504598in}m{0.31435984in}m{0.31365985in}m{0.31845984in}}
(14) &
[[\textit{Hai} &
\itshape bloppa &
\textit{sing}]\textsubscript{NP} &
\textit{der}]\textsubscript{NP} &
\itshape ga-r &
\itshape diakang. &
 &
 &
\\
 &
\scshape 2sg.poss &
weapon &
\scshape art &
some &
3\textsc{sg-}with &
descend &
 &
 &
\\
 &
\multicolumn{9}{m{5.9074593in}}{ {\textquoteleft}Bring down some of your weapons.{\textquoteright} [publia152]}\\
\end{supertabular}
\end{flushleft}
Nominal plurality is expressed by the plural word \textit{maru(ng)}, (15).\footnote{\textit{Marung} has cognate forms in three AP languages: Klon \textit{maang}, Kula\textit{ }\textit{araman} (with liquid nasal metathesis) and Sawila \textit{maarang} (Schapper and Huber, ms.)} The use of numerals is illustrated in (16), and (17)-(18) show that numeral and plural word do not co-occur in a single NP.

Western Pantar (Holton 2012)

\begin{flushleft}
\tablehead{}
\begin{supertabular}{m{0.39135984in}m{0.6622598in}m{0.76295984in}m{0.5094598in}m{0.67885983in}m{0.5476598in}m{0.42125985in}m{0.35315984in}m{0.35255986in}m{0.35665986in}}
(15) &
\itshape Bal  &
\itshape marung &
\itshape mea &
\itshape tang &
\itshape pering. &
 &
 &
 &
\\
 &
ball &
\scshape pl &
table &
on &
pour &
 &
 &
 &
\\
 &
\multicolumn{9}{m{5.2747602in}}{{\textquoteleft}A bunch of balls are spread out on the table.{\textquoteright}}\\
\end{supertabular}
\end{flushleft}
Western Pantar (Holton 2012)

\begin{flushleft}
\tablehead{}
\begin{supertabular}{m{0.38865986in}m{0.6073598in}m{0.69765985in}m{0.50255984in}m{0.6420598in}m{0.5476598in}m{0.43165985in}m{0.36775985in}m{0.5393598in}m{0.31155986in}}
\label{bkm:Ref334530759}(16) &
\itshape Bal &
\itshape ara  &
\itshape atiga, &
\itshape kalla &
\itshape yasing, &
\itshape mea  &
\itshape tang &
\textit{ti}\textit{{\textglotstop}}\textit{ang.} &
\\
 &
ball &
large &
three &
small &
five &
table &
on  &
set &
\\
 &
\multicolumn{9}{m{5.2775598in}}{{\textquoteleft}Three large balls and five small balls are sitting on the table.{\textquoteright} }\\
\end{supertabular}
\end{flushleft}
Western Pantar (Holton 2012)

\begin{flushleft}
\tablehead{}
\begin{supertabular}{m{0.39065984in}m{0.39485985in}m{0.65875983in}m{0.76085985in}m{0.5080598in}m{0.36225986in}m{0.6754598in}m{0.5455598in}m{0.44135985in}m{0.34975985in}m{0.35045984in}m{0.35595986in}}
 (17) &
a. * &
\textit{ke}\textit{{\textglotstop}}\textit{e } &
\itshape kealaku &
\itshape maru &
b. * &
\textit{ke}\textit{{\textglotstop}}\textit{e} &
\itshape bina &
\itshape maru &
 &
 &
\\
 &
 &
fish  &
twenty &
\scshape pl &
 &
fish &
\scshape clf &
\scshape pl &
 &
 &
\\
 &
 &
\multicolumn{3}{m{2.0851598in}}{Intended: \textsc{{\textquoteleft}}twenty fish{\textquoteright}} &
 &
\multicolumn{6}{m{3.1122599in}}{Intended: {\textquoteleft}twenty fish{\textquoteright}}\\
\end{supertabular}
\end{flushleft}
Western Pantar (Holton 2012)

\begin{flushleft}
\tablehead{}
\begin{supertabular}{m{0.38025984in}m{0.39555985in}m{0.60455984in}m{0.6823598in}m{0.6316598in}m{0.40385985in}m{0.6170598in}m{0.5163598in}m{0.40875986in}m{0.28585985in}m{0.28515986in}m{0.29065984in}}
(18) &
a. * &
\textit{ke}\textit{{\textglotstop}}\textit{e } &
\itshape maru &
\itshape kealaku &
b. * &
\textit{ke}\textit{[241?]}\textit{e} &
\itshape maru  &
\itshape bina &
 &
 &
\\
 &
 &
fish  &
\scshape pl &
twenty &
 &
fish &
\scshape pl &
\scshape clf &
 &
 &
\\
 &
 &
\multicolumn{3}{m{2.0760598in}}{Intended: {\textquoteleft}twenty fish{\textquoteright}} &
 &
\multicolumn{6}{m{2.7975597in}}{Intended: {\textquoteleft}twenty fish{\textquoteright}}\\
\end{supertabular}
\end{flushleft}
\textit{Maru(ng)} cannot substitute for a whole NP and function independently as a verbal argument, compare (19a) and (19b). 

Western Pantar (Holton 2012)

\begin{flushleft}
\tablehead{}
\begin{supertabular}{m{0.38935986in}m{0.43445984in}m{0.6747598in}m{0.5900598in}m{0.5455598in}m{0.5448598in}m{0.5400598in}m{0.41565984in}m{0.34765986in}m{0.34835985in}m{0.35045984in}}
(19)  &
a. &
\itshape Raya &
\itshape marung &
\itshape lama &
\itshape ta. &
 &
 &
 &
 &
\\
 &
 &
chief &
\scshape pl &
walk &
\scshape ipfv &
 &
 &
 &
 &
\\
 &
 &
\multicolumn{9}{m{4.98736in}}{{\textquoteleft}The chiefs walk.{\textquoteright}}\\
\end{supertabular}
\end{flushleft}
\begin{flushleft}
\tablehead{}
\begin{supertabular}{m{0.38935986in}m{0.40455985in}m{0.65805984in}m{0.7358598in}m{0.49835983in}m{0.6309598in}m{0.51155984in}m{0.39485985in}m{0.33095986in}m{0.33025986in}m{0.33585986in}}
 &
b. * &
\itshape Marung &
\itshape lama &
\itshape ta. &
 &
 &
 &
 &
 &
\\
 &
 &
\scshape pl &
walk &
\scshape ipfv &
 &
 &
 &
 &
 &
\\
 &
 &
\multicolumn{9}{m{5.0566597in}}{Intended: {\textquoteleft}They walk.{\textquoteright}}\\
\end{supertabular}
\end{flushleft}
\ \ In sum, Western Pantar \textit{marung }can only be used as a nominal attribute within an NP. It is in complementary distribution with adjectival quantifiers and numerical expressions and lacks nominal properties.

\subsection[3.2\ \ Teiwa]{3.2\ \ Teiwa}
The template of the Teiwa NP is presented in \textbf{(}20). The NP is maximally composed of a head noun (N) followed by an attributive (\textsc{Attr) }noun, derived nominal or adjective\textsc{,} followed by a numeral phrase (indicated by \{\})consisting of either a numeral with an optional classifier (\textsc{(Clf) Num) }or a plural word with an optional quantifier (\textsc{Pl (Q)), }a demonstrative \textsc{(Dem)} and a demonstrative particle in the article (\textsc{Art) }slot.

(20)\ \ Template of the Teiwa NP\footnote{Teiwa has no relative clauses; nominal referents are focused with the focus particle \textit{la }(Klamer 2010). }

\ \ [\textsc{N  Attr\{(Clf) Num / Pl (Q)\}  Dem Art ]}\textsc{\textsubscript{NP}}

In the \textsc{Dem }slot, we often find \textit{ga}\textit{[241?]}\textit{an }(glossed as {\textquoteleft}that.\textsc{knwn}{\textquoteright}), a 3\textsc{sg }object pronoun that also functions as a demonstrative modifier of nouns. In the \textsc{Art} slot are the demonstrative particles \textit{u }{\textquoteleft}\textsc{distal{\textquoteright}} and \textit{a }{\textquoteleft}\textsc{proximate{\textquoteright}}. These particles occupy the NP-final position, marking definiteness and/or the location of NP referent with respect to the speaker.

The plural word has its own slot within the NP. It cannot combine with numeral constituents as those in (21a); compare (21b) with (22a-c). However, \textit{non }can be combined with a quantifier in an NP, as shown in (23) and (24). Note that \textit{dum }{\textquoteleft}many/much{\textquoteright} is used contrastively here.

Teiwa (Klamer, Teiwa corpus) 

\begin{flushleft}
\tablehead{}
\begin{supertabular}{m{0.48795983in}m{0.31505984in}m{0.51225984in}m{0.5573598in}m{0.6101598in}m{0.6622598in}m{0.31505984in}m{0.5573598in}m{0.40665984in}}
\label{bkm:Ref354653126}(21)  &
a. &
\itshape war &
\itshape (bag)  &
\itshape haraq &
 &
b. &
\itshape war &
\itshape non\\
 &
 &
rock &
\scshape clf &
two &
 &
 &
rock &
\scshape pl\\
 &
 &
\multicolumn{4}{m{2.57826in}}{\textsc{{\textquoteleft}}two rocks{\textquoteright}} &
 &
\multicolumn{2}{m{1.0427599in}}{{\textquoteleft}rocks{\textquoteright}}\\
\end{supertabular}
\end{flushleft}
Teiwa (Klamer, Teiwa corpus) 

\begin{flushleft}
\tablehead{}
\begin{supertabular}{m{0.48795983in}m{0.32195985in}m{0.7990598in}m{0.47195986in}m{0.47755983in}m{0.58655983in}m{0.31505984in}m{0.7247598in}m{0.42475984in}m{0.42475984in}m{0.42815986in}}
\label{bkm:Ref335059934}(22) &
a. * &
\itshape war &
\itshape haraq &
\itshape non &
 &
 &
 &
 &
 &
\\
 &
 &
rock &
two &
\scshape pl &
 &
 &
 &
 &
 &
\\
 &
 &
\multicolumn{9}{m{5.2825594in}}{Intended: {\textquoteleft}two rocks{\textquoteright}}\\
 &
 &
\multicolumn{9}{m{5.2825594in}}{}\\
 &
b. * &
\itshape war &
\itshape bag &
\itshape haraq &
\itshape non &
 &
 &
 &
 &
\\
 &
 &
rock &
\scshape clf &
two &
\scshape pl &
 &
 &
 &
 &
\\
 &
 &
\multicolumn{9}{m{5.2825594in}}{Intended: {\textquoteleft}two rocks{\textquoteright}}\\
 &
 &
\multicolumn{9}{m{5.2825594in}}{}\\
 &
c. * &
\itshape war &
\itshape bag &
\itshape non &
 &
 &
 &
 &
 &
\\
 &
 &
rock &
\scshape clf &
\scshape pl &
 &
 &
 &
 &
 &
\\
 &
 &
\multicolumn{9}{m{5.2825594in}}{Intended: {\textquoteleft}two rocks{\textquoteright}}\\
\end{supertabular}
\end{flushleft}
Teiwa (Klamer, Teiwa corpus) 

\begin{flushleft}
\tablehead{}
\begin{supertabular}{m{0.48445985in}m{0.6636598in}m{0.49905983in}m{0.5052598in}m{0.66705984in}m{0.5052598in}m{0.5163598in}m{0.48515984in}m{0.48515984in}m{1.2698599in}}
\label{bkm:Ref354653269}(23) &
\itshape Hala &
[\textit{qavif} &
\itshape non &
\textit{dum}]\textsubscript{ NP} &
\itshape pin &
\textit{aria}\textit{{\textglotstop}}\textit{?} &
 &
 &
\\
 &
someone &
goat &
\scshape pl &
many &
hold &
arrive &
 &
 &
\\
 &
\multicolumn{9}{m{6.22676in}}{{\textquoteleft}Were many [rather than few] goats brought here?{\textquoteright} }\\
\end{supertabular}
\end{flushleft}
Teiwa (Klamer, Teiwa corpus) 

\begin{flushleft}
\tablehead{}
\begin{supertabular}{m{0.48445985in}m{0.5990598in}m{0.49835983in}m{0.66705984in}m{0.5663598in}m{0.5052598in}m{0.48585984in}m{0.48515984in}m{0.48585984in}m{0.48935983in}}
(24) &
[\textit{Wat} &
\itshape non &
\textit{dum}]\textsubscript{ NP} &
\itshape usan &
\itshape ma! &
 &
 &
 &
\\
 &
coconut &
\scshape pl &
many &
pick.up &
come &
 &
 &
 &
\\
 &
\multicolumn{9}{m{5.41226in}}{{\textquoteleft}Pick up the many coconuts.{\textquoteright} [situation: there are many coconuts in a pile of various kinds of fruits, and the order is to pick up these, not the rest]}\\
\end{supertabular}
\end{flushleft}
\textit{Non }does not substitute for an NP and cannot function independently as a verbal argument, either with or without the distal demonstrative particle \textit{u }that functions as an (grammatically optional) article in (25b-c). It must always remain part of the NP, as shown by the ungrammaticality of (25d). 

Teiwa (Klamer, Teiwa corpus) 

\begin{flushleft}
\tablehead{}
\begin{supertabular}{m{0.42475984in}m{0.46365985in}m{0.7129598in}m{0.47195986in}m{0.47685984in}m{0.7580598in}m{0.44275984in}m{0.42545983in}m{0.42475984in}m{0.42545983in}m{0.42885986in}}
(25) &
a. &
[\textit{G-oqai  } &
\itshape non &
\textit{u}]\textsubscript{ NP} &
\itshape min-an &
\itshape tau. &
 &
 &
 &
\\
 &
 &
\textsc{3sg}{}-child &
\scshape pl &
\scshape dist &
die-\textsc{real} &
\scshape prf &
 &
 &
 &
\\
 &
 &
\multicolumn{9}{m{5.1970596in}}{{\textquoteleft}Her children (lit. those her children) have died.{\textquoteright}}\\
\end{supertabular}
\end{flushleft}
\begin{flushleft}
\tablehead{}
\begin{supertabular}{m{0.42195985in}m{0.46365985in}m{0.71015984in}m{0.47195986in}m{0.7587598in}m{0.7566598in}m{0.44065985in}m{0.42405984in}m{0.42405984in}m{0.42405984in}m{0.42615983in}}
 &
b. * &
[\textit{Non} &
\textit{u}]\textsubscript{ NP} &
\itshape min-an &
\itshape tau. &
 &
 &
 &
 &
\\
 &
 &
\scshape pl &
\scshape dist &
die-\textsc{real} &
\scshape prf &
 &
 &
 &
 &
\\
 &
 &
\multicolumn{9}{m{5.46646in}}{Intended: {\textquoteleft}They have died.{\textquoteright}}\\
\end{supertabular}
\end{flushleft}
\begin{flushleft}
\tablehead{}
\begin{supertabular}{m{0.42265984in}m{0.46365985in}m{0.70945984in}m{0.7587598in}m{0.7566598in}m{0.7538598in}m{0.44005987in}m{0.42195985in}m{0.42265984in}m{0.42195985in}m{0.42545983in}}
 &
c. * &
[\textit{Non}]\textsubscript{ NP} &
\itshape min-an &
\itshape tau. &
 &
 &
 &
 &
 &
\\
 &
 &
\scshape pl &
die-\textsc{real} &
\scshape prf &
 &
 &
 &
 &
 &
\\
 &
 &
\multicolumn{9}{m{5.74076in}}{Intended: {\textquoteleft}They have died.{\textquoteright}}\\
\end{supertabular}
\end{flushleft}
\begin{flushleft}
\tablehead{}
\begin{supertabular}{m{0.42265984in}m{0.46365985in}m{0.70945984in}m{0.7587598in}m{0.7566598in}m{0.7538598in}m{0.44005987in}m{0.42195985in}m{0.42265984in}m{0.42195985in}m{0.42545983in}}
 &
d. * &
[\textit{G-oqai  } &
\textit{u}]\textsubscript{NP} &
\itshape non &
\itshape min-an &
\itshape tau. &
 &
 &
 &
\\
 &
 &
\textsc{3sg}{}-child &
\scshape dist &
\scshape pl &
die-\textsc{real} &
\scshape prf &
 &
 &
 &
\\
 &
 &
\multicolumn{9}{m{5.74076in}}{Intended: {\textquoteleft}Her children (they) have died.{\textquoteright}}\\
\end{supertabular}
\end{flushleft}
Just as Western Pantar \textit{maru(ng)}, Teiwa \textit{non }can occur in an NP that stands in apposition with a pronoun (26):

Teiwa (Klamer, Teiwa corpus) 

\begin{flushleft}
\tablehead{}
\begin{supertabular}{m{0.48445985in}m{0.7490598in}m{0.6302598in}m{0.5059598in}m{0.5663598in}m{0.97335976in}m{0.48515984in}m{0.48515984in}m{0.48515984in}m{0.49005982in}}
(26) &
[\textit{Kemi}  &
\textit{non}]\textsubscript{ NP} &
\itshape iman  &
\itshape xap &
\itshape gu-uyan &
\itshape mat... &
 &
 &
\\
 &
ancestor &
\scshape pl &
they &
bride  &
3.\textsc{obj-}search &
take &
 &
 &
\\
 &
\multicolumn{9}{m{6.0004597in}}{{\textquoteleft}(Our) ancestors (they) searched for brides...{\textquoteright}}\\
\end{supertabular}
\end{flushleft}
It is possible for an NP with \textit{non }to be part of the subject of numeral predication if the numeral predicate also contains a classifier, as illustrated in (27), where \textit{bag }is the generic numeral classifier (Klamer, in press) and combines with \textit{tiaam }{\textquoteleft}six{\textquoteright}. The plural word \textit{non }is part of the subject NP, and is grammatically optional. Subjects pluralised with \textit{non }can thus occur with a numeral predicate. 

However, an NP with \textit{non }cannot be the subject of a quantifier predication with \textit{dum }{\textquoteleft}many/much{\textquoteright}, compare (28a-b). This is because the Teiwa plural word \textit{non }often has the connotation of {\textquoteleft}many{\textquoteright} and {\textquoteleft}plenty{\textquoteright} (see section 4.2). A subject NP like the one in (28) already implies that there are {\textquoteleft}many/plenty goats{\textquoteright}, so that combining it a predicate {\textquoteleft}be many{\textquoteright} in (28b) is semantically redundant. 

Teiwa (Klamer, Teiwa corpus) 

\begin{flushleft}
\tablehead{}
\begin{supertabular}{m{0.48375985in}m{0.7490598in}m{0.6629598in}m{0.5059598in}m{0.5663598in}m{0.8622598in}m{0.48585984in}m{0.48585984in}m{0.48585984in}m{0.48865983in}}
(27) &
[\textit{Ga-qavif} &
\textit{(non)}]\textsubscript{ NP} &
[\textit{un} &
\itshape bag &
\textit{tiaam}]\textsubscript{ Pred} &
 &
 &
 &
\\
 &
\textsc{3sg}{}-goat &
\scshape pl &
\scshape cont &
\scshape clf &
six &
 &
 &
 &
\\
 &
\multicolumn{9}{m{5.92276in}}{{\textquoteleft}His goats are six.{\textquoteright}}\\
\end{supertabular}
\end{flushleft}
Teiwa (Klamer, Teiwa corpus) 

\begin{flushleft}
\tablehead{}
\begin{supertabular}{m{0.48445985in}m{-0.0023401603in}m{0.40805984in}m{0.07475985in}m{0.5615598in}m{0.22815984in}m{0.25455984in}m{0.16505983in}m{0.31775984in}m{0.27405986in}m{0.20875984in}m{0.27955985in}m{0.20315984in}m{0.22405985in}m{0.48515984in}m{0.48585984in}m{0.48515984in}m{0.48865983in}}
(28) &
\multicolumn{2}{m{0.48445985in}}{a.} &
\multicolumn{3}{m{1.0219599in}}{[\textit{Ga-qavif}] \textsubscript{NP}} &
\multicolumn{2}{m{0.49835983in}}{[\textit{un}} &
\multicolumn{2}{m{0.6705598in}}{\textit{dum}]\textsubscript{ Pred}\textit{ }} &
\multicolumn{2}{m{0.5670598in}}{} &
\multicolumn{2}{m{0.5059598in}}{} &
 &
 &
 &
\\
 &
\multicolumn{2}{m{0.48445985in}}{} &
\multicolumn{3}{m{1.0219599in}}{\textsc{3sg}{}-goat} &
\multicolumn{2}{m{0.49835983in}}{\scshape cont} &
\multicolumn{2}{m{0.6705598in}}{many} &
\multicolumn{2}{m{0.5670598in}}{} &
\multicolumn{2}{m{0.5059598in}}{} &
 &
 &
 &
\\
 &
\multicolumn{2}{m{0.48445985in}}{} &
\multicolumn{15}{m{5.83866in}}{{\textquoteleft}His goats are many.{\textquoteright}}\\
 &
\multicolumn{2}{m{0.48445985in}}{} &
\multicolumn{15}{m{5.83866in}}{}\\
\multicolumn{2}{m{0.5608598in}}{} &
\multicolumn{2}{m{0.5615598in}}{b. *} &
[\textit{Ga-qavif} &
\multicolumn{2}{m{0.56145984in}}{\textit{non}]\textsubscript{ NP}} &
\multicolumn{2}{m{0.5615598in}}{[\textit{un}} &
\multicolumn{2}{m{0.56155986in}}{\textit{dum}]\textsubscript{Pred}\textit{ }} &
\multicolumn{2}{m{0.56145984in}}{} &
\multicolumn{5}{m{2.4838598in}}{}\\
\multicolumn{2}{m{0.5608598in}}{} &
\multicolumn{2}{m{0.5615598in}}{} &
\textsc{3sg}{}-goat &
\multicolumn{2}{m{0.56145984in}}{\scshape pl} &
\multicolumn{2}{m{0.5615598in}}{\scshape cont} &
\multicolumn{2}{m{0.56155986in}}{many} &
\multicolumn{2}{m{0.56145984in}}{} &
\multicolumn{5}{m{2.4838598in}}{}\\
\multicolumn{2}{m{0.5608598in}}{} &
\multicolumn{2}{m{0.5615598in}}{} &
\multicolumn{14}{m{5.6851597in}}{Intended: {\textquoteleft}His many/plenty goats are many.{\textquoteright}}\\
\end{supertabular}
\end{flushleft}
The fact that \textit{non} does not combine with a numeral in a single NP suggests that it patterns with the numeral word class. However, unlike numerals, \textit{non }cannot combine with a classifier. On the other hand, \textit{non} can combine with the quantifier \textit{dum} {\textquoteleft}much/many{\textquoteright} in a single NP, which a numeral cannot do. However, at the same time, \textit{non }does not pattern with the class of quantifiers for two reasons. First, such quantifiers can occur as predicates, while \textit{non }cannot, (29a-b); and second, non-numeral quantifiers can occur both inside the NP (30a) as well as outside of it, adjacent to the verb (30b), while \textit{non }must remain within the NP. In (30c) the NP contains \textit{non, }and the ungrammaticality of (30d) shows that \textit{non }cannot occur in the position adjacent to the verb.

Teiwa (Klamer, Teiwa corpus) 

\begin{flushleft}
\tablehead{}
\begin{supertabular}{m{0.48445985in}m{0.48445985in}m{0.7497598in}m{0.49835983in}m{0.6698598in}m{0.5663598in}m{0.5059598in}m{0.48515984in}m{0.48515984in}m{0.48515984in}m{0.49065986in}}
\label{bkm:Ref334184526}(29) &
a. &
\itshape Masar &
[\textit{un} &
\textit{dum}]\textsubscript{ Pred} &
 &
 &
 &
 &
 &
\\
 &
 &
male &
\scshape cont &
many &
 &
 &
 &
 &
 &
\\
 &
 &
\multicolumn{9}{m{5.56636in}}{{\textquoteleft}There are many men.{\textquoteright} (Lit. {\textquoteleft}Males are [being] many.{\textquoteright})}\\
\end{supertabular}
\end{flushleft}
\begin{flushleft}
\tablehead{}
\begin{supertabular}{m{0.48445985in}m{0.48515984in}m{0.7490598in}m{0.49905983in}m{0.5052598in}m{0.5663598in}m{0.5052598in}m{0.48585984in}m{0.48585984in}m{0.48515984in}m{0.49005982in}}
 &
b. * &
\itshape Masar &
[\textit{un} &
\textit{non}]\textit{.} &
 &
 &
 &
 &
 &
\\
 &
 &
male &
\scshape cont &
\scshape pl  &
 &
 &
 &
 &
 &
\\
 &
 &
\multicolumn{9}{m{5.4018598in}}{Intended: {\textquoteleft}There are many/several males.{\textquoteright}}\\
\end{supertabular}
\end{flushleft}
Teiwa (Klamer, Teiwa corpus) 

\begin{flushleft}
\tablehead{}
\begin{supertabular}{m{0.46435985in}m{0.44205984in}m{0.6948598in}m{0.49005982in}m{0.77265984in}m{0.6643598in}m{0.49765983in}m{0.42755982in}m{0.42815986in}m{0.42815986in}m{0.43165985in}}
\label{bkm:Ref334184556}(30) &
a. &
[\textit{Qavif} &
\itshape dum &
\textit{ga}\textit{{\textglotstop}}\textit{an}]\textsubscript{NP} &
\itshape hala  &
\itshape tatax. &
 &
 &
 &
\\
 &
 &
goat &
many &
that.\textsc{knw}\textsc{n} &
someone &
chop &
 &
 &
 &
\\
 &
 &
\multicolumn{9}{m{5.4650593in}}{{\textquoteleft}Many (known) goats were chopped up.{\textquoteright}}\\
\end{supertabular}
\end{flushleft}
\begin{flushleft}
\tablehead{}
\begin{supertabular}{m{0.42475984in}m{0.44065985in}m{0.6906598in}m{0.77265984in}m{0.6643598in}m{0.5483598in}m{0.49695984in}m{0.42545983in}m{0.42545983in}m{0.42545983in}m{0.42685983in}}
 &
b. &
[\textit{Qavif} &
\textit{ga}\textit{{\textglotstop}}\textit{an}]\textsubscript{NP} &
\itshape hala  &
\itshape dum &
\itshape tatax. &
 &
 &
 &
\\
 &
 &
goat &
that.\textsc{knwn} &
someone &
many &
chop &
 &
 &
 &
\\
 &
 &
\multicolumn{9}{m{5.50616in}}{{\textquoteleft}Many of these (known) goats were chopped up.{\textquoteright}}\\
\end{supertabular}
\end{flushleft}
\begin{flushleft}
\tablehead{}
\begin{supertabular}{m{0.44345984in}m{0.45455983in}m{0.70875984in}m{0.45595983in}m{0.77265984in}m{0.6643598in}m{0.46155986in}m{0.44415984in}m{0.44415984in}m{0.44345984in}m{0.44835988in}}
 &
c. &
[\textit{Qavif} &
\itshape non &
\textit{ga}\textit{{\textglotstop}}\textit{an}]\textsubscript{NP} &
\itshape hala  &
\itshape tatax. &
 &
 &
 &
\\
 &
 &
goat &
\scshape pl &
that.\textsc{knwn} &
someone &
chop &
 &
 &
 &
\\
 &
 &
\multicolumn{9}{m{5.4733605in}}{{\textquoteleft}These (known) goats were chopped up by someone.{\textquoteright}}\\
\end{supertabular}
\end{flushleft}
\begin{flushleft}
\tablehead{}
\begin{supertabular}{m{0.42755982in}m{0.44485983in}m{0.6934598in}m{0.77335984in}m{0.6636598in}m{0.5344598in}m{0.49065986in}m{0.42755982in}m{0.42755982in}m{0.42755982in}m{0.43095985in}}
 &
d. * &
[\textit{Qavif} &
\textit{ga}\textit{{\textglotstop}}\textit{an}]\textsubscript{NP} &
\itshape hala  &
\itshape non &
\itshape tatax &
 &
 &
 &
\\
 &
 &
goat &
that.\textsc{knwn} &
someone &
\scshape pl &
chop &
 &
 &
 &
\\
 &
 &
\multicolumn{9}{m{5.49916in}}{Intended: {\textquoteleft}These (known) goats were chopped up.{\textquoteright}}\\
\end{supertabular}
\end{flushleft}
In sum, Teiwa \textit{non }does not have any nominal properties, shares some of the distributional properties of numerals and quantifiers, and constitutes its own syntactic class.\footnote{In addition to the plural word, Teiwa has four dedicated pronoun series for referents of different quantificational types: (i) the dual paradigm (\textit{we two}, etc.), (ii) the {\textquotedblleft}X and they{\textquotedblright} paradigm (\textit{you (sg/pl) and they},\textit{ s/he/they and they};\textit{ I/we (incl/excl) and they}), (iii) the {\textquotedblleft}X alone{\textquotedblright} paradigm (\textit{I alone},\textit{ you alone},\textit{ }etc.) and (iv) the {\textquotedblleft}X as a group of ...{\textquotedblright} paradigm (\textit{we/you/they as a group of x numbers}) (Klamer 2010: 82-85). The plural word cannot co-occur with these pronouns. Teiwa has no associative plural word. To express associative plural notions, a form from the special pronoun series {\textquotedblleft}X and they{\textquotedblright} is used, 
e.g., \textit{Rini  }\textbf{\textit{i-qap}}\textit{ a-kawan aria{\textquoteright} wad  }{\textquoteleft}Rini \textbf{3-and.they} 3-friend arrive today{\textquoteright},  {\textquoteleft}\textit{Today} \textit{Rini arrived with her friends}{\textquoteright}.}

\subsection[3.3\ \ Kamang]{3.3\ \ Kamang}
The template of the Kamang noun phrase (NP) is presented in (31). The NP is maximally composed of a head noun (\textsc{N}) followed by its attribute (\textsc{Attr),} a numeral phrase \textsc{(Num)}, a relative clause (\textsc{Rc}), a demonstrative \textsc{(Dem)} and an article \textsc{(Art)}. The article marks the right edge of an NP and is used to nominalise (i.e., create NPs from) clauses and other non-nominal phrases in the language. In addition, a Kamang NP can occur with a range of items co-referential with it in a slot outside the NP, called here the NP-appositional (\textsc{Appos)} slot (discussed further below). The apposition between an NP and an item in the NP-appositional slot is syntactically tight: there is no intonational break or pause between NP and appositional item, and no item may intervene between them. For more details on the status of the \textsc{Appos }slot or for discussion of the other NP slots, see Schapper (to appear). 

(31)\ \ Template of the Kamang NP (Schapper to appear)

\ \ [\textsc{N}\textsc{\textsubscript{HEAD}}\textsc{  Attr  NumP  Rc  Dem  Art]}\textsc{\textsubscript{NP}}\textsc{ }\textsc{\textsubscript{Appos}}

The Kamang plural word \textit{nung} is conspicuously absent from the above template. In Kamang \textit{nung} does not occur within the NP, but directly follows it. That is, it occurs to the right of the NP article, where one is expressed. For example, in (32) and (33) \textit{nung} follows the specific ({\textquoteleft}\textsc{spec}{\textquoteright}) and definite ({\textquoteleft}\textsc{def}{\textquoteright}) articles respectively. The alternative order with the article following \textit{nung} is not grammatical: *\textit{nung=a} {\textquoteleft}\textsc{pl=spec}{\textquoteright} and *\textit{nung=ak} {\textquoteleft}\textsc{pl=def{\textquoteright}}. In short, \textit{nung} only occurs in the NP-appositional slot.

Kamang (Schapper, fieldnotes)

\begin{flushleft}
\tablehead{}
\begin{supertabular}{m{0.47955987in}m{0.7497598in}m{1.1990598in}m{0.48795983in}m{0.8941598in}m{0.48725984in}m{0.46775988in}m{0.46775988in}m{0.46775988in}m{0.47405985in}}
(32) &
\itshape Almakang &
\itshape laising-laung=a &
\itshape nung &
\textit{ye}\textit{[241?]}\textit{{}-baa} &
\itshape sue. &
 &
 &
 &
\\
 &
people &
youthful=\textsc{spec} &
\scshape pl &
\textsc{3.sben}{}-say &
arrive &
 &
 &
 &
\\
 &
\multicolumn{9}{m{6.3254604in}}{{\textquoteleft}Go tell the young people to come.{\textquoteright}}\\
\end{supertabular}
\end{flushleft}
Kamang (Schapper, fieldnotes)

\begin{flushleft}
\tablehead{}
\begin{supertabular}{m{0.47885987in}m{0.7858598in}m{0.5150598in}m{0.48725984in}m{0.88515985in}m{0.48235986in}m{0.46365985in}m{0.46365985in}m{0.46435985in}m{0.46565983in}}
(33) &
\itshape Muut=ak &
\itshape nung &
\itshape iduka. &
 &
 &
 &
 &
 &
\\
 &
citrus=\textsc{def} &
\scshape pl &
sweet &
 &
 &
 &
 &
 &
\\
 &
\multicolumn{9}{m{5.6429596in}}{{\textquoteleft}The citrus fruits are sweet.{\textquoteright}}\\
\end{supertabular}
\end{flushleft}
By contrast, other Kamang quantifiers can occur within the NP, i.e., to the left of the NP-defining article. Non-numeral quantifiers such as \textit{adu} {\textquoteleft}many/much{\textquoteright} occupy the \textsc{Attr} slot within the NP and cannot float out of it, as seen in (34). 

Kamang (Schapper, fieldnotes)

\begin{flushleft}
\tablehead{}
\begin{supertabular}{m{0.34835985in}m{0.18725985in}m{0.6726598in}m{0.93445987in}m{1.2420598in}m{0.21635985in}m{1.0816599in}m{1.0191599in}}
(34)  &
a. &
\itshape sibe &
\itshape adu=a  &
 &
b. &
*\textit{sibe=a} &
\itshape adu\\
 &
 &
chicken &
many=\textsc{spec} &
 &
 &
chicken=\textsc{spec} &
many\\
 &
 &
\multicolumn{3}{m{3.00666in}}{\textsc{{\textquoteleft}}the many chickens{\textquoteright}} &
 &
\multicolumn{2}{m{2.17956in}}{Intended: \textsc{{\textquoteleft}}the many chickens{\textquoteright}}\\
\end{supertabular}
\end{flushleft}
Kamang does not have a syntactic class of non-numeral quantifiers; items denoting \textit{many}, \textit{few}, \textit{a little}, etc. are adjectives and occur in the \textsc{Attr} slot of the NP. Numeral quantifiers occur with a classifier in the \textsc{NumP}. The unmarked position for the \textsc{NumP} is within the NP to the left of the article (35a), and the marked position is post-posed into the NP-appositional slot outside the NP (35b). The latter position is less frequent and pragmatically marked, functioning to topicalise the enumeration of the NP referent.

Kamang (Schapper, fieldnotes)

\begin{flushleft}
\tablehead{}
\begin{supertabular}{m{0.34835985in}m{0.18725985in}m{0.6733598in}m{0.6865598in}m{0.8511598in}m{0.21635985in}m{0.21565986in}m{1.0816599in}m{0.5254598in}m{0.65875983in}}
(35)  &
a. &
\itshape sibe &
[\textit{uh\ \ }\textit{ } &
\textit{su}]\textsc{\textsubscript{NumP}}\textit{=a} &
 &
b. &
\itshape sibe=a\ \  &
[\textit{uh} &
\textit{su}]\textsc{\textsubscript{NumP}}\\
 &
 &
chicken &
\scshape clf  &
three=\textsc{spec} &
 &
 &
chicken=\textsc{spec} &
\scshape clf &
three\\
 &
 &
\multicolumn{3}{m{2.3685598in}}{{\textquoteleft}the three chickens{\textquoteright}} &
 &
 &
\multicolumn{3}{m{2.4233599in}}{{\textquoteleft}the chickens, the three ones{\textquoteright}}\\
\end{supertabular}
\end{flushleft}
The plural word shares distributional properties in common not only with a \textsc{NumP} but also with a pronoun, since the NP-appositional position can also host a pronoun. In (36) we see that a pronoun (36a) and a plural word (36b) respectively can both occur in the slot following an NP. In these examples, the parts of the free translations in curly brackets are the semantics contributed by the items in the appositional slot.

Kamang (Schapper, fieldnotes)

\begin{flushleft}
\tablehead{}
\begin{supertabular}{m{0.34835985in}m{0.27955985in}m{0.04135984in}m{0.82685983in}m{0.13865983in}m{0.6101598in}m{0.10665984in}m{0.6413598in}m{0.8275598in}m{0.8275598in}}
\multicolumn{3}{m{0.8267599in}}{(36) } &
a. &
\multicolumn{2}{m{0.8275598in}}{\itshape almakang=ak} &
\multicolumn{2}{m{0.8267598in}}{\itshape gera } &
 &
\\
\multicolumn{3}{m{0.8267599in}}{} &
 &
\multicolumn{2}{m{0.8275598in}}{people=\textsc{def}} &
\multicolumn{2}{m{0.8267598in}}{\scshape 3.contr} &
 &
\\
\multicolumn{3}{m{0.8267599in}}{} &
 &
\multicolumn{2}{m{0.8275598in}}{{\textquoteleft}the \{specific group of\} people \{not some other group\}{\textquoteright}} &
\multicolumn{4}{m{2.63936in}}{}\\
\multicolumn{3}{m{0.8267599in}}{} &
 &
\multicolumn{2}{m{0.8275598in}}{} &
\multicolumn{4}{m{2.63936in}}{}\\
 &
b. &
\multicolumn{3}{m{1.1643598in}}{\itshape almakang=ak\ \ } &
\multicolumn{2}{m{0.7955598in}}{\itshape nung} &
\multicolumn{3}{m{2.45396in}}{}\\
 &
 &
\multicolumn{3}{m{1.1643598in}}{people=\textsc{def}} &
\multicolumn{2}{m{0.7955598in}}{\scshape pl} &
\multicolumn{3}{m{2.45396in}}{}\\
 &
 &
\multicolumn{5}{m{2.0386598in}}{{\textquoteleft}the \{multiple\} people{\textquoteright}} &
\multicolumn{3}{m{2.45396in}}{}\\
\end{supertabular}
\end{flushleft}
The Kamang plural word has a distribution similar to that of an NP in two respects. Firstly, \textit{nung} can substitute for a whole NP, where reference is sufficiently clear. For instance, in (37) \textit{nung} is the sole element representing the S of the verb \textit{sue} {\textquoteleft}come{\textquoteright}. Secondly, like an NP, a plural word can itself occur with a pronoun in the NP appositional slot where no NP is expressed, as in (38).

Kamang (Schapper, fieldnotes)

\begin{flushleft}
\tablehead{}
\begin{supertabular}{m{0.44765988in}m{0.7962598in}m{0.46015987in}m{0.5156598in}m{0.6754598in}m{0.48025987in}m{0.6344598in}m{0.46435985in}m{0.41435984in}m{0.38375986in}}
 (37) &
[\textit{Nung}]\textsubscript{ NP} &
\itshape sue. &
 &
 &
 &
 &
 &
 &
\\
 &
\scshape pl &
arrive &
 &
 &
 &
 &
 &
 &
\\
 &
\multicolumn{9}{m{5.45466in}}{{\textquoteleft}\{Multiple\} (people) arrived.{\textquoteright}}\\
\end{supertabular}
\end{flushleft}
Kamang (Schapper, fieldnotes)

\begin{flushleft}
\tablehead{}
\begin{supertabular}{m{0.44765988in}m{0.7691598in}m{0.7226598in}m{0.5170598in}m{0.6754598in}m{0.48025987in}m{0.6344598in}m{0.46435985in}m{0.41435984in}m{0.38235986in}}
(38) &
[\textit{Nung}]\textsubscript{ NP} &
\textit{gera}\textsubscript{APPOS} &
\itshape sue. &
 &
 &
 &
 &
 &
\\
 &
\scshape pl &
\scshape 3.contr &
arrive &
 &
 &
 &
 &
 &
\\
 &
\multicolumn{9}{m{5.69006in}}{{\textquoteleft}\{Multiple other\} (people) arrived.{\textquoteright}}\\
\end{supertabular}
\end{flushleft}
\textit{Nung} is not compatible with any other quantificational items. That is, despite its occurring outside the NP, marking an NP with \textit{nung} means that other quantificational items cannot occur in the NP. This is seen in the examples in (39) where \textit{nung} cannot grammatically co-occur with the numeral quantifier \textit{su} {\textquoteleft}three{\textquoteright} (39a) and with the non-numeral quantifier \textit{adu} {\textquoteleft}many{\textquoteright} (39b). 

Kamang (Schapper, fieldnotes)

\begin{flushleft}
\tablehead{}
\begin{supertabular}{m{0.34905985in}m{0.045559846in}m{0.15455985in}m{0.24005985in}m{0.41085985in}m{-0.016240157in}m{0.33445984in}m{0.060859837in}m{0.34835985in}m{0.5990598in}}
(39)  &
\multicolumn{2}{m{0.27885985in}}{a.} &
\multicolumn{2}{m{0.72965986in}}{*\textit{sibe\ \   }} &
\multicolumn{2}{m{0.39695984in}}{\itshape uh} &
\multicolumn{2}{m{0.48795983in}}{\itshape su} &
\itshape nung\\
 &
\multicolumn{2}{m{0.27885985in}}{} &
\multicolumn{2}{m{0.72965986in}}{chicken} &
\multicolumn{2}{m{0.39695984in}}{\scshape clf} &
\multicolumn{2}{m{0.48795983in}}{three} &
\scshape pl\\
 &
\multicolumn{2}{m{0.27885985in}}{} &
\multicolumn{7}{m{2.4498599in}}{Intended: {\textquoteleft}three chickens{\textquoteright}}\\
 &
\multicolumn{2}{m{0.27885985in}}{} &
\multicolumn{6}{m{1.7720599in}}{} &
\\
\multicolumn{2}{m{0.47335985in}}{} &
\multicolumn{2}{m{0.47335985in}}{b.} &
\multicolumn{2}{m{0.47335985in}}{*\textit{sibe}\textit{\ \ }} &
\multicolumn{2}{m{0.47405985in}}{\itshape adu} &
\multicolumn{2}{m{1.0261599in}}{\itshape nung}\\
\multicolumn{2}{m{0.47335985in}}{} &
\multicolumn{2}{m{0.47335985in}}{} &
\multicolumn{2}{m{0.47335985in}}{chicken} &
\multicolumn{2}{m{0.47405985in}}{many  } &
\multicolumn{2}{m{1.0261599in}}{\scshape pl}\\
\multicolumn{2}{m{0.47335985in}}{} &
\multicolumn{2}{m{0.47335985in}}{} &
\multicolumn{6}{m{2.13106in}}{Intended: {\textquoteleft}many chickens{\textquoteright}}\\
\end{supertabular}
\end{flushleft}
In addition to the plural word, Kamang has multiple dedicated quantificational pronoun series to signal different quantificational types.\footnote{There are four {\textquotedblleft}quantifying{\textquotedblright} pronominal paradigms in Kamang: (i) the {\textquotedblleft}alone{\textquotedblright} paradigm (\textit{I alone/on my own}, \textit{we alone/on our own}, etc.), (ii) the dual paradigm (\textit{we two}, etc.- only in non-singular numbers), (iii) the {\textquotedblleft}all{\textquotedblright} paradigm (\textit{we all}, etc.- only in non-singular numbers), and (iv) the {\textquotedblleft}group{\textquotedblright} paradigm (\textit{we together in a group}, etc.- only in non-singular numbers). See Schapper (to appear) for full set of Kamang pronominal paradigms.} For instance, we see the third person pronouns forms for group plurality and universal quantification in (40) and (41) respectively. The plural word cannot co-occur with these pronouns. 

Kamang (Schapper, fieldnotes)

\begin{flushleft}
\tablehead{}
\begin{supertabular}{m{0.34835985in}m{0.32405984in}m{0.75185984in}m{0.75185984in}m{0.75185984in}m{0.75185984in}}
\multicolumn{2}{m{0.75115985in}}{(40) } &
\textit{Geifu}\textit{  } &
\itshape loo &
\itshape maa. &
\\
\multicolumn{2}{m{0.75115985in}}{} &
3.\textsc{group} &
walk &
go &
\\
\multicolumn{2}{m{0.75115985in}}{} &
\multicolumn{4}{m{3.24366in}}{{\textquoteleft}They go together (as a group).{\textquoteright}}\\
\multicolumn{2}{m{0.75115985in}}{} &
\multicolumn{4}{m{3.24366in}}{}\\
\multicolumn{2}{m{0.75115985in}}{(41)} &
\textit{Gaima}\textit{\ \ } &
\itshape bisa &
\itshape wo-ra=bo &
\itshape pilan.\\
\multicolumn{2}{m{0.75115985in}}{} &
3.\textsc{all} &
can   &
3.\textsc{loc}{}-wear=\textsc{lnk} &
lego-lego\\
 &
\multicolumn{5}{m{3.64646in}}{{\textquoteleft}They all can wear (them) and dance in a lego-lego.{\textquoteright}}\\
\end{supertabular}
\end{flushleft}
The use of quantificational pronouns with NPs is illustrated in (42) and (43). We see in these examples that the quantificational pronouns fill the appositional slot in the same manner as the plural word \textit{nung} and signal the plurality of the referents of the preceding NP. 

Kamang (Schapper, fieldnotes)

\begin{flushleft}
\tablehead{}
\begin{supertabular}{m{0.43445984in}m{0.39765984in}m{0.22125986in}m{0.6108598in}m{0.11435985in}m{0.71845984in}m{0.36365986in}m{0.46845987in}m{0.6198598in}m{0.21565986in}}
\multicolumn{2}{m{0.91085976in}}{(42) } &
\multicolumn{2}{m{0.9108598in}}{[\textit{Mane}} &
\multicolumn{2}{m{0.9115599in}}{\textit{ang}]\textsubscript{NP}} &
\multicolumn{2}{m{0.9108599in}}{\textit{geifu}\textsubscript{ APPOS}\textit{  }} &
\multicolumn{2}{m{0.91425985in}}{\itshape mauu.}\\
\multicolumn{2}{m{0.91085976in}}{} &
\multicolumn{2}{m{0.9108598in}}{village} &
\multicolumn{2}{m{0.9115599in}}{\scshape dem} &
\multicolumn{2}{m{0.9108599in}}{3.\textsc{group}} &
\multicolumn{2}{m{0.91425985in}}{war}\\
\multicolumn{2}{m{0.91085976in}}{} &
\multicolumn{8}{m{3.88376in}}{{\textquoteleft}Those villages make war together (against another village).{\textquoteright}}\\
\multicolumn{2}{m{0.91085976in}}{} &
\multicolumn{8}{m{3.88376in}}{}\\
(43) &
\multicolumn{2}{m{0.6976598in}}{[\textit{Arita\ \ }} &
\multicolumn{2}{m{0.8039598in}}{\textit{pang}]\textsubscript{NP}} &
\multicolumn{2}{m{1.16086in}}{\textit{gaima}\textsubscript{ APPOS}} &
\multicolumn{2}{m{1.1670599in}}{\itshape luaa-ra} &
\textit{lai-ma}\textit{.}\\
 &
\multicolumn{2}{m{0.6976598in}}{leaf} &
\multicolumn{2}{m{0.8039598in}}{\textsc{dem}  } &
\multicolumn{2}{m{1.16086in}}{3.\textsc{all}} &
\multicolumn{2}{m{1.1670599in}}{whither-\textsc{aux}} &
finished-\textsc{pfv}\\
\multicolumn{2}{m{0.91085976in}}{} &
\multicolumn{8}{m{3.88376in}}{{\textquoteleft}All the leaves have withered completely.{\textquoteright}}\\
\end{supertabular}
\end{flushleft}
Finally, Kamang has a suffix marking associative plurality, \textit{{}-lee} {\textquoteleft}\textsc{assoc}{\textquoteright}. This suffix can occur on kin terms or proper names, as in (44) and (45) respectively. Nouns marked by \textit{{}-lee} cannot be modified by any other NP elements. The plural word\textit{ nung} does not occur in such contexts.

Kamang (Schapper, fieldnotes)

\begin{flushleft}
\tablehead{}
\begin{supertabular}{m{0.52195984in}m{0.10315984in}m{0.7045598in}m{0.7045598in}m{0.7045598in}m{0.7045598in}m{0.7052598in}}
\multicolumn{2}{m{0.70385987in}}{(44) } &
\textit{..., }\textit{ } &
\itshape ge-dum-lee &
\itshape see &
\itshape silanta &
\itshape malii\\
\multicolumn{2}{m{0.70385987in}}{} &
 &
3.\textsc{gen}{}-child-\textsc{assoc} &
arrive &
mourn &
mourn\\
 &
\multicolumn{6}{m{4.0203595in}}{{\textquoteleft}{\dots}, her children and their associates come to mourn.{\textquoteright} }\\
\end{supertabular}
\end{flushleft}
Kamang (Schapper, fieldnotes)

\begin{flushleft}
\tablehead{}
\begin{supertabular}{m{0.52195984in}m{0.055959843in}m{0.9823598in}m{0.7865598in}m{0.40045986in}m{0.6080598in}m{0.5080598in}}
(45)  &
\multicolumn{2}{m{1.11706in}}{\itshape Marten-lee} &
\itshape n-at &
\itshape tak. &
 &
\\
 &
\multicolumn{2}{m{1.11706in}}{Marten-\textsc{assoc}} &
1\textsc{sg}{}-from &
run &
 &
\\
\multicolumn{2}{m{0.65665984in}}{} &
\multicolumn{5}{m{3.60046in}}{{\textquoteleft}Marten and his associates run away from me.{\textquoteright}}\\
\end{supertabular}
\end{flushleft}
So, the Kamang plural word occurs outside the NP and shares distributional properties of pronouns. The semantics of the plural word also intersects with pronouns, in particular, the quantificational pronouns whose functions are to denote different number features.

\subsection[3.4\ \ Abui]{3.4\ \ Abui}
The template of the Abui NP is presented in (46).\footnote{The morphosyntactic analysis and glossing of Abui presented here is that of Schapper, and differs from that presented in Kratochv\'il (2007). Examples are individually marked as to source.} The NP is composed of a head noun (N) followed by its attribute (\textsc{Attr). }The Abui plural word \textit{loku }is not etymologically related to the plural word that is reconstructable for pAP. It has a variable position with respect to the relative clause (\textsc{Rc}), being able to either precede or follow the plural word. The plural word occurs inside the NP and thus always occur to left to the determiner (\textsc{Det).}

(46)\ \ Template of the Abui NP

[\textsc{N  Attr  \{Pl  Rc  /  Rc  Pl \} Det}]\textsubscript{NP}

The variable position of \textit{loku }in relation to the relative clause is illustrated in (47) and (48). In (47) \textit{loku} appears after the relative clause but before the demonstrative \textit{yo}. In (48) \textit{loku} precedes both the relative clause and the article \textit{nu}. The two plural word positions are mere variants of one another; extensive elicitation and the examination of corpus data have revealed no difference in the scope or semantics correlating with the plural word{\textquoteright}s position, although corpus frequency and speaker judgments point to the position preceding the relative clause as being preferred. 

Abui (Kratochv\'il, Abui corpus)

\begin{flushleft}
\tablehead{}
\begin{supertabular}{m{0.46505985in}m{0.06705984in}m{0.55315983in}m{-0.021140158in}m{0.6115598in}m{0.31435984in}m{0.21775985in}m{0.18095985in}m{0.35185984in}m{0.57335985in}m{-0.041240156in}m{0.5705598in}m{-0.038440157in}m{0.42685983in}m{0.10595984in}m{0.29205984in}m{0.24005985in}m{0.10525984in}m{0.42815986in}}
(47) &
\multicolumn{2}{m{0.6989598in}}{[\textit{...oto}} &
\multicolumn{3}{m{1.0622599in}}{\itshape he-amakaang} &
\multicolumn{2}{m{0.47745988in}}{[\textit{ba}} &
\multicolumn{2}{m{1.0039599in}}{\itshape h-omi} &
\multicolumn{2}{m{0.6080598in}}{\textit{mia}]\textsubscript{RC}} &
\multicolumn{2}{m{0.46715984in}}{\itshape loku} &
\multicolumn{2}{m{0.47675982in}}{\textit{yo}]\textsubscript{NP}} &
\multicolumn{2}{m{0.42405984in}}{} &
\\
 &
\multicolumn{2}{m{0.6989598in}}{car} &
\multicolumn{3}{m{1.0622599in}}{\textsc{3.gen}{}-person} &
\multicolumn{2}{m{0.47745988in}}{\scshape rel} &
\multicolumn{2}{m{1.0039599in}}{\textsc{3.gen}{}-inside} &
\multicolumn{2}{m{0.6080598in}}{in} &
\multicolumn{2}{m{0.46715984in}}{\scshape pl} &
\multicolumn{2}{m{0.47675982in}}{\scshape dem} &
\multicolumn{2}{m{0.42405984in}}{} &
\\
 &
\multicolumn{18}{m{6.2767596in}}{{\textquoteleft}...those people who were inside the car}\\
 &
\multicolumn{18}{m{6.2767596in}}{}\\
\multicolumn{2}{m{0.6108598in}}{} &
\multicolumn{2}{m{0.61075985in}}{\itshape mi} &
\itshape pak &
\multicolumn{2}{m{0.6108598in}}{\itshape mahoi-ni} &
\multicolumn{2}{m{0.6115598in}}{} &
\multicolumn{2}{m{0.6108599in}}{} &
\multicolumn{2}{m{0.6108598in}}{} &
\multicolumn{2}{m{0.6115598in}}{} &
\multicolumn{2}{m{0.6108598in}}{} &
\multicolumn{2}{m{0.61215985in}}{}\\
\multicolumn{2}{m{0.6108598in}}{} &
\multicolumn{2}{m{0.61075985in}}{take} &
cliff &
\multicolumn{2}{m{0.6108598in}}{gather-\textsc{pfv}} &
\multicolumn{2}{m{0.6115598in}}{} &
\multicolumn{2}{m{0.6108599in}}{} &
\multicolumn{2}{m{0.6108598in}}{} &
\multicolumn{2}{m{0.6115598in}}{} &
\multicolumn{2}{m{0.6108598in}}{} &
\multicolumn{2}{m{0.61215985in}}{}\\
\multicolumn{2}{m{0.6108598in}}{} &
\multicolumn{17}{m{6.1309595in}}{were taken over the [edge of the] cliff.{\textquoteright} }\\
\end{supertabular}
\end{flushleft}
Abui (Kratochv\'il, Abui corpus)

\begin{flushleft}
\tablehead{}
\begin{supertabular}{m{0.42885986in}m{0.07055984in}m{0.46985987in}m{0.029559843in}m{0.30735984in}m{0.19205984in}m{0.15735984in}m{0.34205985in}m{0.16085985in}m{0.33935985in}m{0.40525985in}m{0.5462598in}m{0.46565983in}m{1.1573598in}m{0.31775984in}}
 (48) &
\multicolumn{2}{m{0.6191598in}}{[\textit{Sieng}} &
\multicolumn{2}{m{0.41565984in}}{\itshape loku} &
\multicolumn{2}{m{0.42815986in}}{[\textit{ba}} &
\multicolumn{2}{m{0.58165985in}}{\itshape uti} &
\multicolumn{2}{m{0.8233598in}}{\textit{mia}]\textsubscript{ RC}} &
\textit{nu}]\textsubscript{ NP} &
\itshape sik &
\itshape bakon-i &
\\
 &
\multicolumn{2}{m{0.6191598in}}{rice} &
\multicolumn{2}{m{0.41565984in}}{\scshape pl} &
\multicolumn{2}{m{0.42815986in}}{\scshape rel} &
\multicolumn{2}{m{0.58165985in}}{garden} &
\multicolumn{2}{m{0.8233598in}}{in} &
\scshape art &
pluck &
rip.off.\textsc{pfv-pfv} &
\\
 &
\multicolumn{14}{m{5.98496in}}{{\textquoteleft}Pluck off [all] the rice that is in the garden [and]}\\
 &
\multicolumn{14}{m{5.98496in}}{}\\
\multicolumn{2}{m{0.5781598in}}{} &
\multicolumn{2}{m{0.57815987in}}{\itshape mi} &
\multicolumn{2}{m{0.5781598in}}{\itshape melang} &
\multicolumn{2}{m{0.5781598in}}{\itshape sei.} &
\multicolumn{2}{m{0.5789598in}}{} &
\multicolumn{5}{m{3.2072597in}}{}\\
\multicolumn{2}{m{0.5781598in}}{} &
\multicolumn{2}{m{0.57815987in}}{take} &
\multicolumn{2}{m{0.5781598in}}{village} &
\multicolumn{2}{m{0.5781598in}}{come.down} &
\multicolumn{2}{m{0.5789598in}}{} &
\multicolumn{5}{m{3.2072597in}}{}\\
 &
\multicolumn{14}{m{5.98496in}}{take it down to the village.{\textquoteright} }\\
\end{supertabular}
\end{flushleft}
\textit{Loku} cannot co-occur in an NP together with any quantifiers; numeral (49a) or non-numeral (49b). However, it is possible for an NP with \textit{loku} to be the subject of both numeral and non-numeral quantifier predications (50a-b). This indicates that, whilst double marking of quantification/plurality is not permitted within the NP, there is no semantic redundancy in the quantificational values of the Abui plural word and other quantifiers. In this respect, Abui \textit{loku }differs from Teiwa \textit{non} (section 3.2). 

Abui  (Schapper, fieldnotes)

\begin{flushleft}
\tablehead{}
\begin{supertabular}{m{0.34835985in}m{0.23935986in}m{0.13655984in}m{0.45185986in}m{0.66705984in}m{0.66705984in}m{0.66705984in}m{0.66705984in}m{0.66775984in}}
\multicolumn{2}{m{0.66645986in}}{(49) } &
\multicolumn{2}{m{0.66715986in}}{a. *} &
\itshape He-wiil &
\itshape taama &
\itshape loku &
\itshape nu &
\itshape mon-i.\\
\multicolumn{2}{m{0.66645986in}}{} &
\multicolumn{2}{m{0.66715986in}}{} &
\textsc{3.gen}{}-child &
six &
\scshape pl &
\scshape art &
die.\textsc{pfv}{}-\textsc{pfv}\\
\multicolumn{2}{m{0.66645986in}}{} &
\multicolumn{2}{m{0.66715986in}}{} &
\multicolumn{5}{m{3.65096in}}{Intended: {\textquoteleft}His six children died.{\textquoteright} }\\
\multicolumn{2}{m{0.66645986in}}{} &
\multicolumn{2}{m{0.66715986in}}{} &
\multicolumn{5}{m{3.65096in}}{}\\
\multicolumn{2}{m{0.66645986in}}{} &
\multicolumn{2}{m{0.66715986in}}{b. *} &
\itshape He-wiil &
\itshape faring &
\itshape loku &
\itshape nu &
\itshape mon-i.\\
\multicolumn{2}{m{0.66645986in}}{} &
\multicolumn{2}{m{0.66715986in}}{} &
\textsc{3.gen}{}-child &
many &
\scshape pl &
\scshape art &
die.\textsc{pfv}{}-\textsc{pfv}\\
 &
\multicolumn{2}{m{0.45465985in}}{} &
\multicolumn{6}{m{4.18156in}}{Intended: {\textquoteleft}His many children died.{\textquoteright} }\\
\end{supertabular}
\end{flushleft}
Abui  (Schapper, fieldnotes)

\begin{flushleft}
\tablehead{}
\begin{supertabular}{m{0.34835985in}m{0.45455983in}m{0.032359846in}m{0.99345976in}m{0.99275976in}m{0.99345976in}m{0.99275976in}m{0.9941599in}}
\multicolumn{3}{m{0.99275976in}}{(50) } &
a. &
\itshape He-wiil &
\itshape loku &
\itshape nu &
\itshape taama.\\
\multicolumn{3}{m{0.99275976in}}{} &
 &
\textsc{3.gen}{}-child &
\scshape pl &
\scshape art &
six\\
\multicolumn{3}{m{0.99275976in}}{} &
 &
\multicolumn{4}{m{4.20936in}}{{\textquoteleft}His children were six.{\textquoteright} i.e., {\textquoteleft}He had six children.{\textquoteright} }\\
\multicolumn{3}{m{0.99275976in}}{} &
 &
 &
\multicolumn{3}{m{3.1378598in}}{}\\
\multicolumn{3}{m{0.99275976in}}{} &
b. &
\itshape He-wiil &
\itshape loku &
\itshape nu &
\itshape faring.\\
\multicolumn{3}{m{0.99275976in}}{} &
 &
\textsc{3.gen}{}-child &
\scshape pl &
\scshape art &
many\\
 &
 &
\multicolumn{6}{m{5.392659in}}{{\textquoteleft}His children were many.{\textquoteright} i.e., {\textquoteleft}He had many children.{\textquoteright}}\\
\end{supertabular}
\end{flushleft}
\textit{Loku} can be used to modify a third person pronoun, as in (51) and (52). Abui has no number distinction in the third person of its pronominal series. By using \textit{loku }the plural reference can be made explicit. 

Abui (Kratochv\'il, Abui corpus)

\begin{flushleft}
\tablehead{}
\begin{supertabular}{m{0.34835985in}m{0.31985986in}m{0.7476598in}m{0.7476598in}m{0.7476598in}m{0.7476598in}m{0.7476598in}m{0.7483598in}}
\multicolumn{2}{m{0.74695987in}}{(51)} &
\itshape Hel &
\itshape loku &
\itshape abui &
\itshape yaa &
\itshape ut &
\itshape teak.\\
\multicolumn{2}{m{0.74695987in}}{} &
3 &
\textsc{pl}   &
mountain &
go &
garden &
watch\\
 &
\multicolumn{7}{m{5.2789598in}}{{\textquoteleft}They went to the mountains to check the garden.{\textquoteright} }\\
\end{supertabular}
\end{flushleft}
Abui (Kratochv\'il, Abui corpus)

\begin{flushleft}
\tablehead{}
\begin{supertabular}{m{0.34835985in}m{0.44275984in}m{0.42125985in}m{0.9212598in}m{1.1212599in}m{2.05736in}}
(52) &
\itshape Hel &
\itshape loku &
\itshape he-sepatu &
\textit{he-tawida}. &
\\
 &
3 &
\textsc{pl}   &
\textsc{3.gen}{}-shoe &
\textsc{3.gen}{}-be.alike &
\\
 &
\multicolumn{5}{m{5.27886in}}{{\textquoteleft}They have the same shoes.{\textquoteright} }\\
\end{supertabular}
\end{flushleft}
\textit{Loku }must co-occur with a noun or with the third person pronoun \textit{hel}. It cannot stand alone in an NP. \textit{ }

In addition to the general plural word \textit{loku}, Abui has an associative plural word, \textit{we }{\textquoteleft}\textsc{assoc{\textquoteright}}. This item only appears marking proper names for humans and has the meaning {\textquoteleft}[name] and people associated with [name]{\textquoteright} and occurs directly after the noun it modifies, as in (53a). When \textit{loku} is used in the same context (53b), the reading is not one of associative plurality, but of individualised plurality. \textit{Loku} and \textit{we }can co-occur, and either can precede the other, as shown in (53c).

\clearpage
Abui (Kratochv\'il, p.c.) 

\begin{flushleft}
\tablehead{}
\begin{supertabular}{m{0.39555985in}m{0.29765984in}m{0.13725984in}m{0.5559598in}m{0.037959844in}m{0.65525985in}m{-0.061340157in}m{0.7552598in}m{0.77265984in}m{0.77195984in}m{0.77195984in}m{0.77265984in}m{0.7768598in}}
\multicolumn{2}{m{0.77195984in}}{(53) } &
\multicolumn{2}{m{0.77195984in}}{a.} &
\multicolumn{2}{m{0.77195984in}}{\itshape Benny} &
\multicolumn{2}{m{0.77265984in}}{\textit{w}\textit{e}} &
\itshape ut &
 &
\itshape yaa. &
 &
\\
\multicolumn{2}{m{0.77195984in}}{} &
\multicolumn{2}{m{0.77195984in}}{} &
\multicolumn{2}{m{0.77195984in}}{Benny} &
\multicolumn{2}{m{0.77265984in}}{\scshape assoc} &
garden &
 &
go.to &
 &
\\
\multicolumn{2}{m{0.77195984in}}{} &
\multicolumn{2}{m{0.77195984in}}{} &
\multicolumn{8}{m{5.0275598in}}{{\textquoteleft}Benny and his associates go to the garden.{\textquoteright} } &
\\
\multicolumn{2}{m{0.77195984in}}{} &
\multicolumn{2}{m{0.77195984in}}{} &
\multicolumn{2}{m{0.77195984in}}{} &
\multicolumn{2}{m{0.77265984in}}{} &
 &
\multicolumn{3}{m{2.47406in}}{} &
\\
\multicolumn{2}{m{0.77195984in}}{} &
\multicolumn{2}{m{0.77195984in}}{b.} &
\multicolumn{2}{m{0.77195984in}}{\itshape Benny} &
\multicolumn{2}{m{0.77265984in}}{\itshape loku} &
\itshape ut &
 &
\itshape yaa. &
 &
\\
\multicolumn{2}{m{0.77195984in}}{} &
\multicolumn{2}{m{0.77195984in}}{} &
\multicolumn{2}{m{0.77195984in}}{Benny} &
\multicolumn{2}{m{0.77265984in}}{\scshape pl} &
garden &
 &
go.to &
 &
\\
\multicolumn{2}{m{0.77195984in}}{} &
\multicolumn{2}{m{0.77195984in}}{} &
\multicolumn{8}{m{5.0275598in}}{{\textquoteleft}Different individuals called Benny go to the garden.{\textquoteright} } &
\\
 &
\multicolumn{2}{m{0.51365983in}}{} &
\multicolumn{2}{m{0.6726598in}}{} &
\multicolumn{2}{m{0.6726599in}}{} &
\multicolumn{5}{m{4.1594596in}}{} &
\\
\multicolumn{2}{m{0.77195984in}}{} &
\multicolumn{2}{m{0.77195984in}}{c.} &
\multicolumn{2}{m{0.77195984in}}{\itshape Benny} &
\multicolumn{2}{m{0.77265984in}}{\itshape loku } &
\itshape we &
/ &
\textit{Benny}  \textit{we} &
\itshape loku &
\itshape ut\\
\multicolumn{2}{m{0.77195984in}}{} &
\multicolumn{2}{m{0.77195984in}}{} &
\multicolumn{2}{m{0.77195984in}}{Benny} &
\multicolumn{2}{m{0.77265984in}}{\scshape pl } &
\scshape assoc &
 &
Benny\textsc{  assoc} &
\scshape pl &
garden\\
 &
\multicolumn{2}{m{0.51365983in}}{} &
\multicolumn{9}{m{5.6622596in}}{{\textquoteleft}Two or more people called Benny go to the garden.{\textquoteright} } &
\\
\end{supertabular}
\end{flushleft}
Connected to its individualising semantics, \textit{loku }may be used with verbs to make expressions for collections of people. Examples are given in (54). 

Abui (Kratochvil 2007: 155)

\begin{flushleft}
\tablehead{}
\begin{supertabular}{m{0.37055984in}m{0.43235984in}m{0.83515984in}m{0.44205984in}m{0.08445985in}m{0.48585984in}m{3.60526in}}
(54)  &
a. &
\itshape pe  &
\itshape loku &
 &
 &
\\
 &
 &
near &
\scshape pl &
 &
 &
\\
 &
 &
\multicolumn{5}{m{5.76776in}}{lit. {\textquoteleft}the near ones{\textquoteright}; i.e. {\textquoteleft}neighbours{\textquoteright}}\\
 &
b. &
\itshape firai   &
\itshape loku &
 &
 &
\\
 &
 &
run &
\scshape pl &
 &
 &
\\
 &
 &
\multicolumn{5}{m{5.76776in}}{lit. {\textquoteleft}the running ones{\textquoteright}; i.e. {\textquoteleft}runners{\textquoteright}}\\
 &
c. &
\itshape walangra  &
\itshape loku &
 &
 &
\\
 &
 &
fresh &
\scshape pl &
 &
 &
\\
 &
 &
\multicolumn{5}{m{5.76776in}}{lit. {\textquoteleft}the new ones{\textquoteright}; i.e. {\textquoteleft}the newcomers, the Malays{\textquoteright} }\\
\end{supertabular}
\end{flushleft}
Abui differs from the more western languages (such as Western Pantar and Teiwa) in that it has two plural words marking different kinds of plurality. 

\subsection[3.5\ \ Wersing]{3.5\ \ Wersing}
The template for the Wersing noun phrase (NP) is given in (55). Modifiers follow the head noun of the NP (\textsc{N}\textsubscript{\MakeUppercase{head}}). They are an attribute (\textsc{Attr}), a numeral (\textsc{Num}) or the plural word\textsc{ (Pl), }and a\textsc{ }relative clause (\textsc{Rc). R}ight-most in the NP is a determiner (\textsc{Det}). See Schapper and Hendery (to appear) for details and full illustration of the Wersing NP.

(55)\ \ Template of Wersing NP (Schapper and Hendery to appear)

 \ \ [\textsc{N}\textsubscript{\MakeUppercase{head  }}\textsc{Attr Num/Pl  Rc  Det]}\textsc{\textsubscript{NP}}

The Wersing plural word is \textit{deing}. As is clear from the template above, it occurs in the NP in the same slot as a numeral. It cannot be used in combination with a numeral or any non-numeral quantifier (which are typically simple intransitive verbs that appear in the \textsc{Attr} slot), as illustrated in (56).  

Wersing (Schapper and Hendery, Wersing corpus)

\begin{flushleft}
\tablehead{}
\begin{supertabular}{m{0.34905985in}m{0.098359846in}m{0.27685985in}m{0.17125985in}m{0.42055985in}m{0.026859842in}m{0.49345985in}m{-0.045440156in}m{0.45385984in}m{0.5997598in}}
(56)  &
\multicolumn{2}{m{0.45395985in}}{a. *} &
\multicolumn{2}{m{0.6705598in}}{\textit{aning}\textit{\ \   }} &
\multicolumn{2}{m{0.5990598in}}{\itshape weting} &
\multicolumn{2}{m{0.48715982in}}{\itshape deing} &
\\
 &
\multicolumn{2}{m{0.45395985in}}{} &
\multicolumn{2}{m{0.6705598in}}{person} &
\multicolumn{2}{m{0.5990598in}}{five} &
\multicolumn{2}{m{0.48715982in}}{\scshape pl} &
\\
 &
\multicolumn{2}{m{0.45395985in}}{} &
\multicolumn{6}{m{1.9142599in}}{Intended: {\textquoteleft}five people{\textquoteright}} &
\\
 &
\multicolumn{2}{m{0.45395985in}}{} &
\multicolumn{6}{m{1.9142599in}}{} &
\\
\multicolumn{2}{m{0.5261598in}}{} &
\multicolumn{2}{m{0.5268598in}}{b. *} &
\multicolumn{2}{m{0.5261598in}}{\itshape aning\ \ } &
\multicolumn{2}{m{0.5267598in}}{\itshape bal} &
\multicolumn{2}{m{1.1323599in}}{\itshape deing}\\
\multicolumn{2}{m{0.5261598in}}{} &
\multicolumn{2}{m{0.5268598in}}{} &
\multicolumn{2}{m{0.5261598in}}{person} &
\multicolumn{2}{m{0.5267598in}}{many  } &
\multicolumn{2}{m{1.1323599in}}{\scshape pl}\\
\multicolumn{2}{m{0.5261598in}}{} &
\multicolumn{2}{m{0.5268598in}}{} &
\multicolumn{6}{m{2.3427598in}}{Intended:\textsc{ {\textquoteleft}}many people{\textquoteright}}\\
\end{supertabular}
\end{flushleft}
\textit{Deing} need not occur with an overt noun in the NP, but can stand alone so long as the referent can be retrieved from the discourse context. So, for instance, the head noun \textit{gis} in (57a) can be elided, as in the following examples (57b-d). What is more, the NP can be reduced to the plural word (57d) where there is neither noun head nor article.

Wersing (Schapper and Hendery, Wersing corpus)

\begin{flushleft}
\tablehead{}
\begin{supertabular}{m{0.34905985in}m{0.19485986in}m{0.18025985in}m{0.36435986in}m{0.49005982in}m{0.053859837in}m{0.46635982in}m{0.07815985in}m{0.64695984in}m{0.5997598in}}
(57)  &
\multicolumn{2}{m{0.4538599in}}{a.} &
\multicolumn{2}{m{0.93315977in}}{\textit{g}\textit{{}-is}\textit{\ \   }} &
\multicolumn{2}{m{0.5989598in}}{\itshape kebai} &
\multicolumn{2}{m{0.80385983in}}{\itshape dein=a  } &
\\
 &
\multicolumn{2}{m{0.4538599in}}{} &
\multicolumn{2}{m{0.93315977in}}{3-content} &
\multicolumn{2}{m{0.5989598in}}{young} &
\multicolumn{2}{m{0.80385983in}}{\scshape pl=art} &
\\
 &
\multicolumn{2}{m{0.4538599in}}{} &
\multicolumn{6}{m{2.4934597in}}{{\textquoteleft}their (coconut) young flesh{\textquoteright}} &
\\
 &
\multicolumn{2}{m{0.4538599in}}{} &
\multicolumn{6}{m{2.4934597in}}{} &
\\
\multicolumn{2}{m{0.62265986in}}{} &
\multicolumn{2}{m{0.62335986in}}{b.} &
\multicolumn{2}{m{0.6226598in}}{\itshape kebai\ \ } &
\multicolumn{2}{m{0.6232598in}}{\itshape dein=a  } &
\multicolumn{2}{m{1.32546in}}{}\\
\multicolumn{2}{m{0.62265986in}}{} &
\multicolumn{2}{m{0.62335986in}}{} &
\multicolumn{2}{m{0.6226598in}}{young} &
\multicolumn{2}{m{0.6232598in}}{\scshape pl=art} &
\multicolumn{2}{m{1.32546in}}{}\\
\multicolumn{2}{m{0.62265986in}}{} &
\multicolumn{2}{m{0.62335986in}}{} &
\multicolumn{6}{m{2.72886in}}{{\textquoteleft}the young (flesh){\textquoteright}}\\
\multicolumn{2}{m{0.62265986in}}{} &
\multicolumn{2}{m{0.62335986in}}{} &
\multicolumn{6}{m{2.72886in}}{}\\
 &
\multicolumn{2}{m{0.4538599in}}{c.} &
\multicolumn{2}{m{0.93315977in}}{\itshape dein=a  } &
\multicolumn{2}{m{0.5989598in}}{} &
\multicolumn{2}{m{0.80385983in}}{} &
\\
 &
\multicolumn{2}{m{0.4538599in}}{} &
\multicolumn{2}{m{0.93315977in}}{\scshape pl=art} &
\multicolumn{2}{m{0.5989598in}}{} &
\multicolumn{2}{m{0.80385983in}}{} &
\\
 &
\multicolumn{2}{m{0.4538599in}}{} &
\multicolumn{6}{m{2.4934597in}}{{\textquoteleft}the (young flesh){\textquoteright}} &
\\
 &
\multicolumn{2}{m{0.4538599in}}{} &
\multicolumn{6}{m{2.4934597in}}{} &
\\
\multicolumn{2}{m{0.62265986in}}{} &
\multicolumn{2}{m{0.62335986in}}{d.} &
\multicolumn{6}{m{2.72886in}}{\itshape deing}\\
\multicolumn{2}{m{0.62265986in}}{} &
\multicolumn{2}{m{0.62335986in}}{} &
\multicolumn{6}{m{2.72886in}}{\scshape pl}\\
\multicolumn{2}{m{0.62265986in}}{} &
\multicolumn{2}{m{0.62335986in}}{} &
\multicolumn{6}{m{2.72886in}}{{\textquoteleft}the (young flesh){\textquoteright}}\\
\end{supertabular}
\end{flushleft}
Like Kamang and the other eastern Alor languages, and Teiwa on Pantar, Wersing has multiple pronominal paradigms dedicated to denoting particular quantities of referents, for instance, universal quantification ({\textquoteleft}\textsc{all{\textquoteright})} (58) and group plurality ({\textquoteleft}\textsc{group}{\textquoteright}) (59).\footnote{There are five {\textquotedblleft}quantifying{\textquotedblright} pronominal paradigms in Wersing: (i) the {\textquotedblleft}alone{\textquotedblright} paradigm (\textit{I alone (no one else)}, etc.), (ii) the {\textquotedblleft}independent{\textquotedblright} paradigm (\textit{I on my own without help}, etc.), (iii) the dual paradigm (\textit{we two}, etc.- only in non-singular numbers), (iv) the {\textquotedblleft}all{\textquotedblright} paradigm (\textit{we all}, etc.- only in non-singular numbers), and (v) the {\textquotedblleft}group{\textquotedblright} paradigm (\textit{we together in a group}, etc.- only in non-singular numbers) (Schapper and Hendery to appear)
.} Such quantificational pronouns also play an important role in marking plurality of NP referents in Wersing. In (60) we see, for instance, the 3\textsuperscript{rd} person pronoun \textit{genaing} being used to signal the plurality of the referents of the preceding NP.\footnote{A pronoun of this paradigm can also be marked with \textit{{}-le}, as in: \textit{Aning ge-naingle kamar ming=te nanal te-mekeng} (\textsc{3.all-pl }room be.in=\textsc{conj }thing \textsc{recp}{}-exchange) {\textquoteleft}All of those who are in the room exchange things{\textquoteright}. The -\textit{le} suffix does not appear on nouns or any other pronominal series in Wersing; it is likely cognate with the Kamang associative plural marker \textit{{}-lee} (see section 3.3).}

Wersing (Schapper and Hendery, Wersing corpus)

\begin{flushleft}
\tablehead{}
\begin{supertabular}{m{0.34835985in}m{0.48235986in}m{0.9101598in}m{0.9094598in}m{0.9101598in}m{0.9101598in}m{0.9101598in}}
\multicolumn{2}{m{0.9094598in}}{(58)} &
\itshape Tanaing &
\itshape dra &
\itshape bo! &
 &
\\
\multicolumn{2}{m{0.9094598in}}{} &
\scshape 1pl.incl.all &
\textsc{pl}   &
\scshape emph &
 &
\\
 &
\multicolumn{6}{m{5.42616in}}{{\textquoteleft}Let{\textquoteright}s sing.{\textquoteright}}\\
\end{supertabular}
\end{flushleft}
Wersing (Schapper  and Hendery, Wersing corpus)

\begin{flushleft}
\tablehead{}
\begin{supertabular}{m{0.34835985in}m{0.48235986in}m{0.9101598in}m{0.9094598in}m{0.9101598in}m{0.9101598in}m{0.9101598in}}
\multicolumn{2}{m{0.9094598in}}{(59)} &
\itshape Nyawi &
\itshape nyi-mit &
\itshape o! &
 &
\\
\multicolumn{2}{m{0.9094598in}}{} &
\textsc{1pl.excl.}\textsc{group} &
\textsc{1pl.excl}{}-sit &
\scshape exclam &
 &
\\
 &
\multicolumn{6}{m{5.42616in}}{{\textquoteleft}Let{\textquoteright}s sit together!{\textquoteright}}\\
\end{supertabular}
\end{flushleft}
Wersing (Schapper  and Hendery, Wersing corpus)

\begin{flushleft}
\tablehead{}
\begin{supertabular}{m{0.34905985in}m{0.49765983in}m{0.9254598in}m{0.9254598in}m{0.9254598in}m{0.9254598in}m{0.9261598in}}
\multicolumn{2}{m{0.9254598in}}{(60)} &
\itshape Ge-siriping  &
\itshape genaing &
\itshape beteng &
\itshape ge-dai. &
\\
\multicolumn{2}{m{0.9254598in}}{} &
\textsc{3-}root &
\textsc{3.all}   &
pull &
3-come.up &
\\
 &
\multicolumn{6}{m{5.5193596in}}{{\textquoteleft}All its roots were pulled right up.{\textquoteright}}\\
\end{supertabular}
\end{flushleft}
Wersing \textit{d}\textit{eing} can nevertheless mark plurality for non-singular numbers of topic pronouns, as in (61) and (62). In this respect, then, the Wersing plural word is not like a pronoun as in Kamang, but a distinct item which can modify any NP head, nominal or pronominal.

Wersing (Schapper  and Hendery, Wersing corpus)

\begin{flushleft}
\tablehead{}
\begin{supertabular}{m{0.34835985in}m{0.22265986in}m{0.22755986in}m{0.6184598in}m{0.63795984in}m{0.8740598in}m{0.8712598in}m{0.6733598in}}
(61) &
\multicolumn{2}{m{0.5289598in}}{\itshape Gai} &
\itshape dein=a &
\itshape mona &
\itshape min-a. &
 &
\\
 &
\multicolumn{2}{m{0.5289598in}}{\scshape 3.top} &
\textsc{pl}=\textsc{art} &
\scshape across &
be.at-\textsc{real} &
 &
\\
\multicolumn{2}{m{0.6497598in}}{} &
\multicolumn{6}{m{4.29636in}}{{\textquoteleft}They are all over there.{\textquoteright}}\\
\multicolumn{2}{m{0.6497598in}}{} &
\multicolumn{6}{m{4.29636in}}{}\\
\end{supertabular}
\end{flushleft}
Wersing (Schapper  and Hendery, Wersing corpus)

\begin{flushleft}
\tablehead{}
\begin{supertabular}{m{0.34835985in}m{1.0684599in}m{0.5462598in}m{0.5455598in}m{1.8636599in}m{0.6733598in}}
(62) &
\itshape Nyai &
\itshape deing &
\raggedleft \itshape o= &
\itshape min-a. &
\\
 &
\scshape 1pl.excl.top &
\textsc{pl}   &
\raggedleft \scshape here= &
be.at-\textsc{real} &
\\
 &
\multicolumn{2}{m{1.6934599in}}{\textsc{{\textquoteleft}}We are all here.{\textquoteright}} &
 &
 &
\\
\end{supertabular}
\end{flushleft}
Wersing has a further plural word, \textit{naing}, which marks associative plurality. This form has been observed only marking personal names, as in (63) and (64). As an associative plural word, it doesn{\textquoteright}t have the ability to stand in for a NP. Like \textit{deing}, \textit{naing }cannot occur with other quantifiers, numeral and non-numeral.

Wersing (Malikosa n.d.)

\begin{flushleft}
\tablehead{}
\begin{supertabular}{m{0.34905985in}m{0.24005985in}m{0.20945984in}m{0.6177598in}m{0.7656598in}m{0.8740598in}m{0.8712598in}m{0.6733598in}}
(63) &
\multicolumn{2}{m{0.5282598in}}{\itshape Petrus} &
\itshape naing &
\itshape g-aumeng &
\itshape ga-pang &
\itshape ge-pai. &
\\
 &
\multicolumn{2}{m{0.5282598in}}{Peter} &
\scshape assoc &
3-fear &
3-dead &
3-make &
\\
\multicolumn{2}{m{0.66785985in}}{} &
\multicolumn{6}{m{4.40526in}}{{\textquoteleft}Peter and the others were afraid to die.{\textquoteright} }\\
\end{supertabular}
\end{flushleft}
Wersing (Malikosa n.d.)

\begin{flushleft}
\tablehead{}
\begin{supertabular}{m{0.34905985in}m{0.24005985in}m{0.21505985in}m{0.5462598in}m{0.6733598in}m{0.8733598in}m{0.9420598in}m{0.76155984in}}
(64) &
\multicolumn{2}{m{0.5338598in}}{\itshape Yesus} &
\itshape naing &
\itshape lailol &
\itshape gewai &
\itshape Kapernaum &
\itshape taing.\\
 &
\multicolumn{2}{m{0.5338598in}}{Jesus} &
\scshape assoc &
walk &
3-go &
Kapernaum &
reach\\
\multicolumn{2}{m{0.66785985in}}{} &
\multicolumn{6}{m{4.4053597in}}{{\textquoteleft}Jesus and the others walked onto Kapernaum.{\textquoteright}\ \ }\\
\end{supertabular}
\end{flushleft}
\subsection[3.6\ \ Summary ]{3.6\ \ Summary }
Most Alor-Pantar languages have inherited a plural word, but they show much variation in the syntactic properties of this word. Table 2 presents a summary of the variable syntax discussed in the previous sections. The table reveals the gradient differences between plural words in Alor-Pantar languages. Kamang stands out from the other four languages for the fact that the plural word is not part of the NP. Of the languages that do have their plural word in the NP, the plural word cannot typically stand alone in the NP, but requires another, nominal, element be present. In Wersing, however, this is only the case for the associative plural word \textit{deing}; its plural number word can form independent NPs. Alor-Pantar plural words are prohibited from co-occurring with quantifiers. No language allows co-occurrence with a numeral quantifier and only Teiwa permits co-occurrence with a non-numeral quantifier.

{\centering
Table 2: Variable syntax of five Alor-Pantar plural words
\par}

\begin{flushleft}
\tablehead{}
\begin{supertabular}{|m{2.75116in}|m{0.51155984in}|m{0.8080598in}|m{0.70805985in}|m{0.42335984in}|m{0.70045984in}|}
\hline
 &
\bfseries Teiwa  &
\bfseries W Pantar &
\bfseries Kamang  &
\bfseries Abui  &
\bfseries Wersing\\\hline
Is plural word part of NP? &
yes &
yes &
no &
yes &
yes\\\hline
Can plural word stand alone in NP? &
no &
no &
{}-{}- &
no &
yes\\\hline
Can the plural word and non-numeral quantifier co-occur? &
yes &
no &
no &
no &
no\\\hline
Can plural word and numeral co-occur? &
no  &
no &
no &
no  &
no\\\hline
\end{supertabular}
\end{flushleft}
\ \ These different properties mean that in all five languages, plural word(s) constitutes a word class of its own, with only partial overlap with other morpho-syntatic classes of words. In Western Pantar, the plural word shares much with adjectival quantifiers and numerical expressions. In Teiwa, the plural word patterns mostly with non-numeral quantifiers. In Kamang and Wersing, plural words pattern similarly to quantificational pronouns in denoting the number of a preceding noun. However, Wersing \textit{deing} behaves much more like a nominal element. Nominal properties are also visible in the Abui word \textit{loku}, particularly in its frequent use with verbs to form expressions for collections of people.

\ \ In short, Alor-Pantar plural words are a morpho-syntactically diverse group of items that are seemingly united only by their semantic commonalities. Yet, as we will see in the following sections, even the semantics of plurality reveal more variability than might have been expected.

\section[4\ \ Semantics of plural words in Alor{}-Pantar ]{4\ \ Semantics of plural words in Alor-Pantar }
In all five languages, the plural words code plurality alongside other notions. In this section, we review three additional connotations of the plural word.

\subsection[4.1\ \ Completeness]{4.1\ \ Completeness}
The Western Pantar plural word \textit{maru(ng) }typically imparts a sense of entirety, completeness, and comprehensiveness, as in (65):

Western Pantar (Holton, Western Pantar corpus)

\begin{flushleft}
\tablehead{}
\begin{supertabular}{m{0.43165985in}m{0.059459843in}m{0.5191598in}m{-0.027340159in}m{0.5698598in}m{0.31565985in}m{0.17615986in}m{0.39905986in}m{0.09205984in}m{0.5156598in}m{-0.023840155in}m{0.5254598in}m{-0.034340158in}m{0.5705598in}m{-0.048140157in}m{0.44975987in}m{0.01085984in}m{0.28165984in}m{0.21085986in}}
(65) &
\multicolumn{2}{m{0.6573598in}}{\itshape Ping} &
\multicolumn{3}{m{1.0156598in}}{\itshape pi} &
\multicolumn{2}{m{0.6539598in}}{\itshape mappu} &
\multicolumn{2}{m{0.68645984in}}{\itshape maiyang,} &
\multicolumn{2}{m{0.5803598in}}{\itshape lokke} &
\multicolumn{3}{m{0.6455598in}}{\itshape maiyang} &
\itshape saiga &
\multicolumn{2}{m{0.37125984in}}{\itshape si,} &
\\
 &
\multicolumn{2}{m{0.6573598in}}{\textsc{1pl.incl} } &
\multicolumn{3}{m{1.0156598in}}{\scshape 1pl.incl.poss} &
\multicolumn{2}{m{0.6539598in}}{fishpond} &
\multicolumn{2}{m{0.68645984in}}{place} &
\multicolumn{2}{m{0.5803598in}}{fishtrap} &
\multicolumn{3}{m{0.6455598in}}{place} &
\scshape dem &
\multicolumn{2}{m{0.37125984in}}{\scshape art} &
\\
 &
\multicolumn{18}{m{5.9011602in}}{{\textquoteleft}We placed our fishponds, placed our fish traps, }\\
 &
\multicolumn{18}{m{5.9011602in}}{}\\
\multicolumn{2}{m{0.5698598in}}{} &
\multicolumn{2}{m{0.5705598in}}{\itshape gai} &
\textit{ke}\textit{{\textglotstop}}\textit{e} &
\multicolumn{2}{m{0.5705598in}}{\itshape maru} &
\multicolumn{2}{m{0.5698598in}}{\itshape si} &
\multicolumn{2}{m{0.5705598in}}{\itshape aname} &
\multicolumn{2}{m{0.5698598in}}{\itshape ging} &
\itshape haggi &
\multicolumn{3}{m{0.5699598in}}{\itshape kanna.} &
\multicolumn{2}{m{0.5712598in}}{}\\
\multicolumn{2}{m{0.5698598in}}{} &
\multicolumn{2}{m{0.5705598in}}{\scshape 3.poss} &
fish &
\multicolumn{2}{m{0.5705598in}}{\scshape pl} &
\multicolumn{2}{m{0.5698598in}}{\scshape art } &
\multicolumn{2}{m{0.5705598in}}{person} &
\multicolumn{2}{m{0.5698598in}}{\scshape 3pl.act} &
take &
\multicolumn{3}{m{0.5699598in}}{already} &
\multicolumn{2}{m{0.5712598in}}{}\\
\multicolumn{2}{m{0.5698598in}}{} &
\multicolumn{17}{m{5.76296in}}{and then people took all the fish.{\textquoteright} }\\
\end{supertabular}
\end{flushleft}
Its sense of comprehensiveness and entirety explains why NPs pluralised with \textit{maru(ng) }can be the subject of the nominal predicate \textit{gaterannang} {\textquoteleft}all{\textquoteright} expressing universal quantity, as in (66), while combinations of \textit{marung} and mid-range quantifiers such as \textit{haweri} {\textquoteleft}many{\textquoteright} are absent in the Western Pantar corpus. It also explains why \textit{marung }is not compatible with a numeral predicate, as in (67), as these indicate a quantity of a certain number rather than universal quantity. 

Western Pantar (Holton 2012)

\begin{flushleft}
\tablehead{}
\begin{supertabular}{m{0.37475985in}m{0.6108598in}m{0.6927598in}m{0.45255986in}m{0.9226598in}m{0.44205984in}m{0.43235984in}m{0.6226598in}m{0.24135986in}m{0.99275976in}}
(66) &
[\textit{Aname } &
\textit{marung}] &
\itshape ging &
\itshape gaterannang &
\itshape dia  &
\itshape wang &
\itshape pidding. &
 &
\\
 &
people &
\scshape pl &
they &
all  &
go &
exist &
spread &
 &
\\
 &
\multicolumn{9}{m{6.039959in}}{{\textquoteleft}All the people spread out [to look for them]{\textquoteright} (Holton 2012) 

(Lit. {\textquoteleft}All people they were all going spreading ...{\textquoteright})}\\
\end{supertabular}
\end{flushleft}
Western Pantar (Holton 2012)

\begin{flushleft}
\tablehead{}
\begin{supertabular}{m{0.47335985in}m{0.6122598in}m{0.6969598in}m{0.45455983in}m{0.9226598in}m{0.44625983in}m{0.43235984in}m{0.6226598in}m{0.24975985in}m{0.25525984in}}
(67) * &
[\textit{Aname } &
\textit{marung}] &
\itshape ging &
\itshape kealaku &
\itshape dia  &
\itshape wang &
\itshape pidding. &
 &
\\
 &
people &
\scshape pl &
they &
twenty &
go &
exist &
spread &
 &
\\
 &
\multicolumn{9}{m{5.3226595in}}{Intended: {\textquoteleft}All people they were twenty going spreading...{\textquoteright} }\\
\end{supertabular}
\end{flushleft}
Finally, \textit{marung} is used with count nouns, and cannot combine with mass noun such as \textit{halia }{\textquoteleft}water{\textquoteright}, (68). In this respect, \textit{marung }contrasts with the plural words in Abui, Wersing, Kamang and Teiwa, which can combine with mass nouns (sections 4.2,  4.3.1). 

Western Pantar (Holton, p.c.)

\begin{flushleft}
\tablehead{}
\begin{supertabular}{m{0.5552598in}m{0.6615598in}m{0.76365983in}m{0.5094598in}m{0.67885983in}m{0.5476598in}m{0.42195985in}m{0.35315984in}m{0.35255986in}m{0.35595986in}}
(68) * &
\itshape halia &
\itshape marung &
 &
 &
 &
 &
 &
 &
\\
 &
water &
\scshape pl &
 &
 &
 &
 &
 &
 &
\\
 &
\multicolumn{9}{m{5.2747602in}}{Intended: {\textquoteleft}several containers of water{\textquoteright};  {\textquoteleft}multiple waters{\textquoteright}}\\
\end{supertabular}
\end{flushleft}
The connotation of comprehensiveness is also found in Abui \textit{loku}. That is, the inclusion of \textit{loku} signals that the whole mass of saliva was subject to the swarming of the birds in (69) and that all the available corn had to be stowed away (70) in an orderly fashion, so as to use the maximum capacity of the basket.

Abui (Kratochv\'il, Abui corpus)

\begin{flushleft}
\tablehead{}
\begin{supertabular}{m{0.37055984in}m{0.58375984in}m{0.42615983in}m{0.5191598in}m{0.8483598in}m{0.58515984in}m{0.6025598in}m{0.44275984in}m{0.39485985in}m{0.36565986in}}
(69) &
\textit{...}\textit{ kuya} &
\itshape do &
\itshape sila &
\itshape nahang &
\itshape oro &
 &
 &
 &
\\
 &
bird &
\scshape dem &
much &
everywhere &
\scshape across &
 &
 &
 &
\\
 &
\multicolumn{9}{m{5.39836in}}{{\textquoteleft}Those birds were everywhere there,}\\
\end{supertabular}
\end{flushleft}
\begin{flushleft}
\tablehead{}
\begin{supertabular}{m{0.37055984in}m{1.0795599in}m{0.9240598in}m{0.36435986in}m{0.36985984in}m{0.98095983in}m{0.077559836in}m{0.077559836in}m{0.077559836in}m{0.07955984in}}
 &
\itshape he-ya &
\itshape he-puyung &
\itshape loku &
\itshape do  &
\itshape he-afai. &
 &
 &
 &
\\
 &
\textsc{3.gen}{}-mother &
\textsc{3.gen-}saliva &
\scshape pl &
\scshape dem &
\textsc{3.gen-}swarm &
 &
 &
 &
\\
 &
\multicolumn{9}{m{4.66096in}}{swarming over the saliva of his mother.{\textquoteright}}\\
\end{supertabular}
\end{flushleft}
Abui (Kratochv\'il, fieldnotes)

\begin{flushleft}
\tablehead{}
\begin{supertabular}{m{0.46015987in}m{0.5580598in}m{0.47335985in}m{0.47815987in}m{0.52125984in}m{0.85385984in}m{0.84345984in}m{0.41365984in}m{0.41435984in}m{0.41705984in}}
(70) &
\itshape Fat &
\itshape loku &
\itshape mi &
\itshape ba &
\itshape buot &
\itshape he-rei &
 &
 &
\\
 &
corn &
\scshape pl &
take &
\scshape conj &
back.basket &
\textsc{3.gen-}stow &
 &
 &
\\
 &
\multicolumn{9}{m{5.6031604in}}{{\textquoteleft}Stow all the corn in the basket.{\textquoteright} }\\
\end{supertabular}
\end{flushleft}
The sense of comprehensive quantity expressed by \textit{loku} ({\textquoteleft}all{\textquoteright}) is relative to the situation at hand ({\textquoteleft}all that is there{\textquoteright}). As a result, \textit{loku }can occur with the universal quantifier \textit{tafuda }{\textquoteleft}all{\textquoteright}, as in (71). 

\clearpage
Abui (Kratochv\'il, Abui corpus)

\begin{flushleft}
\tablehead{}
\begin{supertabular}{m{0.33305985in}m{-0.05024016in}m{0.30455986in}m{0.14905983in}m{0.10595984in}m{0.33305985in}m{0.33375984in}m{0.17265984in}m{0.08095985in}m{0.33375984in}m{0.33305985in}m{-0.043340158in}m{0.29765984in}m{0.13935983in}m{0.11565984in}m{0.33305985in}m{0.10875984in}m{0.07955984in}m{-0.012740158in}m{0.013659842in}m{0.08935984in}}
\multicolumn{2}{m{0.36155984in}}{(71)} &
\multicolumn{2}{m{0.53235984in}}{\itshape Ama} &
\multicolumn{4}{m{1.1816598in}}{[\textit{ne-mea}} &
\multicolumn{4}{m{0.9406598in}}{\textit{loku}]} &
\multicolumn{2}{m{0.5157598in}}{\itshape tafuda} &
\multicolumn{3}{m{0.7149598in}}{\itshape takaf-i} &
 &
\multicolumn{2}{m{0.07965984in}}{} &
\\
\multicolumn{2}{m{0.36155984in}}{} &
\multicolumn{2}{m{0.53235984in}}{person} &
\multicolumn{4}{m{1.1816598in}}{\textsc{1sg.gen-}mango} &
\multicolumn{4}{m{0.9406598in}}{\scshape pl} &
\multicolumn{2}{m{0.5157598in}}{all} &
\multicolumn{3}{m{0.7149598in}}{steal-\textsc{pfv}} &
 &
\multicolumn{2}{m{0.07965984in}}{} &
\\
\multicolumn{2}{m{0.36155984in}}{} &
\multicolumn{19}{m{4.6851597in}}{{\textquoteleft}All my mangos got stolen, }\\
\multicolumn{2}{m{0.36155984in}}{} &
\multicolumn{19}{m{4.6851597in}}{}\\
 &
\multicolumn{2}{m{0.33305985in}}{\itshape do} &
\multicolumn{2}{m{0.33375984in}}{\itshape n-omi} &
\itshape he-ukda &
 &
\multicolumn{2}{m{0.33235985in}}{} &
 &
 &
\multicolumn{2}{m{0.33305985in}}{} &
\multicolumn{2}{m{0.33375984in}}{} &
 &
\multicolumn{3}{m{0.33305982in}}{} &
\multicolumn{2}{m{0.18175986in}}{}\\
 &
\multicolumn{2}{m{0.33305985in}}{\scshape dem} &
\multicolumn{2}{m{0.33375984in}}{\textsc{1sg.gen-}inside} &
3.\textsc{gen-}shock &
 &
\multicolumn{2}{m{0.33235985in}}{} &
 &
 &
\multicolumn{2}{m{0.33305985in}}{} &
\multicolumn{2}{m{0.33375984in}}{} &
 &
\multicolumn{3}{m{0.33305982in}}{} &
\multicolumn{2}{m{0.18175986in}}{}\\
 &
\multicolumn{20}{m{4.71366in}}{it really shocked me.{\textquoteright} (Kratochv\'il, Abui corpus)}\\
\end{supertabular}
\end{flushleft}
In Wersing, the sense of comprehensiveness is found when the plural word is used together with an already plural topic pronoun. For instance, in (72) the use of \textit{deing} implies that the whole set of those who were expected are present. 

Wersing  (Schapper, fieldnotes)

\begin{flushleft}
\tablehead{}
\begin{supertabular}{m{0.34905985in}m{1.0677599in}m{0.5462598in}m{0.5462598in}m{0.8733598in}m{0.6726598in}}
(72) &
\itshape Tai &
\itshape deing &
\itshape o &
\itshape min-a. &
\\
 &
\scshape 1pl.incl.top &
\textsc{pl}   &
\scshape here &
be.at-\textsc{real} &
\\
 &
\multicolumn{5}{m{4.0212603in}}{\textsc{{\textquoteleft}}We are all here.{\textquoteright}}\\
\end{supertabular}
\end{flushleft}
\subsection[4.2\ \ Abundance]{4.2\ \ Abundance}
In Teiwa and Wersing, using the plural word can add the sense that the referent occurs in particular abundance. 

While the core semantics of Teiwa \textit{non }is plural {\textquoteleft}more than one{\textquoteright} or {\textquoteleft}several{\textquoteright}, it often has the connotation of {\textquoteleft}many, plenty{\textquoteright}, as in (73). This is not true for all plural words in AP languages.\textit{ }

Teiwa (Klamer, Teiwa corpus)

\begin{flushleft}
\tablehead{}
\begin{supertabular}{m{0.36635986in}m{0.20185986in}m{0.6150598in}m{0.36915985in}m{0.16845983in}m{0.25315985in}m{0.5483598in}m{0.6226598in}m{0.7066598in}m{0.28025985in}m{0.28095984in}m{0.5455598in}m{0.36365986in}m{0.13655984in}m{0.13795984in}}
\label{bkm:Ref335061853}(73) &
a. &
\itshape in  &
\itshape non &
 &
b.  &
\itshape in  &
\itshape bun &
\itshape non &
 &
c. &
\itshape wou  &
\itshape non &
 &
\\
 &
 &
it.thing &
\scshape pl &
 &
 &
it.thing &
bamboo &
\scshape pl &
 &
 &
mango &
\scshape pl &
 &
\\
 &
 &
\multicolumn{3}{m{1.3101599in}}{{\textquoteleft}plenty of things{\textquoteright}} &
 &
\multicolumn{3}{m{2.0351598in}}{{\textquoteleft}plenty of pieces of bamboo{\textquoteright}} &
 &
 &
\multicolumn{4}{m{1.4199598in}}{{\textquoteleft}plenty of mangos{\textquoteright}}\\
\end{supertabular}
\end{flushleft}
Especially when combining with nouns referring to utensils or consumables, the plurality of \textit{non }often has the connotation {\textquoteleft}plenty{\textquoteright}. A similar reading is imposed when \textit{non }combines with small objects such as flowers or insects. As these come in sets of conventionally large numbers, the use of \textit{non }implies that their set is larger than expected. For instance, \textit{haliwai non }in (74) refers to black ants as crawling into the sarong in unexpected numbers. 

Teiwa (Klamer, Teiwa corpus)

\begin{flushleft}
\tablehead{}
\begin{supertabular}{m{0.42815986in}m{0.40595984in}m{0.6295598in}m{0.6775598in}m{0.40455985in}m{0.6073598in}m{0.48725984in}m{0.47195986in}m{0.43165985in}m{0.44135985in}m{0.70185983in}}
(74) &
\itshape ...a &
\itshape mis-an &
\itshape haliwai &
\itshape non &
\itshape daa &
\itshape nuan &
\itshape gom &
\itshape ma &
\itshape yiri  &
\itshape u si,...\\
 &
\scshape 3sg &
sit-\textsc{real} &
black.ant &
\scshape pl &
ascend &
cloth &
inside &
come &
crawl &
\scshape dist sim\\
 &
\multicolumn{10}{m{5.96776in}}{{\textquoteleft}...(while) he sat (unexpectedly many) black ants came crawling into his sarong,...{\textquoteright}}\\
\end{supertabular}
\end{flushleft}
There are other specific readings that \textit{non }may get, varying according to the type of nominal referent and the pragmatics of the situation. For example, when \textit{non }combines with objects such as seeds, chairs, or rocks, it may imply that they occur in a set that has an unusual configuration which is more disorderly than the conventional one, such as when seeds are spilled across the floor rather than in a bag or a pile, or when chairs are scattered around the room instead of organised around a table. Finally, \textit{non }may also code that the set is non-homogeneous, e.g., \textit{war non }may refer to {\textquoteleft}several rocks{\textquoteright}, but also to {\textquoteleft}rocks of various kinds and sizes{\textquoteright}. 

Wersing also reflects this sense, when referring to inanimates, especially where they have little individuation. In (75) the use of \textit{deing} to modify \textit{wor} {\textquoteleft}rock{\textquoteright} and \textit{inipak} {\textquoteleft}sand{\textquoteright} suggests that an abundance of these items are swept up by the wind. Without the plural word, there would be no indication of the amount of rock and sand moved by the wind. 

Wersing (Schapper, fieldnotes)

\begin{flushleft}
\tablehead{}
\begin{supertabular}{m{0.34835985in}m{0.19345984in}m{0.5163598in}m{0.02545984in}m{0.44135985in}m{0.10115985in}m{0.18725985in}m{0.35455984in}m{0.6212598in}m{-0.062740155in}m{0.54275984in}m{-0.016240157in}m{0.38865986in}m{0.15385985in}m{0.33865985in}m{0.9031598in}}
(75) &
\multicolumn{2}{m{0.7885598in}}{\itshape Tumur} &
\multicolumn{2}{m{0.5455598in}}{\itshape lapong} &
\multicolumn{2}{m{0.36715987in}}{\itshape gai} &
\multicolumn{3}{m{1.0705599in}}{\itshape ge-tati=sa,} &
\itshape wor &
\multicolumn{2}{m{0.45115986in}}{\itshape anta} &
\multicolumn{2}{m{0.5712598in}}{\itshape inipak} &
\itshape lang=mi\\
 &
\multicolumn{2}{m{0.7885598in}}{east.wind} &
\multicolumn{2}{m{0.5455598in}}{wind} &
\multicolumn{2}{m{0.36715987in}}{3} &
\multicolumn{3}{m{1.0705599in}}{3-stand=\textsc{conj}} &
rock &
\multicolumn{2}{m{0.45115986in}}{or} &
\multicolumn{2}{m{0.5712598in}}{sand} &
beach=\textsc{loc}\\
\multicolumn{2}{m{0.6205598in}}{} &
\multicolumn{2}{m{0.6205598in}}{\itshape dein=a} &
\multicolumn{2}{m{0.6212598in}}{\itshape ge-poing} &
\multicolumn{2}{m{0.6205598in}}{\itshape ge-dai} &
\itshape medi &
\multicolumn{3}{m{0.6212598in}}{\itshape aruku} &
\multicolumn{2}{m{0.62125987in}}{\textit{le-ge-ti.}\textit{  }} &
\multicolumn{2}{m{1.3205599in}}{}\\
\multicolumn{2}{m{0.6205598in}}{} &
\multicolumn{2}{m{0.6205598in}}{\scshape pl=art} &
\multicolumn{2}{m{0.6212598in}}{\textsc{3-}hit} &
\multicolumn{2}{m{0.6205598in}}{3-go.up} &
take &
\multicolumn{3}{m{0.6212598in}}{dry.land} &
\multicolumn{2}{m{0.62125987in}}{\textsc{appl-3-}lie} &
\multicolumn{2}{m{1.3205599in}}{}\\
\multicolumn{2}{m{0.6205598in}}{} &
\multicolumn{14}{m{5.5191603in}}{{\textquoteleft}When the east wind blows, a mass of rocks and sand from the beach is lifted up and deposited on dry land (beyond the beach).{\textquoteright} }\\
\end{supertabular}
\end{flushleft}
Such senses of abundance have not been observed with the plural word in Western Pantar or Kamang.

\subsection[4.3\ \ Individuation]{4.3\ \ Individuation}
The use of a plural word often imposes an individuated reading of a referent, that is, that the referent is not an undifferentiated mass but rather is composed of an internally cohesive set of individuals of the same type. For instance, consider the contrast between the \textit{we} and the \textit{loku} plural in Abui in (76a-b), repeated from example (53) in section 3.4. The associative plural \textit{we} gives a reading of a closely-knit group of individuals centred on one prominent individual, Benny. By contrast, the \textit{loku} plural, when it is used in the same context, imposes a referentially heterogeneous or individualised reading whereby multiple distinct people of the same name are being referred to. This difference is also characteristic of the Wersing plural words \textit{deing} {\textquoteleft}\textsc{pl}{\textquoteright} and \textit{naing} {\textquoteleft}\textsc{assoc}{\textquoteright}. 

Abui (Schapper, fieldnotes)

\begin{flushleft}
\tablehead{}
\begin{supertabular}{m{0.34835985in}m{0.27955985in}m{0.05805985in}m{0.84415984in}m{0.84415984in}m{0.84415984in}m{0.84345984in}m{0.84485984in}}
\multicolumn{3}{m{0.84345984in}}{(76) } &
a. &
\itshape Benny &
\textit{w}\textit{e} &
\itshape ut &
\itshape yaa.\\
\multicolumn{3}{m{0.84345984in}}{} &
 &
Benny &
\scshape assoc &
garden &
go.to\\
 &
 &
\multicolumn{6}{m{4.6725597in}}{{\textquoteleft}Benny and his associates go to the garden.{\textquoteright}}\\
\multicolumn{3}{m{0.84345984in}}{} &
 &
 &
\multicolumn{3}{m{2.68996in}}{}\\
\multicolumn{3}{m{0.84345984in}}{} &
b. &
\itshape Benny &
\itshape loku &
\itshape ut &
\itshape yaa.\\
\multicolumn{3}{m{0.84345984in}}{} &
 &
Benny &
\scshape pl &
garden &
go.to\\
 &
 &
\multicolumn{6}{m{4.6725597in}}{{\textquoteleft}Different individuals called Benny go to the garden.{\textquoteright}}\\
\end{supertabular}
\end{flushleft}
There are two contexts in which we find a particular tendency of plural words in AP to impose individualised readings on the nouns they modify. These are discussed in the following subsections. 

\subsubsection[4.3.1\ \ Individuation of mass to count]{4.3.1\ \ Individuation of mass to count}
While they are typically used with count nouns, plural words may combine with mass nouns, provided these are recategorised. Combining a plural word with a mass noun indicates that it is interpreted as a count noun. For instance, Teiwa \textit{yir }{\textquoteleft}water{\textquoteright} is interpreted as a mass in (77a), but gets an individuated reading in (77b) when it combines with \textit{non}. In Kamang (78a) the noun \textit{ili }{\textquoteleft}water{\textquoteright} combined with \textit{nung }is individuated just like when it combines with the numeral \textit{nok }{\textquoteleft}one{\textquoteright} (78b).\footnote{As we saw in section 4.1, Western Pantar \textit{maru(ng) }does not have this individuating function due to the sense of comprehensiveness and completeness of the word.}

Teiwa (Klamer, Teiwa corpus)

\begin{flushleft}
\tablehead{}
\begin{supertabular}{m{0.40595984in}m{0.43795988in}m{0.5365598in}m{0.44065985in}m{0.47125986in}m{0.6143598in}m{0.44345984in}m{0.57685983in}m{0.42545983in}m{0.37955984in}m{0.35255986in}}
(77) &
a. &
\itshape Na &
\itshape yir &
\itshape ma &
\itshape gelas &
\textit{mia}\textit{[241?]}\textit{.} &
 &
 &
 &
\\
 &
 &
\scshape 1sg &
water &
\scshape obl &
glass &
fill &
 &
 &
 &
\\
 &
 &
\multicolumn{9}{m{4.8706603in}}{{\textquoteleft}I fill the glass with water.{\textquoteright}}\\
\end{supertabular}
\end{flushleft}
\begin{flushleft}
\tablehead{}
\begin{supertabular}{m{0.40595984in}m{0.43725982in}m{0.5365598in}m{0.44065985in}m{0.47195986in}m{0.6136598in}m{0.43935987in}m{0.57685983in}m{0.42545983in}m{0.37955984in}m{0.35255986in}}
 &
b. &
\itshape Na &
\itshape yir &
\itshape non &
\itshape ma &
\itshape drom &
\textit{mia}\textit{[241?]}\textit{.} &
 &
 &
\\
 &
 &
\scshape 1sg &
water &
\scshape pl &
\scshape obl &
drum &
fill &
 &
 &
\\
 &
 &
\multicolumn{9}{m{4.8665605in}}{{\textquoteleft}I fill the drum with several containers of water.{\textquoteright}}\\
\end{supertabular}
\end{flushleft}
Kamang (Schapper, fieldnotes)

\begin{flushleft}
\tablehead{}
\begin{supertabular}{m{0.34905985in}m{0.18725985in}m{0.6726598in}m{0.68795985in}m{0.8504598in}m{0.21635985in}m{0.21635985in}m{0.63865983in}m{0.41085985in}m{0.65735984in}}
(78)  &
a. &
\itshape ili &
\itshape nung &
 &
 &
b. &
\itshape ili &
\itshape nok &
\\
 &
 &
water &
\textsc{pl}\textsc{ } &
 &
 &
 &
water &
one &
\\
 &
 &
\multicolumn{3}{m{2.3685598in}}{{\textquoteleft}\{multiple individual\} waters{\textquoteright}} &
 &
 &
\multicolumn{3}{m{1.8643599in}}{{\textquoteleft}a water{\textquoteright}}\\
\end{supertabular}
\end{flushleft}
The plural words in Abui and Wersing also occur together with mass nouns with readings of abundance, as discussed already in section 4.2. Western Pantar \textit{marung }cannot combine with mass nouns.

\subsubsection[4.3.2\ \ Clan or place name to members ]{\textup{4.3.2\ \ Clan or place name to members}\textup{ }}
When Abui \textit{loku }is combined with the name of a clan or a place name, the expression refers to the members belonging to that clan (79) or issuing from that place (80), a use that can be extended to the question word \textit{te }{\textquoteleft}where{\textquoteright} (81).

Abui (Kratochvil 2007: 165)

\begin{flushleft}
\tablehead{}
\begin{supertabular}{m{0.34905985in}m{1.2837598in}m{0.63865983in}m{0.07615984in}}
(79) &
\itshape Afui Ata &
\itshape loku &
\\
 &
clan.name &
\scshape pl &
\\
 &
\multicolumn{2}{m{2.00116in}}{{\textquoteleft}people of the Afui Ata clan{\textquoteright} } &
\\
\end{supertabular}
\end{flushleft}
Abui (Kratochvil 2007: 166)

\begin{flushleft}
\tablehead{}
\begin{supertabular}{m{0.34905985in}m{0.8636598in}m{0.57335985in}m{0.07615984in}}
(80) &
\itshape Kafola  &
\itshape loku &
\\
 &
Kabola  &
\scshape pl &
\\
 &
\multicolumn{2}{m{1.51576in}}{\textsc{{\textquoteleft}}people from Kabola{\textquoteright} } &
\\
\end{supertabular}
\end{flushleft}
Abui (Kratochvil, fieldnotes)

\begin{flushleft}
\tablehead{}
\begin{supertabular}{m{0.34905985in}m{0.6219598in}m{0.47815987in}m{0.40045986in}m{0.9101598in}}
(81) &
\itshape Edo &
\itshape te &
\itshape loku, &
\itshape naana?\\
 &
\scshape 2sg.foc &
where &
\scshape pl &
older.sibling\\
 &
\multicolumn{4}{m{2.64696in}}{{\textquoteleft}Where are you from, bro?{\textquoteright} }\\
\end{supertabular}
\end{flushleft}
A similar use is attested for Teiwa \textit{non }when it is used to make an ethnonym from a clan name (82). However, Teiwa \textit{non }cannot combine with place names. 

\textit{ }Teiwa (Klamer, Teiwa corpus)

\begin{flushleft}
\tablehead{}
\begin{supertabular}{m{0.44975987in}m{0.75115985in}m{0.43095985in}m{0.77335984in}m{0.6865598in}m{0.48795983in}m{0.6448598in}m{0.47265986in}m{0.42055985in}m{0.38865986in}}
(82) &
\itshape Teiwa &
\itshape non &
\textit{ga}\textit{{\textglotstop}}\textit{an} &
\textit{ita}\textit{{\textglotstop}}\textit{a } &
\itshape ma &
\itshape gi? &
 &
 &
\\
 &
clan.name &
\scshape pl &
that.\textsc{knwn} &
where &
\scshape obl &
go &
 &
 &
\\
 &
\multicolumn{9}{m{5.6866603in}}{{\textquoteleft}Where did that group of Teiwa [people] go to?{\textquoteright}}\\
\end{supertabular}
\end{flushleft}
This function of the plural word is not known to occur in Western Pantar, Kamang or Wersing. In Kamang this kind of plurality is encoded by the combination of a place name with a group plural pronoun, as in (83). 

Kamang (Schapper, fieldnotes)

\begin{flushleft}
\tablehead{}
\begin{supertabular}{m{0.44835988in}m{0.31225985in}m{0.15875983in}m{1.6518599in}m{0.77125984in}m{0.8309598in}m{0.48655984in}m{1.1372598in}}
(83) &
\multicolumn{2}{m{0.5497598in}}{\itshape Ga  } &
\itshape wo-suk-si=bo   &
\itshape gafaa &
\itshape Takailubui &
\itshape geifu   &
\textit{mauu-h=a},...\\
 &
\multicolumn{2}{m{0.5497598in}}{3.\textsc{agt}} &
\textsc{3.loc-}think\textsc{{}-ipfv=lnk} &
3.\textsc{alone} &
Takailubui &
3.\textsc{grp} &
war-\textsc{purp}=\textsc{spec}\\
\multicolumn{2}{m{0.8393598in}}{} &
\multicolumn{6}{m{5.43036in}}{{\textquoteleft}They think that if they alone make war against the people of Takailubui,...{\textquoteright}}\\
\end{supertabular}
\end{flushleft}
\subsection[4.4\ \ Partitive]{4.4\ \ Partitive}
Plural words also occur in contexts of partitive plural reference. This means that the plural can be used to pick out a part or group of referents from a larger set. 

The Kamang plural word \textit{nung }can be used for partitive plural reference, often with contrast between different subsets of referents. For instance, in (84) \textit{nung} is used twice to divide the set of citruses into the multitude that are sweet and the multitude that are sour. Similarly, in (86) \textit{nung} is used twice to contrast the sub-set of people who went to Molpui with the sub-set that went to the nearby village.

Kamang (Stokhof 1982: 40)

\begin{flushleft}
\tablehead{}
\begin{supertabular}{m{0.44975987in}m{0.7483598in}m{0.43235984in}m{0.5254598in}m{0.8323598in}m{0.48795983in}m{0.64555985in}m{0.47195986in}m{0.41985986in}m{0.38935986in}}
(84) &
\itshape Muut=ak &
\itshape nung &
\itshape iduka, &
\itshape ah=a &
\itshape nung &
\itshape alesei. &
 &
 &
\\
 &
citrus=\textsc{pl} &
\scshape pl &
sweet &
\scshape cnct=spec &
\scshape pl &
sour &
 &
 &
\\
 &
\multicolumn{9}{m{5.58316in}}{{\textquoteleft}Some of these citrus fruits, others are sour.{\textquoteright}}\\
\end{supertabular}
\end{flushleft}
Kamang (Stokhof 1978: 57)

\begin{flushleft}
\tablehead{}
\begin{supertabular}{m{0.34905985in}m{0.43235984in}m{0.6427598in}m{1.0045599in}m{0.34905985in}m{0.5525598in}m{0.7344598in}m{0.98095983in}m{0.9858598in}m{0.08095985in}}
(85) &
\itshape Nung &
\itshape gera &
\itshape ye-iyaa &
\itshape ai &
\itshape Molpui &
\itshape wo-oi &
\itshape ye-te, &
 &
\\
 &
\scshape pl &
\scshape 3.contr &
\textsc{3.gen}{}-return &
take &
M. &
3.\textsc{loc}{}-to &
\textsc{3.gen}{}-go.up &
 &
\\
 &
\multicolumn{9}{m{6.39346in}}{{\textquoteleft}Some of them went home going up to Molpui,}\\
\end{supertabular}
\end{flushleft}
\begin{flushleft}
\tablehead{}
\begin{supertabular}{m{0.41915986in}m{0.58585984in}m{0.6427598in}m{0.5705598in}m{0.9643598in}m{0.45455983in}m{0.5990598in}m{0.44005987in}m{0.39275986in}m{0.36435986in}}
 &
\itshape nung &
\itshape gera &
\itshape yeeisol &
\itshape ye-iyaa &
\itshape ai &
 &
 &
 &
\\
 &
\scshape pl &
\scshape 3.contr &
straight &
\textsc{3.gen}{}-return &
take &
 &
 &
 &
\\
 &
\multicolumn{9}{m{5.64426in}}{others went straight home going}\\
\end{supertabular}
\end{flushleft}
\begin{flushleft}
\tablehead{}
\begin{supertabular}{m{0.38445985in}m{0.5788598in}m{1.0566599in}m{1.3962599in}m{0.8733598in}m{0.41635984in}m{0.5455598in}m{0.40385985in}m{0.36085984in}m{0.33515984in}}
 &
\itshape mane &
\itshape wo-oi &
\itshape ye-wete. &
 &
 &
 &
 &
 &
\\
 &
village &
3.\textsc{loc}{}-towards &
\textsc{3.gen}{}-go.up.across &
 &
 &
 &
 &
 &
\\
 &
\multicolumn{9}{m{6.59686in}}{up across to the village.{\textquoteright}}\\
\end{supertabular}
\end{flushleft}
The Wersing plural word can be used also in partitive plural reference, but does not typically make explicit contrasts between subsets using the plural word over multiple NPs. For instance, in (87) \textit{deing} refers to a subset of candle nuts that have not yet been crushed, the other set is not explicitly mentioned but must simply be inferred from the discourse context. In (88), the other member of the whole (namely the speaker himself) is singular and so is not marked with the plural word, but he is contrasted with the set of others who are teaching other languages. This second plural set is accordingly marked with \textit{deing}.

Wersing (Schapper, fieldnotes)\textsc{\ \ }

\begin{flushleft}
\tablehead{}
\begin{supertabular}{m{0.34835985in}m{0.081659846in}m{0.5087598in}m{0.5094598in}m{0.5094598in}}
\multicolumn{2}{m{0.5087598in}}{(86)} &
\itshape Deing &
\itshape de &
\itshape naung.\\
\multicolumn{2}{m{0.5087598in}}{} &
\scshape pl &
\scshape ipfv &
\scshape neg\\
 &
\multicolumn{4}{m{1.8455598in}}{{\textquoteleft}Some are still not done.{\textquoteright}}\\
\end{supertabular}
\end{flushleft}
Wersing (Schapper, fieldnotes)\ \ \ \ \ \ 

\begin{flushleft}
\tablehead{}
\begin{supertabular}{m{0.36505985in}m{0.35525984in}m{0.8101598in}m{0.42545983in}m{0.28585985in}m{0.34415984in}m{0.18865985in}m{0.10875984in}m{0.24555984in}m{0.24135986in}m{0.15385985in}m{0.47685984in}m{0.10735985in}m{0.057359837in}m{0.7997598in}m{-0.062740155in}m{0.43935987in}m{0.26505986in}m{0.50255984in}}
(87) &
\multicolumn{2}{m{1.2441598in}}{\itshape Naida  } &
\itshape Abui &
\multicolumn{3}{m{0.9761599in}}{\itshape ge-lomu} &
\multicolumn{3}{m{0.7531598in}}{\itshape ong} &
\multicolumn{3}{m{0.89555985in}}{\itshape ge-tenara,} &
\multicolumn{4}{m{1.46996in}}{} &
 &
\\
 &
\multicolumn{2}{m{1.2441598in}}{\scshape 1pl.excl.top} &
Abui &
\multicolumn{3}{m{0.9761599in}}{3-language} &
\multicolumn{3}{m{0.7531598in}}{use} &
\multicolumn{3}{m{0.89555985in}}{3-teach\ \ } &
\multicolumn{4}{m{1.46996in}}{} &
 &
\\
 &
\multicolumn{17}{m{6.5019593in}}{{\textquoteleft}I will teach them Abui} &
\\
\multicolumn{2}{m{0.7990598in}}{} &
\multicolumn{16}{m{6.0679593in}}{} &
\\
 &
\multicolumn{2}{m{1.2441598in}}{\itshape pang=sa} &
\multicolumn{3}{m{1.2129599in}}{\itshape te-nong} &
\multicolumn{3}{m{0.7004598in}}{\itshape aumang} &
\multicolumn{3}{m{1.0295599in}}{\itshape dein=a} &
\multicolumn{4}{m{1.13796in}}{} &
 &
\multicolumn{2}{m{0.84635985in}}{}\\
 &
\multicolumn{2}{m{1.2441598in}}{\scshape dem=conj} &
\multicolumn{3}{m{1.2129599in}}{\textsc{1pl.incl}{}-friend} &
\multicolumn{3}{m{0.7004598in}}{other} &
\multicolumn{3}{m{1.0295599in}}{\scshape pl=art} &
\multicolumn{4}{m{1.13796in}}{} &
 &
\multicolumn{2}{m{0.84635985in}}{}\\
 &
\multicolumn{18}{m{7.0832596in}}{and other friends of ours}\\
 &
\multicolumn{18}{m{7.0832596in}}{}\\
\multicolumn{2}{m{0.7990598in}}{} &
\itshape Pantara &
\multicolumn{2}{m{0.7900598in}}{\itshape ge-lomu} &
\multicolumn{3}{m{0.7990598in}}{\itshape ong} &
\multicolumn{3}{m{0.7982598in}}{\itshape ge-tenara} &
\multicolumn{3}{m{0.7990598in}}{\itshape war  } &
\itshape Sawila   &
\multicolumn{3}{m{0.7991598in}}{\itshape ge-lomu.} &
\\
\multicolumn{2}{m{0.7990598in}}{} &
Pantar &
\multicolumn{2}{m{0.7900598in}}{3-language} &
\multicolumn{3}{m{0.7990598in}}{use} &
\multicolumn{3}{m{0.7982598in}}{3-teach} &
\multicolumn{3}{m{0.7990598in}}{and} &
Sawila &
\multicolumn{3}{m{0.7991598in}}{3-language} &
\\
\multicolumn{2}{m{0.7990598in}}{} &
\multicolumn{16}{m{6.0679593in}}{will teach them Pantar and Sawila.{\textquoteright}} &
\\
\end{supertabular}
\end{flushleft}
Such a contrastive use of the plural word has not been attested in Western Pantar and Abui, but may be a sense present in Teiwa \textit{non}, see (22) and (23) (section 3.2).

\subsection[4.5\ \ Vocative]{4.5\ \ Vocative}
A term of address, relation or kin can be also marked with a plural to express a plural vocative. Western Pantar \textit{marung} has a vocative use in (89). Teiwa \textit{non }can be used in vocatives with kin terms, for instance, when starting a speech (90) or in a hortative (91). 

Western Pantar (Holton, Western Pantar corpus)

\begin{flushleft}
\tablehead{}
\begin{supertabular}{m{0.37125984in}m{0.65525985in}m{0.7365598in}m{0.48795983in}m{0.6726598in}m{0.5573598in}m{0.41915986in}m{0.8163598in}m{0.31085986in}m{0.31435984in}}
(88) &
\itshape Wenang  &
\itshape marung &
\itshape hing &
\itshape yadda &
\itshape mising,  &
\itshape nang &
\textit{na-ti}\textit{[241?]}\textit{ang.} &
 &
\\
 &
Mr  &
\scshape pl &
\scshape pl &
\scshape not.yet &
sit &
I &
\textsc{1sg}{}-sleep &
 &
\\
 &
\multicolumn{9}{m{5.6004596in}}{{\textquoteleft}You all keep sitting, I{\textquoteright}m going to sleep.{\textquoteright}}\\
\end{supertabular}
\end{flushleft}
Teiwa (Klamer, Teiwa corpus)

\begin{flushleft}
\tablehead{}
\begin{supertabular}{m{0.34835985in}m{1.4490598in}m{0.32125986in}m{0.41635984in}m{0.075459845in}m{0.075459845in}m{0.075459845in}m{0.07475985in}m{0.075459845in}m{0.07815985in}}
(89) &
\itshape Na-rat qai &
\itshape non &
\itshape oh! &
 &
 &
 &
 &
 &
\\
 &
1\textsc{sg}.\textsc{poss}{}-grandchild &
\scshape pl &
\scshape excl &
 &
 &
 &
 &
 &
\\
 &
\multicolumn{9}{m{3.2713602in}}{{\textquoteleft}Oh my grandchildren!{\textquoteright}}\\
\end{supertabular}
\end{flushleft}
Teiwa (Klamer, Teiwa corpus)

\begin{flushleft}
\tablehead{}
\begin{supertabular}{m{0.41085985in}m{1.9351599in}m{0.40455985in}m{0.5080598in}m{0.6761598in}m{0.37545985in}m{0.5163598in}m{0.31985986in}m{0.28865984in}m{0.27125984in}}
(90) &
\itshape Na-gas qai &
\itshape non, &
\itshape tup &
\itshape pi &
\itshape gi &
\itshape ina. &
 &
 &
\\
 &
\textsc{1sg-}female.younger.sibling &
\scshape pl &
get.up &
\scshape 1pl.incl &
go &
eat &
 &
 &
\\
 &
\multicolumn{9}{m{5.92546in}}{{\textquoteleft}My (female) friends, let{\textquoteright}s get up to eat.{\textquoteright}}\\
\end{supertabular}
\end{flushleft}
Abui \textit{loku }also can be present in vocative contexts with relational nouns (91) or kin terms (92).

Abui (Schapper, fieldnotes)\textsc{\ \ }

\begin{flushleft}
\tablehead{}
\begin{supertabular}{m{0.39005986in}m{1.0900599in}m{0.40045986in}m{0.31225985in}m{0.36705986in}m{0.8691598in}}
 (91) &
\itshape Ne-feela &
\itshape loku, &
\itshape yaa &
\itshape fat &
\itshape ho-aneek.\\
 &
\textsc{1sg.gen-}friend &
\scshape pl &
go &
corn &
\textsc{3.loc}{}-weed\\
 &
\multicolumn{5}{m{3.35396in}}{{\textquoteleft}My friends, go weed the corn.{\textquoteright}}\\
\end{supertabular}
\end{flushleft}
Abui (Schapper, fieldnotes)\textsc{\ \ }

\begin{flushleft}
\tablehead{}
\begin{supertabular}{m{0.43865988in}m{1.5559598in}m{0.42755982in}m{0.5150598in}m{0.6226598in}m{0.44485983in}m{0.58515984in}m{0.43095985in}m{0.38375986in}m{0.35665986in}}
(92) &
\itshape Ne-fing &
\itshape loku, &
\itshape me! &
 &
 &
 &
 &
 &
\\
 &
\textsc{1sg.gen-}elder.sibling &
\scshape pl &
come &
 &
 &
 &
 &
 &
\\
 &
\multicolumn{9}{m{5.9525595in}}{{\textquoteleft}My siblings, come on already.{\textquoteright}}\\
\end{supertabular}
\end{flushleft}
There is no reason to expect that plural words should not be usable in vocatives. Yet, the plural word is not found in Kamang or Wersing vocatives. In Kamang, there are a range of special vocatives for calling (a) child(ren) or (b) friend(s). A Kamang vocative suffix, when used, means that a noun cannot be further modified, for instance, with the plural word.

\subsection[4.6\ \ Summary ]{4.6\ \ Summary }
Plural words code more than plurality; they have additional connotations and usages which vary across the languages as summarised in Table 3. 

{\centering
Table 3: Semantics of Alor-Pantar plural words
\par}

\begin{flushleft}
\tablehead{}
\begin{supertabular}{|m{2.34006in}|m{0.6809598in}|m{0.6809598in}|m{0.6809598in}|m{0.7788598in}|m{0.8094598in}|}
\hline
 &
\bfseries Teiwa  &
\bfseries Abui  &
\bfseries Wersing &
\bfseries W Pantar &
\bfseries Kamang \\\hline
Completeness  &
no &
yes &
yes {\dag} &
yes &
no\\\hline
Vocative &
yes &
yes &
no &
yes &
no\\\hline
Individuation: mass {\textgreater} count &
yes &
yes &
no &
no &
yes\\\hline
Individuation: name {\textgreater} members   &
yes &
yes &
no &
no &
no\\\hline
Abundance &
yes &
no &
yes {\ddag} &
no &
no\\\hline
Partitive &
no &
no &
yes &
no &
yes\\\hline
\end{supertabular}
\end{flushleft}
{\dag} On topic pronouns only. {\ddag} With inanimates only.

\section[5\ \ Typological perspectives on plural words in AP languages]{5\ \ Typological perspectives on plural words in AP languages}
We saw in section 1 that a good deal of what was known of the typology of plural words is due to Matthew Dryer{\textquoteright}s work, in particular Dryer (1989, 2011) and to a lesser extent Dryer (2007). Dryer (2011) documents the use of plural words in the coding of nominal plurality. In doing so, Dryer wanted to prove the existence of a phenomenon that was not generally recognised, and his definitions reflect that. As mentioned in section 1, for Dryer, to be a plural word an item must be the prime indicator of plurality, and in the pure case they have this as their unique function. Based on this constrained characterisation, Dryer shows that plural words nevertheless show considerable diversity. 

\ \ First, while being by definition non-affixal, they vary according to their degree of phonological independence. Second, they show great variety in the word class to which they belong; they may be integrated (to a greater or lesser degree) into another class, or form a unique class. The examples from the Alor-Pantar languages show vividly the variety of plural words in this regard: in all of them, plural words form a unique class on their own, which is however integrated into another class - but which class is variable across the languages. For instance, in Teiwa, the plural word is part of the noun phrase and behaves largely like a nominal quantifier, while in Kamang, rather than actually being part of the noun phrase, the plural word distributes as a noun phrase itself. 

\ \ Third, plural words may have different values. In this respect they are perhaps poorly named. Dryer (1989: 869) suggests that {\textquotedblleft}grammatical number words{\textquotedblright} would be a better term, since he gives instances of singular words and dual words. This is an area where Alor-Pantar languages indicate how the typology can be taken forward. When we look at the full range of {\textquotedblleft}ordinary{\textquotedblright} number values, those associated with affixal morphology, we distinguish {\textquoteleft}determinate{\textquoteright} and {\textquoteleft}indeterminate{\textquoteright} number values (Corbett 2000: 39-41). Determinate number values are those where only one form is appropriate, given the speaker{\textquoteright}s knowledge of the real world. If a language has an obligatory dual, for instance, this would be a determinate number value since to refer to two distinct entities this would be the required choice. However, values such as paucal or greater plural are not like 
this; there is an additional element to the choice. We find this same distinction in the Alor-Pantar number words: for instance, Teiwa \textit{non }signals not just plurality but has the connotation of abundance (like the greater plural).

\ \ Fourth, a key part of the typology of number systems is the items to which the values can apply. Two systems may be alike in their values (say both have singular and plural) but may differ dramatically in that in one language almost all nominals have singular and plural available, while in the other plurality may be restricted to a small (top) segment of the Animacy Hierarchy. The data from Alor-Pantar languages are important in showing how this type of differentiation applies also with number words. With affixal number, we find instances of recategorisation; these are found particularly where a mass noun is recategorised as a count noun, and then has singular and plural available. We see this equally in Alor-Pantar languages such as Kamang where \textit{nung }is used with \textit{ili }{\textquoteleft}water{\textquoteright}, when recategorised as a count noun. 

\ \ Furthermore, number words are not restricted to appearing with nouns. In Abui, plural \textit{loku }can occur with a third person pronoun; \textit{hel }is the third singular pronoun, which can be pluralised by \textit{loku}. While this is of great interest, other languages go further. A fine example is Miskitu, a Misumalpan language of Nicaragua and Honduras. Number is marked by number words (Green ms., Andrew Koontz-Garboden, p.c.), singular (\textit{kum}) and plural (\textit{nani}). Pronouns take the plural word, rather like nouns: 

Miskitu (Green ms., Andrew Koontz-Garboden, p.c.)

(113) \ \ \textit{Yang\ \ nani\ \ kauhw-ri.}

\ \ 1\ \ \textsc{pl}\ \ fall-1.\textsc{pst.indf}

\ \ {\textquoteleft}We (exclusive) fell.{\textquoteright}

This example, like all those cited above from Alor-Pantar languages, helps to extend the typology of number words; as we gather a fuller picture, the typology of number words becomes increasingly like that of affixal number.

\clearpage\section[6\ \ Conclusions]{6\ \ Conclusions}
Proto-Alor-Pantar had a plural word of the shape *\textit{non}. Some daughter languages inherited this form, others innovated one or more plural words. In none of the five AP languages investigated here do restrictions apply on the type of referents that can be pluralised with the plural word, and all of them prohibit a combination of the plural word and a numeral in a single constituent. 

\ \ The syntax of the plural word varies. In each language investigated here the word constitutes a class of its own. In Western Pantar, the plural word shares much with adjectival quantifiers and numerical expressions, in Teiwa it patterns mostly with non-numeral quantifies, and in Kamang, Abui and Wersing plural words function very much like nouns. The plural words in the five languages behave differently, so that it is not possible to establish a category of plural word that is cross-linguistically uniform.

\ \ The plural words all code plurality, but in all five languages they have additional connotations, such as expressing a sense of completeness or abundance. A plural word may also function to impose an individuated reading of a referent, or to pick out a part or group of referents from a larger set. Plural words are used to express plural vocatives. None of the additional senses and functions of the plural words is shared across all of the five languages. 

\ \ What our study shows is that, even amongst five typologically similar and genetically closely related languages whose ancestor had a plural word, the original plural word has drifted in different syntactic directions and developed additional semantic dimensions, showing a degree of variation that is higher than any other inherited word class in the languages. 

\section[References]{References}
Baird, Louise. 2008. \textit{A grammar of Klon: A non-Austronesian language of Alor, Indonesia}. Canberra: Pacific Linguistics.

Corbett, Greville G. 2000. \textit{Number}. Cambridge: Cambridge University Press.

Dryer, Matthew S. 1989. Plural words.  \textit{Linguistics} 27: 865-895.

Dryer, Matthew S. 2007. Noun phrase structure. In Shopen, Timothy (ed.), \textit{Language Typology and Syntactic Description. Volume II: Complex Constructions.} Cambridge: Cambridge University Press, 151-205.

Dryer, Matthew S. 2011. Coding of Nominal Plurality. In Dryer, Matthew S. and Haspelmath, Martin (eds.), The World Atlas of Language Structures Online. Munich: Max Planck Digital Library, chapter 33. Available online at http://wals.info/chapter/33. Accessed on 2011-09-01.

Green, Tom. ms. Covert clause structure in the Miskitu noun phrase. Unpublished paper, 1992, Cambridge, MA: MIT.

Haan, Johnson Wellem. 2001. \textit{A grammar of Adang: A Papuan language spoken on the Island of Alor, East Nusa Tenggara, Indonesia.} PhD Thesis, University of Sydney.

Holton, Gary and Mahalel Lamma Koly. 2008. \textit{Kamus pengantar Bahasa Pantar Barat}. Kupang: UBB. 

Holton, Gary. 2012. Number in the Papuan outliers of East Nusantara. Paper presented at the International Conference on Austronesian Linguistics, Bali, 5 July 2012.

Holton, Gary. To appear. Western Pantar. In Antoinette Schapper (ed.), \textit{Papuan languages of Timor-Alor-Pantar: Sketch grammars}. Berlin/New York: De Gruyter Mouton.

Klamer, Marian. 2010. \textit{A grammar of Teiwa}. Berlin / New York: De Gruyter Mouton.

Klamer, Marian. In press. The history of Numeral classifiers in Teiwa (Papuan). In Gerrit J. Dimmendaal and Anne Storch (eds.), \textit{Number: Constructions and Semantics. Case studies from Africa, India, Amazonia and Oceania}. Amsterdam: Benjamins. 

Klamer, Marian. To appear. Kaera. In Antoinette Schapper\textit{ }(ed.), \textit{Papuan languages of Timor-Alor-Pantar. Sketch grammars}. Berlin/New York: De Gruyter Mouton.

Kratochv\'il, Franti\v{s}ek. 2007. \textit{A grammar of Abui}. PhD dissertation, Leiden University. Utrecht: LOT Publications.

Kratochv\'il, Franti\v{s}ek and Benidiktus Delpada. 2008. \textit{Kamus pengantar Bahasa Abui} [Abui-Indonesian-English dictionary]. Kupang, Indonesia: UBB-GMIT.

Kratochv\'il, Franti\v{s}ek. n.d. Dictionary of Sawila. Unpublished Lexique-Pro database. Singapore, Nanyang Technological University. 

Malikosa, Anderias. n.d. Yesus Sakku Geleworo Kana. Unpublished manuscript. UBB-GMIT Kupang, Indonesia.

Robinson, Laura and John Haan. To appear. Adang. In Antoinette Schapper\textit{ }(ed.), \textit{Papuan languages of Timor-Alor-Pantar. Sketch grammars}. Berlin/New York: De Gruyter Mouton.

Schapper, Antoinette. 2011. Crossing the border: Historical and linguistic divides among the Bunaq in central Timor. \textit{Wacana, Journal of the Humanities of Indonesia} 13.1: 29-49.

Schapper, Antoinette. Ms. Grammar of Kamang, a Papuan language of central Alor. Unpublished manuscript, Leiden University.

Schapper, Antoinette. To appear. Kamang. In Antoinette Schapper\textit{ }(ed.), \textit{Papuan languages of Timor-Alor-Pantar. Sketch grammars. }Berlin/New York: De Gruyter Mouton.\textit{ }

Schapper, Antoinette and Rachel Hendery. To appear. Wersing. In Antoinette Schapper (ed.), \textit{Papuan languages of Timor-Alor-Pantar. Sketch grammars. }Berlin/New York: De Gruyter Mouton.

Schapper, Antoinette and Juliette Huber. Ms. A reconstruction of the Timor-Alor-Pantar family: Phonology and morphology. Manuscript, Leiden University and Lund University.

Schapper, Antoinette and Marian Klamer. 2011. Plural words in the Alor-Pantar languages. In Peter K. Austin, Oliver Bond, David Nathan and Lutz Marten (eds.), \textit{Proceedings of Conference on Language Documentation and Linguistic Theory 3}:247-256.  

Stokhof, W. A. L. 1978. Woisika text, \textit{Miscellaneous Studies in Indonesian and Languages in Indonesia} 5: 34-57. Jakarta:  NUSA.

Stokhof, W. A. L. 1982. \textit{Woisika riddles}. Canberra: Pacific Linguistics.


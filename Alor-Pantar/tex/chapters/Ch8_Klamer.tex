\clearpage\setcounter{page}{1}\pagestyle{Standard}
{\centering
\textit{Chapter 8}
\par}

{\centering
\textbf{Numeral Words and Arithmetic Operations }
\par}

{\centering
\textbf{in the Alor-Pantar languages }
\par}

{\centering
\emph{Marian Klamer, Antoinette Schapper, Greville Corbett, Gary Holton, }
\par}

{\centering
\emph{Franti}\emph{\v{s}ek}\emph{ Kratochv\'il, Laura Robinson}
\par}

\setcounter{tocdepth}{3}
\renewcommand\contentsname{}
\tableofcontents
\clearpage{\centering
\textbf{\textit{Chapter 8}}
\par}

{\centering
\textbf{Numeral Words and Arithmetic Operations }
\par}

{\centering
\textbf{in the Alor-Pantar languages }
\par}

{\centering
\emph{Marian Klamer, Antoinette Schapper, Greville Corbett, Gary Holton, }
\par}

{\centering
\emph{Franti}\emph{\v{s}ek}\emph{ Kratochv\'il, Laura Robinson}
\par}

\emph{\textup{Abstract: }}The indigenous numerals of the AP languages, as well as the indigenous structures for arithmetic operations are currently under pressure from Indonesian, and will inevitably be replaced with Indonesian forms and structures. This chapter presents a documentary record of the forms and patterns currently in use to express numerals and arithmetic operations in the Alor-Pantar languages. We describe the structure of cardinal, ordinal and distributive numerals, and how operations of addition, subtraction, multiplication, division, and fractions are expressed.

\section[1 \ \ \ \ Introduction]{1 \ \ \ \ Introduction}
\hypertarget{RefHeading105228871885726}{}\hypertarget{Toc376958929}{}Numeral systems are more endangered than languages. Cultural or commercial superiority of one group over another often results in borrowing of numerals, or replacements of parts or all of a numeral system, even in a language that itself is not endangered (Comrie 2005). In the Alor-Pantar (AP) context, the national language, Indonesian, plays a dominant role in education and commerce, and this will inevitably lead to the replacement of the numerals and the arithmetic expressions with Indonesian equivalents. It is therefore crucial to keep a record of the forms and patterns as they are currently used for future reference, and this chapter aims to be such a documentary record. 

The patterns described in this chapter fall into two broad classes, pertaining to two distinct linguistic levels: the word (section 2) and the clause or sentence (section 3). At the word level we describe how numeral words are created, discussing the structure of cardinals (2.1), ordinals (2.2) and distributives (2.3). At the clause and sentence level, we describe the constructions that contain numerals and function to express the arithmetic of addition (3.1), subtraction (3.2), multiplication (3.3), division (3.4), and fractions (3.5). Section 4 presents a summary and conclusions. 

Details on the data on which this chapter is based is given in the  Sources section at the end of this chapter. Adang, Blagar, Kamang and Abui are each very diverse internally. The data presented in this chapter are from the Dolap dialect of Blagar, the Takailubui dialect of Kamang, the Lawahing dialect of Adang, and the Takalelang dialect of Abui. These dialect names refer to the place where the variety is spoken.

\clearpage\section[2 \ \ \ \ Operations to create numeral words ]{2 \ \ \ \ Operations to create numeral words }
\hypertarget{RefHeading105230871885726}{}Most of the cardinals in AP languages are historically morphologically complex forms. Within and across the languages we find variation in choice of numeral base, the type of operations invoked for the interpretation of the composite elements, and the ways in which these operations are expressed (section 2.1). Ordinals in AP languages are possessive constructions that are derived from cardinals, where the ordered entity is the grammatical possessor of the cardinal (section 2.2). Distributive numerals in AP languages are also derived from the cardinal, by reduplicating it partially or fully. When the cardinal contains more than one morpheme, generally only the right-most formative is reduplicated (section 2.3).  In all cases, the numeral words follow the noun they quantify. Cardinals may be preceded by a classifier, if the language has them. 

\subsection[2.1 \ \ Cardinal numerals]{\textup{2.1 \ \ Cardinal numerals}}
\hypertarget{RefHeading105232871885726}{}By cardinal numerals, we understand the set of numerals used in attributive quantification of nouns (e.g., {\textquoteleft}three dogs{\textquoteright}). In enumeration, the numeral follows the noun in all AP languages (N NUM), as in Teiwa \textit{yaf haraq }{\textquoteleft}house two{\textquoteright} {\textgreater} {\textquoteleft}two houses{\textquoteright}. If a language uses a sortal or mensural classifier, the classifier occurs between the noun and the numeral (N CLF NUM). The same cardinals that are used in enumeration are also used for non-referential counting (\textit{one, two, three, four, five, etc.}), and all the AP languages use the same numeral forms to count small animates (ants, flies, bees, or house lizards), large animates (children, dogs, or pigs), and inanimates (houses, rocks, stars, or coconut trees).

In all the AP languages we surveyed, the cardinal numbers {\textquoteleft}one{\textquoteright} to {\textquoteleft}five{\textquoteright} are morphologically simple forms, as illustrated in Table 1. The composition of {\textquoteleft}six{\textquoteright} varies. Most of the AP languages have a monomorphemic {\textquoteleft}six{\textquoteright}, an example is Teiwa \textit{tiaam}. Bi-morphemic forms for {\textquoteleft}six{\textquoteright} are composed of (reflexes) of {\textquoteleft}five{\textquoteright} and {\textquoteleft}one{\textquoteright}, e.g., Kula \textit{yawaten sona}. The cardinals {\textquoteleft}seven{\textquoteright} and higher consist of minimally two formatives in all AP languages. Often, these forms involve reflexes of {\textquoteleft}five{\textquoteright}, {\textquoteleft}one{\textquoteright}, {\textquoteleft}two{\textquoteright}, {\textquoteleft}three{\textquoteright} and {\textquoteleft}four{\textquoteright}, as illustrated in Table 1, though other patterns are also attested (Schapper and 
Klamer, this volume). 

{\centering
Table 1: {\textquoteleft}One{\textquoteright} through {\textquoteleft}nine{\textquoteright} in Teiwa (Pantar) and Kula (East Alor)
\par}

\begin{flushleft}
\tablehead{}
\begin{supertabular}{m{0.72815984in}m{0.7219598in}m{1.0344598in}m{1.1906599in}}
\hline
\textbf{Cardinal} &
\textbf{Analysis} &
\textbf{Teiwa} &
\textbf{Kula}\\\hline
1 &
1 &
\textit{nuk} &
\textit{sona}\\
2 &
2 &
\textit{(ha)raq} &
\textit{yakwu}\\
3 &
3 &
\textit{yerig} &
\textit{tu}\\
4 &
4 &
\textit{ut} &
\textit{arasiku}\\
5 &
5 &
\textit{yusan} &
\textit{yawatena}\\
6 &
5  1 &
 &
\textit{yawaten sona}\\
6 &
6 &
\textit{tiaam} &
\\
7  &
5  2 &
\textit{yes raq} &
\textit{yawaten yakwu}\\
8  &
5  3 &
\textit{yes nerig} &
\textit{yawaten tu}\\
9 &
5  4 &
\textit{yes na{\textquoteright}ut} &
\textit{yawaten arasiku}\\\hline
\end{supertabular}
\end{flushleft}
From the above it can be inferred that the AP languages have at most six mono-morphemic numerals. This number is significantly fewer than the number we find in many European languages. Present-day English, for example, has twelve mono-morphemic cardinal numerals (Von Mengden 2010: 26). 

Both within and across the AP languages we find variation in the way cardinals are composed (cf. Stump 2010). First, in choice of numeral base: in all systems both quinary ({\textquoteleft}base-five{\textquoteright}) and decimal ({\textquoteleft}base-ten{\textquoteright}) bases are used. Table 1 includes examples of numerals with a quinary base (\textit{yes }in Teiwa, \textit{yawaten }in Kula). A decimal base is used in numerals {\textquoteleft}ten{\textquoteright} and above; an illustration is Teiwa \textit{qaar }in \textit{qaar nuk} {\textquoteleft}ten{\textquoteright} and \textit{qaar raq} {\textquoteleft}twenty{\textquoteright}. 

Second, the type of operations invoked for the interpretation of the morphemes that make up the compound numerals vary between addition (Abui \textit{yeting}\textit{ }\textit{buti} {\textquoteleft}nine{\textquoteright}{\textless} \textit{yeting}\textit{ }{\textquoteleft}five{\textquoteright} + \textit{buti} {\textquoteleft}four{\textquoteright}), subtraction (Adang  \textit{ti}\textit{{\textglotstop}}\textit{i }\textit{nu} {\textquoteleft}nine{\textquoteright} {\textless} \textit{ti}\textit{{\textglotstop}}\textit{i }(semantically opaque), \textit{nu} {\textquoteleft}one{\textquoteright} {\textless} {\textquoteleft}minus one{\textquoteright}) and multiplication (Western Pantar \textit{ke atiga }{\textquoteleft}thirty{\textquoteright} {\textless} \textit{ke }{\textquoteleft}ten{\textquoteright} x \textit{atiga }{\textquoteleft}three{\textquoteright}). Of these, subtraction is the least frequent.

Third, different types of operations are involved in the derivation of cardinals: typically they involve simple juxtaposition of bases (e.g., Abui \textit{kar nuku} {\textquoteleft}ten{\textquoteright} {\textless} \textit{kar} {\textquoteleft}ten{\textquoteright}, \textit{nuku} {\textquoteleft}one{\textquoteright}), but in some cases, a lexeme is added that expresses the operation (e.g., the operator \textit{wal} signifying addition in numerals 11-19, e.g., Abui \textit{kar nuku }\textbf{\textit{wal}}\textit{ nuku} {\textquoteleft}eleven{\textquoteright}. 

The number compounds in AP languages are all exocentric, that is, they lack a morphological head. In this respect they contrast with nominal compounds, which are typically endocentric (e.g., Teiwa \textit{xam yir} {\textquoteleft}milk{\textquoteright} {\textless} \textit{xam} {\textquoteleft}breast{\textquoteright}, \textit{yir} {\textquoteleft}water{\textquoteright}, where the rightmost element is the head). As both nominal and numeral compounds have stress on their final member, we can analyse both types of compounds as prosodically right-headed across the board in Alor-Pantar. 

In the Pantar languages in particular, the synchronic morphological make-up of numeral compounds can be rather obscure. For instance, Teiwa \textit{yesnerig} {\textquoteleft}eight{\textquoteright} is not a transparent compound of synchronic \textit{yusan} {\textquoteleft}five{\textquoteright} + \textit{yerig} {\textquoteleft}three{\textquoteright}. In contrast, the languages of Central and East Alor have more transparent numeral compounds, for instance Abui \textit{yetingsua} {\textquoteleft}eight{\textquoteright} {\textless} \textit{yeting} {\textquoteleft}five{\textquoteright} + \textit{sua} {\textquoteleft}three{\textquoteright}. Phonologically, however, in all the languages of the sample discussed here, we can still recognise compound forms because they consist of two stressed phonological words, the second of which has primary stress. (We return to this issue in section 2.3.2 below.) 

We have not attested an AP language with a number word for {\textquoteleft}null{\textquoteright} or {\textquoteleft}zero{\textquoteright}. The absence of entities is rather expressed predicatively, using a word meaning {\textquoteleft}(be) empty{\textquoteright}, such as Teiwa \textit{hasak} in ().\footnote{ Compare proto-Alor-Pantar *hasak (Holton and Robinson, this volume), reflected in Western Pantar \textit{hakkas}, Kaera \textit{isik}, Abui \textit{taka}, Kamang \textit{saka}.} In the Teiwa idioms in (), a subject precedes a nominal predicate that is headed by the place pronoun \textit{i} {\textquoteleft}it.(place){\textquoteright}, so that absent entities are expressed as {\textquotedblleft}X is (an) empty place(s){\textquotedblright}, compare (a-b).

Teiwa 

\label{bkm:Ref355363568}() a.\ \ \textit{Guru \ \   \ \ }[\textit{i \ \ \ \ hasak}]\textsc{\textsubscript{pred}}\textit{ }

\ \ \ \ teacher \textsc{(ind)}\ \  it.(place)  \ \ empty{\textquoteright} 

\ \ \ \ {\textquoteleft}No / zero teachers{\textquoteright} 

\ \ b. \ \ \textit{Yaf \ \ \ \ }[\textit{i \ \ \ \ hasak}]\textsc{\textsubscript{pred}}\textit{ }

\ \ \ \ house  \ \ \ \ it.(place) \ \ empty

\ \ \ \ {\textquoteleft}No / zero people{\textquoteright} 

In sum, AP languages have up to six morphologically simple cardinals; in all AP languages, the non-borrowed cardinals {\textquoteleft}seven{\textquoteright} and up are morphologically complex. Most cardinals are compounds, consisting of two or more morphemes in apposition, the second of which gets word stress. The definition of the morphological structure of these compounds varies along three dimensions: the choice of base, the arithmetic operations invoked for the interpretation of the cardinals that make up the numeral, and the ways in which these arithmetic operations are expressed. 

\subsection[2.2 \ \ Ordinal numerals]{\textup{2.2 \ \ Ordinal numerals}}
\hypertarget{RefHeading105234871885726}{}Ordinal numerals are words that identify the position that a given member of a set occupies relative to other members of the same set (e.g., {\textquoteleft}the third dog{\textquoteright}). The main function of ordinal numerals is thus to indicate the position of an entity in an ordered sequence. 

All AP languages have distinct forms for cardinal and ordinal numerals, and all of them have ordinal numbers associated with any cardinal from {\textquoteleft}two{\textquoteright} and above. Ordinals in AP languages are derived from cardinals, which is a cross-linguistically common pattern (Stolz \& Veselinova 2013). Variation exists only in the expression of {\textquoteleft}first{\textquoteright}, which in some of the languages is unrelated to the numeral {\textquoteleft}one{\textquoteright}, as discussed below. 

The derivation of ordinals involves a third person possessive pronoun or prefix at the left periphery of the cardinal numeral. The ordered entity functions grammatically as the possessor of the cardinal number. For example, Kamang \textit{dum yeok} {\textquoteleft}child 3.\textsc{poss}{}-two{\textquoteright}, lit. {\textquoteleft}child its-two{\textquoteright} {\textgreater} {\textquoteleft}second child{\textquoteright}. 

Within the ordinal possessive constructions, three areal patterns are discernible. The first pattern is that of the languages of Pantar and the Straits, where the possessive ordinal construction includes an additional element specific for ordinals. The second pattern is found in Central-East Alor, where ordinals are also expressed like possessive constructions, but without including an additional ordinal element. The third pattern is found in Kula and Sawila in East Alor, where the ordinals involve an applicative verb. We discuss the three patterns in turn.

In the languages of Pantar and the Straits, possessive constructions like those in () are the base for ordinal constructions such as those in (), where the elements \textit{maing},\textit{ ma }or \textit{mi }occur between the possessor prefix and the numeral. (Full paradigms of ordinal constructions are presented in the Appendix.) 

{\bfseries
\label{bkm:Ref342649616}\textmd{(}\textmd{)}\textmd{ \ \ Nominal possessive construction }\textmd{in Pantar-Straits}}

\ \ W Pantar\ \ \ \ \textit{aname\ \ gai\ \ bla}

\ \ Teiwa\ \ \ \ \ \ \textit{masar\ \ ga-\ \ yaf}

\ \ Kaera\ \ \ \ \ \ \textit{masik\ \ ge-\ \ ma}

\ \ Blagar\ \ \ \ \ \ \textit{mehal\ \ }\textit{{\textglotstop}}\textit{e-\ \ hava}

\ \ Adang \ \ \ \ \ \ \textit{nami\ \ o-\ \ bang}

\ \ \ \ \ \ \ \ man\ \ \textsc{3.poss}\ \ house\ \ 

\textit{\ \ \ \ \ \ \ \ }{\textquoteleft}the man{\textquoteright}s house{\textquoteright}\ \ 

\clearpage
\label{bkm:Ref342649632}() \ \ Ordinal construction in Pantar-Straits 

\ \ W Pantar\ \ \ \ \textit{aname\ \ gai\ \ maing\ \ atiga}

\ \ Teiwa\ \ \ \ \ \ \textit{masar\ \ ga-\ \ ma-\ \ yerig}

\ \ Kaera\ \ \ \ \ \ \textit{masik\ \ ge-\ \ mi-\ \ tug}

\ \ Blagar\ \ \ \ \ \ \textit{mehal\ \ }\textit{{\textglotstop}}\textit{e-\ \ mi-\ \ tue}

\ \ Adang \ \ \ \ \ \ \textit{nami\ \ o-\ \ mi-\ \ towo\ \ {}-mi}

\ \ \ \ \ \ \ \ man\ \ \textsc{3.poss\ \ ord-\ \ }three\ \ {}-\textsc{ord}

\ \ \ \ \ \ \ \ {\textquoteleft}the third man{\textquoteright}

In Western Pantar, the ordinal element is a free form \textit{maing}; in Teiwa, Kaera, Blagar, and Adang it is a bound morpheme (\textit{ma- }or \textit{mi-}). The ordinal elements are formally similar to existing words in the respective languages: Western Pantar \textit{mayang }{\textquoteleft}to place{\textquoteright},\textit{ }Teiwa\textit{ ma }{\textquoteleft}come, \textsc{obl}{\textquoteright}, and Kaera/Blagar/Adang \textit{mi }{\textquoteleft}\textsc{obl{\textquoteright} (}{\textless}pAP *mai {\textquoteleft}come{\textquoteright} and *mi {\textquoteleft}be in/on{\textquoteright} (Holton and Robinson, this volume). Synchronically, the semantic and syntactic link between these free forms and the ordinal markers is obscure. It may be that the ordinal morphemes express notions that are (historically) related to notions of placement or location at a particular numeral rank. However, their position preceding the numeral does not parallel the position of verbs and oblique markers, which in AP languages 
always \textit{follow} their nominal complement. Note however, that the ordinals in Adang involve two identical morphemes: one preceding and one following the numeral. This might reflect an earlier structure where the ordinal marker followed the numeral, paralleling the position of case markers and verbs.

The second areal pattern of ordinal constructions is found in Central-East Alor, where ordinal constructions are also possessive constructions but now without an ordinal element included. Compare the constructions in () and (). The basic possessive construction in () includes a possessor, an alienable possessive prefix and a possessum. In the ordinal constructions in (), the ranked entity is the possessor of the numeral indicating the rank.

\label{bkm:Ref342649962}() \ \ Nominal possessive construction in Central-East Alor 

\ \ Kamang\ \ \textit{lami\ \ ge-\ \ kadii}

\ \ Abui\ \ \ \ \textit{neng\ \ he-\ \ fala}

\ \ \ \ \ \ man\ \ 3\textsc{.poss-\ \ }house

\ \ {\textquoteleft}the man{\textquoteright}s house{\textquoteright}

\label{bkm:Ref342309284}()  \ \ Ordinal construction in Central-East Alor

\ \ Kamang\ \ \textit{lami\ \ ge-}\ \ \textit{su}

\ \ Abui\ \ \ \ \textit{neng\ \ he-\ \ sua}

\ \ \ \ \ \ man\ \ 3\textsc{.poss-\ \ }three

\ \ {\textquoteleft}the third man{\textquoteright}

In East Alor, ordinal structures that diverge from both these areal patterns are found in Kula and Sawila. Kula (Nick Williams, p.c. 2013) and Sawila ordinals employ applicative verbs involving the cognate prefixes \textit{we-/wii-}, illustated in () and (). In Kula ordinals this verb combines with a possessive structure, (). In Sawila, possessive constructions  are not used in ordinals, (). 

Kula

\label{bkm:Ref355275027}()  \ \ \textit{wanta \ \ gi-we-araasiku}

\ \ day \ \ 3.\textsc{poss}{}-\textsc{appl}{}-four

\ \ {\textquoteleft}the fourth day{\textquoteright}

Kula

\label{bkm:Ref358037937}() \ \ \textit{Maria \ \ gi-skola}

\ \ Maria \ \ 3.\textsc{poss-}school

\ \ {\textquoteleft}Maria{\textquoteright}s school{\textquoteright}

Sawila

\label{bkm:Ref355277550}()  \ \ \textit{imyalara \ \ wii-tua}

\ \ man\ \ \ \ \textsc{appl-}three

\ \ {\textquoteleft}the third man{\textquoteright}

Sawila\ \ 

\label{bkm:Ref342737790}()  \ \ \textit{imyalara \ \ gi}\textbf{\textit{{}-}}\textit{araasing}

\ \ man\ \ \ \ 3.\textsc{poss-}house\ \ \ \ 

\ \ {\textquoteleft}the man{\textquoteright}s house{\textquoteright}

In all AP languages, the ordinals for {\textquoteleft}second{\textquoteright} and higher are regularly derived. There is no limit in the creation of ordinals on the basis of higher, morphologically more complex, cardinals.

Some variation exists, however, in the expression of {\textquoteleft}first{\textquoteright}. Adang and Kamang form {\textquoteleft}first{\textquoteright} by the regular process used for {\textquoteleft}second{\textquoteright} and above. Teiwa and Abui use forms for {\textquoteleft}first{\textquoteright} that are unrelated to the numeral {\textquoteleft}one{\textquoteright}, compare (a-b) and (a-b). In Teiwa, the regular derivation from \textit{nuk }does not exist, (b); in Abui, it does exist, but has a different meaning ({\textquoteleft}the only/single/particular{\textquoteright}), (b). Western Pantar has two options to express {\textquoteleft}first{\textquoteright}. One is to use the regular construction derived from \textit{(a)nuku} {\textquoteleft}one{\textquoteright}, as in (a) while the other option is to use a different root \textit{ye }(b) with an unclear etymology. There is a functional difference between Western Pantar ordinal based on \textit{anuku} which is often used in predicative contexts ({\
textquoteleft}you are the first{\textquoteright}),\textit{ }and \textit{ye}, which is preferred in attributive contexts ({\textquoteleft}my first child{\textquoteright}). 

Teiwa 

\label{bkm:Ref342651040}() a. \ \ \textit{uy\ \ \ \ ga-xol}

\ \ \ \ person\ \ \ \ 3\textsc{.poss-}first

\ \ \ \ {\textquoteleft}first person{\textquoteright}

\ \  \ \ b. *\ \ \textit{uy\ \ \ \ ga- ma\ \ \ \ nuk}

\ \ \ \ \ \ person\ \ \ \ 3\textsc{.poss-ord\ \ }one

Abui

\label{bkm:Ref342651068}() a. \ \ \textit{ama \ \ \ \ he-teitu}

\ \ \ \ \ \ person\ \ \ \ 3\textsc{.poss-\ \ }first  

\ \ \ \ {\textquoteleft}first person{\textquoteright}

\ \  \ \ b. \ \ \textit{ama \ \ \ \ he-nuku}

\ \ \ \ \ \ person\ \ \ \ 3\textsc{.poss-\ \ }one

\ \ \ \ \ \ {\textquoteleft}(the) only/single person,  particular person{\textquoteright}

Western Pantar

\label{bkm:Ref342651067}() a. \ \ \textit{aname\ \ \ \ gai\ \ maing \ \ anuku}

\ \ \ \ person\ \ \ \ 3\textsc{.poss \ \ ord\ \ }one

\ \ \ \ b. \ \ \textit{aname\ \ \ \ gai\ \ maing \ \ ye}

\ \ \ \ person\ \ \ \ 3\textsc{.poss \ \ ord\ \ }one

\ \ \ \ \ \ {\textquoteleft}first person{\textquoteright}

In sum, the AP languages regularly derive ordinals from numerals with a possessor morpheme, so that syntactically the ordinal construction is a possessed nominal phrase. Apart from the third person possessor morpheme, which is used across the board, ordinals vary in structure when we go from west to east. In the western languages (Pantar-Straits-West Alor) special morphemes are employed which may be etymologically related to free forms encoding locations, though synchronically, this relation is not transparent. In the eastern languages, ordinals involve an applicative morpheme. At least three of the AP languages have an ordinal {\textquoteleft}first{\textquoteright} involving a root that is different from the cardinal {\textquoteleft}one{\textquoteright}. This is in line with the cross-linguistic tendency for languages with ordinals unrelated to cardinals to confine them to the lowest numerals (Stolz and Veselinova 2013).

\subsection[2.3 \ \ Distributive numerals]{\textup{2.3 \ \ Distributive numerals}}
\hypertarget{RefHeading105236871885726}{}\subsection[2.3.1 \ \ Forms and distribution of distributives ]{\textmd{2.3.1 \ \ Forms and distribution of distributives}\textmd{ }}
\hypertarget{RefHeading105238871885726}{}Distributive numerals function to express notions such as {\textquoteleft}one by one{\textquoteright} or {\textquoteleft}in groups of three{\textquoteright}. AP languages create distributive numerals by reduplication of the cardinal numeral, or a part of it. Cross-linguistically, reduplication is the most common strategy to form distributives: in about 33\% of the 251 languages in Gil{\textquoteright}s (2013) sample, distributives are created in this way. As Gil points out, the reduplicative strategy is iconically motivated: repeated copies of the cardinal correspond to multiple sets of objects.

Distributive numerals follow the noun or pronoun they modify, as illustrated in ()-(). Distributives can modify different clausal arguments; for example, an actor subject in () and () or a patient object in ().

Teiwa

\label{bkm:Ref342655549}() \ \ \textit{Iman\ \ nuk}\textit{\~{}}\textit{nuk\ \ aria-n}

\textsc{\ \ \ \ }\textsc{3pl} \ \ \textsc{rdp}\textit{\~{}}one \ \ arrive-\textsc{real}

\ \ \ \ {\textquoteleft}They arrived one by one{\textquoteright} 

Abui

\label{bkm:Ref342738939}()  \ \ \textit{Ama\ \ \ \ rifi-rifi\ \ \ \ }

\ \ \ \ person\ \ \ \ \textsc{rdp\~{}}thousand\ \ 

\textit{\ \ \ \ sei\ \ \ \   \ \ hel\ \ buku\ \ nu\ \ he-waalri}

\ \ \ \ come.down.\textsc{cont}  \ \ \textsc{top \ \ }land\ \ \textsc{spec} \ \ 3.\textsc{ loc-}gather.in.\textsc{compl} 

\ \ \ \ {\textquoteleft}People came in thousands to that place{\textquoteright} 

Teiwa

\label{bkm:Ref342655553}() \textit{\ \ Yi \ \ ma \ \ gula \ \ yerig\~{}yerig \ \ mat}

\ \ \ \ 2\textsc{pl\ \ }come\ \ sweet\ \ \textsc{rdp}\textit{\~{}}three\ \ take

\ \ \ \ {\textquoteleft}You take three sweets each{\textquoteright}

In some AP languages distributives may float outside the NP to a position adjacent to the verb; an example is Adang. The exact restrictions and possibilities of such constructions across the AP languages remain a topic for future research; here we focus on the morphological shape of the distributives.

Adang 

()  \ \ \textit{Sunui{\textltailn} \ \ papan \ \ du \ \ teweng }\textit{al-alu} [all\'u]

\ \ 3\textsc{pl\ \ }\ \ board\ \ \textsc{def} \ \ carry \ \ \textsc{rdp}\textit{\~{}}two

\ \ {\textquoteleft}They carry the board two by two (i.e., two at a time){\textquoteright}

The following sections describe how distributives are derived: the regular patterns are discussed in section 2.3.2, and the irregularities in section 2.3.3.  Full paradigms of distributives in five languages of our sample are given in the Appendix.

\subsection[2.3.2 \ \ Regular distributive formation ]{\textmd{2.3.2 \ \ Regular distributive formation}\textmd{ }}
\hypertarget{RefHeading105240871885726}{}Regular distributive formation in Alor-Pantar involves reduplication of (a part of) the cardinal number. In complex numerals it is usually the right-most element, the prosodic head (section 2.1), that is the base for the reduplication. The result is a distributive form that contains word-internal reduplication. 

Even in languages where the morphological make-up of compound cardinals is synchronically opaque, such as Teiwa, distributive reduplication splits the cardinal in two parts, and only the rightmost element, the prosodic head, is reduplicated; see the numerals {\textquoteleft}seven{\textquoteright} to {\textquoteleft}nine{\textquoteright} in Table 2. Also in numerals that contain an operator expressing addition, it is the right-most morpheme that is reduplicated, see ()-() below.

Table 2: Teiwa cardinals and distributives

\begin{center}
\tablehead{}
\begin{supertabular}{m{0.43585983in}m{2.12406in}m{2.39006in}}
\hline
 &
Cardinal &
Distributive\\\hline
1 &
\textit{nuk} &
\textit{nuk\~{}nuk}\\
2 &
\textit{raq} &
\textit{raq\~{}raq}\\
3 &
\textit{yerig} &
\textit{yerig\~{}yerig}\\
4 &
\textit{{\textglotstop}}\textit{ut} &
\textit{{\textglotstop}}\textit{ut\~{}}\textit{{\textglotstop}}\textit{ut}\\
5 &
\textit{yusan} &
\textit{yusan\~{}yusan}\\
6 &
\textit{tiaam} &
\textit{tiaam\~{}tiaam}\\
7 &
\textit{yes}\textbf{\textit{raq}} &
\textit{yes}\textbf{\textit{raq}}\textit{\~{}raq}\\
8 &
\textit{yesn}\textbf{\textit{erig}} &
\textit{yesne}\textbf{\textit{rig}}\textit{\~{}rig}\\
9 &
\textit{yesna}\textbf{\textit{{\textglotstop}}}\textbf{\textit{ut}} &
\textit{yesna}\textbf{\textit{{\textglotstop}}}\textbf{\textit{ut}}\textit{\~{}}\textit{{\textglotstop}}\textit{ut}\\
\end{supertabular}
\end{center}
In Adang, distributives are formed by partial reduplication, as shown in Table 3. In the mono-morphemic forms {\textquoteleft}one{\textquoteright} through {\textquoteleft}six{\textquoteright}, reduplication copies the first two segments (CV or VC) of the cardinal. Note that this analysis assumes that distributive {\textquoteleft}two{\textquoteright} \textit{allo }and {\textquoteleft}five{\textquoteright} \textit{iwwihing }are (historical) contractions of \textit{al-alu} and \textit{iw-iwing}. Numerals {\textquoteleft}seven{\textquoteright} to {\textquoteleft}nine{\textquoteright} are subtractive compound forms, in which the right-most element is the base for the reduplication (cf. \textit{to }{\textless} \textit{towo} {\textquoteleft}three{\textquoteright}, \textit{lo {\textless} alu }{\textquoteleft}two{\textquoteright}, \textit{nu {\textless} nu} {\textquoteleft}one{\textquoteright}). 

Table 3: Adang cardinals and distributives

\begin{center}
\tablehead{}
\begin{supertabular}{m{0.48725984in}m{0.9670598in}m{1.6712599in}}
\hline
 &
Cardinal  &
Distributive\\\hline
1 &
\textit{nu} &
\textit{nu\~{}nu}\\
2 &
\textit{alu }[alu] &
\textit{al}\textit{\~{}}\textit{lo }[al:o]\footnotemark{} \\
3 &
\textit{towo} &
\textit{to\~{}towo}\\
4 &
\textit{{\textglotstop}}\textit{ut} &
\textit{{\textglotstop}}\textit{u\~{}}\textit{{\textglotstop}}\textit{ut}\\
5 &
\textit{iwihing} &
\textit{iw\~{}wihing }\\
6 &
\textit{talang} &
\textit{ta\~{}talang}\\
7 &
\textit{wit}\textbf{\textit{to}} &
\textit{wit}\textbf{\textit{to}}\textit{\~{}to }\\
8 &
\textit{tur}\textbf{\textit{lo}} &
\textit{tur}\textbf{\textit{lo}}\textit{\~{}lo }\\
9 &
\textit{ti}\textit{{\textglotstop}}\textit{i}\textbf{\textit{nu}} &
\textit{ti}\textit{{\textglotstop}}\textit{i}\textbf{\textit{nu}}\textit{\~{}nu }\\
\end{supertabular}
\end{center}
\footnotetext{ Synchronically, the vowel in the distributive \textit{allo }has a distinct quality from the vowel in the cardinal.}
Across the AP family, the formation of distributives by reduplicating (parts of) cardinals is a productive process. It applies not only to frequent or morphologically simple numerals such as {\textquoteleft}one{\textquoteright} or {\textquoteleft}two{\textquoteright}, but also to less frequent and morphologically complex numerals like {\textquoteleft}27{\textquoteright} in (), {\textquoteleft}201{\textquoteright} in (), and {\textquoteleft}1054{\textquoteright} in (14). It must be noted that, while it is difficult to imagine a distributive context for numerals like these, speakers are able to mechanically derive their distributive form. 

\label{bkm:Ref342661452}()\ \ Abui

\textit{\ \ Kar \ \ ayoku \ \ wal \ \ yeting \ \ ayok\~{}ayok-da } 

\ \ ten \ \ two \ \ \textsc{add} \ \ five \ \ \textsc{rdp}\textit{\~{}}two-\textsc{ }get.\textsc{cont}

\ \ {\textquoteleft}in groups of 27{\textquoteright} 

\label{bkm:Ref342656299}() Distributive for {\textquoteleft}201{\textquoteright} 

\ \ \ \ \ \ 100\ \ \ \ 2\ \ \ \ \textsc{add\ \ \ \ rdp}\~{}1

\ \ W Pantar:\textit{ \ \ }\textit{ratu\ \ \ \ alaku\ \ \ \ wali \ \ \ \ ye\~{}ye}

\ \ Teiwa:\ \ \ \ \textit{ratu\ \ \ \ raq\ \ \ \ rug\ \ \ \ nuk}\textit{\~{}}\textit{nuk}

\ \ Abui:\ \ \ \ \textit{aisaha \ \ \ \ ayoku\ \ \ \ wal\ \ \ \ nuk\~{}nukda }

\ \ Kamang:\ \ \textit{ataak \ \ \ \ ok \ \ \ \ waal\ \ \ \ no\~{}nok}

\label{bkm:Ref342744393}() Distributive for {\textquoteleft}1054{\textquoteright}

\ \ \ \ \ \ 1000\ \ \ \ 1\ \ 10\ \ 5\ \ \ \ \textsc{add\ \ rdp}\~{}4

\ \ W Pantar:\textit{ \ \ }\textit{(a)ribu \ \ nuk \ \ ke\ \ yasing \ \ \ \ wali\ \ atu}\textit{\~{}}\textit{atu }

\ \ Teiwa:\ \ \ \ \textit{ribu \ \ \ \ nuk\ \ qaar\ \ yusan\ \ \ \ rug\ \ ut}\textit{\~{}}\textit{ut}

\ \ Abui:\ \ \ \ \textit{rifi\ \ \ \ nuku\ \ kar\ \ yeting\ \ \ \ wal\ \ buk\~{}bukna}

\ \ Kamang:\textit{ \ \ }\textit{ribu\ \ \ \ nok\ \ ataak\ \ wesing\ \ \ \ waal\ \ bye\~{}biat}

In sum, distributives are productively derived from cardinals by reduplicating part of or the whole cardinal base. In morphologically complex forms, the right-most element is the prosodic head and the reduplicative base. 

\subsection[2.3.3 \ \ Irregularities in distributive formation]{\textmd{2.3.3 \ \ Irregularities in distributive formation}}
\hypertarget{RefHeading105242871885726}{}Exceptions to the regular derivations are mainly found in the formation of the morphologically complex low numerals {\textquoteleft}six{\textquoteright} to {\textquoteleft}nine{\textquoteright}. The irregularities include: (i) irregular segmental changes in reduplicated forms; (ii) irregular patterns of partial viz. full reduplication; and (iii) irregular choice of reduplicative base. 

Abui shows the greatest amount of formal difference between its cardinal and distributive numerals, as shown in Table 4. The distributives are reduplicated verbal constructions: their verbal status is clear from the suffixes -\textit{da/-na/-ra }which encode light verbs and (continuative) aspect (Kratochv\'il 2007). In Table 4, the parts printed in bold show the irregular relation between Abui cardinals and the numeral morphemes used in distributives.

\ \ \ \ \ \ Table 4: Abui cardinals and distributives

\begin{center}
\tablehead{}
\begin{supertabular}{m{0.48725984in}m{0.9670598in}m{1.6712599in}}
\hline
 &
Cardinal  &
Distributive\\\hline
1 &
\textit{nuk}\textbf{\textit{u}} &
\textit{nuk\~{}nuk-da}\\
2 &
ayok\textbf{u} &
\textit{ayok\~{}ayok-da }\\
3 &
\textit{su}\textbf{\textit{a}} &
\textit{su}\textbf{\textit{i}}\textit{\~{}sui-da}\\
4 &
\textit{bu}\textbf{\textit{ti}} &
\textit{bu}\textbf{\textit{k}}\textit{\~{}buk-na}\\
5 &
\textit{ye}\textbf{\textit{ting}} &
\textit{ye}\textbf{\textit{k}}\textit{\~{}yek-na}\\
6 &
\textit{tala}\textbf{\textit{ama}} &
\textit{tala}\textbf{\textit{n}}\textit{\~{}talan-ra}\\
\end{supertabular}
\end{center}
\ \ In Kamang distributives, the reduplicant varies in shape. In the numerals {\textquoteleft}one{\textquoteright} to {\textquoteleft}four{\textquoteright} and {\textquoteleft}six{\textquoteright}, a morpheme with the shape (C)VV is reduplicated, while in the numeral {\textquoteleft}five{\textquoteright} and the complex numerals built on it---{\textquoteleft}seven{\textquoteright} through {\textquoteleft}nine{\textquoteright}---the reduplicant has the shape CVCV. This is shown in ():

\label{bkm:Ref342656818}() Kamang distributive numeral formation

\ \ Cardinal\ \ \ \ Reduplicant shape\ \ Distributive 

\textit{\ \ nok\ \ }{\textquoteleft}1{\textquoteright}\ \ \textit{\ \ }CV\ \ \textit{\ \ \ \ no\~{}nok}

\textit{\ \ ok\ \ }{\textquoteleft}2{\textquoteright}\textit{ \ \ \ \ }V\textit{\ \ \ \ \ \ o}\textit{{\textglotstop}}\textit{\~{}ok}\footnote{ The glottal stop in this form is phonetic. It is required to break up the sequence of like vowels in separate syllables. Speakers insist on including it in writing in order to distinguish /o/ from /o:/, orthographically \{oo\}. }

\textit{\ \ su\ \ }{\textquoteleft}3{\textquoteright}\textit{  \ \ \ \ }CV\textit{\ \ \ \ \ \ su\~{}su}

\textit{\ \ biat\ \ }{\textquoteleft}4{\textquoteright}\ \ \textit{\ \ }CVV\textit{\ \ \ \ \ \ bie\~{}biat}

\textit{\ \ wesing\ \ }{\textquoteleft}5{\textquoteright}\textit{\ \ \ \ }CVCV\textit{\ \ \ \ \ \ wesi\~{}wesing}

\textit{\ \ taama\ \ }{\textquoteleft}6{\textquoteright}\textit{\ \ \ \ }CVV\textit{\ \ \ \ \ \ taa\~{}taama}

Kamang has an irregular choice of reduplicative base. Compare the reduplicants (in bold-face) in the numerals {\textquoteleft}seven{\textquoteright} to {\textquoteleft}nine{\textquoteright} in (17). We see that Abui reduplicates only the right-most numeral, resulting in word-internal reduplication, which is consistent with the regular distributive pattern in AP languages (section 2.3.2). By contrast, Kamang reduplicates the initial element \textit{wesing}. As main stress is on the final syllable of the numerals in Kamang just as it is in Abui, we analyse this as a choice of reduplicative base in Kamang distributives which diverges from the overall pattern of AP languages. 

() \ \ Reduplication of base-5 numerals in Abui and Kamang

\ \ \ \ Abui\ \ \ \ \ \ \ \ Kamang

\ \ {\textquoteleft}7{\textquoteright}:\ \ \textit{yeting}\textbf{\textit{ayok}}\textit{\~{}ayokda} \ \ \ \ \textbf{\textit{wesi}}\textit{\~{}wesingok}

\ \ {\textquoteleft}8{\textquoteright}:\ \ \textit{yeting}\textbf{\textit{sui}}\textit{\~{}suida\ \ \ \ }\textbf{\textit{wesi}}\textit{\~{}wesingsu}

\ \ {\textquoteleft}9{\textquoteright}:\ \ \textit{yeting}\textbf{\textit{buk}}\textit{\~{}bukna \ \ \ \ }\textbf{\textit{wesi}}\textit{\~{}wesingbiat}

In sum, AP languages derive distributive numerals by partial or full reduplication of the cardinal. In complex numerals, the right-most element is the prosodic head and as a rule this item is the base for the reduplication. Exceptions to the regular derivations of distributives are mainly found in the formation of the morphologically complex low numerals {\textquoteleft}six{\textquoteright} through {\textquoteleft}nine{\textquoteright} in Central-East Alor, and include segmental changes in reduplicated forms (Abui); irregular patterns of reduplication (Abui, Kamang), and an irregular choice of reduplicative base (Kamang). 

\section[3 \ \ \ \ Structures expressing arithmetic operations ]{3 \ \ \ \ Structures expressing arithmetic operations }
\hypertarget{RefHeading105244871885726}{}To complete the catalogue of numeral expressions in AP languages, this section presents the basic arithmetic operations in which numbers are combined. We describe addition (section 3.1), subtraction (section 3.2), multiplication (section 3.3), division (section 3.4), and fractions (section 3.5). To elicit math constructions from speakers was generally easy and not forced at all. This is remarkable in light of the fact that for none of the languages is it the case that children acquire or use these arithmetic expressions in school: the language of education in Alor-Pantar is Indonesian. 

\subsection[3.1 \ \ Addition]{\textup{3.1 \ \ Addition}}
\hypertarget{RefHeading105246871885726}{}Across Alor-Pantar, addition takes the shape of imperative sentences involving more than one verb. In such constructions, the agent or actor is not expressed and the added numerals are the arguments of verbs in a serial construction. The number that represents the sum amount is a predicate that follows a clause-coordinating element. Languages may abbreviate the expression by omitting a verb or the clause-coordinator. Examples () through () illustrate {\textquoteleft}three plus three is six{\textquoteright}:

Western Pantar 

\begin{flushleft}
\tablehead{}
\begin{supertabular}{m{0.43165985in}m{0.6754598in}m{0.6865598in}m{0.67815983in}m{0.65735984in}m{0.7698598in}m{0.7552598in}m{0.8420598in}}
\label{bkm:Ref342663714}() &
\textit{Atiga} &
\textit{ma}\footnotemark{} &
\textit{atiga} &
\textit{tang} &
\textit{tiggung} &
\textit{(allang)} &
\textit{hisnakkung}\\
 &
three  &
come &
three &
on &
add &
(then) &
six\\
 &
\multicolumn{7}{m{5.53716in}}{{\textquoteleft}Bring three, add on three, (then) [get] six.{\textquoteright}}\\
\end{supertabular}
\end{flushleft}
\footnotetext{ \textit{Ma }may be omitted; in that case there must be a pause between both occurrences of \textit{atiga}.}
\clearpage
Teiwa

\begin{flushleft}
\tablehead{}
\begin{supertabular}{m{0.43165985in}m{0.6754598in}m{0.6865598in}m{0.67815983in}m{0.65735984in}m{0.7698598in}m{0.44625983in}m{1.1511599in}}
() &
\textit{Yerig} &
\textit{ma} &
\textit{yerig} &
\textit{taxa{\textquoteright} } &
\textit{si} &
\textit{a} &
\textit{tiaam}\\
 &
three &
come &
three &
add &
\textsc{sim} &
\textsc{3sg} &
six\\
 &
\multicolumn{7}{m{5.5372596in}}{{\textquoteleft}Add three with three so it{\textquoteright}s six.{\textquoteright}}\\
\end{supertabular}
\end{flushleft}
Adang 

\begin{flushleft}
\tablehead{}
\begin{supertabular}{m{0.39005986in}m{0.44205984in}m{0.34905985in}m{0.40455985in}m{0.31225985in}m{0.5393598in}}
() &
\textit{Towo} &
\textit{med} &
\textit{towo} &
\textit{ta} &
\textit{talang.}\\
 &
three  &
take  &
three  &
add  &
six\\
 &
\multicolumn{5}{m{2.3622599in}}{{\textquoteleft}Take three add three (it{\textquoteright}s) six.{\textquoteright}}\\
\end{supertabular}
\end{flushleft}
Abui 

\begin{flushleft}
\tablehead{}
\begin{supertabular}{m{0.34905985in}m{0.21155986in}m{0.63935983in}m{0.6400598in}m{0.63935983in}m{0.6400598in}m{0.63935983in}m{0.48515984in}m{0.075459845in}}
\multicolumn{2}{m{0.63935983in}}{()} &
\textit{Sua} &
\textit{mi} &
\textit{sua-ng} &
\textit{h-ai} &
\textit{maiye} &
\multicolumn{2}{m{0.63935983in}}{\textit{talaama}}\\
\multicolumn{2}{m{0.63935983in}}{} &
three &
take &
three-\textsc{see} &
\textsc{3.pat}{}-add.to &
if &
\multicolumn{2}{m{0.63935983in}}{six}\\
 &
\multicolumn{7}{m{4.36736in}}{{\textquoteleft}If you add three to three, it is six.{\textquoteright}} &
\\
\end{supertabular}
\end{flushleft}
Kamang

\begin{flushleft}
\tablehead{}
\begin{supertabular}{m{0.39005986in}m{0.40455985in}m{0.34905985in}m{0.40455985in}m{0.7566598in}m{0.63935983in}m{0.40185985in}m{0.48865983in}}
\label{bkm:Ref342663723}() &
\textit{Su} &
\textit{me} &
\textit{su} &
\textit{wo-tte} &
\textit{an-ing} &
\textit{bo} &
\textit{taama}\\
 &
three  &
take &
three &
3.\textsc{loc}{}-add &
thus-\textsc{set}  &
\textsc{conj} &
six\\
 &
\multicolumn{6}{m{3.34976in}}{{\textquoteleft}Add three to three makes six.{\textquoteright}} &
\\
\end{supertabular}
\end{flushleft}
\subsection[3.2 \ \ Subtraction]{\textup{3.2 \ \ Subtraction}}
\hypertarget{RefHeading105248871885726}{}Just like addition, subtraction is also expressed in imperative sentences. Syntactically, the subtrahend (i.e., the numeral subtracted) is expressed as the complement of transitive verbs such as {\textquoteleft}throw away X{\textquoteright}, {\textquoteleft}split off X{\textquoteright}, {\textquoteleft}move X{\textquoteright}, or {\textquoteleft}take X{\textquoteright}. The grammatical role of the minuend (i.e., the numeral subtracted from) is less clear. As is the case with the sum of addition, the result of the subtraction typically occurs as the predicate of a separate clause, following a clause coordinating element. Examples () through () illustrate {\textquoteleft}five minus two is three{\textquoteright}:

Western Pantar

\begin{flushleft}
\tablehead{}
\begin{supertabular}{m{0.43165985in}m{0.72885984in}m{0.7066598in}m{0.85315984in}m{0.7150598in}m{0.76295984in}m{0.7143598in}m{0.55315983in}}
\label{bkm:Ref358042906}() &
\textit{Yasing} &
\textit{alaku} &
\textit{sussung} &
\textit{allang} &
\textit{(gang)} &
\textit{atiga.} &
\\
 &
five &
two &
throw.away &
then &
(\textsc{3sg)} &
three &
\\
 &
\multicolumn{7}{m{5.5066595in}}{{\textquoteleft}Discard two from five then there are three.{\textquoteright} }\\
\end{supertabular}
\end{flushleft}
Teiwa

\begin{flushleft}
\tablehead{}
\begin{supertabular}{m{0.34835985in}m{0.54345983in}m{0.34905985in}m{0.50185984in}m{0.075459845in}m{0.075459845in}m{0.07615984in}m{0.075459845in}m{0.075459845in}m{0.07615984in}}
() &
\textit{Muxui} &
\textit{kam} &
\textit{yusan,} &
 &
 &
 &
 &
 &
\\
 &
banana &
\textsc{clf} &
five &
 &
 &
 &
 &
 &
\\
 &
\multicolumn{7}{m{2.16936in}}{{\textquoteleft}Five bananas, } &
 &
\\
\end{supertabular}
\end{flushleft}
\begin{flushleft}
\tablehead{}
\begin{supertabular}{m{0.30315986in}m{0.06565984in}m{0.44765988in}m{0.44835988in}m{0.44765988in}m{0.44835988in}m{0.44765988in}m{0.44835988in}m{0.44765988in}m{0.44765988in}m{0.44835988in}}
\multicolumn{2}{m{0.44755986in}}{} &
\textit{hara}\textit{q} &
\textit{ma} &
\textit{ga-}\textit{fa{\textquoteright} } &
\textit{mai } &
\textit{ha} &
\textit{si,} &
\textit{kam} &
\textit{yerig } &
\textit{qai.}\\
\multicolumn{2}{m{0.44755986in}}{} &
two &
come &
\textsc{3sg-}split.off &
save &
then &
\textsc{sim} &
\textsc{clf} &
three &
only \\
 &
\multicolumn{10}{m{4.8060603in}}{split off two [to] save then only three [are left].{\textquoteright} }\\
\end{supertabular}
\end{flushleft}
Adang

\begin{flushleft}
\tablehead{}
\begin{supertabular}{m{0.34905985in}m{0.5809598in}m{0.8573598in}m{0.54345983in}m{0.32125986in}m{0.40455985in}}
(34) &
\textit{Iwihing} &
\textit{a-no{\textquoteright} } &
\textit{kurung} &
\textit{alu} &
\textit{towo}\\
 &
five &
\textsc{caus}{}-affect &
less &
two &
three\\
 &
\multicolumn{5}{m{3.02256in}}{{\textquoteleft}Five minus two is three.{\textquoteright}}\\
\end{supertabular}
\end{flushleft}
Abui 

\begin{flushleft}
\tablehead{}
\begin{supertabular}{m{0.39005986in}m{0.49695984in}m{0.38725984in}m{0.6913598in}m{1.0281599in}m{0.46915987in}m{1.1031599in}m{0.40665984in}}
() &
\textit{Yeting} &
\textit{nu} &
\textit{ayoku-ng} &
\textit{ha-bel} &
\textit{maiye} &
\textit{he-pot} &
\textit{sua}\\
 &
five &
\textsc{spec}  &
two-\textsc{see} &
3.\textsc{pat}{}-subtract &
if &
\textsc{3.al}{}-remainder &
three\\
 &
\multicolumn{7}{m{5.05516in}}{{\textquoteleft}If two is subtracted from five, the remainder is three.{\textquoteright} }\\
\end{supertabular}
\end{flushleft}
\begin{flushleft}
\tablehead{}
\begin{supertabular}{m{0.34835985in}m{0.49765983in}m{0.38725984in}m{0.46915987in}m{0.6830598in}m{0.46915987in}m{0.40455985in}m{0.07685984in}}
() &
\textit{Yeting} &
\textit{nu} &
\textit{ayoku} &
\textit{mi-a} &
\textit{maiye} &
\textit{sua} &
\\
 &
five &
\textsc{spec} &
two &
take-\textsc{dur} &
if &
three &
\\
 &
\multicolumn{6}{m{3.30456in}}{{\textquoteleft}If two is taken from five, it is three.{\textquoteright}} &
\\
\end{supertabular}
\end{flushleft}
Kamang

\begin{flushleft}
\tablehead{}
\begin{supertabular}{m{0.37055984in}m{0.6233598in}m{0.32055986in}m{0.79275984in}m{0.63935983in}m{0.40185985in}m{0.40455985in}m{0.07685984in}}
\label{bkm:Ref358115306}() &
\textit{Wesing} &
\textit{ok} &
\textit{wo-met} &
\textit{an-ing} &
\textit{bo} &
\textit{su.} &
\\
 &
five &
two  &
3.\textsc{loc-}take &
thus\textsc{{}-set} &
\textsc{conj} &
three &
\\
 &
\multicolumn{7}{m{3.7317598in}}{{\textquoteleft}Take two from five thus there are three.{\textquoteright} }\\
\end{supertabular}
\end{flushleft}
\subsection[3.3 \ \ Multiplication]{\textup{3.3 \ \ Multiplication}}
\hypertarget{RefHeading105250871885726}{}The strategy used in multiplication is variable. All languages start with the multiplicant, but its shape differs. In Western Pantar and Teiwa it is an underived cardinal followed by a demonstrative, while in Abui it is a morphologically derived distributive (section 2.3). Examples () through ()  illustrate {\textquoteleft}five times four is twenty{\textquoteright}:

Western Pantar

\begin{flushleft}
\tablehead{}
\begin{supertabular}{m{0.37055984in}m{0.5983598in}m{0.5663598in}m{0.7393598in}m{0.56365985in}m{0.70115983in}m{0.6733598in}m{0.6150598in}m{0.7552598in}}
\label{bkm:Ref358043144}() &
\textit{Attu} &
\textit{si} &
\textit{gaunung} &
\textit{me} &
\textit{yasing } &
\textit{allang} &
\textit{(gang)} &
\textit{ke alaku.}\\
 &
four &
that &
just &
on &
five &
then &
3\textsc{sg} &
ten two \\
 &
\multicolumn{8}{m{5.76376in}}{{\textquoteleft}Five on just that four then (it{\textquoteright}s) twenty.{\textquoteright}}\\
\end{supertabular}
\end{flushleft}
Teiwa 

\begin{flushleft}
\tablehead{}
\begin{supertabular}{m{0.34905985in}m{0.34905985in}m{0.46015987in}m{0.8413598in}m{0.7462598in}m{0.30595985in}m{0.32335985in}m{0.70115983in}m{0.07685984in}}
() &
\textit{Ut} &
\textit{ga{\textquoteright}an} &
\textit{tag-an} &
\textit{ma-yusan} &
\textit{si,} &
\textit{a } &
\textit{qaar raq.} &
\\
 &
four &
that  &
count-\textsc{real} &
come-five &
\textsc{sim} &
\textsc{3sg} &
ten two &
\\
 &
\multicolumn{8}{m{4.35536in}}{{\textquoteleft}Count these four five times and it{\textquoteright}s twenty.{\textquoteright}}\\
\end{supertabular}
\end{flushleft}
Abui 

\begin{flushleft}
\tablehead{}
\begin{supertabular}{m{0.38655984in}m{1.3073599in}m{1.3184599in}m{0.38725984in}m{0.44135985in}m{0.7337598in}m{0.46915987in}m{0.8316598in}}
() &
\textit{Buk\~{}bukna} &
\textit{ha-lakda} &
\textit{nu} &
\textit{ming} &
\textit{yekna} &
\textit{maiye} &
\textit{kar ayoku.}\\
 &
\textsc{rdp}\~{}group.of.four &
3.\textsc{pat}{}-count.\textsc{cont} &
\textsc{spec} &
about &
five.times &
if &
ten two \\
 &
\multicolumn{7}{m{5.96146in}}{{\textquoteleft}If a group of four is counted five times, it is twenty.{\textquoteright}}\\
\end{supertabular}
\end{flushleft}
Kamang expresses multiplication with an applicative verb derived from a cardinal base by prefixing \textit{mi-}. (Compare Teiwa, where the applicative derivation is used for fractions, see section 3.5). 

\clearpage
Kamang

\begin{flushleft}
\tablehead{}
\begin{supertabular}{m{0.39005986in}m{0.7580598in}m{0.7566598in}m{0.63935983in}m{0.40185985in}m{0.44205984in}m{0.32055986in}m{0.07685984in}}
\label{bkm:Ref358043179}() &
\textit{Biat=a} &
\textit{mi-wesing} &
\textit{an-ing} &
\textit{bo} &
\textit{ataak } &
\textit{ok.} &
\\
 &
four=\textsc{spec} &
\textsc{appl-}five &
thus\textsc{{}-set} &
\textsc{conj} &
ten &
two &
\\
 &
\multicolumn{7}{m{3.8678603in}}{{\textquoteleft}Five times these four makes twenty.{\textquoteright}}\\
\end{supertabular}
\end{flushleft}
\subsection[3.4 \ \ Division ]{\textup{3.4 \ \ Division}\textup{ }}
\hypertarget{RefHeading105252871885726}{}Expressions for division involve the transitive verbs {\textquoteleft}split{\textquoteright} and {\textquoteleft}divide{\textquoteright} in Western Pantar, Teiwa, and Adang. The following examples illustrate {\textquoteleft}ten divided by two is five{\textquoteright}: 

Western  Pantar

\begin{flushleft}
\tablehead{}
\begin{supertabular}{m{0.43165985in}m{0.45115986in}m{0.9622598in}m{0.9622598in}m{0.9622598in}m{0.9622598in}m{0.9622598in}}
\multicolumn{2}{m{0.96155983in}}{\label{bkm:Ref342664493}()} &
\textit{Ke anuku} &
\textit{daai} &
\textit{alaku} &
\textit{allang} &
\textit{yasing}\\
\multicolumn{2}{m{0.96155983in}}{} &
ten one &
split &
two &
then &
five\\
 &
\multicolumn{6}{m{5.6561594in}}{{\textquoteleft}Ten split (by) two then (it{\textquoteright}s) five.{\textquoteright}}\\
\end{supertabular}
\end{flushleft}
Teiwa

\begin{flushleft}
\tablehead{}
\begin{supertabular}{m{0.34835985in}m{0.42335984in}m{0.31155986in}m{0.48795983in}m{0.5893598in}m{0.46985987in}m{0.30525985in}m{0.50185984in}m{0.075459845in}m{0.077559836in}}
() &
\textit{Qaar} &
\textit{nuk} &
\textit{paxai} &
\textit{g-et} &
\textit{haraq} &
\textit{si} &
\textit{yusan.} &
 &
\\
 &
ten &
one &
divide &
3sg-eye &
two &
\textsc{sim} &
five &
 &
\\
 &
\multicolumn{9}{m{3.8721602in}}{{\textquoteleft}Ten divided in two parts (lit. eyes), then [it{\textquoteright}s] five.{\textquoteright}}\\
\end{supertabular}
\end{flushleft}
Adang

\begin{flushleft}
\tablehead{}
\begin{supertabular}{m{0.39005986in}m{0.5573598in}m{0.6816598in}m{0.6927598in}m{0.57335985in}}
\label{bkm:Ref342664505}() &
\textit{{\textquoteright}Air nu} &
\textit{{\textquoteright}aba{\textquoteright}ang} &
\textit{o-alu} &
\textit{iwihing}\\
 &
ten one &
divide &
\textsc{poss-}two &
five\\
 &
\multicolumn{4}{m{2.74136in}}{{\textquoteleft}Ten divided (by) two (is) five.{\textquoteright}\ \ }\\
\end{supertabular}
\end{flushleft}
Note that the order of the verb relative to its complement {\textquoteleft}two{\textquoteright} in ()-() is unexpected, as it goes against the canonical AP object-verb order, found in subtraction (section 3.2). Note that the equivalent expression in Indonesian/Malay is \textit{sepuluh bagi dua (adalah)} \textit{lima} lit. {\textquoteleft}ten divide two (is) five{\textquoteright}, with verb complement order. It may be the case that the constructions in ()-() are calques from Indonesian/Malay. 

Abui divisions are expressed as imperative sentences with regular serial verb constructions, where the result follows a coordinating element, see (). Kamang expresses a fraction by marking the dividing numeral with \textit{wo-}, the same prefix that is used to express, for instance, fractions resulting from an action, e.g., \textit{bo{\textquoteright}ne wo-ok} {\textquoteleft}hit into two pieces{\textquoteright}.

Abui

\begin{flushleft}
\tablehead{}
\begin{supertabular}{m{0.34835985in}m{0.33095986in}m{0.39555985in}m{0.38655984in}m{0.34905985in}m{0.46915987in}m{0.9323599in}m{0.46915987in}m{1.2559599in}}
\label{bkm:Ref358116296}() &
\textit{Kar} &
\textit{nuku} &
\textit{nu} &
\textit{mi} &
\textit{ayoku} &
\textit{he-yeng} &
\textit{maiye} &
\textit{yek-yekna}\\
 &
ten &
one &
\textsc{spec} &
take &
two &
3.\textsc{loc-}divide &
if &
\textsc{rdp-}group.of.five\\
 &
\multicolumn{8}{m{5.13996in}}{\textsc{{\textquoteleft}I}f a ten is divided into two (you get) a group of five.{\textquoteright}}\\
\end{supertabular}
\end{flushleft}
Kamang

\begin{flushleft}
\tablehead{}
\begin{supertabular}{m{0.34835985in}m{0.46015987in}m{0.7219598in}m{0.7650598in}m{0.63935983in}m{0.40185985in}m{0.5344598in}m{0.07685984in}}
() &
\textit{Ataak} &
\textit{nok=a} &
\textit{wo-ok} &
\textit{an-ing} &
\textit{bo} &
\textit{wesing} &
\\
 &
ten &
one=\textsc{spec} &
3.\textsc{loc-}two &
thus\textsc{{}-set} &
\textsc{conj} &
five &
\\
 &
\multicolumn{7}{m{4.07216in}}{{\textquoteleft}Ten divided into two makes five.{\textquoteright} }\\
\end{supertabular}
\end{flushleft}
\subsection[3.5 \ \ Fractions]{\textup{3.5 \ \ Fractions}}
\hypertarget{RefHeading105254871885726}{}Expressions for fractions show much variety across the AP languages. Western Pantar and Teiwa express fractions using a verb, while Kamang uses fraction adverbs, and no fractions appear to exist in Abui. 

Western Pantar derives fractions productively with the verb {\textquoteleft}divide{\textquoteright}, (). In Teiwa, expressions for fractions contain an applicative verb derived from a cardinal base by prefixing \textit{g-un-, }a fossilized combination of a 3\textsc{sg} object prefix and an applicative prefix \textit{un-}. The fraction verb occurs as second verb in a serial verb construction, (): 

Western Pantar 

\begin{flushleft}
\tablehead{}
\begin{supertabular}{m{0.34835985in}m{0.053859837in}m{0.48165986in}m{0.48165986in}m{0.48165986in}m{0.48165986in}m{-0.0037401542in}m{0.34905985in}m{-0.021140158in}m{0.017759845in}m{0.075459845in}m{0.075459845in}m{0.07685984in}}
\multicolumn{2}{m{0.48095986in}}{\label{bkm:Ref342746525}()} &
\textit{Ye} &
\textit{daai} &
\textit{atiga,} &
\textit{ye } &
\multicolumn{3}{m{0.48165986in}}{\textit{daai}} &
\multicolumn{4}{m{0.48175988in}}{\textit{attu}}\\
\multicolumn{2}{m{0.48095986in}}{} &
one  &
divide &
three  &
one &
\multicolumn{3}{m{0.48165986in}}{divide} &
\multicolumn{4}{m{0.48175988in}}{four}\\
 &
\multicolumn{6}{m{2.37046in}}{{\textquoteleft}one third, one fourth{\textquoteright} } &
 &
\multicolumn{2}{m{0.07535984in}}{} &
 &
 &
\\
\end{supertabular}
\end{flushleft}
Teiwa

\begin{flushleft}
\tablehead{}
\begin{supertabular}{m{0.38375986in}m{0.09135985in}m{0.55385983in}m{0.5545598in}m{0.55385983in}m{0.5545598in}m{0.55385983in}m{0.5545598in}m{0.55385983in}m{0.55385983in}m{0.5545598in}m{0.55385983in}}
\multicolumn{2}{m{0.55385983in}}{\label{bkm:Ref342746583}()} &
\textit{Taxaran} &
\textit{g-} &
\textit{un-} &
\textit{yerig,} &
\textit{g-} &
\textit{un-} &
\textit{ut,} &
\textit{g-} &
\textit{un-} &
\textit{qaar nuk}\\
\multicolumn{2}{m{0.55385983in}}{} &
divide &
\textsc{3sg-} &
\textsc{appl} &
three &
\textsc{3sg-} &
\textsc{appl} &
four &
\textsc{3sg-} &
\textsc{appl-} &
ten one\\
 &
\multicolumn{11}{m{6.42016in}}{{\textquoteleft}a third, a fourth, a tenth{\textquoteright} (lit. {\textquoteleft}Divide into three, four, ten{\textquoteright})}\\
\end{supertabular}
\end{flushleft}
Adang

\begin{flushleft}
\tablehead{}
\begin{supertabular}{m{0.34905985in}m{0.31155986in}m{0.6823598in}m{0.71985984in}}
(45) &
\textit{Nu} &
\textit{{\textquoteright}aba{\textquoteright}ang} &
\textit{o-ut}\\
 &
one &
divide &
\textsc{poss}{}-four\\
 &
\multicolumn{3}{m{1.8712599in}}{{\textquoteleft}one fourth{\textquoteright}}\\
\end{supertabular}
\end{flushleft}
In Kamang, fractions are verbs derived by prefixing \textit{wo-} {\textquoteleft}3.LOC{\textquoteright} to the numeral base, as in (). In (), the derived verb is part of a resultative serial verb construction.

Kamang

\begin{flushleft}
\tablehead{}
\begin{supertabular}{m{0.47335985in}m{0.7656598in}m{0.07615984in}m{0.8483598in}m{0.83515984in}m{0.71915984in}m{0.075459845in}m{0.075459845in}m{0.07685984in}}
\label{bkm:Ref342746224}() &
\textit{wo-ok,} &
 &
\textit{wo-su,} &
\textit{wo-biat,} &
\textit{wo-ataak} &
 &
 &
\\
 &
3.\textsc{loc-}two &
 &
3.\textsc{loc-}three &
3.\textsc{loc-} four &
3.\textsc{loc-}ten\textsc{ } &
 &
 &
\\
 &
\multicolumn{8}{m{4.02346in}}{{\textquoteleft}half, a third, a fourth/quarter, a tenth{\textquoteright} }\\
\end{supertabular}
\end{flushleft}
Kamang

\begin{flushleft}
\tablehead{}
\begin{supertabular}{m{0.47335985in}m{0.39555985in}m{0.5254598in}m{0.32055986in}m{0.42335984in}m{0.91845983in}m{0.34905985in}m{0.08445985in}m{0.5893598in}}
\label{bkm:Ref342746258}() &
\textit{Nala} &
\textit{le} &
\textit{nok} &
\textit{katee} &
\textit{wo-biat} &
 &
 &
\\
 &
\textsc{1sg} &
mango &
one &
eat &
3.\textsc{loc-} four &
 &
 &
\\
 &
\multicolumn{8}{m{4.1574597in}}{{\textquoteleft}I eat a fourth of the mango{\textquoteright}, {\textquoteleft}I eat the mango in fourths{\textquoteright}}\\
\end{supertabular}
\end{flushleft}
Abui does not seem to have a construction dedicated to derive fractions. It does have a word for {\textquoteleft}half{\textquoteright} that is unrelated to {\textquoteleft}two{\textquoteright}:\ \ 

Abui

\begin{flushleft}
\tablehead{}
\begin{supertabular}{m{0.34905985in}m{0.84345984in}m{0.49695984in}m{0.39555985in}m{0.5254598in}m{0.075459845in}m{0.07475985in}m{0.075459845in}m{0.077559836in}}
() &
\textit{Nalama} &
\textit{pingai} &
\textit{nuku} &
\textit{ahama} &
 &
 &
 &
\\
 &
cooked.rice &
plate &
one &
half &
 &
 &
 &
\\
 &
\multicolumn{8}{m{3.1158602in}}{{\textquoteleft}One and a half platesof rice{\textquoteright}  }\\
\end{supertabular}
\end{flushleft}
Words for {\textquoteleft}half{\textquoteright} that are unrelated to {\textquoteleft}two{\textquoteright} are also found in Western Pantar,  Teiwa, and Adang.  In Western Pantar, {\textquoteleft}half{\textquoteright} can be a nominal \textit{gamme }{\textquoteleft}half, portion{\textquoteright}, but also a fraction involving the verb {\textquoteleft}divide{\textquoteright}, compare ()-(). In Teiwa, {\textquoteleft}half{\textquoteright} may be a nominal (\textit{qaas} {\textquoteleft}side, half{\textquoteright}, \textit{abaq} {\textquoteleft}half{\textquoteright} in ()-()), but may also be expressed by an applicative verb derived from {\textquoteleft}two{\textquoteright}, as in ().

Western Pantar

\begin{flushleft}
\tablehead{}
\begin{supertabular}{m{0.47335985in}m{0.44205984in}m{0.54275984in}m{0.55315983in}m{0.26565984in}m{0.07615984in}}
\label{bkm:Ref342746707}() &
\textit{Gang} &
\textit{maggi} &
\textit{gamme} &
\textit{na} &
\\
 &
3sg &
banana &
half &
eat &
\\
 &
\multicolumn{5}{m{2.1947598in}}{{\textquoteleft}He ate half a banana.{\textquoteright}}\\
\end{supertabular}
\end{flushleft}
Western Pantar

\begin{flushleft}
\tablehead{}
\begin{supertabular}{m{0.48795983in}m{0.39625984in}m{0.48725984in}m{0.44205984in}m{0.08585984in}}
\label{bkm:Ref342746708}() &
\textit{ye} &
\textit{daai} &
\textit{alaku} &
\\
 &
one &
divide &
two &
\\
 &
\multicolumn{4}{m{1.6476599in}}{{\textquoteleft}half{\textquoteright} }\\
\end{supertabular}
\end{flushleft}
Teiwa

\begin{flushleft}
\tablehead{}
\begin{supertabular}{m{0.47335985in}m{0.32265985in}m{0.5254598in}m{0.46015987in}m{0.42335984in}m{0.38655984in}m{0.66705984in}}
\label{bkm:Ref342746924}() &
\textit{Ha} &
\textit{wou} &
\textit{ga{\textquoteright}an} &
\textit{tu{\textquoteright}un} &
\textit{qaas} &
\textit{na-mian}\\
 &
\textsc{2sg} &
mango &
that &
peel &
side &
\textsc{1sg}{}-give\\
 &
\multicolumn{6}{m{3.1789598in}}{{\textquoteleft}Peel that mango [and] give me half.{\textquoteright}}\\
\end{supertabular}
\end{flushleft}
Teiwa

\begin{flushleft}
\tablehead{}
\begin{supertabular}{m{0.43305984in}m{0.44065985in}m{0.43305984in}m{0.42815986in}m{0.085159846in}m{0.08445985in}m{0.085159846in}m{0.08445985in}m{0.085159846in}m{0.08445985in}m{0.085159846in}m{0.08445985in}m{0.08725984in}}
\label{bkm:Ref342746926}() &
\textit{Yir} &
\textit{sluan} &
\textit{abaq} &
 &
 &
 &
 &
 &
 &
 &
 &
\\
 &
water &
glass &
half &
 &
 &
 &
 &
 &
 &
 &
 &
\\
 &
\multicolumn{12}{m{2.9337592in}}{{\textquoteleft}half a glass of water{\textquoteright}}\\
\end{supertabular}
\end{flushleft}
Teiwa

\begin{flushleft}
\tablehead{}
\begin{supertabular}{m{0.43235984in}m{0.63655984in}m{0.42335984in}m{0.82405984in}m{0.32125986in}m{0.085159846in}m{0.085159846in}m{0.085159846in}m{0.085159846in}m{0.08445985in}m{0.085159846in}m{0.085159846in}m{0.08585984in}}
\label{bkm:Ref342746963}() &
\textit{Taxaran} &
\textit{g-} &
\textit{un-raq} &
 &
 &
 &
 &
 &
 &
 &
 &
\\
 &
divide &
\textsc{3sg-} &
\textsc{appl-} two &
 &
 &
 &
 &
 &
 &
 &
 &
\\
 &
\multicolumn{12}{m{3.7526593in}}{{\textquoteleft}half{\textquoteright} }\\
\end{supertabular}
\end{flushleft}
Adang

\begin{flushleft}
\tablehead{}
\begin{supertabular}{m{0.34905985in}m{0.32265985in}m{0.5254598in}m{0.6170598in}m{0.36775985in}m{0.34005985in}m{0.37615985in}m{0.7025598in}}
(53) &
\textit{na} &
\textit{be} &
\textit{bo{\textquoteright}oden} &
\textit{solo} &
\textit{{\textquoteleft}adi} &
\textit{no{\textquoteright}o} &
\textit{me-nani?}\\
 &
1\textsc{sg} &
mango &
half &
only &
eat &
can &
or-not\\
 &
\multicolumn{7}{m{3.72416in}}{{\textquoteleft}Can I only eat half a mango?{\textquoteright}}\\
\end{supertabular}
\end{flushleft}
\section[4 \ \ \ \ Summary and conclusions]{4 \ \ \ \ Summary and conclusions}
\hypertarget{RefHeading105256871885726}{}The majority of cardinal numerals in AP languages are morphologically complex expressions---most are compounds. These forms have quinary or decimal bases, though mathematical operations always employ a decimal base. No AP language has a numeral {\textquoteleft}null{\textquoteright} or a word for {\textquoteleft}zero{\textquoteright}---the absence of entities is expressed predicatively instead.

Ordinals are derived from cardinals by means of a third person possessor morpheme. Syntactically, ordinals are possessive phrases where the ranked numeral is possessed by the ranked item. In the languages of Pantar, the Straits and West Alor, ordinal constructions also contain a dedicated ordinal morpheme; an applicative morpheme is used in the ordinals of languages of Central and East Alor. 

Most languages derive distributives from cardinals by reduplicating part or whole of the cardinal. In complex forms, the right-most lexeme, which is the prosodic head of the compound, is taken as the base for the reduplication. This applies even to those forms that are synchronically morphologically opaque. Kamang is exceptional in that it reduplicates the left-most element of the compound rather than the prosodic head, and in Abui, distributives and cardinals are only indirectly related. 

Across the languages, there is more homogeneity in the expressions of addition and subtraction than there is in the expression of multiplication and division. Addition and subtraction typically take the shape of imperative sentences. In additive expressions, the added numerals each have their own predicate. The second numeral is often the grammatical object of a transitive verb ({\textquoteleft}add X{\textquoteright}) that has an implied subject, the imperative addressee. In subtraction, the subtrahend is also the object of a transitive verb ({\textquoteleft}throw away X{\textquoteright}) but the grammatical role of the {\textquoteleft}minuend{\textquoteright} is less clear. In both addition and subtraction, the result follows a clause coordinating element.

The strategies used in multiplication, division, and fractions vary significantly across the languages. While all the languages express multiplication by a multiplicant followed by a verb, the morpho-syntactic shape of the multiplicant and the choice of verb differ. In expressions for division, the number of verbs involved range from zero to two, and word orders in the western languages go against the head-final order that is typical for AP and follow the order of Indonesian/Malay, suggesting they may be calques. Across the AP languages, the expression of fractions shows the largest variety. The lack of homogeneity in the expressions for multiplication, division and fractions suggests that these expressions are more labile than those for addition and subtraction, which is probably due to their lower frequency in everyday language.

The indigenous numeral forms of the AP languages, as well as the indigenous structures for arithmetic operations are currently under pressure from Indonesian as the language of interethnic trade and national education. This will inevitably lead to their replacement with Indonesian forms and constructions. This chapter keeps a snapshot of them for future generations. 

\subsection[Sources]{\textup{Sources}}
\hypertarget{RefHeading105258871885726}{}The data sets on which this paper is based were collected from 2010-2012 by the authors. We used a questionnaire on numerals designed in 2010 by Marian Klamer and Antoinette Schapper for the purpose of documenting the numerals and numeral systems in AP languages (see Appendix B). The core dataset discussed in this chapter thus comes from questionnaires filled in for Teiwa (by Klamer and Robinson), Western Pantar (by Holton), Adang (by Robinson), Abui (by Kratochv\'il and Schapper), and Kamang (by Schapper). Comparative information on additional languages was provided through personal communication with Hein Steinhauer (Blagar), Nick Williams (Kula), Franti\emph{\textup{\v{s}}}ek Kratochv\'il (Sawila) and Marian Klamer (Kaera).

\section[References]{References}
\hypertarget{RefHeading105260871885726}{}Comrie, Bernard. 2005. {\textquoteleft}Endangered numeral systems{\textquoteright}. In Jan Wohlgemuth and Tyko Dirksmeyer (eds.) \textit{Bedrohte Vielfalt: Aspekte des Sprach(en)tods [Endangered Diversity: Aspects of Language Death]}, 203--230. Berlin: Wei{\ss}ensee Verlag, 203-230.

Gil, David. 2013. Distributive Numerals. In: Matthew Dryer and Martin Haspelmath (eds.) \textit{The World Atlas of Language Structures Online}. Munich: Max Planck Digital Library, chapter 54. Available online at \href{http://wals.info/chapter/54. Accessed 06 May 2013}{http://wals.info/chapter/54. Accessed 06 May 2013}.

Holton, Gary and Laura Robinson, this volume. The internal history of the Alor-Pantar language family. In Marian Klamer, ed., \textit{Alor-Pantar languages: History and typology}. Language Science Press.

Kratochv\'il, Franti\emph{\textup{\v{s}}}ek. 2007. \textit{A grammar of Abui, a Papuan language of Alor.} PhD dissertation, Leiden University. Utrecht: LOT Publications.

Schapper and Klamer, this volume. Numeral systems. In Marian Klamer, ed., \textit{Alor-Pantar languages: History and typology}. Language Science Press.

Stolz, Thomas, and Ljuba Veselinova. 2013. Ordinal Numerals. In Matthew  Dryer and Martin Haspelmath (eds.) \textit{The World Atlas of Language Structures Online}. Munich: Max Planck Digital Library, chapter 53. Online at \href{http://wals.info/feature/54}{http://wals.info/chapter/5}3. Accessed 06 May 2013.

Stump,  Gregory. 2010.\textit{ }The derivation of compound ordinal numerals: Implications for morphological theory. \textit{Word Structure} 3.2 (2010): 205--233.

Von Mengden, Ferdinand. 2010. Cardinal Numerals: Old English from a Cross-Linguistic Perspective. Berlin New York: De Gruyter Mouton.

\clearpage\section[Appendix A: Ordinal and Distributive Numerals ]{Appendix A: Ordinal and Distributive Numerals }
\hypertarget{RefHeading105262871885726}{} Table A. Western Pantar ordinals in a construction with \textit{bla }{\textquoteleft}house{\textquoteright} and \textit{aname} {\textquoteleft}person{\textquoteright}

\begin{flushleft}
\tablehead{}
\begin{supertabular}{m{0.48795983in}m{1.0531598in}m{0.42125985in}m{0.9205598in}m{2.4094598in}}
1\textsuperscript{st} &
\textit{bla/aname} &
\textit{gai} &
\textit{maing} &
\textit{ye}\\
 &
\textit{bla/aname} &
\textit{gai} &
\textit{maing} &
\textit{anuku}\\
2\textsuperscript{nd} &
\textit{bla/aname} &
\textit{gai} &
\textit{maing} &
\textit{alaku}\\
3\textsuperscript{rd} &
\textit{bla/aname} &
\textit{gai} &
\textit{maing} &
\textit{atiga}\\
4\textsuperscript{th} &
\textit{bla/aname} &
\textit{gai} &
\textit{maing} &
\textit{at\'u}\\
5\textsuperscript{th} &
\textit{bla/aname} &
\textit{gai} &
\textit{maing} &
\textit{yasing}\\
6\textsuperscript{th} &
\textit{bla/aname} &
\textit{gai} &
\textit{maing} &
\textit{hisnakkung}\\
7\textsuperscript{th} &
\textit{bla/aname} &
\textit{gai} &
\textit{maing} &
\textit{betalaku}\\
8\textsuperscript{th} &
\textit{bla/aname} &
\textit{gai} &
\textit{maing} &
\textit{betiga}\\
9\textsuperscript{th} &
\textit{bla/aname} &
\textit{gai} &
\textit{maing} &
\textit{anuku tannang}\\
10\textsuperscript{th} &
\textit{bla/aname} &
\textit{gai} &
\textit{maing} &
\textit{ke anuku}\\
100\textsuperscript{th} &
\textit{bla/aname} &
\textit{gai} &
\textit{maing} &
\textit{ratu}\\
\end{supertabular}
\end{flushleft}
Table B. Teiwa ordinals with \textit{yaf }{\textquoteleft}house{\textquoteright} and \textit{uy }{\textquoteleft}person{\textquoteright} 

\begin{flushleft}
\tablehead{}
\begin{supertabular}{m{0.48235986in}m{0.5809598in}m{0.52195984in}m{0.7372598in}m{2.9698598in}}
1\textsuperscript{st} &
\textit{yaf/uy} &
\textit{ga-} &
 &
\textit{xol}\footnotemark{}\\
2\textsuperscript{nd} &
\textit{yaf/uy} &
\textit{ga-} &
\textit{ma-} &
\textit{ga-mar }[gama{\textquoteright}gamar]\textit{ }{\textquoteleft}3s-ORD-3s-take{\textquoteright}\textit{ }\\
 &
\textit{yaf/uy} &
\textit{ga-} &
\textit{ma-} &
\textit{raq }\\
3\textsuperscript{rd} &
\textit{yaf/uy} &
\textit{ga-} &
\textit{ma-} &
\textit{yerig}\\
4\textsuperscript{th} &
\textit{yaf/uy} &
\textit{ga-} &
\textit{ma-} &
\textit{ut}\\
5\textsuperscript{th} &
\textit{yaf/uy} &
\textit{ga-} &
\textit{ma-} &
\textit{yusan }\\
6\textsuperscript{th} &
\textit{yaf/uy} &
\textit{ga-} &
\textit{ma-} &
\textit{tiaam }\\
7\textsuperscript{th} &
\textit{yaf/uy} &
\textit{ga-} &
\textit{ma-} &
\textit{yes raq }\\
8\textsuperscript{th} &
\textit{yaf/uy} &
\textit{ga-} &
\textit{ma-} &
\textit{yes nerig }\\
9\textsuperscript{th} &
\textit{yaf/uy} &
\textit{ga-} &
\textit{ma-} &
\textit{yes na}\textit{{\textglotstop}}\textit{ut }\\
10\textsuperscript{th} &
\textit{yaf/uy} &
\textit{ga-} &
\textit{ma-} &
\textit{qaar nuk }\\
100th &
\textit{yaf/uy} &
\textit{ga-} &
\textit{ma-} &
\textit{ratu nuk }\\
\end{supertabular}
\end{flushleft}
\footnotetext{ Teiwa \textit{ga-nuk }means {\textquoteleft}one from a group{\textquoteright}, \textit{ga-ma-nuk }is not a Teiwa word.}
Table C. Kaera ordinals with \textit{ma }{\textquoteleft}house{\textquoteright} and \textit{ui }{\textquoteleft}person{\textquoteright}

\begin{flushleft}
\tablehead{}
\begin{supertabular}{m{0.6205598in}m{0.5462598in}m{0.42125985in}m{0.6705598in}m{3.0337598in}}
1\textsuperscript{st} &
\textit{ma/ui} &
\textit{(ge-)} &
 &
\textit{tuning  }(\textit{tuni }{\textquoteleft}gate{\textquoteright}, \textit{tuning }{\textquoteleft}placenta{\textquoteright})\\
2\textsuperscript{nd} &
\textit{ma/ui} &
\textit{ge-} &
\textit{mi} &
\textit{(a)raxo}\\
3\textsuperscript{rd} &
\textit{ma/ui} &
\textit{ge-} &
\textit{mi} &
\textit{(u)tug}\\
4\textsuperscript{th} &
\textit{ma/ui} &
\textit{ge-} &
\textit{mi} &
\textit{ut}\\
5\textsuperscript{th} &
\textit{ma/ui} &
\textit{ge-} &
\textit{mi} &
\textit{isim}\\
6\textsuperscript{th} &
\textit{ma/ui} &
\textit{ge-} &
\textit{mi} &
\textit{tiam}\\
7\textsuperscript{th} &
\textit{ma/ui} &
\textit{ge-} &
\textit{mi} &
\textit{yesraxo }\\
8\textsuperscript{th} &
\textit{ma/ui} &
\textit{ge-  } &
\textit{mi} &
\textit{yentug}\\
9\textsuperscript{th} &
\textit{ma/ui} &
\textit{ge-} &
\textit{mi} &
\textit{yeniut}\\
10\textsuperscript{th} &
\textit{ma/ui} &
\textit{ge-} &
\textit{mi} &
\textit{xar nuko}\\
100\textsuperscript{th} &
\textit{ma/ui} &
\textit{ge-} &
\textit{mi} &
\textit{ratu nuko}\\
\end{supertabular}
\end{flushleft}
Table D. Adang ordinals with \textit{bang }{\textquoteleft}house{\textquoteright} and \textit{nami }{\textquoteleft}person{\textquoteright}

\begin{flushleft}
\tablehead{}
\begin{supertabular}{m{0.49695984in}m{0.7844598in}m{0.30045986in}m{0.42885986in}m{1.0344598in}m{1.5559598in}}
1\textsuperscript{st} &
\textit{bang/nami} &
\textit{o-} &
\textit{mi-} &
\textit{nu} &
\textit{mi}\\
2\textsuperscript{nd} &
\textit{bang/nami} &
\textit{o-} &
\textit{mi-} &
\textit{alu} &
\textit{mi}\\
3\textsuperscript{rd} &
\textit{bang/nami} &
\textit{o-} &
\textit{mi-} &
\textit{towo} &
\textit{mi}\\
4\textsuperscript{th} &
\textit{bang/nami} &
\textit{o-} &
\textit{mi-} &
\textit{ut} &
\textit{mi}\\
5\textsuperscript{th} &
\textit{bang/nami} &
\textit{o-} &
\textit{mi-} &
\textit{(i)wihing} &
\textit{mi}\\
6\textsuperscript{th} &
\textit{bang/nami} &
\textit{o-} &
\textit{mi-} &
\textit{talang} &
\textit{mi}\\
7\textsuperscript{th} &
\textit{bang/nami} &
\textit{o-} &
\textit{mi-} &
\textit{witto} &
\textit{mi}\\
8\textsuperscript{th} &
\textit{bang/nami} &
\textit{o-} &
\textit{mi-} &
\textit{turlo} &
\textit{mi}\\
9\textsuperscript{th} &
\textit{bang/nami} &
\textit{o-} &
\textit{mi-} &
\textit{ti}\textit{{\textglotstop}}\textit{inu} &
\textit{mi}\\
10\textsuperscript{th} &
\textit{bang/nami} &
\textit{o-} &
\textit{mi-} &
\textit{{\textglotstop}}\textit{{\quotesinglbase}}\textit{air nu} &
\textit{mi}\\
100\textsuperscript{th} &
\textit{bang/nami} &
\textit{o-} &
\textit{mi-} &
\textit{rat nu} &
\textit{mi}\\
\end{supertabular}
\end{flushleft}
Table E. Abui ordinals with \textit{fala }{\textquoteleft}house{\textquoteright} and \textit{ama }{\textquoteleft}person{\textquoteright}

\begin{flushleft}
\tablehead{}
\begin{supertabular}{m{0.5191598in}m{0.7761598in}m{0.41295984in}m{3.07196in}}
1\textsuperscript{st} &
\textit{fala/ama} &
\textit{he-} &
\textit{teitu}\\
 &
\textit{fala/ama} &
\textit{he-} &
\textit{nuku }\\
2\textsuperscript{nd} &
\textit{fala/ama} &
\textit{he-} &
\textit{ayoku}\\
3\textsuperscript{rd} &
\textit{fala/ama} &
\textit{he-} &
\textit{sua}\\
4\textsuperscript{th} &
\textit{fala/ama} &
\textit{he-} &
\textit{buti}\\
5\textsuperscript{th} &
\textit{fala/ama} &
\textit{he-} &
\textit{yeting}\\
6\textsuperscript{th} &
\textit{fala/ama} &
\textit{he-} &
\textit{talaama}\\
7\textsuperscript{th} &
\textit{fala/ama} &
\textit{he-} &
\textit{yeting ayoku}\\
8\textsuperscript{th} &
\textit{fala/ama} &
\textit{he-} &
\textit{yeting sua}\\
9\textsuperscript{th} &
\textit{fala/ama} &
\textit{he-} &
\textit{yeting buti}\\
10\textsuperscript{th} &
\textit{fala/ama} &
\textit{he-} &
\textit{kar nuku}\\
100\textsuperscript{th} &
\textit{fala/ama} &
\textit{he-} &
\textit{aisaha nuku}\\
\end{supertabular}
\end{flushleft}
Table F. Kamang ordinals for \textit{kadii} {\textquoteleft}house{\textquoteright} and \textit{alma }{\textquoteleft}person{\textquoteright}

\begin{flushleft}
\tablehead{}
\begin{supertabular}{m{0.5254598in}m{0.93025976in}m{0.42055985in}m{5.01776in}}
1\textsuperscript{st} &
\textit{kadii / alma} &
\textit{ye-} &
\textit{nok}\\
2\textsuperscript{nd} &
\textit{kadii / alma} &
\textit{ye-} &
\textit{ok}\\
3\textsuperscript{rd} &
\textit{kadii / alma} &
\textit{ye-} &
\textit{su}\\
4\textsuperscript{th} &
\textit{kadii / alma} &
\textit{ye-} &
\textit{biat}\\
5\textsuperscript{th} &
\textit{kadii / alma} &
\textit{ye-} &
\textit{wesing}\\
6\textsuperscript{th} &
\textit{kadii / alma} &
\textit{ye-} &
\textit{taama}\\
7\textsuperscript{th} &
\textit{kadii / alma} &
\textit{ye-} &
\textit{wesing ok}\\
8\textsuperscript{th} &
\textit{kadii / alma} &
\textit{ye-} &
\textit{wesing su}\\
9\textsuperscript{th} &
\textit{kadii / alma} &
\textit{ye-} &
\textit{wesing biat}\\
10\textsuperscript{th} &
\textit{kadii / alma} &
\textit{ye-} &
\textit{ataak nok}\\
100\textsuperscript{ th}  &
\textit{kadii / alma} &
\textit{ye-} &
\textit{asaka nok}\\
\end{supertabular}
\end{flushleft}
Table G. Sawila ordinals with \textit{araasing }{\textquoteleft}house{\textquoteright} and \textit{imyalara }{\textquoteleft}man{\textquoteright} 

\begin{flushleft}
\tablehead{}
\begin{supertabular}{m{0.5170598in}m{1.3122599in}m{0.33095986in}m{3.50526in}}
1\textsuperscript{st} &
\textit{araasing/imyalara} &
\textit{wii-} &
\textit{suna}\\
2\textsuperscript{nd} &
\textit{araasing/imyalara} &
\textit{wii-} &
\textit{yaku}\\
3\textsuperscript{rd} &
\textit{araasing/imyalara} &
\textit{wii-} &
\textit{tuo}\\
4\textsuperscript{th} &
\textit{araasing/imyalara} &
\textit{wii-} &
\textit{araasiiku}\\
5\textsuperscript{th} &
\textit{araasing/imyalara} &
\textit{wii-} &
\textit{yooting}\\
6\textsuperscript{th} &
\textit{araasing/imyalara} &
\textit{wii-} &
\textit{yootsuna}\\
7\textsuperscript{th} &
\textit{araasing/imyalara} &
\textit{wii-} &
\textit{yootingyaku}\\
8\textsuperscript{th} &
\textit{araasing/imyalara} &
\textit{wii-} &
\textit{yootingtuo}\\
9\textsuperscript{th} &
\textit{araasing/imyalara} &
\textit{wii-} &
\textit{yootingaraasiiku}\\
10\textsuperscript{th} &
\textit{araasing/imyalara} &
\textit{wii-} &
\textit{adaaku}\\
100\textsuperscript{th} &
\textit{araasing/imyalara} &
\textit{wii-} &
\textit{asaka}\\
\end{supertabular}
\end{flushleft}
Table H. Distributive numerals in Pantar-West Alor languages

\begin{center}
\tablehead{}
\begin{supertabular}{m{0.48725984in}m{1.6233599in}m{1.6705599in}m{1.5719599in}}
\hline
 &
Western Pantar &
Teiwa &
Adang-Lawahing\\\hline
1 &
\textit{ye\~{}ye} &
\textit{nuk\~{}nuk} &
\textit{nu-nu}\\
2 &
\textit{alaku\~{}alaku} &
\textit{raq\~{}raq} &
\textit{al-lo }\\
3 &
\textit{atiga\~{}atiga} &
\textit{yerig\~{}yerig} &
\textit{to-towo}\\
4 &
\textit{atu\~{}atu} &
\textit{{\textglotstop}}\textit{ut\~{}}\textit{{\textglotstop}}\textit{ut} &
\textit{u-ut}\\
5 &
\textit{yasing\~{}yasing} &
\textit{yusan\~{}yusan} &
\textit{iw-wihing}\\
6 &
\textit{hisnakkung\~{}nakkung} &
\textit{tiaam\~{}tiaam} &
\textit{ta-talang}\\
7 &
\textit{betalaku}\textit{\~{}}\textit{talaku} &
\textit{yesraq\~{}raq} &
\textit{witto-to }\\
8 &
\textit{betiga}\textit{\~{}}\textit{tiga} &
\textit{yesnerig\~{}rig} &
\textit{turlo-lo }\\
9 &
\textit{anuktannang\~{}tannang} &
\textit{yesna}\textit{{\textglotstop}}\textit{ut\~{}}\textit{{\textglotstop}}\textit{ut} &
\textit{ti{\textquoteright}inu-nu }\\
10 &
\textit{ke anuku\~{}nuku} &
\textit{qaar nuk\~{}nuk} &
\textit{{\textglotstop}}\textit{ air nu-nu}\\
11 &
\textit{ke anuku wali ye\~{}ye} &
\textit{qaar nuk rug nuk\~{}nuk } &
\textit{{\textglotstop}}\textit{ air nu waling nu-nu}\\
100 &
\textit{ratu}\textit{\~{}}\textit{ratu} &
\textit{ratu nuk\~{}nuk} &
\textit{rat nu-nu}\\
1000 &
\textit{aribu}\textit{\~{}}\textit{aribu} &
\textit{ribu nuk\~{}nuk} &
\textit{rib nu-nu}\\\hline
\end{supertabular}
\end{center}
Table I. Distributive numerals in Central-East Alor languages

\begin{center}
\tablehead{}
\begin{supertabular}{m{0.66775984in}m{2.0462599in}m{1.7038599in}}
\hline
 &
Abui &
Kamang\\\hline
1 &
\textit{nuk\~{}nukda} &
\textit{no\~{}nok, nokda\~{}nokda}\\
2 &
\textit{ayok\~{}ayokda} &
\textit{o\~{}ok\ \ }\\
3 &
\textit{sui\~{}suida} &
\textit{su\~{}su}\\
4 &
\textit{buk\~{}bukna} &
\textit{bye\~{}biat}\\
5 &
\textit{yek\~{}yekna} &
\textit{wesi\~{}wesing}\\
6 &
\textit{talan\~{}talanra} &
\textit{taa\~{}taama}\\
7 &
\textit{yeting ayok\~{}ayokda} &
\textit{wesi\~{}wesingok}\\
8 &
\textit{yeting sui\~{}suida} &
\textit{wesi\~{}wesingsu}\\
9 &
\textit{yeting buk\~{}bukna} &
\textit{wesi\~{}wesingbiat}\\
10 &
\textit{kar nuk\~{}nukda} &
\textit{ataak no\~{}nok}\\
11 &
\textit{kar nuku wal nuk\~{}nukda} &
\textit{ataak nok waal no\~{}nok}\\
100 &
\textit{aisaha nuk}\textit{\~{}}\textit{nukda} &
\textit{asaka no}\textit{\~{}}\textit{nok}\\
1000 &
\textit{rifi nuk}\textit{\~{}}\textit{nukda} &
\textit{ribu no}\textit{\~{}}\textit{nok}\\\hline
\end{supertabular}
\end{center}
\clearpage\section[Appendix B: Numeral Questionnaire used in the field]{Appendix B: Numeral Questionnaire used in the field}
\hypertarget{RefHeading105264871885726}{}{\centering
\textbf{Numerals }
\par}

{\centering
Marian Klamer and Antoinette Schapper
\par}

{\centering
Leiden University, 2010
\par}

It is preferred to elicit the data for this questionnaire using words and constructions in the language of investigation as much as possible. The Malay examples below are not given as prompts to be translated, but rather as additional background for you to help you steer a discussion in Malay. Expressions containing numerals and ordinals, and morphological derivations relating to numerals and ordinals in the AP languages are expected to be quite different from what they are in Malay. 

\textbf{I. Tasks}

1. Ask a person to count in sequence from 1-20 and record this.

2. Elicit 1-100 on paper. Appendix 1: answer sheet.

3. Elicit higher cardinals 2000, 3000,..., 10.000. Appendix 2: answer sheet. 

4. Elicit 100-1000 on paper. Suggestion: You could give (a) speaker(s) an empty  notebook to work on this at their leisure at home. After they have written up all the numbers, please go over it, to check 

\begin{enumerate}
\item \begin{enumerate}
\item \begin{enumerate}
\item if the writing is legible 
\item if you know which letter is used for which sound
\item if this letter-sound correspondence in their orthography is consistent (or consistent enough to be used by us) 
\item if there are any (possible) morphemes or morpheme boundaries that need additional elicitation or discussion --these notes can go with the manuscript.
\end{enumerate}
\end{enumerate}
\end{enumerate}
5. Elicit expressions for basic calculations if any exist: 

\begin{itemize}
\item 3 + 3 = 6: \textit{3 tambah 3 sama dengan }4
\item 5 -- 2 = 3: \textit{lima kurang dua sama dengan tiga}
\item 4 x 5 = 20: \textit{empat kali lima sama dengan dua puluh}
\item 10 : 2 = 5: \textit{sepuluh bagi 2 sama dengan lima} 
\end{itemize}
6. If expressions for basic calculations don{\textquoteright}t exist, or if they are borrowed or calqued from Malay, can consultants think of any other strategies how such basic calculations can be done? Situations to suggest could include:

\begin{itemize}
\item talking about the number of children alive in a family (e.g. 8 children born, 3 died as babies, 5 are still alive), 
\item counting / adding / subtracting pupils in a class setting
\item sigarettes in a packet
\item members of the church who have newly arrived / have left / died
\item multiplying/dividing rupiahs earned by a group of people
\item measuring land to buy or sell e.g. to build a house on
\item etc. 
\end{itemize}
7. Elicit the years 1978, 1999, 2010. If there is no consensus or consistency across speakers, please note down any differences you notice. 

8. Elicit fractions, if they exist

\begin{itemize}
\item half
\item one third
\item quarter
\item try smaller fractions?
\item a tenth
\end{itemize}
Please ask for examples in context, e.g. \textit{Saya bisa makan setengah buah manggo saja }{\textquoteleft}I can only eat half a manggo{\textquoteright},\textit{ Tolong berikan sepertiga/seperempat (bagian) saja }{\textquoteleft}Please give me a third/quarter only{\textquoteright}.

9. If expressions for fractions don{\textquoteright}t exist, can consultants think of other ways to talk about parts of fruits, subgroups of people, parts of piece of land?

10. Ordinals: Elicit 1\textsuperscript{st}{}-10\textsuperscript{th. }e.g., \textit{Saya lihat barisan anak di muka rumah. Yang pertama bernama... yang kedua... yang ketiga... }etc.

Please try also for higher ones: contrast \textit{Anggota gereja yang ketiga }{\textquoteleft}the third member of the church{\textquoteright}  with \textit{anggota yang kesepuluh, yang keratus, yang keseribu} ... It is best to use a local language prompt here, as the higher ones are ungrammatical in Malay!

\textbf{II. Points for further elicitation}

1. Is there a word for zero?

2. Is there an idigenous word for million/\textit{jutah}? 

3. Are there indigenous numbers higher than million?

4. Distinguish non-referential counting (1, 2, 3, {\dots} 10) and enumeration (\textit{satu ekor ayam, dua orang, tiga buku, sepuluh rumah}): are different numerals used?

5. Check if there is a contrast in counting small animates versus large animates and animate vs. inanimate entities: 

\begin{itemize}
\item Small animates

\begin{itemize}
\item ant/\textit{semut }
\item fly/\textit{lalat}
\item bee/\textit{lebah}
\item house lizzard/\textit{cecak}  
\end{itemize}
\item Large animates 

\begin{itemize}
\item child/\textit{anak} 
\item dog/\textit{anjing}
\item pig/\textit{babi} 
\end{itemize}
\item Inanimates

\begin{itemize}
\item house/\textit{rumah} 
\item rock/\textit{batu karang} 
\item star/\textit{bintang} 
\item coconut tree/\textit{pohon kelapa}
\end{itemize}
\end{itemize}
6. Note down the distribution of cardinals as part of NP (in {\textquoteleft}attributive{\textquoteright} function), for example in a context like:

\textit{Ada }\textit{tiga orang}\textit{ di rumah. }\textit{Dua orang}\textit{ pergi ke kota, }\textit{satu orang}\textit{ tinggal di rumah. }{\textquoteleft}There are three people at home. Two went to town, one stayed at home.{\textquoteright} 

\begin{itemize}
\item Is the position of numeral w.r.t. noun fixed or is there variability? E.g. \textit{Orang tiga vs tiga orang }in the above example.
\item If there is variability, check if it is related to higher vs. lower cardinals. E.g. Malay 

\begin{itemize}
\item \textit{Ada }\textit{dua orang}\textit{ di rumah }\textbf{\textit{vs}}\textit{. ada }\textit{orang dua}\textit{ di rumah}
\item \textit{Ada }\textit{sebelas orang}\textit{ di rumah }\textbf{\textit{vs}}\textit{ ada }\textit{orang sebelas}\textit{ di rumah}
\item \textit{Ada lima puluh orang di rumah }\textbf{\textit{vs.}}\textit{ ada orang lima puluh di rumah}
\end{itemize}
\item What is the position of the numeral in the NP if it contains a demonstrative? E.g. \textit{Those}\textit{ }\textit{five}\textit{ girls...  }

\begin{itemize}
\item \textit{Dua}\textit{ orang }\textit{itu}\textit{ ada di rumah, Orang }\textit{dua itu}\textit{ ada di rumah, }\textit{Sebelas}\textit{ orang }\textit{itu}\textit{ ada di rumah, Orang }\textit{sebelas itu}\textit{ ada di rumah, }etc.
\end{itemize}
\end{itemize}
7. Is there any agreement morphology between numeral and noun?

8. Note down the distribution of cardinals as predicate (in {\textquoteleft}predicative{\textquoteright} function), if they are used as such, e.g.:  

\begin{itemize}
\item \textit{Waktu itu kami }\textit{masih bertiga}\textit{ }{\textquoteleft}At that time we were still three{\textquoteright}; 
\item \textit{Mereka datang }\textit{berlima}\textit{, }\textit{berdua}\textit{ mereka pergi }{\textquoteleft}They came with five and left with two{\textquoteright}
\end{itemize}
9. If cardinals may be used in predicative function, can a higher numeral also be used as such? Note that this not generally possible in Malay, where the predicative \textit{ber-} construction is not productively used with higher numerals: \textit{*Waktu itu kami }\textit{berduapuluh}\textit{. }Instead one would say \textit{Waktu itu kami duapuluh orang }{\textquoteleft}We were twenty at the time{\textquoteright}. 

\begin{itemize}
\item Check e.g. 12, 15, 20, 35, 50, 76, 95. 
\end{itemize}
10. In Malay, certain particular high cardinals do appear in the \textit{ber-} construction: \textit{Kami akan datang berseribu }{\textquoteleft}we will come (as) a (group of) thousand{\textquoteright}. So perhaps a language does not treat all higher cardinals in the same way. 

\begin{itemize}
\item Check e.g. 1000, 2000, 100, 500, 1 000 000, 2 000 000
\end{itemize}
11. Can cardinals be used as elliptical for a fuller NP (subject or object): \textit{Mau berapa buah pisang? Saya mau }\textit{dua}\textit{ (dua buah/ dua pisang)}

12. Can cardinals be used as abstract entities, e.g. in contexts like: 

\begin{itemize}
\item \textit{Nomor HP saya mulai dengan angka/nomor tiga }{\textquoteleft}My mobile phone number starts with digit/number three{\textquoteright} 
\item \textit{Waktu mengajar anak menulis guru bilang:} {\textquotedblleft}\textit{Coba menulis angka/nomor dua dan angka duabelas sekarang{\textquotedblright} } {\textquoteleft}When the teacher taught the children to write, he said: {\textquotedblleft} Please write digit/number two and number/digit 12 now{\textquoteright}\textit{. }
\end{itemize}
Try the same with some higher numerals: 

\begin{itemize}
\item Guru bilang kepada anak: \textit{Angka dua puluh itu}\textit{ masih terlalu kecil  }{\textquoteleft}that number 20 is still too small{\textquoteright} 
\end{itemize}
Try the same for \textit{angka} \textit{limabelas, tigapuluh, seratus, seribu, dua ribu, (se)jutah}.

13. \textbf{Reduplication of cardinals:} Can numerals be reduplicated? If so, give some examples in sentential context. 

\begin{itemize}
\item Try 1, 2, 3, 4, 5, 10, 12, 17, 15, 20, 50, 100, 500, 1000.
\end{itemize}
14. What does the reduplication mean? E.g. Malay \textit{beribu-ribu orang datang ke kota itu }{\textquoteleft}People came in thousands to that town{\textquoteright} (vs. \textit{ribuan orang }{\textquoteleft}thousands of people{\textquoteright}) 

15. Does reduplicated {\textquoteleft}one{\textquoteright} have any special meaning? E.g. Bunaq \textit{uen\~{}uen} means {\textquoteleft}same, equal{\textquoteright}; Kamang \textit{no-nok }{\textquoteleft}one by one{\textquoteright}.

16. Where do numeral reduplications occur: before or after the noun? Before or after the verb? Please provide some example sentences.

17. Do reduplicated numerals occur as part of NPs in {\textquoteleft}attributive{\textquoteright} function (as in Malay \textit{beribu-ribu orang})? Or do they occur in {\textquoteleft}predicative{\textquoteright} function? 

18. Check reduplication of NPs encoding subject/actor vs NPs encoding object/undergoer: 

\begin{itemize}
\item \textit{Dua orang laki-laki membawa papan. }\textit{Satu demi satu}\textit{ mereka membawa papan }= one carrier at the time vs. 
\item \textit{Dua orang laki-laki membawa papan. }\textit{Mereka membawa papan }\textit{satu demi satu}\textit{ }= one plank at the time
\end{itemize}
19. Note down the \textbf{distribution of Ordinals}: 

\begin{itemize}
\item as part of NP: 

\begin{itemize}
\item \textit{Orang pertama}\textit{ yang membeli tv adalah Markus }{\textquoteleft}The first person to buy a radio was Markus{\textquoteright}  
\end{itemize}
\item as sth. similar to a non-verbal predicate: 

\begin{itemize}
\item \textit{Lidia }\textit{adalah}\textit{ }\textit{orang pertama yang pergi ke Kupang}\textit{ }{\textquoteleft}Lidia was the first person who went to Kupang{\textquoteright} 
\end{itemize}
\item with an inanimate noun: 

\begin{itemize}
\item \textit{Mereka masuk }\textit{jalan kedua}
\item \textit{Kepala desa membangunkan }\textit{rumahnya kedua}\textit{ (}or\textit{ }\textit{rumah keduanya}\textit{) pada tahun yang lalu  }
\end{itemize}
\item modifying the predicate, in adverbial-like function: 

\begin{itemize}
\item \textit{Mereka pergi ke Kupang }\textit{pada kali yang kedua}.
\end{itemize}
\end{itemize}
20. Are the ordinals etymologically clearly related to cardinals? e.g. Indonesian ordinal \textit{pertama} is not derived from cardinal \textit{satu.}

21. Are there any words that are used like ordinals but have no numeral or ordinal root?

22. \textbf{Plural marking}: Is plural marked with an affix? 

23. Does the language have a plural word? E.g. \textit{non }{\textquoteleft}PLURAL{\textquoteright} in Teiwa.

A plural word is a morpheme whose meaning and function is similar to that of plural affixes in other languages, but which is a separate word that functions as a modifier of the noun.  Plural words are overrepresented in isolating or analytic languages, in languages with classifiers, and in head-marking languages (cf. M. Dryer, Plural words, \textit{Linguistics }27 (1989), 865-895.)

\textbf{\textit{Questions 24-29 only apply when the language has a plural word:}}

24. If the language has a plural word, do you observe any animacy or size effects in the use of the plural word? Check:

\begin{itemize}
\item \textit{orang perempuan }
\item \textit{kakak perempuan }
\item \textit{anak laki-laki }
\item \textit{babi, anjing }
\item \textit{tikus, burung }
\item \textit{nyamuk, semut, lebah, lalat }
\item \textit{batu kecil, jarum, kancing}
\item \textit{kendi, panci, mok}
\item \textit{batu karang, pohon kelapa}
\item \textit{bintang}
\item \textit{rumah}
\end{itemize}
25. Plural words as {\textquoteleft}numerals{\textquoteright}: Can plural word and numeral co-occur? (If so, this could be evidence that they belong to different categories.)

26. Can plural word and non-numeral quantifiers (\textit{beberapa, semua, sedikit, banyak}) co-occur? 

27. Can plural word and possessor noun co-occur? 

28. Can plural word and possessive prefix co-occur? 

29. Plural words are reported to derive from e.g. third person plural pronoun, plural article, words meaning \textit{all} or \textit{many}, nouns meaning \textit{group} or \textit{set}, classifier,{\dots} etc. Do you have ideas about the possible diachronic origin of the plural word in the language of study?

30. \textbf{Quantifiers (non-numeral)} \textit{semua, banyak, sedikit, beberapa}

What does the quantifier inventory look like for

\begin{itemize}
\item Countable objects 

\begin{itemize}
\item \textit{orang, babi, anjing, rumah, kursi, gelas}
\end{itemize}
\item Uncountable objects or masses 

\begin{itemize}
\item \textit{garam, gula, air, nasi, jagung (?), semut, lebah, lalat, }
\item \textit{gunung-gunung (?), awan-awan (?)}
\end{itemize}
\item Liquids 

\begin{itemize}
\item \textit{air, air susu, anggur, arak, teh}
\end{itemize}
\item Edibles 

\begin{itemize}
\item \textit{buah pinang, daun papaya, daging babi, ikan}  
\end{itemize}
\end{itemize}
31. Do particular semantics play a role in the interpretation of the value of the quantifiers? (e.g. (un)expected/(un)wanted value, e.g. many people come to church, more than expected, or when only a little bit of gas is sold less than expected (\textit{misalnya kalau }\textit{banyak orang}\textit{ datang ke kereja, lebih dari harapan }(\textit{atau }\textit{hanya}\textit{ }\textit{sedikit minyak}\textit{ dijual}, \textit{kurang dari harapan).}

32. \textbf{Classifiers}: We will make a separate questionnaire \& stimuli for this at a later stage. If you have made some observations about the classifiers, please include them here. 

\textbf{Appendix 1: Numerals 1-100}

\begin{flushleft}
\tablehead{}
\begin{supertabular}{|m{0.6205598in}|m{2.6712599in}|m{2.6719599in}|}
\hline
No &
Language: .... &
Notes\\\hline
1 &
 &
\\\hline
2 &
 &
\\\hline
\end{supertabular}
\end{flushleft}
\textit{..... }

\begin{flushleft}
\tablehead{}
\begin{supertabular}{|m{0.6205598in}|m{2.6712599in}|m{2.6719599in}|}
\hline
99 &
 &
\\\hline
100 &
 &
\\\hline
\end{supertabular}
\end{flushleft}
\textbf{Appendix 2: Higher cardinals}

\begin{flushleft}
\tablehead{}
\begin{supertabular}{|m{0.9955599in}|m{3.0302598in}|m{2.01296in}|}
\hline
No &
Language: ...... &
Notes\\\hline
 &
 &
\\\hline
1000 &
 &
\\\hline
2000 &
 &
\\\hline
\end{supertabular}
\end{flushleft}
\textit{..... }

\begin{flushleft}
\tablehead{}
\begin{supertabular}{|m{0.9955599in}|m{3.0302598in}|m{2.01296in}|}
\hline
9000 &
 &
\\\hline
10 000 &
 &
\\\hline
\end{supertabular}
\end{flushleft}

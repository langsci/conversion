
\clearpage\setcounter{page}{1}\pagestyle{Standard}
{\centering\itshape
[Warning: Draw object ignored]Chapter 1
\par}

\begin{center}
 [Warning: Image ignored] % Unhandled or unsupported graphics:
%\includegraphics[width=5.9996in,height=4.3268in,width=\textwidth]{Ch1KlamerIntroduction-img1.png}

\end{center}
\begin{center}
 [Warning: Image ignored] % Unhandled or unsupported graphics:
%\includegraphics[width=7.5921in,height=2.4795in,width=\textwidth]{Ch1KlamerIntroduction-img2.jpg}

\end{center}
\begin{center}
 [Warning: Image ignored] % Unhandled or unsupported graphics:
%\includegraphics[width=6.3004in,height=3.7531in,width=\textwidth]{Ch1KlamerIntroduction-img3.png}

\end{center}
{\centering\bfseries
The Alor Pantar languages: 
\par}

{\centering\bfseries
Linguistic context, history and typology  
\par}

\textit{[NB. This is a }\textbf{\textit{first draft}}\textit{ that still needs revision]}

{\centering\bfseries\itshape
Chapter 1
\par}

{\centering\bfseries
The Alor Pantar languages: 
\par}

{\centering\bfseries
Linguistic context, history and typology  
\par}

{\centering
Marian Klamer
\par}

Abstract: This chapter presents an introduction to the Alor Pantar languages, and to the chapters of the volume. It discusses the current linguistic ecology of Alor and Pantar, the history of research on the languages, presents an overview of the history of research in the area and describes the state of the art of the (pre-)history of speaker groups on the islands. A typological overview of the family is presented, followed by a discussion of specific sets of lexical items. Throughout the chapter I provide pointers to individual chapters of the volume that contain more detailed information or references. 

{\bfseries
\ \ \hypertarget{Toc376957260}{}Introduction\footnote{Parts of this overview were culled from Klamer 2010, 2011, Baird et al. ms., Holton and Klamer to appear. }}

The languages of the Alor Pantar (AP) family constitute a group of twenty Papuan languages spoken on the islands of Alor and Pantar, located just north of Timor, at the end of the Lesser Sunda island chain in eastern Indonesia, see Map 1. This outlier {\textquoteleft}Papuan{\textquoteright} group is located some 1000 kilometers west of the New Guinea mainland. The term {\textquoteleft}Papuan{\textquoteright} is used here as a cover term for the hundreds of languages spoken in New Guinea and its vicinity that are not Austronesian (Ross 2005:15), and it is considered synonymous with non-Austronesian. The label {\textquoteleft}Papuan{\textquoteright} says nothing about the genealogical ties between the languages. 

\ \ The Alor Pantar languages form a family that is clearly distinct from the Austronesian languages spoken on the islands surrounding Alor and Pantar, but much is still unknown about their history: Where did they originally come from? Are they related to other languages or language groups, and if so, to which ones? Typologically, the AP languages are also very different from their Austronesian neighbours, as their syntax is head-final rather than a head-initial. Typologically; they show an interesting variety of alignment patterns, and the family has some cross-linguistically rare features.  

\ \ This volume studies the history and typology of the AP languages. Each chapter compares a set of AP languages by their lexicon, syntax or morphology, with the aim to uncover linguistic history and discover typological patterns that inform linguistic theory. 

\ \ As an introduction to the volume, this chapter places the AP languages in their current linguistic context (section 2), followed by an overview of the history of research in the area (section 3). Then I describe the state of the art of the (pre-)history of speaker groups on Alor and Pantar (section 4). A typological overview of the family is presented in section 5, followed by information on the lexicon in section 6. In section 7, I summarize the chapter and outline challenges for future research in the area. The chapter ends with a description of the empirical basis for the research that is reported in this volume (section 8). Throughout this introduction, cross-references to the volume chapters will be given, to enable the reader to focus on those chapters that s/he is most interested in. 

{\centering
Map 1: Alor and Pantar in Indonesia
\par}

  [Warning: Image ignored] % Unhandled or unsupported graphics:
%\includegraphics[width=6in,height=4.3272in,width=\textwidth]{Ch1KlamerIntroduction-img4.png}
 

\begin{center}
\begin{minipage}{0.7752in}
New Guinea
\end{minipage}
\end{center}
\begin{center}
\begin{minipage}{0.5484in}
Java
\end{minipage}
\end{center}
\begin{center}
\begin{minipage}{0.702in}
Timor
\end{minipage}
\end{center}
\begin{center}
\begin{minipage}{1.0173in}
Australia
\end{minipage}
\end{center}
\clearpage{\bfseries
\ \ Current linguistic situation on Alor and Pantar}

There are \~{}20 indigenous Papuan languages spoken (section 2.1) alongside one large indigenous Austronesian language commonly referred to as {\textquoteleft}Alorese{\textquoteright} (section 2.2). Virtually all speakers of these indigenous languages also speak the local Malay variety and/or the national language Indonesian on a regular basis for trade, education and governmental business (section 2.3). 

\subsection[2.1 \ \ The Papuan languages of Alor and Pantar ]{2.1 \ \ The Papuan languages of Alor and Pantar }
The Papuan languages of Alor and Pantar as they are currently known are listed alphabetically in Table 1, and presented geographically on Map 2. Together they form the Alor-Pantar family. The Alor Pantar family is related to five Papuan languages spoken on Timor and Kisar which are listed in Table 2, and presented geographically on Map 3.

\ \ Like all classifications, the classification in Table 1 is preliminary. The status of Reta and Tereweng as separate languages awaits further investigation. Also, it is likely that the central-eastern part of Alor is linguistically richer than the map suggests, but untill a more principled survey of the area has been done, we stick with the labels Abui and Kamang, while acknowledging that there may be multiple languages within each of these regions. 

\ \ Some of the names and classifications of earlier works (e.g. Stokhof 1975, Grimes et. al. 1997, Lewis 2009) do not agree with what is presented here (see also section 3). One reason may be that a language variety may either be referred to by the name of the village where it is spoken, or by the name of the ancestor village of the major clan that speaks the language, or by the clan name. The classification in Table 1 aims toward more {\textquoteleft}lumping{\textquoteright} than {\textquoteleft}splitting{\textquoteright}. The traditional criterion of mutual intelligibility is extremely difficult to apply, as the languages have undergone an extended period of contact and multi-lingualism. 

\ \ For those languages which have been the subject of recent investigation a reference to a grammar or grammatical sketch that is (about to be) published is included in the table. Further references to published work on the languages are presented in section 3.

\clearpage{\centering
Table : The languages of the Alor-Pantar family. (Population estimates from fieldworker/source; starred (*) population estimates from Lewis 2009).
\par}

\begin{center}
\tablehead{}
\begin{supertabular}{m{1.1983598in}m{0.54275984in}m{0.9497598in}m{0.9115598in}m{2.1538599in}}
\hline
\bfseries Language &
{\bfseries ISO }

\bfseries 639-3 &
{\bfseries Alternate }

\bfseries Name(s) &
\textbf{Population}\footnotemark{} &
\bfseries References (selected)\\\hline
Abui (A\textsc{b)} &
abz &
Papuna &
17000 &
Kratochv\'il (2007)\\
Adang (A\textsc{d)} &
adn &
 &
7000 &
Haan (2001), Robinson and Haan (to appear)\\
Blagar (B\textsc{l)} &
beu &
Pura &
10000 &
Steinhauer (to appear)\\
Deing (D\textsc{e)} &
{}-{}- &
Diang, Tewa &
 &
\\
Hamap ((H\textsc{m)} &
hmu &
 &
1300* &
\\
Kabola (K\textsc{b)} &
klz &
 &
3900* &
Stokhof (1987)\\
Kaera (K\textsc{e)}  &
{}-{}- &
 &
5500 &
Klamer (to appear b)\\
Kafoa (K\textsc{f)} &
kpu &
 &
1000* &
Baird (to appear)\\
Kamang (K\textsc{m)} &
woi &
Woisika &
6000 &
Stokhof (1977), Schapper (to appear)\\
Kiramang (K\textsc{r)} &
kvd-kir &
 &
4240* &
\\
Klon (K\textsc{l)} &
kyo &
Kelon &
5000 &
Baird (2008)\\
Kui (K\textsc{i)} &
kvd &
 &
4240* &
\\
Kula (K\textsc{u)} &
tpg &
Tanglapui &
5000* &
Williams and Donohue (to appear), Donohue (1996)\\
Nedebang (N\textsc{d)} &
nec &
Klamu &
1380* &
\\
Reta (R\textsc{t)}  &
ret &
Reta &
 &
\\
Sar (S\textsc{r)} &
{}-{}- &
Teiwa? &
 &
\\
Sawila (S\textsc{w)} &
swt &
 &
3000 &
Kratochv\'il (to appear)\\
Teiwa (T\textsc{w)} &
twe &
Tewa &
4000 &
Klamer (2010)\\
Wersing (W\textsc{e)} &
kvw &
Kolana &
3700* &
Schapper and Hendery (to appear)\\
Western 

Pantar (WP) &
lev &
Lamma &
10300 &
Holton (2010, to appear a)\\\hline
\end{supertabular}
\end{center}
\footnotetext{Population estimates from fieldworker/source; starred (*) estimates from Lewis (2009).}
\clearpage{\centering
Map 2: The languages of the Alor-Pantar family, and Austronesian Alorese.
\par}

  [Warning: Image ignored] % Unhandled or unsupported graphics:
%\includegraphics[width=7.5925in,height=2.4799in,width=\textwidth]{Ch1KlamerIntroduction-img5.jpg}
 

\clearpage{\centering
Table 2: The Papuan languages of Timor-Kisar . (Population estimates from fieldworker/source; starred (*) population estimates from Lewis 2009).
\par}

\begin{center}
\tablehead{}
\begin{supertabular}{m{0.8059598in}m{0.54345983in}m{0.9490598in}m{0.9115598in}m{2.29486in}}
\bfseries Language &
{\bfseries ISO }

\bfseries 639-3 &
{\bfseries Alternate }

\bfseries Name(s) &
\bfseries Population &
\bfseries References (selected)\\
Fataluku &
ddg &
 &
30000* &
van Engelenhoven (2009, 2010)\\
Makalero &
mkz &
Maklere &
6500 &
Huber (2011)\\
Mkasai &
mkz &
Makasae &
70000* &
Huber (2008)\\
Oirata &
oia &
 &
1220* &
de Josselin de Jong (1937)\\\hline
\end{supertabular}
\end{center}
{\centering
Map 3: The Papuan languages of Timor-Alor-Pantar.
\par}

{\centering
(Areas where Austronesian languages are spoken are left white.)
\par}

  [Warning: Image ignored] % Unhandled or unsupported graphics:
%\includegraphics[width=6.3008in,height=3.7535in,width=\textwidth]{Ch1KlamerIntroduction-img6.png}
 

\clearpage\subsection[2.2 \ \ Indigenous Austronesian languages on Alor and Pantar ]{2.2 \ \ Indigenous Austronesian languages on Alor and Pantar }
The major indigenous Austronesian language spoken on Alor and Pantar is Alorese, also refered to as {\textquoteleft}Bahasa Alor{\textquoteright}, {\textquoteleft}Alor{\textquoteright}, or {\textquoteleft}Coastal Alorese{\textquoteright}. Klamer (2011) is a sketch of the language. Alorese has 25,000 speakers, who live in pockets along the coasts of western Pantar and the Kabola peninsula of Alor island, as well as on the islets Ternate and Buaya (Stokhof 1975:8-9, Grimes et.al. 1997, Lewis 2009). There are reports that Alorese was used as the language of wider communication in the Alor Pantar region till at least the mid 1970{\textquoteright}s (see Stokhof 1975:8), but as such it did not make inroads into the central mountainous areas of Pantar or Alor, and its lingua franca function may have been limited to Pantar and the Straits in between Pantar and Alor.

\ \ The vocabulary of Alorese is clearly (Malayo-Polynesian) Austronesian. On the basis of a short word list, Stokhof (1975:9) and Steinhauer (1993:645) suggest that the language spoken on the Alor and Pantar coasts is a dialect of Lamaholot. Lamaholot is an Austronesian language spoken on the islands west of Pantar: Lembata, Solor, and east Flores.\footnote{Barnes (2001:275) and Blust (2009:82, 2013: 87) represent Lamaholot as a language spoken on Alor and Pantar, but the language spoken on Alor and Pantar is Alorese (Klamer 2011).} Recent research however indicates that Alorese and Lamaholot show significant differences in lexicon as well as grammar: 40-50\% of the basic vocabulary in Alorese and Lamaholot is different, severely hindering mutual intelligibility; the languages have different sets of pronouns and different possessive constructions; and, most striklingly, Alorese lacks all the inflectional and derivational morphology that is present in Lamaholot (Klamer 2011, 2012). The evidence clearly 
suggests that Alorese should be considered a language in its own right. 

\ \ Oral history and ethnographic observations (Anonymous 1914, Lemoine 1969, Rodemeijer 2006) report local traditions about non-indigenous, Austronesian groups arriving in \textstylepagenumber{the northern coastal parts of Pantar around 1,300 AD whose }descendants colonized the coasts of north-western Pantar and west Alor. Some of the locations mentioned are home to speakers of Alorese.\footnote{Pandai, Baranusa and Alor are locations where today Austronesian Alorese is spoken. H\"agerdal (2012: 38) cites evidence that Kui and Blagar were part of the a league of princedoms with Pandai, Baranusa, and Alor. }\ \  

\ \ Apart from Alorese, there are also languages spoken by more recent Austronesian immigrants. For instance, Bajau (or Bajo) is the language of the nomadic communities located through most of Indonesia which are also referred to as {\textquoteleft}sea gypsies{\textquoteright} (cf. Verheijen 1986). One or more groups of Bajau came from Sulawesi, through Flores, and settled on the coast near Kabir, on Pantar island, in the 1950{\textquoteright}s. A second wave of Bajau speakers arrived in 1999, from East Timor. Bajau communities are also found on Alor. 

\subsection[2.3 \ \ Indonesian and Alor Malay]{2.3 \ \ Indonesian and Alor Malay}
Indonesian has been introduced relatively recently in Alor Pantar, roughly correlating with the increasing number of Indonesian primary schools established in rural areas since 1960-1970. Today, speakers of the Alor Pantar languages employ Indonesian and/or the local variety of Malay as language of trade, education, and governmental business. 

\ \ The Alor Malay variety was already in use in the Alor-Pantar archipelago before standard Indonesian was introduced. Alor Malay is based on the variety spoken in the provincial capital on Timor, referred to as Kupang Malay (Jacob and Grimes 2003; Baird, Klamer, Kratochv\'il, ms.). 

\ \ Alor and Pantar were under (remote) Portuguese control till 1860, and Dutch colonial influence only became apparent in the first decades of the 20\textsuperscript{th }century (see section 4). In 1945, Van Gaalen reports that {\textquoteleft}[on the Kabola peninsula] the majority of the people can speak Malay{\textquoteright} (1945:30). It was probably there, and in the main town Kalabahi where most of the (few) Dutch government schools were located. But the influence of the Dutch schools must have been fairly limited, because in 1937 (after being a Dutch colony for over 60 years), only 7.5\% of the children on Alor were going to school (2,089 out of a total population of 28,063 boys and girls) (Van Gaalen 1945:24, 41a). Du Bois (1960:17) comments on the situation of Malay in schools in central Alor as being desolate, and notes in passing that in her reserach location Atimelang there were only about 20 boys who understood Malay (possibly implying that girls were not attending the school). The picture 
emerges that many areas in central and east Alor remained mostly unexposed to Malay. On the other hand, certain areas that were converted to Christianity before WWII may have been exposed to Malay earlier through the churches; this may have been the case in the Teiwa and Kaera speaking areas in Pantar, and the Apui area in central Alor. \ \ One result of the language contact between Alor Malay, Indonesian and the local Papuan languages is a rapidly on-going language shift from vernacular to languages of wider communication. None of the Alor Pantar languages is {\textquotedblleft}safe,{\textquotedblright} and most are definitely endangered, in that many children are not learning the language in the home. Local languages are not used or taught in schools, as primary school teachers often have a different language background, and orthographies and dictionaries have only recently been produced for some of the languages. Language shift to Indonesian/Malay is often accelerated by urbanization and the practice of 
schooling children in urban centers away from their home vernacular language areas. For instance, children from Pantar or east Alor go to school in Kalabahi or Kupang. Language attitudes play an additional role, as the local languages lack prestige value.

{\bfseries
\ \ History of research on the Alor and Pantar languages}

Initial anthropological and linguistic work on Alor was carried out by Du Bois (1960 [1944]) and Nicolspyer (1940), both working in the Abui area in central Alor. Between 1970-2000 research based in Leiden University resulted in a number of publications on Alor and Pantar languages. Stokhof (1975) is a 100-item word lists of 17 Alor Pantar varieties. Stokhof published language materials on Woisika, which is the language referred to as Kamang in Table 1 (Stokhof 1977, 1978, 1979, 1982, 1983), on Abui (Stokhof 1984) and on Kabola (Stokhof 1987). Publications on Blagar are by Steinhauer (1977, 1991, 1993, 1995, 1999, 2010, 2012). Donohue published an article on Tanglapui, the language referred to as Kula in Table 1) (Donohue 1996). The \textit{Pusat Pembinaan dan Pengembangan Bahasa} produced survey word lists of the languages of Alor (Martis et al. 2000) and a sketch of Lamma (Nitbani et al. 2001), one of the dialects of the language listed as Western Pantar in  Table 1. A grammar of Adang was completed by a 
native speaker of the language (Haan 2001). 

\ \ Between 2003-2007, research took place in Pantar and the western part of Alor, through a project at Leiden University that was funded with a grant from the Netherlands Organisation of Scientific Research.\footnote{Innovative research ({\textquotedblleft}Vernieuwingsimpuls{\textquotedblright}) project \textit{Linguistic variation in Eastern Indonesia: The Alor Pantar project}, lead by Marian Klamer at Leiden University.} Results of this project include work on Klon (Baird 2005, 2008, 2010), Kafoa (Baird, to appear), Abui (Kratochv\'il 2007, Kratochv\'il and Delpada 2008a, Kratochv\'il and Delpada 2008b, Klamer and Kratochv\'il 2006, 2010; Kratochv\'il 2011a, 2011b), Teiwa (Klamer 2010a, 2010b, 2010c, 2012, Klamer and Kratochv\'il 2006, Klamer 2011, Klamer 2012), Kaera (Klamer 2010b, to appear), Sawila (Kratochv\'il to appear) and Alorese (Klamer 2011, 2012). At the same time, Gary Holton from the University of Alaska at Fairbanks, documented Western Pantar (Holton 2008, 2010a, 2011, Holton and Lamma Koly 
2008, to appear a, b). 

\ \ In 2009, a fund from the European Science Foundation enabled a further research project on Alor Pantar languages, now involving a group of seven researchers from the University of Alaska Fairbanks, the University of Surrey, and Leiden University. The chapters in the present volume all report on research carried out between 2009-2013 as part of  this project.

{\bfseries
\ \ History of Alor and Pantar languages and their speakers}

\subsection[4.1 \ \ Prehistory]{4.1 \ \ Prehistory}
The Papuan languages of Alor and Pantar belong to a single group (Holton et al. 2010, Holton et al. 2012, Holton and Robinson, \textit{this volume} a), which spread over the two islands several millennia ago. The Alor Pantar language group then forms with the Papuan languages of Timor the Timor Alor Pantar family (Schapper et al. \textit{this volume}).

\ \ One hypothesis holds that the Timor Alor Pantar family is a sub-branch of the Trans New-Guinea family. That is, it ultimately descends from immigrants from the New Guinea highlands who arrived in the Lesser Sundas 4,500-4,000 BP (Bellwood 1997:123, Ross 2005:42, Pawley 2005). However, recent comparative historical research (Robinson and Holton 2012, Holton and Robinson, \textit{this volume} b) shows no lexical evidence to support an affiliation with the Trans New-Guinea languages (cf. Wurm, Voorhoeve, McElhanon 1975, Ross 2005).

\ \ Another hypothesis holds that the Papuans in the Lesser Sundas descend from arrivals 20,000 BP (Summerhayes 2007). While this possibility cannot be excluded, the level of lexical and grammatical similarity in the AP family does not support an age of more than several millennia, and the reconstructed vocabulary of proto-AP appears to contain Austronesian loan words (e.g. {\textquoteleft}betel nut{\textquoteright}, Holton et al. 2012). Ancient Austronesian loans found across the Alor Pantar family following regular sound changes suggest that the AP family split up after being in contact with the Austronesian languages in the area. As the Austronesians are commonly assumed to have arrived in the area \~{}3,000 Before Present (BP) (Pawley 2005:100, Spriggs 2011), this which would give the Alor Pantar family a maximum age of \~{}3,000 years.

\ \ As yet, no archeological data on the Alor Pantar archipelago is available. Archaeological research in Indonesia has been largely determined by the aim to trace the Austronesian dispersal through the archipelago, with a focus on the western islands Borneo, Sulawesi and Java (Mahirta 2006).\footnote{The most important site in eastern Indonesia is Liang Bua in central Flores (Morwood et al. 2004), located several hundreds of kilometers west of the Pantar.  Mid 2012 a site was opened in Pain Haka, east Flores (investigators Simanjuntak, Galipaud, Buckley). Results are expected towards the end of 2015.} What archeological evidence we have on the Lesser Sunda islands relates to large islands such as Flores and Timor, and it suggests that the large islands were settled by Austronesians prior to smaller and more isolated islands such as Pantar and Alor. 

\ \ Archaeological and anthropological studies in East Timor (O{\textquoteright}Connor 2003, 2007, McWilliam 2004) show that the chronology of Papuan and Austronesian influence can differ per location, and that populations that now speak a Papuan language may have been Austronesian originally. Similarly, Austronesian languages may have been adopted by originally Papuan speakers.

\ \ Human genetic studies support a connection between populations of the Lesser Sundas with Papuan populations of New Guinea and Austronesians from Asia (Lansing et al. 2011, Xu et al. 2012). The Papuan (or {\textquotedblleft}Melanesian{\textquotedblright})-Asian admixture is estimated to have begun about 5,000 years BP in the western part of eastern Indonesia, decreasing to 3,000 years BP in the eastern part. This associates the Papuan-Asian admixture with Austronesian expansion (Xu et al. 2012). Debate is ongoing on the importance and details of Austronesian expansion in Island South East Asia, but consensus exists that eastern Indonesia shows a {\textquotedblleft}complex migration history{\textquotedblright} (Lansing et al. 2011: 263). 

\subsection[4.2 \ \ Historical records on Alor and Pantar]{4.2 \ \ Historical records on Alor and Pantar}
To date, few if any records exist on the history of the Papuan groups of Alor and Pantar. Most of the written historical records refer to the large neighboring islands of Flores and Timor, and to contacts between groups on Flores and Timor on the one hand, and the coastal populations of Pantar and Alor on the other (Barnes 1996, De Roever 2002, Steenbrink 2003, H\"agerdal 2010a,b, 2011, 2012, and references). It is very likely that these coastal populations were the Austronesian Alorese (section 2.2). 

\ \ One of the earliest written records is a Portuguese missionary text written after 1642, where Pantar (referred to as {\textquotedblleft}Galiyao{\textquotedblright}\footnote{L\textstylepagenumber{inguistic research on Pantar by Gary Holton (2010b) has shown that }Galiyao is used in various local Papuan languages as the indigenous name for the island of Pantar. The name originates from Western Pantar \textit{Gale Awa}, literally {\textquoteleft}living body{\textquoteright}\textstylepagenumber{. {\textquotedblleft}}\textstyleCharChar{The appropriateness of this name is evidenced by the presence of an active volcano which dominates southern Pantar. This volcano regularly erupts, often raining ash and pyroclastic flows onto villages of the region. Even when it is not erupting, the volcano ominously vents sulfur gas and smoke from its crater. In a very real sense, the volcano is a living body.{\textquotedblright} (Holton 2010). For discussions of how the term Galiyao refers to (parts of) Pantar, see Le Roux 
1929:47, Barnes 1982:407, Dietrich 1984, Rodemeier 1995, Barnes 2001:277, Rodemeier 2006.}}) is mentioned as a place inhabited by pagans and Muslims, together with Lewotolok and Kedang on Lembata island, located west of Pantar. Alor (referred to as {\textquotedblleft}Malua{\textquotedblright}) is described as an unattractive place, with few opportunities for trade and a heathen cannibal population (H\"agerdal 2012:101). It is certain that in ancient times there was traffic back on forth between Alor-Pantar, Timor, and the islands west of Pantar: traders in  Kalikur, a port in north Lembata, heard from Alor traders about famous Timorese warriors which were brought to Kedang, also in north Lembata, to suppress villages of the island{\textquoteright}s interior (Barnes 1974:10, 12; Le Roux 1929:14). 

\ \ Ships of the Dutch East India Company (VOC) rarely ventured to Pantar and Alor. It was traders from Portugal who bought local products in exchange for iron, cutlasses, and axes (Van Galen; see H\"agerdal 2010:17). In the early 18\textsuperscript{th} century, Portugal attempted to establish a base on Alor. Some fifty black Portuguese soldiers (originally from Africa) travelled from Larantuka in East Flores, landed in Pandai (north Pantar) in 1717 and built a church and a settlement there (Coolhaas 1979:297, Rodemeier 2006:78). The Portuguese made some {\textquoteleft}treaties{\textquoteright} with local rulers, but their influence remained limited to some coastal regions in north Pantar and west Alor.\footnote{The Portuguese {\textquotedblleft}[handed] out Portuguese flags to some coastal rulers, among others those of Koei, Mataroe, Batoelolong, Kolana{\textquotedblright} (Van Gaalen 1945: 2).} 

\ \ The Alor archipelago was part of an areal trade network. For example, in 1851, every year more than 100 vessels came to the island, with traders from Buton and Kupang (buying rice and corn), as well as Bugis and Makassar (buying wax) (Van Lynden 1851:333). In 1853, the Portuguese gave up their claim on the Alor archipelago in exchange for the Dutch Pulau Kambing (currently known as the island Ata\'uru), located just north of Dili in East Timor. However, the Kui speaking areas on the southern coast of Alor remained in close contact with the Portuguese, thus prompting a Dutch military action in 1855, when the Dutch steamship Vesuvius destroyed the Kui village with its guns (H\"agerdal 2010:18{}-19). Overall, however, the Dutch involvement with Alor remained limited for decades. The Dutch stationed a Posthouder at the mouth of the Kabola bay around 1861 and basically left it at that.

\ \ Only in 1910, under Governor-General Van Heutz, did the Dutch start a military campaign to put local rulers under Dutch control. Until 1945, there were regular revolts from local rulers (see the reports in Van Gaalen 1945:2-9). The presence of Dutch law was generally accepted only after WWII, after much bloodshed. Today, the Dutch cultural influence is most visible in the town of Kalabahi and in the Kabola peninsula.

\ \ Chinese traders have been active in the area since the end of the 19\textsuperscript{th} century (Du Bois 1960:16). These traders have likely arrived from Kupang or more remote communities, bringing with them Kupang Malay or trade Malay. Nicolspeyer (1940:1) reports that by late 1930s there was a 200 member strong Chinese trader community in Kalabahi engaged in the production and trade in copra. The relationship between the Chinese community and the local population must have been friendly, judging from the oral accounts of mountain population offering Chinese people refuge during the Japanese occupation in WWII. 

\ \ In contrast, Nicolspeyer (1940:8) describes the trade relations between the highlanders and the coastal populations  {}- likely to be Alorese - in west and central Alor as mutually distrusting and hostile. Traditionally, the Alorese clans exchanged fish and woven cloth for food crops with the inland populations (cf. Anonymous 1914:76, 81-82). Given the small size of individual Alorese clans - Anonymous (1914:89-90) mentions settlements of only 200-300 people - , they probably exchanged women with the exogamous Papuan populations around them, or bought them as slaves.

\ \ In the east of Alor, there was contact with populations on Ata\'uru and Timor. Until 1965 it was not uncommon to sail from the southcoast of Alor to Ata\'uru island on fishing trips. The oral accounts of these contacts are supported by genealogies and origin myths, as well as by a number of Portuguese loanwords such as the Sawila verb \textit{siribisi} originating in the Portuguese \textit{serviso} {\textquoteleft}work{\textquoteright} (Kratochv\'il, field notes). In addition, many songs in central-east Alor mention place names such as Likusaen and Maubara, which are located in the north of Timor in the area where today Tokodede is spoken (Wellfelt and Schapper 2013).

\ \ In 1965-1966, hundreds and possibly thousands of highlanders in Alor and Pantar were marched to Kalabahi and killed by the Indonesian forces and associated vigilantes in the convulsions after the alleged communist coup. Oral accounts of the atrocities still circulate among the population and the terror is palpable whenever such accounts are shared.

\subsection[4.3 \ \ Contact]{4.3 \ \ Contact}
All of the Alor Pantar languages show some traces of contact with Austronesian languages, but in general, borrowing from Austronesian has not been very intense. Contact with Malay and Indonesian is a relatively recent phenomenon in most of the Alor-Pantar languages. Comparing \~{}160 vocabulary items in 13 AP languages, Robinson (2012) found Austronesian loan percentages to range between 3.8\% (in Kamang and Western Pantar) and 11.3\% (in Blagar), while the majority of AP languages has only 5-7\% of Austronesian loans.

\ \ Of course, lexical borrowing within the Alor Pantar family occurs as well. An example is Western Pantar \textit{bagis }{\textquoteleft}to wail{\textquoteright}, borrowed from Deing \textit{bagis }{\textquoteleft}to cry{\textquoteright} (Holton and Robinson \textit{this volume}). In situations where speakers of sister languages are also geographical neighbors and in contact with each other, it is however notoriously difficult to distinguish loans from cognates.

{\bfseries
\ \ Typological overview}

This section presents a general overview of the structural features of AP languages. The aim is to introduce the reader to the phonology, morphology and syntax of the AP languages, pointing out patterns that are crosslinguistically common and patterns that are rare. Where appropriate, I refer to chapters in this volume for further discussion or illustration. 

\subsection[5.1 \ \ Phonology]{5.1 \ \ Phonology}
The sizes of the vowel and consonant inventories of the AP languages are transitional between the smaller vowel systems and large consonant systems of insular Southeast Asia, and the more complex vowel systems but much more reduced consonant inventories to the east, in the wider New Guinea/Oceania region (cf. Hajek 2010).

\ \ The vowel systems in Alor-Pantar involve the five cardinal vowels, possibly adding distinctions in mid vowels (e.g. Klon) and/or in length (e.g. Teiwa, Abui, Kamang). The proto-Alor Pantar consonant inventory (Holton et al. 2012, Holton and Robinson, \textit{this volume} a) is shown in Table 3.

{\centering
Table 3: Reconstructed pAP consonant inventory
\par}

\begin{center}
\tablehead{}
\begin{supertabular}{m{0.7795598in}m{0.5545598in}m{0.57335985in}m{0.6913598in}m{0.5365598in}m{0.64555985in}m{0.7073598in}}
\hline
 &
\centering \scshape labial &
\centering \scshape Apical &
\centering \scshape Palatal &
\centering \scshape Velar &
\centering \scshape Uvular &
\centering\arraybslash \scshape Glottal\\\hline
\scshape Stop &
\centering \textstyleIPA{p  b} &
\centering \textstyleIPA{t  d} &
 &
\centering \textstyleIPA{k  g} &
\centering \textstyleIPA{q} &
\\
\scshape Fricative &
 &
\centering \textstyleIPA{s} &
 &
 &
 &
\centering\arraybslash \textstyleIPA{h}\\
\scshape Nasal &
\centering \textstyleIPA{m} &
\centering \textstyleIPA{n} &
 &
 &
 &
\\
\scshape Glide &
\centering \textstyleIPA{w} &
 &
\centering \textstyleIPA{j} &
 &
 &
\\
\scshape Liquid &
 &
\centering \textstyleIPA{l (r)} &
 &
 &
 &
\\\hline
\end{supertabular}
\end{center}
In proto-AP, *r may have been an allophone of *j since both segments occur in complementary distribution in the reconstructed phonology: *r does not occur in initial position, *j does. However, all of the modern AP languages distinguish /r/ from /l/, and at least two liquids must be reconstructed to proto-Timor-Alor-Pantar, the immediate parent of pAP. If it is the case that Papuan languages usually lack an /r/ \~{} /l/ distinction (Foley 1986), then the languages of Timor Alor Pantar are atypical.

\ \ Within the Alor Pantar family, consonant inventories are largest in Pantar, where Teiwa has 20 consonants, and West Pantar has 16 consonants plus 11 geminates. The inventories decrease in size towards the eastern part of Alor, where Abui has 16 (native) consonants, and Kamang 14. 

\ \ While the consonant inventories of the AP languages are rather similar to each other, some variation is found in the number of fricatives and nasals. In Pantar we find consonants unique to the family: the Western Pantar geminate stops, the Teiwa pharyngeal fricative /[127?]/ the Teiwa uvular stop /q/, the Kaera velar fricative /x/, and the Blagar implosive voiced bilabial stop /{\texthtb}/. The pharyngeal fricative in particular is cross-linguistically rather uncommon (found in 2.4\% of  Maddieson{\textquoteright}s (2005) sample). 

\subsection[5.2 \ \ Constituent order]{5.2 \ \ Constituent order}
Overall, the AP languages are syntactically right-headed (see also chapter 4). Basic transitive clauses are verb-final, with Agent-Patient-Verb (APV) and Subject-Verb (SV) order. A refers to the more agent-like argument of a transitive verb, P to the more patient-like argument of a transitive verb, and S to the single argument of an intransitive verb. () illustrates an intransitive clause followed by a transitive one. PAV is a pragmatically motivated variant in many of the Alor Pantar languages.

Teiwa (Klamer 2010: 25) 

\begin{flushleft}
\tablehead{}
\begin{supertabular}{m{0.43235984in}m{0.40455985in}m{0.32265985in}m{0.32055986in}m{0.46915987in}m{0.35315984in}m{0.7122598in}m{0.32265985in}m{0.32055986in}m{0.41365984in}m{0.44555983in}}
\label{bkm:Ref336875300}() &
 &
S &
 &
V &
V &
V &
A &
 &
P &
V\\
 &
\itshape Qau &
\itshape a &
\itshape ta &
\itshape ewar &
\itshape mis. &
\itshape Mis-an &
\itshape a &
\itshape ta &
\itshape man &
\itshape pi{\textquoteright}i.\\
 &
good &
3\textsc{sg} &
\scshape top &
return &
sit &
sit-\textsc{real} &
3\textsc{sg} &
\scshape top  &
grass &
twine\\
 &
\multicolumn{10}{m{4.7934604in}}{{\textquoteleft}So she sits down again. Sitting, she twines grass.{\textquoteright}}\\
\end{supertabular}
\end{flushleft}
\ \ In adpositional phrases, postpositions follow their complement, as illustrated in section 5.7 below. Clausal negators follow the predicate:  

Kaera (Klamer, to appear b)

\begin{flushleft}
\tablehead{}
\begin{supertabular}{m{0.26505986in}m{0.44205984in}m{0.5156598in}m{0.43235984in}m{0.40945986in}m{0.075459845in}m{0.07475985in}m{0.075459845in}m{0.075459845in}m{0.077559836in}}
(2) &
\itshape Gang &
\itshape masu &
\itshape ma &
\itshape bino. &
 &
 &
 &
 &
\\
 &
\scshape 3sg &
maybe &
come &
\scshape neg &
 &
 &
 &
 &
\\
 &
\multicolumn{9}{m{2.8081603in}}{{\textquoteleft}He may not come.{\textquoteright} }\\
\end{supertabular}
\end{flushleft}
\ \ In nominal phrases, determiners such as articles and demonstratives follow the noun (see Klamer, Schapper and Corbett, \textit{this volume}). All AP languages have clause-final conjunctions; often these are combined with clause-initial ones, as shown in (3), where clause final \textit{a }{\textquoteleft}and{\textquoteright} combines with clause-initial \textit{xabi }{\textquoteleft}then{\textquoteright}: 

Kaera (Klamer, to appear b)

\begin{flushleft}
\tablehead{}
\begin{supertabular}{m{0.26505986in}m{0.44205984in}m{1.0191599in}m{0.32125986in}m{0.34905985in}m{0.31225985in}m{0.075459845in}m{0.075459845in}m{0.077559836in}}
(3) &
\itshape Gang &
\itshape ge-topi &
\itshape gu &
\itshape med &
\itshape a, &
 &
 &
\\
 &
\scshape 3sg &
\textsc{3sg.alien-}hat &
that &
take &
and &
 &
 &
\\
 &
\multicolumn{6}{m{2.91296in}}{{\textquoteleft}He takes that hat of his and } &
 &
\\
\end{supertabular}
\end{flushleft}
\begin{flushleft}
\tablehead{}
\begin{supertabular}{m{0.07475985in}m{0.45115986in}m{0.7233598in}m{0.40595984in}m{0.34975985in}m{0.33655986in}m{0.6205598in}m{0.46155986in}m{0.6115598in}m{0.32195985in}m{0.65735984in}}
 &
\itshape xabi &
\itshape mampelei &
\itshape utug &
\itshape met &
\itshape mi &
\itshape kunang &
\itshape masik  &
\itshape namung &
\itshape gu &
\itshape gi-ng.\\
 &
then &
mango &
three &
take &
\scshape loc &
children &
male &
\scshape pl &
that &
\textsc{3pl}{}-give\\
 &
\multicolumn{10}{m{5.6484604in}}{then takes three mangoes to give to the boys.{\textquoteright}}\\
\end{supertabular}
\end{flushleft}
\subsection[5.3 \ \ Pronominal indexing and morphological alignment]{5.3 \ \ Pronominal indexing and morphological alignment}
The term {\textquoteleft}pronominal indexing{\textquoteright} is used here (and in Feddent et al, \textit{this volume}) to describe a structure where there is a pronominal affix on the verb and a co-referent. Noun Phrase (NP) or free pronoun occur optionally (indicated by brackets) in the same clause, as in (4)). Co-reference is indicated by the index \textit{k}. There is no pronominal indexing in (5). In the Alor-Pantar languages pronominal indices are exclusively prefixal.

(4)\ \ \ \ (NP\textit{\textsubscript{k }}/free pronoun\textit{\textsubscript{k}})\ \ \ \ prefix\textit{\textsubscript{k}}{}-verb

(5)\ \ \ \ (NP/free pronoun)\ \ \ \ verb

The pronominal indices found on the verbs in the Alor-Pantar languages are all very similar in form, pointing to a common historical origin. They are reconstructed for pAP as in Table 4 (see Holton and Robinson, \textit{this volume} a, b; Schapper et al. \textit{this volume}). The initial consonant encodes person, and the theme vowels \textit{a }and \textit{i} encode singular and plural number.  

{\centering
Table 4: Reconstructed pAP pronominal verb prefixes 
\par}

\begin{center}
\tablehead{}
\begin{supertabular}{|m{1.6344599in}|m{0.56365985in}|}
\hline
\scshape 1sg &
*na-\\\hline
\scshape 2sg &
*ha-\\\hline
\scshape 3sg &
*ga-\\\hline
\scshape common/distributive &
*ta-\\\hline
\scshape 1pl.incl &
*pi-\\\hline
\scshape 1pl.excl &
*ni-\ \ \\\hline
\scshape 2pl &
*hi-\\\hline
\scshape 3pl &
*gi-\\\hline
\end{supertabular}
\end{center}
\ \ All AP languages distinguish inclusive from exclusive forms. All the modern AP languages also have reflexes of pAP *ta-, a prefix with a common or impersonal referent (compare {\textquoteleft}one{\textquoteright}\textit{ }in English \textit{One should consider this}), and a reading that is often distributive or reflexive \textit{({\textquoteleft}each one, each other{\textquoteright}}). In table 4, this prefix is grouped with the singular forms because it carries the singular theme vowel \textit{a}.

Going from west to east, we find increasingly complex systems of grammatical relations involving multiple paradigms of pronominal indices. For example, Teiwa (Pantar) has one paradigm of object prefixes (which is almost identical to the pAP paradigm in Table 5), Klon in West Alor has three paradigms (Table x), and Abui (Central Alor) has five (Table 6). Prefixes with theme vowels \textit{e} and \textit{o }reflect the pAP genitive and locative morphemes.

{\centering
Table 5: Klon prefixes (Baird 2008: 69, 39)
\par}

\begin{center}
\tablehead{}
\begin{supertabular}{m{0.6087598in}m{0.5344598in}m{0.5344598in}m{0.5351598in}}
\hline
 &
I  &
II  &
III\\\hline
1\textsc{sg} &
\itshape n- &
\itshape ne- &
\itshape no-\\
2\textsc{sg} &
\itshape V-/ {\O}- &
\itshape e- &
\itshape o-\\
3 &
\itshape g-  &
\itshape ge- &
\itshape go-\\
\scshape 1pl.ex &
\itshape ng- &
\itshape nge- &
\itshape ngo-\\
\scshape 1pl.in &
\itshape t- &
\itshape te- &
\itshape to-\\
\scshape 2pl &
\itshape i- &
\itshape ege- &
\itshape ogo-\\
\scshape recipr &
\itshape t- &
\itshape te- &
\itshape to-\\\hline
\end{supertabular}
\end{center}
{\centering
Table 6: Abui prefixes (Kratochv\'il 2007: 78, 2011:591)
\par}

\begin{center}
\tablehead{}
\begin{supertabular}{m{0.5254598in}m{0.29275984in}m{0.28515986in}m{0.29275984in}m{0.35805985in}m{0.37755984in}}
\hhline{~-----}
 &
I  &
II &
III &
IV &
V\\\hhline{~-----}
1\textsc{sg} &
\itshape na- &
\itshape ne- &
\itshape no- &
\itshape nee- &
\itshape noo-\\
2\textsc{sg} &
\itshape a- &
\itshape e- &
\itshape o- &
\itshape ee- &
\itshape oo-\\
3 &
\itshape ha- &
\itshape he- &
\itshape ho- &
\itshape hee- &
\itshape hoo-\\
\scshape 1pl.ex &
\itshape ni- &
\itshape ni- &
\itshape nu- &
\itshape nii- &
\itshape nuu-\\
\scshape 1pl.in &
\itshape pi- &
\itshape pi- &
\itshape pu- &
\itshape pii- &
\itshape puu-\\
\scshape 2pl &
\itshape ri- &
\itshape ri- &
\itshape ru- &
\itshape rii- &
\itshape ruu-\\
\scshape distr &
\itshape ta- &
\itshape te- &
\itshape to- &
\itshape tee- &
\itshape too-\\\hhline{~-----}
\end{supertabular}
\end{center}
\ \ In AP languages, the use of these different pronominal sets is not so much determined by the grammatical role of their referent (e.g. being an object), but is mostly triggered by semantic factors. Most AP languages index P on the verb, and not A, as in (6). A and S are typically expressed as free lexical NPs or pronouns. Cross-linguistically, this is an uncommon pattern; it occurs in only 7\% of Siewierska{\textquoteright}s (2005) sample.

Teiwa (Klamer, fieldnotes)

\begin{flushleft}
\tablehead{}
\begin{supertabular}{m{0.26565984in}m{0.32265985in}m{0.48795983in}m{0.5552598in}m{0.075459845in}m{0.075459845in}m{0.075459845in}m{0.075459845in}m{0.07615984in}m{0.07615984in}}
(6) &
\itshape Na &
\itshape Maria &
\itshape g-ua{\textquoteright} &
 &
 &
 &
 &
 &
\\
 &
1\textsc{sg} &
M. &
\textsc{3sg-}hit &
 &
 &
 &
 &
 &
\\
 &
\multicolumn{9}{m{2.44996in}}{{\textquoteleft}I hit Maria.{\textquoteright}}\\
\end{supertabular}
\end{flushleft}
\ \ One of the factors determining the indexing of P is animacy. For instance, when the P of the Teiwa verb \textit{mar }{\textquoteleft}take{\textquoteright} is inanimate, it is not indexed on the verb, (7a), but when it  is animate, it is indexed (7b). That is, while a verbal prefix in a AP language typically indexes P, not every P is always indexed on a verb.

Teiwa ( Klamer 2010:91)

\begin{flushleft}
\tablehead{}
\begin{supertabular}{m{0.43235984in}m{0.19625986in}m{0.32335985in}m{0.7400598in}m{0.38095984in}m{0.085159846in}m{0.085159846in}m{0.08445985in}m{0.085159846in}m{0.085159846in}m{0.08445985in}m{0.085159846in}m{0.085159846in}m{0.085159846in}m{0.08445985in}m{0.085159846in}m{0.085159846in}m{0.08725984in}}
(7) &
a. &
\itshape Na &
\itshape ga{\textquoteright}an &
\itshape mar. &
 &
 &
 &
 &
 &
 &
 &
 &
 &
 &
 &
 &
\\
 &
 &
1\textsc{sg} &
3\textsc{sg} &
take &
 &
 &
 &
 &
 &
 &
 &
 &
 &
 &
 &
 &
\\
 &
 &
\multicolumn{16}{m{3.7325587in}}{{\textquoteleft}I take / get it{\textquoteright}}\\
 &
 &
 &
 &
 &
 &
 &
 &
 &
 &
 &
 &
 &
 &
 &
 &
 &
\\
 &
b. &
\itshape Na &
\itshape ga-mar. &
 &
 &
 &
 &
 &
 &
 &
 &
 &
 &
 &
 &
 &
\\
 &
 &
1\textsc{sg} &
3\textsc{sg}{}-take &
 &
 &
 &
 &
 &
 &
 &
 &
 &
 &
 &
 &
 &
\\
 &
 &
\multicolumn{16}{m{3.7325587in}}{{\textquoteleft}I follow him/her{\textquoteright}}\\
\end{supertabular}
\end{flushleft}
In Abui, the different prefixes roughly correspond to semantically different P{\textquoteright}s. For example, in (8)-(12) the P is a patient, location, recipient, benefactive and goal, and the shape of the prefix varies accordingly.

Abui (Kratochv\'il 2007: 592) 

(8)\ \ \textit{Na \ \ a-ruidi}.\ \ \ \ \ \ \ \ \ \  

\textsc{1sg \ \ 2sg}{}-wake up

{\textquoteleft}I woke you up{\textquoteright}\ \ 

(9)\ \ \textit{Di \ \ palootang \ \ mi \ \ ne-l bol.}

3 \ \ rattan \ \ \ \ take \ \ \textsc{1sg-give} hit

{\textquoteleft}He hit me with a rattan (stick){\textquoteright}

(10)\ \ \textit{Fanmalei \ \ no-k \ \ \ \ yai.}

F. \ \ \ \ \textsc{1sg-throw} \ \ laugh

{\textquoteleft}Fanmalei  laughed at me{\textquoteright}

(11)\ \ \textit{Ma \ \ \ \ ne \ \ ee-bol.}

be.\textsc{prox \ \ 1sg \ \ 2sg}{}-hit

{\textquoteleft}Let me hit instead of (i.e. for) you{\textquoteright}

(12)\ \ \textit{Simon \ \ di \ \ noo-dik.}

S. \ \ 3 \ \ \textsc{1sg}{}-prick

{\textquoteleft}Simon is poking me{\textquoteright}

In some AP languages (for instance, Abui, Kamang, Klon) S arguments may be indexed on the verbs. Such arguments are usually more affected and less volitional, although individual languages differ in which semantic factors apply (Fedden et al 2013, \textit{this volume}). Also, lexical verb classes often play a role in the indexing of arguments. 

\ \ Apart from the multiple ways to index P, there is also variation in the morphological alignment type of AP languages. Alignment in the AP languages is defined here relative to pronominal indexing. 

\ \ The prefixes are used in a syntactic (accusative) alignment system, in a semantic alignment system, or in a {\textquoteleft}Split-S{\textquoteright} system. Accusative alignment is defined here as the alignment where S and A are treated alike as opposed to P. Teiwa, Kaera, Blagar and Adang have accusative alignment, only indexing P, while S and A are free forms. An illustration is Blagar, where the same pronoun \textit{{\textglotstop}}\textit{ana }{\textquoteleft}3\textsc{sg}{\textquoteright} can encode A (13) or or S (14), and P is prefixed on the verb (13).

Blagar (Steinhauer to appear) 

(13)\ \ \textit{{\textglotstop}}\textit{ana \ \ uruhi{\ng}\ \ aru }\textit{{\textglotstop}}\textit{{}-atapa-t \ \ \ \ \ \ imina}

\ \ 3\textsc{sg}\ \ deer \ \ two 3-shoot.with.arrow-\textit{t}\ \ die

\ \ {\textquoteleft}S/he killed two deer with bow and arrow.{\textquoteright}

(14)\ \  \textit{{\textglotstop}}\textit{ana\ \ mi \ \ bihi}

\ \ 3\textsc{sg}\ \ in\ \ run

\ \ {\textquoteleft}He/she/it runs inside.{\textquoteright}

In Klon, however, the P prefix can also be used to index S, depending on the class the verb belongs to: one class of verbs always aligns S with A (resulting in accusative alignment), another class always aligns S with P, and a third class of verbs encodes S either as A (free pronoun) or as P (prefix), depending on its affectedness: compare (15a) and (15b).

Klon (Baird 2008: 8) 

\begin{flushleft}
\tablehead{}
\begin{supertabular}{m{0.47335985in}m{0.19625986in}m{0.8650598in}m{0.40455985in}m{0.075459845in}m{0.075459845in}m{0.075459845in}m{0.07475985in}m{0.075459845in}m{0.075459845in}m{0.075459845in}m{0.075459845in}m{0.075459845in}m{0.075459845in}m{0.075459845in}m{0.075459845in}m{0.075459845in}m{0.07955984in}}
(15) &
a. &
\itshape A &
\itshape kaak &
 &
 &
 &
 &
 &
 &
 &
 &
 &
 &
 &
 &
 &
\\
 &
 &
\scshape 2sg &
itchy &
 &
 &
 &
 &
 &
 &
 &
 &
 &
 &
 &
 &
 &
\\
 &
 &
\multicolumn{16}{m{3.5105603in}}{{\textquoteleft}You{\textquoteright}re itchy{\textquoteright}}\\
 &
 &
\multicolumn{16}{m{3.5105603in}}{}\\
 &
b. &
\itshape E-kaak &
 &
 &
 &
 &
 &
 &
 &
 &
 &
 &
 &
 &
 &
 &
\\
 &
 &
\textsc{2sg.II}{}-itchy &
 &
 &
 &
 &
 &
 &
 &
 &
 &
 &
 &
 &
 &
 &
\\
 &
 &
\multicolumn{16}{m{3.5105603in}}{{\textquoteleft}You{\textquoteright}re itchy (and affected){\textquoteright} }\\
\end{supertabular}
\end{flushleft}
Among the Pantar languages, only Western Pantar allows its prefix that is typically reserved to index P, as in (16), to also index S, as in (17).  Some verbs, such as \textit{diti} {\textquoteleft}stab{\textquoteright} in (18)-(19) allow an alternation in the coding of a O or S with either a prefix or a free pronoun, with a difference in the degree of affectedness resulting. 

Western Pantar (Holton 2010a: 105-106) 

(16)  \ \ \textit{Gang\ \ na-niaka.}\ \ 

\ \ 3\textsc{sg}\ \ \textsc{1sg}{}-see\ \ \ \ \ \ 

\ \ {\textquoteleft}S/he saw me{\textquoteright} 

(17)  \ \ \textit{Nang \ \ na-lama \ \ ta.}

\ \ 1\textsc{sg} \ \ \textsc{1sg}{}-descend \ \ IPFV

\ \ {\textquoteleft}I{\textquoteright}m going{\textquoteright}

(18)  \ \ \textit{Nang\ \ \ \ ga-diti.\ \ }\ \ 

1\textsc{sg}\ \ \ \ 3\textsc{sg}{}-stab\ \ 

\ \ {\textquoteleft}I stabbed him{\textquoteright} (superficially)\ \ \ \ 

(19)  \ \ \textit{Nang\ \ gaing \ \ diti.}

\ \ 1\textsc{sg} \ \ 3\textsc{sg}\ \ stab

{\textquoteleft} I stabbed him{\textquoteright} (severely)

Abui and Kamang are often found to index S by use of a prefix. The choice of prefix is determined by a mix of factors, such as the level of affectedness or volitionality of the argument (Fedden et al. 2013, Fedden et al., \textit{this volume}). 

A pattern where two arguments are indexed on a transitive verb is found in Abui, (20)-(21). Unlike what would be expected, these are not transitive constructions expressing actions involving an affix for A and for P, but rather experience constructions where both affixes encode a P. 

Abui (Kratochv\'il 2011: 615)

(20)\ \ \textit{Sieng ma \ \ he-noo-maran-i}

\ \ rice cooked \ \ 3.III-1\textsc{sg}.V-come.up.\textsc{compl-pfv}

\ \ {\textquoteleft}I am satiated with the rice{\textquoteright}

Abui (Kratochv\'il 2011: 617)

(21)\ \ \textit{He\ \ n hee-na-minang} 

\ \ that \ \ 3.IV-1\textsc{sg}.I-remember

{\textquoteleft}I remembered that{\textquoteright}

In sum, free pronouns exist alongside verbal affixes that index person and number of verbal arguments. There is significant variation in the choice of participant that is indexed on the verb. The Alor-Pantar languages are typologically unusual in that they index P but not A, and some of them have rich inventories of prefixes differentiating different types of P. 

\subsection[5.4 \ \ Possession]{5.4 \ \ Possession}
Possession is marked by prefixes on nouns. There are parallels with the argument indexing on verbs, particularly because inalienable possession usually involves possessors linearly preceding the possessed noun in the same way that arguments linearly precede the verb. 

\ \ All AP languages have preposed possessors, and all of them distinguish alienable from inalienable possession. For example, Abui employs distinct possessive prefixes for alienable and inalienable possession, (22). 

Abui (Kratochv\'il 2007) 

\begin{flushleft}
\tablehead{}
\begin{supertabular}{m{0.39135984in}m{0.40385985in}m{1.1177598in}m{0.7900598in}m{0.32265985in}m{1.0656599in}m{0.5184598in}m{1.3406599in}}
(22) &
a. &
\itshape na-min &
 &
b.  &
\itshape ne-fala &
 &
\\
 &
 &
\multicolumn{2}{m{1.9865599in}}{1\textsc{SG}.INAL-nose } &
 &
\multicolumn{2}{m{1.6628599in}}{1\textsc{SG}.AL-house } &
\\
 &
 &
{\textquoteleft}my nose{\textquoteright} &
 &
 &
{\textquoteleft}my house{\textquoteright}  &
 &
\\
\end{supertabular}
\end{flushleft}
When the alienable-inalienable distinction is encoded by prefix choice, then the inalienable has the theme vowel \textit{a}, while the alienable is expressed by a prefix with the theme vowel \textit{e, }as in Abui.  Prefixes with the vowel \textit{a }reflect the proto-AP P-indexing morpheme (Table 4) while prefixes with the vowel \textit{e }reflect the pAP genitive prefix (cf. the prefixes with theme vowels \textit{e} in Klon (Table 5) and Abui (Table 6). 

In Teiwa, alienable and inalienable possession is distinguished by optional versus obligatory use of the same (\textit{a-}vowel) prefix, (23):

Teiwa (Klamer 2010:192) 

\begin{flushleft}
\tablehead{}
\begin{supertabular}{m{0.38935986in}m{0.38445985in}m{1.0740598in}m{1.0865599in}m{0.39345986in}m{1.0018599in}m{1.1281599in}}
(23) &
a. &
\itshape na-yaf &
 &
b.  &
\itshape yaf &
\\
 &
 &
\multicolumn{2}{m{2.2393599in}}{\textsc{1sg.poss-}house} &
 &
house &
\\
 &
 &
\multicolumn{2}{m{2.2393599in}}{{\textquoteleft}my house{\textquoteright}} &
 &
\multicolumn{2}{m{2.2087598in}}{{\textquoteleft}(a) house, houses{\textquoteright}}\\
 &
c. &
na-tan &
 &
d.  * &
tan &
\\
 &
 &
\textsc{1sg.poss-}hand &
 &
 &
hand &
\\
 &
 &
\multicolumn{2}{m{2.2393599in}}{{\textquoteleft}my hand{\textquoteright}} &
 &
\multicolumn{2}{m{2.2087598in}}{not good for {\textquoteleft}(a) hand, hands{\textquoteright}}\\
\end{supertabular}
\end{flushleft}
The variation in the treatment of the alienable-inalienable distinction across the AP family is summarized in Table 7. (Different subscripts indicate different paradigms in a single language.) 

{\centering
Table 7: Encoding of alienable and inalienable possessors in some AP languages. 
\par}

\begin{center}
\tablehead{}
\begin{supertabular}{m{1.0608599in}m{1.1775599in}m{1.4490598in}m{1.5636599in}}
\hline
Location &
Language name &
Alienable possessor &
Inalienable possessor\\\hline
Pantar &
Western Pantar &
free form &
prefix \\
 &
Teiwa &
optional prefix\textsubscript{a}  &
obligatory prefix\textsubscript{a} \\
 &
Kaera  &
prefix\textsubscript{a}; alienable &
prefix\textsubscript{b}; inalienable\\
Pantar Straits &
Blagar &
free form &
prefix\\
Alor &
Klon &
free form &
prefix\textsubscript{a} , prefix\textsubscript{b} \\
 &
Abui &
prefix\textsubscript{a}: alienable &
prefix\textsubscript{b} ; inalienable\\
 &
Kamang &
prefix\textsubscript{a}: alienable &
prefix\textsubscript{b} ; inalienable\\\hline
\end{supertabular}
\end{center}
\subsection[5.5 \ \ Plural number words]{5.5 \ \ Plural number words}
The Alor-Pantar languages exhibit a typologically unusual pattern (Dryer 2005) whereby nominal plurality is indicated via a separate number word; {\textquoteleft}A morpheme whose meaning and function is similar to that of plural affixes in other languages{\textquoteright} (Dryer 1989). An illustration is Teiwa \textit{non }in (24b). 

Teiwa (Klamer, Teiwa corpus) 

\begin{flushleft}
\tablehead{}
\begin{supertabular}{m{0.39135984in}m{0.39135984in}m{0.5254598in}m{0.5615598in}m{0.41845986in}m{0.67815983in}m{0.5483598in}m{0.42195985in}m{0.35255986in}m{0.35315984in}m{0.35665986in}}
(24) &
a. &
\itshape Qavif &
\textit{ita}\textit{[241?]}\textit{a } &
\itshape ma &
\itshape gi? &
 &
 &
 &
 &
\\
 &
 &
goat &
where &
\scshape obl &
go &
 &
 &
 &
 &
\\
 &
 &
\multicolumn{9}{m{4.84626in}}{{\textquoteleft}Where did the goat(s) go?{\textquoteright}}\\
 &
 &
\multicolumn{9}{m{4.84626in}}{}\\
\end{supertabular}
\end{flushleft}
\begin{flushleft}
\tablehead{}
\begin{supertabular}{m{0.39275986in}m{0.39275986in}m{0.5261598in}m{0.40595984in}m{0.56365985in}m{0.6802598in}m{0.5497598in}m{0.42265984in}m{0.35385984in}m{0.35315984in}m{0.35805985in}}
 &
b. &
\itshape Qavif &
\itshape non &
\textit{ita}\textit{[241?]}\textit{a } &
\itshape ma &
\itshape gi? &
 &
 &
 &
\\
 &
 &
goat &
\scshape pl &
where &
\scshape obl &
Go &
 &
 &
 &
\\
 &
 &
\multicolumn{9}{m{4.8434596in}}{{\textquoteleft}Where did the (several) goats go?{\textquoteright};  *{\textquoteleft}Where did the goat go?{\textquoteright} ~}\\
\end{supertabular}
\end{flushleft}
Plural number words are found across Alor-Pantar and the cognates found across the family suggest that pAP had a plural number word *non. Across the AP family, there is significant variation in form, syntax and semantics of the plural number word as described in Klamer et al. (\textit{this volume}). 

. . 

\subsection[5.6 \ \ Serial verb constructions]{5.6 \ \ Serial verb constructions}
Serial verb constructions (SVCs) are analysed here as two or more verbs that occur together in a single clause, which share minimally one argument, and whose shared argument(s) is (are) expressed maximally once. SVCs are distinguished from bi-clausal constructions by the presence of a clause boundary marker in between the clauses in the latter (a conjunction-like element, an intonational break, or a pause). The verbs in a SVC share aspect marking and occur under a single intonation contour without a boundary marker. 

The semantic contrast between a mono-clausal construction with an SVC and a biclausal construction is illustrated by the minimally contrasting pair of Teiwa sentences in (25). Monoclausal (25a) expresses through an SVC the intransitive event of someone who died because he fell down (e.g. from a coconut tree). The biclausal construction in (20b) describes two events in clauses that are linked by the conjunction \textit{ba}: someone is dying (e.g. because of a heart attack) and is falling down (e.g. out of a tree) as a result of this. No such conjunction-like element would occur between the verbs constituting an SVC.

Teiwa (Klamer 2010) 

\begin{flushleft}
\tablehead{}
\begin{supertabular}{m{0.28235984in}m{-0.012740158in}m{0.19625986in}m{-0.05864016in}m{0.19975984in}m{0.0045598447in}m{0.28235984in}m{-0.05864016in}m{0.26295984in}m{0.28235984in}m{0.28305984in}m{-0.016940158in}m{0.22055984in}m{0.28305984in}m{0.28515986in}m{0.075459845in}m{0.075459845in}m{-0.028740156in}m{0.02545984in}m{0.075459845in}m{0.024759844in}m{-0.028040156in}m{0.075459845in}m{0.07815985in}}
\multicolumn{2}{m{0.34835985in}}{(25)} &
a. &
\multicolumn{2}{m{0.21985984in}}{A} &
\multicolumn{3}{m{0.38575983in}}{ta} &
\multicolumn{4}{m{1.0476599in}}{min-an} &
\multicolumn{3}{m{0.9462598in}}{ba{\textquoteright}.} &
 &
 &
\multicolumn{2}{m{0.075459845in}}{} &
 &
\multicolumn{2}{m{0.075459845in}}{} &
 &
\\
\multicolumn{2}{m{0.34835985in}}{} &
 &
\multicolumn{2}{m{0.21985984in}}{3s} &
\multicolumn{3}{m{0.38575983in}}{TOP} &
\multicolumn{4}{m{1.0476599in}}{die-REAL} &
\multicolumn{3}{m{0.9462598in}}{fall.down} &
 &
 &
\multicolumn{2}{m{0.075459845in}}{} &
 &
\multicolumn{2}{m{0.075459845in}}{} &
 &
\\
\multicolumn{2}{m{0.34835985in}}{} &
 &
\multicolumn{21}{m{3.9178596in}}{{\textquoteleft}He died falling down.{\textquoteright}  }\\
\multicolumn{2}{m{0.34835985in}}{} &
 &
\multicolumn{21}{m{3.9178596in}}{}\\
 &
\multicolumn{3}{m{0.28235987in}}{b.} &
\multicolumn{2}{m{0.28305984in}}{A} &
ta &
\multicolumn{2}{m{0.28305984in}}{min-an} &
ba &
ba{\textquoteright}. &
\multicolumn{2}{m{0.28235984in}}{} &
 &
 &
\multicolumn{3}{m{0.27965987in}}{} &
\multicolumn{3}{m{0.28315985in}}{} &
\multicolumn{3}{m{0.28305987in}}{}\\
 &
\multicolumn{3}{m{0.28235987in}}{} &
\multicolumn{2}{m{0.28305984in}}{3s} &
TOP &
\multicolumn{2}{m{0.28305984in}}{die- REAL} &
\scshape conj &
fall.down &
\multicolumn{2}{m{0.28235984in}}{} &
 &
 &
\multicolumn{3}{m{0.27965987in}}{} &
\multicolumn{3}{m{0.28315985in}}{} &
\multicolumn{3}{m{0.28305987in}}{}\\
 &
\multicolumn{3}{m{0.28235987in}}{} &
\multicolumn{20}{m{3.8977597in}}{{\textquoteleft}He died then fell down.{\textquoteright} }\\
\end{supertabular}
\end{flushleft}
SVCs are frequently attested in all AP languages, and they express a wide range of notions, including direction (25a), manner (26), and aspect (27). 

Western Pantar (Holton, to appear b) 

\begin{flushleft}
\tablehead{}
\begin{supertabular}{m{0.43025985in}m{0.45875987in}m{0.9684598in}m{0.9684598in}m{0.9684598in}m{0.9684598in}m{0.9691598in}}
\multicolumn{2}{m{0.9677598in}}{(26)} &
Habbang &
mau &
aname &
horang &
sauke-yabe\\
\multicolumn{2}{m{0.9677598in}}{} &
village &
there &
person &
make.noise &
dance.lego-lego\\
 &
\multicolumn{6}{m{5.6954594in}}{{\textquoteleft}Over there in the village people are making noise dancing lego-lego{\textquoteright}}\\
\end{supertabular}
\end{flushleft}
Teiwa (Klamer 2010: 358) 

\begin{flushleft}
\tablehead{}
\begin{supertabular}{m{0.43235984in}m{0.22195986in}m{0.7351598in}m{0.28095984in}m{0.64905983in}m{0.6497598in}m{0.8525598in}m{0.54135984in}}
(27)   &
A  &
bir-an &
gi &
awan &
awan &
tas-an &
gula{\textquoteright}...\\
 &
3s &
run- real  &
go &
far.away &
far.away &
stand- real &
finish\\
 &
\multicolumn{7}{m{4.4032593in}}{{\textquoteleft}She ran far away [and] stood [still]...{\textquoteright}  }\\
\end{supertabular}
\end{flushleft}
\ \ SVCs in AP languages also serve to introduce event participants, for example in clauses that express a {\textquoteleft}give{\textquoteright}-event. This is due to the fact that the AP languages lack a class of simple ditransitive root verbs. {\textquoteleft}Give{\textquoteright} events involving three participants (actor, recipient, and theme) are typically expressed by means of biclausal or serial verb constructions involving the monotransitive verbs {\textquoteleft}take{\textquoteright} and {\textquoteleft}give{\textquoteright}. {\textquoteleft}Take{\textquoteright} introduces the theme, {\textquoteleft}give{\textquoteright} the recipient, and the clausal sequence or serial verb construction in which the verbs appear is then [actor [theme {\textquoteleft}take{\textquoteright}] [recipient {\textquoteleft}give{\textquoteright}]]. In some of the AP languages (e.g. Kamang) the verb {\textquoteleft}take{\textquoteright} has been semantically bleached and syntactically reduced to a light verb or a 
postposition-like element which encodes oblique constituents.

\ \ The AP {\textquoteleft}give{\textquoteright} constructions are illustrated by the Abui sentences in (28)-(29). In the biclausal construction in (28), the theme is expressed in the first clause as a complement of \textit{mi} {\textquoteleft}take{\textquoteright}, while the recipient is found in the second clause, as a complement of the verb \textit{{}-l / -r} {\textquoteleft}give{\textquoteright} (the consonant alternation encodes an aspectual distinction). In (29), the {\textquoteleft}give{\textquoteright} construction is monoclausal: the NP encoding the recipient \textit{nei yo }{\textquoteleft}mine{\textquoteright} is fronted to a position preceding both {\textquoteleft}give{\textquoteright} and {\textquoteleft}take{\textquoteright}. This would not be possible in the biclausal structure of (28).

Abui (Kratochv\'il, Abui corpus; cited in Klamer and Schapper 2012)

\begin{flushleft}
\tablehead{}
\begin{supertabular}{m{0.34835985in}m{0.34905985in}m{0.34905985in}m{0.40185985in}m{0.36775985in}m{0.40805984in}m{0.8406598in}}
(28)  &
\itshape Hen &
\itshape mi &
\itshape ba &
\itshape Lius &
\itshape la  &
\itshape he-l-e.\\
 &
3 &
take &
\scshape conj &
Lius &
\scshape part &
3-give-\textsc{ipfv}\\
 &
\multicolumn{6}{m{3.1101599in}}{{\textquoteleft}Just give that one to Lius.{\textquoteright} }\\
\end{supertabular}
\end{flushleft}
Abui (Kratochv\'il, Abui corpus; cited in Klamer and Schapper 2012)

\begin{flushleft}
\tablehead{}
\begin{supertabular}{m{0.34835985in}m{0.6809598in}m{0.36505985in}m{0.40805984in}m{0.34905985in}m{0.66565984in}m{0.34005985in}m{0.36505985in}m{0.075459845in}m{0.07815985in}}
(29) &
\itshape Nei &
\itshape yo &
\itshape la &
\itshape mi &
\itshape ne-r &
\itshape te &
\itshape ya! &
 &
\\
 &
\scshape 1sg.poss &
\scshape dem &
\scshape part &
take &
1\textsc{sg}{}-give &
first &
\scshape dem &
 &
\\
 &
\multicolumn{9}{m{3.95746in}}{{\textquoteleft}Just give me mine!{\textquoteright} }\\
\end{supertabular}
\end{flushleft}
\ \ The single argument indexed on the verb {\textquoteleft}give{\textquoteright} is the recipient. The AP languages thus exhibit {\textquoteleft}secundative{\textquoteright} alignment (Malchukov et al. 2010, Klamer and Schapper 2012).

\subsection[5.7 Adpositions]{5.7 Adpositions}
Adpositions in AP languages follow their complement. Many AP languages have adpositions encoding locations that are similar in form to (light) locative verbs, suggesting a historical relation between items in both these word classes. For example, the Kaera postpositions  \textit{mi }{\textquoteleft}in{\textquoteright}, {\textquoteleft}on{\textquoteright}, {\textquoteleft}at{\textquoteright}, {\textquoteleft}into{\textquoteright} (glossed as {\textquoteleft}\textsc{loc{\textquoteright}) } is related to the locative verbs \textit{ming }{\textquoteleft}be at{\textquoteright}, (30b), while \textit{ta }{\textquoteleft}on{\textquoteright} is related to the locative verb \textit{tang }{\textquoteleft}be on{\textquoteright}. 

Kaera (Klamer, to appear b)

\begin{flushleft}
\tablehead{}
\begin{supertabular}{m{0.34835985in}m{0.18725985in}m{0.40455985in}m{0.54345983in}m{0.33445984in}m{0.49695984in}m{0.075459845in}m{0.075459845in}m{0.075459845in}m{0.075459845in}m{0.077559836in}}
(30) &
a. &
\itshape Ging &
[\textit{abang} &
\textit{mi}] &
\itshape mis-o. &
 &
 &
 &
 &
\\
 &
 &
\scshape 3pl &
village &
\scshape loc &
sit-\textsc{fin} &
 &
 &
 &
 &
\\
 &
 &
\multicolumn{9}{m{2.7887602in}}{{\textquoteleft}They stay in the village.{\textquoteright} }\\
\end{supertabular}
\end{flushleft}
\begin{flushleft}
\tablehead{}
\begin{supertabular}{m{0.38935986in}m{0.21635985in}m{0.44555983in}m{0.53095984in}m{0.40805984in}m{0.32475984in}m{1.6400598in}}
 &
b. &
\itshape Ging &
\itshape abang  &
\itshape ming &
\itshape gu,  &
\itshape mis-o.\\
 &
 &
\scshape 3pl &
village &
be.at &
that &
sit-\textsc{fin}\\
 &
 &
\multicolumn{5}{m{3.66436in}}{{\textquoteleft}Those [that] are in the village, [will] stay [there].{\textquoteright}}\\
\end{supertabular}
\end{flushleft}
In Adang too, postpositions share properties with verbs. In (31)-(32), \textit{mi} {\textquoteleft}be in, at{\textquoteright} and \textit{ta} {\textquoteleft}be on (top of){\textquoteright} function as verbs in serial verb constructions. 

Adang (Robinson and Haan, to appear) 

\begin{flushleft}
\tablehead{}
\begin{supertabular}{m{0.43235984in}m{0.40525985in}m{0.6726598in}m{0.31985986in}m{0.5254598in}m{0.35385984in}m{0.71155983in}m{0.32335985in}m{0.31985986in}m{0.41435984in}m{0.44485983in}}
(31) &
\itshape Na &
\itshape [241?]arabah &
\itshape mi &
\itshape mih. &
 &
 &
 &
 &
 &
\\
 &
\scshape 1sg &
Kalabahi &
in  &
sit/live &
 &
 &
 &
 &
 &
\\
 &
\multicolumn{10}{m{5.1997604in}}{{\textquoteleft}I live in Kalabahi.{\textquoteright} }\\
\end{supertabular}
\end{flushleft}
Adang (Robinson and Haan, to appear) 

\begin{flushleft}
\tablehead{}
\begin{supertabular}{m{0.43235984in}m{0.40455985in}m{0.6726598in}m{0.31985986in}m{0.5254598in}m{0.35385984in}m{0.7122598in}m{0.42335984in}m{0.31985986in}m{0.41365984in}m{0.44555983in}}
(32) &
\itshape {\textepsilon}i &
\textit{mat{\textepsilon}}\textit{ } &
\itshape nu &
\itshape tang &
\itshape ta &
\textit{lam}\textit{{\textepsilon}} &
\itshape eh. &
 &
 &
\\
 &
boat &
be.large &
one &
sea &
on &
walk &
\scshape prog &
 &
 &
\\
 &
\multicolumn{10}{m{5.2997603in}}{{\textquoteleft}A large boat is travelling on the sea.{\textquoteright}}\\
\end{supertabular}
\end{flushleft}
There are also AP languages that lack adpositions altogether, Teiwa being a case in point (Klamer 2010). 

\subsection[5.8 Morphological typology ]{5.8 Morphological typology }
Nominal morphology in AP languages is sparse. Nominal inflection is typically limited to possessive prefixing, and the nominal derivation most frequently attested is compounding. Morphologically, verbs are the most complex word class the AP languages. Prefixation to index arguments on verbs is very common (section 5.3). Broadly speaking, the languages of of Pantar are less agglutinative than those of Central and East Alor. For example, while Teiwa (Pantar) has only one person prefix paradigm, Kamang (Central Alor) has six person prefix paradigms, compare Table  8 and Table 9 (Fedden et al., \textit{this volume}). 

{\centering
Table 8: Teiwa person prefixes (Klamer 2010: 77, 78)
\par}

\begin{center}
\tablehead{}
\begin{supertabular}{|m{0.90875983in}|m{0.70945984in}|}
\hline
 &
Prefix\\\hline
\scshape 1sg &
\itshape n(a)-\\\hline
\scshape 2sg &
\itshape h(a)-\\\hline
\scshape 3sg &
\itshape g(a)-\\\hline
\scshape 1pl.excl &
\itshape n(i)-\\\hline
\scshape 1pl.incl &
\itshape p(i)-\\\hline
\scshape 2pl &
\itshape y(i)-\\\hline
\scshape 3pl &
\itshape g(i)-, ga-\\\hline
\end{supertabular}
\end{center}
{\centering
Table 9: Kamang person prefixes (Schapper, to appear)
\par}

\begin{center}
\tablehead{}
\begin{supertabular}{|m{1.7136599in}|m{0.20045984in}m{0.45665982in}|m{0.8191598in}|m{0.8018598in}|m{0.8525598in}|m{0.88865983in}|}
\hhline{~-~~~~~}
\multicolumn{1}{m{1.7136599in}|}{} &
\multicolumn{1}{m{0.20045984in}|}{\centering Prefixes} &
\multicolumn{5}{m{4.1338596in}}{}\\\hline
 &
\multicolumn{2}{m{0.7358598in}|}{\scshape pat} &
\scshape loc &
\scshape gen &
\textsc{ast}\footnotemark{} &
\scshape dat\\\hline
\scshape 1sg &
\multicolumn{2}{m{0.7358598in}|}{\itshape na-} &
\itshape no- &
\itshape ne- &
\itshape noo- &
\itshape nee-\\\hline
\scshape 2sg &
\multicolumn{2}{m{0.7358598in}|}{\itshape a-} &
\itshape o- &
\itshape e- &
\itshape oo- &
\itshape ee-\\\hline
3 &
\multicolumn{2}{m{0.7358598in}|}{\itshape ga-} &
\itshape wo- &
\itshape ge- &
\itshape woo- &
\itshape gee-\\\hline
\scshape 1pl.excl &
\multicolumn{2}{m{0.7358598in}|}{\itshape ni-} &
\itshape nio- &
\itshape ni- &
\itshape nioo- &
\itshape nii-\\\hline
\scshape 1pl.incl &
\multicolumn{2}{m{0.7358598in}|}{\itshape si-} &
\itshape sio- &
\itshape si- &
\itshape sioo- &
\itshape sii-\\\hline
\scshape 2pl &
\multicolumn{2}{m{0.7358598in}|}{\itshape i-} &
\itshape io- &
\itshape i- &
\itshape ioo- &
\itshape ii-\\\hline
\end{supertabular}
\end{center}
\footnotetext{The assistive (\textsc{ast}) refers to the participant who assists in the action.}
\ \ AP languages do not commonly have much derivational morphology. Some AP languages have verbal prefixes that increase valency, including a causative and/or an applicative (e.g. Blagar, Adang, Klon); but in other languages such derivations are either unproductive (Teiwa), or absent altogether (Western Pantar). In the absence of verbal derivation, serial verb constructions are often employed to introduce applicative or instrumental participants, or to express analytical causatives (Abui). Indeed, there is evidence that certain verbal affixes may have developed out of verbs that were originally part of serial verb constructions: in Sawila, for instance, applicative prefixes found on verbs are grammaticalized forms of verbs. An illustration is the applicative prefix \textstyleVernacularword{ma- } in (33b), which is related to the locative verb \textit{ma }{\textquoteleft}be.\textstyleGloss{prox}{\textquoteright} (Kratochv\'il, to appear). 

Sawila (Kratochv\'il, to appear).

\begin{flushleft}
\tablehead{}
\begin{supertabular}{m{0.39415985in}m{0.21225986in}m{1.3198599in}m{0.6719598in}m{0.26435986in}}
(33) &
a.  &
\itshape Laampuru &
\itshape ilo. &
\\
 &
 &
lamp &
be.bright &
\\
 &
 &
\multicolumn{3}{m{2.41366in}}{{\textquoteleft}The lamp is bright{\textquoteright}}\\
\end{supertabular}
\end{flushleft}
\begin{flushleft}
\tablehead{}
\begin{supertabular}{m{0.5247598in}m{0.7670598in}m{1.1712599in}m{0.66845983in}m{0.32055986in}m{0.41435984in}m{0.44275984in}}
b.  &
\itshape Laampuru &
\itshape ma-ilo. &
 &
 &
 &
\\
 &
lamp &
\textstyleGloss{appl}{}- be.bright &
 &
 &
 &
\\
 &
\multicolumn{6}{m{4.1781597in}}{{\textquoteleft}The lamp is brighter / turned up{\textquoteright}}\\
\end{supertabular}
\end{flushleft}
\ \ Tense inflections are often lacking on verbs in AP languages, and inflections for aspect and mood remain rather limited. The languages show very little similarity in Tense-Aspect-Mood inflections: not only are the forms different, but the values they express, and the position the morphemes take with respect to the verbal stem show much variation. For example, (34) shows that aspect in Western Pantar is prefixing, while in Kaera and Kamang it is suffixing. Also, morphemes with overlapping values have very different shapes: compare the perfective of Kaera \textit{{}--i }with Kamang -\textit{ma }and imperfective Kaera -\textit{(i)t }with Kamang \textit{{}-si}. 

(34)  \ \ Western  Pantar: \ \ \ \ i-\ \ Progressive

\ \ \ \ \ \ \ \ \ \ a-\ \ Inceptive

\ \ Kaera: \ \ \ \ \ \ \ \ {}-it, -t\ \ Imperfective

\ \ \ \ \ \ \ \ \ \ {}-i\ \ Perfective

\ \ \ \ \ \ \ \ \ \ {}-ang\ \ Continuative

\ \ Kamang: \ \ \ \ \ \ {}-si\ \ Imperfective

\ \ \ \ \ \ \ \ \ \ {}-ma\ \ Perfective

\ \ \ \ \ \ \ \ \ \ {}-ta \ \ Stative

\ \ In sum, overall, the morphological profile of languages in the AP family is simple compared to Papuan standards. The only affixes that have been reconstructed for proto-AP at this stage, are a paradigm of person prefixes on verbs and a third person possessive prefix on nouns (Holton and Robinson, \textit{this volume} a: Appendix). 

\subsection[5.9 Typological features of AP languages in the Papuan context ]{5.9 Typological features of AP languages in the Papuan context }
Several proposals have been made to characterize the typological profile of Papuan languages. Table 10 presents a list of typological features that have been mentioned most in the literature as typical for Papuan languages (see Foley 1986, Foley 2000, Pawley 2005, Aikhenvald and Stebbins 2007, Klamer et al. 2008, Klamer and Ewing 2010). In the right-most column, I indicate whether or not a feature applies to the AP languages. 

{\centering
Table 10. Structural features in {\textquotedblleft}Papuan{\textquotedblright} and in AP languages
\par}

\begin{flushleft}
\tablehead{}
\begin{supertabular}{m{0.9038598in}m{4.01636in}m{1.4156599in}}
 &
Typical for Papuan languages &
In AP languages?\\
Phonology &
No distinction between r/l &
no \\
Morphology &
Marking of gender &
no\\
 &
Subject marked as suffix on verb &
no\\
 &
Inclusive/exclusive distinction is absent in the pronominal paradigm &
no\\
 &
Morphological distinction between alienable-inalienable nouns  &
yes\\
Syntax &
Object-Verb &
yes\\
 &
Subject-Verb &
yes\\
 &
Postpositions &
yes\\
 &
Gen-Noun  &
yes\\
 &
Clause-final negators &
yes\\
 &
Clause-final conjunctions  &
yes\\
 &
Clause-chaining, switch reference, medial vs.  final verbs &
no \\
 &
Serial verb constructions &
yes\\
\end{supertabular}
\end{flushleft}
\ \ Table 10 clearly suggests that the syntactic typology of AP languages is much like that of other Papuan languages: object-verb order and preposed possessors (Gen-Noun) predominate, and negators and conjunctions are clause final, or at least follow the predicate. A formal distinction between alienable and inalienable possession is made in all languages. Serial verb constructions are found across the group. 

\ \ AP languages are different from other Papuan languages in that they do not exhibit clause-chaining, do not have switch reference systems, never suffix subjects to verbs and generally do not make a formal distinction between medial and final verbs. Gender is not marked in AP languages. Unlike many other Papuan groups, the AP languages do encode clusivity in their pronominal systems and do have a phonemic r-l distinction. 

\ \ All this goes to suggest that the typology of Papuan languages is more diverse than has previously been recognized. Indeed, apart from a broadly similar head-final syntactic profile, there is very little else that the AP languages share with Papuan languages spoken in other regions (see also Holton and Robinson, \textit{this volume} b).  

\clearpage{\bfseries
\ \ Lexicon}

In this section I summarize some of the lexical features that are defining or typical for the AP language group. 

\subsection[6.1 \ \ Cognates and reconstructed vocabulary ]{6.1 \ \ Cognates and reconstructed vocabulary }
Well over a hundred words have been reconstructed for proto-AP (pAP). They are listed in Holton and Robinson (\textit{this volume} a, Appendix). Complementing these data are the additional and revised pAP reconstructions presented in Schapper et al. (\textit{this volume}, Appendix I), which are based on lexical data from a larger language sample. As the focus of that chapter is to investigate the relation between the AP languages and those of Timor and Kisar, it presents reconstructions for proto-Timor (pTIM) as well as 89 pAP and pTIM cognates and forms reconstructed for pTAP (Schapper et al, \textit{this volume}, Appendix II and III).

\subsection[6.2 \ \ Numerals and numeral systems]{6.2 \ \ Numerals and numeral systems}
The indigenous numerals of the AP languages, as well as the indigenous structures for arithmetic operations are currently under pressure from Indonesian, and will inevitably be replaced with Indonesian forms and structures. Future generations may thus be interested in a documentary record of the forms and patterns currently used for cardinal, ordinal and distributive numerals, and the expressions of arithmetic operations (Klamer et al., \textit{this volume}). 

\ \ The numeral system reconstructed for pAP mixes numeral words that have a quinary and a decimal base, as shown in Table 11. That is, numeral {\textquoteleft}5{\textquoteright} is a monomorphemic form, the numeral {\textquoteleft}7{\textquoteright} is expressed with (reflexes of) morphemes for [5 2], {\textquoteleft}8{\textquoteright} as  [5 3], {\textquoteleft}9{\textquoteright} as [5 4], while {\textquoteleft}10{\textquoteright} is [10 1] (Schapper and Klamer, \textit{this volume}). Systems with numeral bases other than 10 such as the one reconstructed for pAP are relatively rare in the world{\textquoteright}s languages. From a typological point of view, the reconstructed form for the numeral {\textquoteleft}6{\textquoteright} is even more interesting, as it is not composed as [5 1], as expected in a quinary system, but is rather a mono-morphemic form:

{\centering
Table 11. Numerals and numeral system of pAP (Schapper and Klamer, \textit{this volume})
\par}

\begin{flushleft}
\tablehead{}
\begin{supertabular}{m{0.48445985in}m{0.59415984in}m{0.6538598in}m{0.6400598in}m{0.73095983in}m{0.59415984in}m{0.43725982in}m{0.43725982in}m{0.43725982in}m{0.5184598in}}
{\textquoteleft}1{\textquoteright} &
{\textquoteleft}2{\textquoteright} &
{\textquoteleft}3{\textquoteright} &
{\textquoteleft}4{\textquoteright} &
{\textquoteleft}5{\textquoteright} &
{\textquoteleft}6{\textquoteright} &
{\textquoteleft}7{\textquoteright} &
{\textquoteleft}8{\textquoteright} &
{\textquoteleft}9{\textquoteright} &
{\textquoteleft}10{\textquoteright}\\
*nuk 

[1] &
*araqu 

[2] &
*(a)tiga 

[3] &
*buta 

[4] &
*yiwesin 

[5] &
*talam 

[6] &
[5 2] &
[5 3] &
[5 4] &
[10 1]\\
\end{supertabular}
\end{flushleft}
\ \ Reflexes of the pAP numeral system are found across Alor and Pantar.  In the region of the Straits between both islands, the languages underwent a separate later development, innovating some forms, as well as introducing a subtractive pattern, representing {\textquoteleft}9{\textquoteright} as [[10] -1] and {\textquoteleft} 8{\textquoteright} as [[10] -2] (Schapper and Klamer, \textit{this volume}). In contrast with the AP languages, the Papuan languages spoken in Timor all have decimal systems. They have also borrowed forms from Austronesian; examples include Makalero {\textquoteleft}4{\textquoteright}, {\textquoteleft}5{\textquoteright}, {\textquoteleft}7{\textquoteright}, {\textquoteleft}9{\textquoteright} (Huber 2011) and Bunaq {\textquoteleft}7{\textquoteright}, {\textquoteleft}8{\textquoteright}, and {\textquoteleft}9{\textquoteright} (Schapper 2010). 

\subsection[6.3 \ \ Numeral classifiers]{6.3 \ \ Numeral classifiers}
Numeral classifiers are found in numeral NPs throughout the AP family. From a Papuan point of view, this is remarkable, as few Papuan languages have classifiers. In AP languages, the classifier always follows the noun and precedes the numeral: \textsc{[Noun - classifier - Numeral]. A}n illustration with the Teiwa general classifier \textit{bag }{\textquoteleft}\textsc{clf} {\textquoteleft} is (35):

Teiwa (Klamer 2013)

(35) \ \ \textit{Qarbau \ \ bag \ \ ut \ \ ga{\textquoteright}an \ \ u}

\ \ water.buffalo\ \ \textsc{clf}\ \ four\ \ \textsc{dem\ \ dist}

\ \ {\textquoteleft}Those four water buffalos{\textquoteright}\ \ \ \ \ \ \ \ \ \ \ \ 

Some of the AP languages have cognate numeral classifiers. For instance, Western Pantar \textit{waya} and Adang \textit{beh} both originate from a noun meaning {\textquoteleft}leaf{\textquoteright}. However, across the AP languages, the classifiers differ significantly in lexeme form as well in their classifying function so that no classifier can be reconstructed for proto-AP. A number of AP languages have a {\textquoteleft}general{\textquoteright} classifier, which functions to classify nouns outside the semantic domains of other classifiers that are semantically more specific (c.f. Zubin and Shimojo 1993). Illustrations are the general classifiers Teiwa \textit{bag }(derived from a lexeme meaning {\textquoteleft}seed{\textquoteright}, Klamer 2013, to appear),\textit{ }and Adang \textit{pa[241?], }derived from a lexeme meaning {\textquoteleft}non-round fruit{\textquoteright} (Robinson and Haan, to appear). Although these {\textquoteleft}general{\textquoteright} classifiers share a common, general 
classifying function, they derive from different lexical sources.

\ \ Individual languages also differ in the number of classifiers they use. For instance, Adang has 14 classifiers, while Kamang has only 2.  In addition, each of the languages uses its classifiers to carve out semantic domains of a quite different nature. By way of illustration, consider the way in which the semantic category of fruits and animals are classified. In Teiwa, fruits are classed according to their shape, while in Adang, fruits are classified together with animals and people (Robinson and Haan, to appear) while Western Pantar classifies fruits with {\textquotedblleft}contents{\textquotedblright} (\textit{hissa}) (Holton, to appear a). On the other hand, Klon (Baird 2008) and Kamang (Schapper, to appear) do not use a classifier with fruits at all. Animals are classed with fruits and humans in Adang, but with inanimate (!) objects in Abui. 

\ \ In sum, numeral classifiers appear to have developed after pAP split up, as no classifier is reconstructable for proto-AP. This is not a surprising finding, as numeral classifier sets are often highly volatile, and typically develop out of other lexical classes, such as nouns. A {\textquoteleft}spontaneous{\textquoteright} innovative development of sets of numeral classifiers is however unusual for a Papuan group, as Papuan languages generally lack classifiers (Aikhenvald 2000:123; Klamer 2013).\footnote{Numeral classifiers are absent from the overviews of Papuan features by Foley (1986, 2000) and Aikhenvald \& Stebbins (2007). Aikhenvald (2000:123)  mentions ten Papuan languages with classifiers in scattered locations of Papua New Guinea: Iwam, Abau (East Sepik province), Chambri, Wogamusin, Chenapian (Lower Sepik), Angave, Tanae (Gulf Province), Folopa (Highlands), Wantoat, Awar\'a (Morobe province).} Indeed, classifiers do not occur in any areal and/or genealogical cluster of Papuan languages, \textit{
except} for three areas located in eastern Indonesia where contact between Austronesian and Papuan languages has been long term and intense: Timor-Alor-Pantar,  Halmahera and the Bird{\textquoteright}s Head of Papua.\footnote{See Holton (to appear), Klamer (2013, to appear) and references cited there.} On the other hand, classifiers are typically found in Austronesian languages, and the Austronesian languages spoken in eastern Indonesia almost universally have them (Klamer 2013, to appear a). I thus hypothesise that long term Austronesian-Papuan contact has resulted in the diffusion of a numeral classification system into AP languages. In addition, it is likely that recent and intensive contact with Indonesian/Malay (Austronesian) has spiraled the development of the {\textquoteleft}general{\textquoteright} classifier type in a good number of Alor Pantar languages as functional copies of the Indonesian general classifier \textit{buah. }While Indonesian classifier \textit{buah }is derived from a noun meaning {\
textquoteleft}fruit{\textquoteright}, it has almost lost any semantic content, and it functions as a general classifier today (Hopper 1986, Chung 2010). 

\ \ Note that the contact did not involve borrowing of lexemes: no similarity in shape or semantics exists between classifiers in any Alor Pantar language and any known classifiers of Austronesian languages spoken in the region, nor with classifiers of  Indonesian/Malay. In particular, reflexes of the reconstructed proto-Malayo-Polynesian classifier *buaq, which are attested throughout the Austronesian family, are not found in the AP languages. Neither has the grammatical structure of Austronesian numeral NPs been copied: in Austronesian NPs, the classifier follows the numeral while the position of the noun varies, thus we find [\textsc{Numeral-Classifier-Noun}] (as in Indonesian \textit{dua buah rumah }{\textquoteleft}two CLF house{\textquoteright}, {\textquoteleft}two houses{\textquoteright}) but also \textsc{[Noun-Numeral- classifier]} (as in colloquial Malay \textit{rumah dua buah} {\textquoteleft}house two CLF{\textquoteright}; Blust 2009:283-284). In contrast, in AP languages, the classifier always 
precedes the numeral: \textsc{[Noun - classifier - Numeral] (}as in (30) above).

\ \ The AP classifiers thus represent neither borrowed forms nor borrowed structures. What speakers may have adopted from Austronesian, however, is the propensity to reanalyze lexemes which they already had at their disposal (such as {\textquoteleft}seed{\textquoteright} or {\textquoteleft}fruit{\textquoteright}) and to grammaticalise these as sortal classifiers in numeral expressions. 

\subsection[6.4 \ \ Kinship terminology]{6.4 \ \ Kinship terminology}
Kinship terms vary between languages according to ancestor-descendant relationships: the more closely related two languages are, the more likely they are to share cognate forms, and the more likely it is that the meanings of the terms coincide. But as a social construct, kinship practice may be influenced by contact, with concomitant changes in the shape or meaning of the kin terms. The Alor Pantar languages show enormous variation in kinship terminology and practice, in spite of the fact that many of the communities are closely bound together through ties of marriage alliance (Holton, \textit{this volume}). The westernmost languages distinguish both maternal and paternal cross-cousins (children of opposite-sex siblings) as ideal marriage partners, while at the opposite extreme in the highlands of Alor are found languages which expressly forbid cross-cousin marriage. Holton (\textit{this volume}) suggests that the current distribution of kinship terminologies suggests a recent drift toward symmetric exchange 
systems which distinguish both maternal and paternal cross-cousins, perhaps under the influence of neighboring Austronesian languages. 

{\bfseries
\ \ Summary and challenges for future research}

The current volume provides information on the linguistic history of the AP languages. 

[xxxxx]

[xxxx]

Apart from the from macro-level observations about general migration patterns, to date, there is no integrated account of the history of the Alor Pantar region. To reveal more of the culture history of the speakers, we need more fine-grained bottom up research of targeted parts of the region, where linguistic research is combined with ethnography, archaeology, geography and musicology. 

{\bfseries
8. \ \ Data sets and archiving: word lists, corpora, field notes}

All authors contributing to this volume have endeavoured to make the empirical basis on which their investigations rest as explicit as possible. To this end, most of the chapters include a section referred to as {\textquoteleft}Sources{\textquoteright}, which lists the various sources (both published and unpublished) that were used for the chapter, their authors/collectors and the year(s) of collection. Roughly three types of data sets have been used: word lists, corpora, and field notes. 

\ \ The word lists used for the chapters on the history of AP languages are part of a lexical database referred to as the Alor Pantar (AP) Lexical Database, an Excel sheet with 400+ words, containing lexical survey data from Nedebang, Western Pantar (Tubbe), Deing, Sar (Adiabang), Kaera (Padangsul), Blagar (Warsalelang), Blagar (Nule), Kabola, Adang (Lawahing), Hamap, Klon, Kafoa, Abui (Atangmelang), Kamang, Kula, Kui, Sawila, Wersing. These word lists were collected between 2003-2011 by the following researchers: Louise Baird, Franti\v{s}ek Kratochv\'il, Gary Holton, Laura Robinson, Antoinette Schapper, Nick Williams, and Marian Klamer. Parts of this (slightly modified)\textstylefootnotereference{ }\footnote{Modifications of the original survey lists include the correction of typo{\textquoteright}s, and, where phonological forms are known, these have replaced the original phonetic forms. The Blagar and Adang data were replaced with data from different dialects: Adang (Pitungbang/Kokar dialect) and Blagar (
Dolabang dialect).} lexical database were used in Holton and Robinson (\textit{this volume} a, b), and Schapper et al. (\textit{this volume}). Where a more extensive lexicon of a language is available, that lexicon was used instead of the lexical survey lists. Thus, the lexical data from Teiwa, Kaera, Western Pantar, Blagar, Adang, Klon, Abui, and Sawila are from Toolbox files mutually shared among the researchers, and a published dictionary was the source for Kamang (Schapper and Manimau 2011).  

\ \ The chapters on the typology of AP languages use corpora as data sets, and build on information collected by researchers in the field ({\textquoteleft}fieldnotes{\textquoteright}). A corpus of an AP language as it is used in this volume typically consists of a Toolbox file containing various spoken texts, which have been transcribed, glossed and translated by the researcher working on the language. Corpus data are cited in the text with reference to the author and the language, and refer to the following sources: 

(36)\ \ Baird, Louise. \textit{Klon corpus}. Leiden University.

Holton, Gary. \textit{Western Pantar corpus}. University of Alaska at Fairbanks.

\ \ Klamer, Marian. \textit{Teiwa corpus}. Leiden University.

Klamer, Marian. \textit{Kaera corpus}. Leiden University.

\ \ Kratochv\'il, Franti\v{s}ek. \textit{Abui corpus}. Nanyang Technological University Singapore.

Kratochv\'il, Franti\v{s}ek. \textit{Sawila corpus}. Nanyang Technological University Singapore.\footnote{The Sawila corpus includes work by the SIL-team members Anderias Malaikosa and Isak Bantara who translated Genesis, the Gospel of Mark, and the Acts into Sawila. }

Schapper, Antoinette. \textit{Kamang corpus}. Leiden University/University of Cologne.

Schapper, Antoinette and Rachel Hendery. \textit{Wersing corpus}. Australian National University.

Sizes of corpora vary from over 100,000 words (Abui: 120,000 words, Sawila: 108,000 words), to less than 50,000 words (Klon: 36,000 words, Teiwa: 26,000 words) to less than 20,000 words (Kaera: 19,000 words). Contents of corpora vary too: some include oral texts as well as elicited sentences collected during the research; in other cases, a researcher may have kept elicited information separate from the oral text corpus.

\ \ The data sets mentioned in this section are currently being archived as part of  {\textquotedblleft}The LAISEANG corpora at The Language Archive{\textquotedblright}. They will be accessible in the near future at \href{https://webmail.campus.leidenuniv.nl/owa/redir.aspx?C=VcF3tDaLq0iIlRczzT0ZXw_xD4uU3tAIZUYVDzSuO7QzFpu5wi759CTWognlXJNmDHNv_KzsnWc.&URL=http://hdl.handle.net/1839/00-0000-0000-0018-CB72-4@view}{http://hdl.handle.net/1839/00-0000-0000-0018-CB72-4@view}. 

{\bfseries
9.\ \ Acknowledgements relating to the volume }

The research presented in this volume was done as part of the research project {\textquoteleft}Alor-Pantar languages: origin and theoretical impact{\textquoteright}, funded through the EuroCORES programme {\textquoteleft}Better Analyses Based on Endangered Languages{\textquoteright} (BABEL) of the European Science Foundation. The project website is: www.alor-pantar.org. Brown, Corbett and Fedden were funded by the Arts and Humanities Research Council (UK) under grant AH/H500251/1; since April 2013 Fedden and Brown were funded by the Arts and Humanities Research Council (UK) under grant AH/K003194/1. Robinson was funded by the National Science Foundation (US) under BCS Grant No. 0936887. Schapper was funded by a grant from the Netherlands Organisation for Scientific Research (NWO) 2009-2012. 

\ \ Chapters in this volume have been reviewed single-blind, by external and internal reviewers. The following colleagues reviewed or commented on chapters of this volume (in alphabetical order): Dunstan Brown, Niclas Burenhult, Mary Darlymple, Bethwyn Evans, Sebastian Fedden, Bill Foley, Jim Fox, Martin Haspelmath, Gary Holton, Andy Pawley, Laura Robinson, Hein Steinhauer, and Peter de Swart.

{\bfseries
References}

Aikhenvald, Alexandra Y. 2000. Classifiers: a typology of noun categorization devices. Oxford: Oxford University Press. 

Aikhenvald, Alexandra Y. and Tonya Stebbins. 2007. Languages of New Guinea. The Vanishing Voices of the Pacific Rim, ed. by O. Miyaoka, O. Sakiyama and M.E. Krauss, 239-66. Oxford: Oxford University Press.

Anonymous. 1914. De eilanden Alor en Pantar, Residentie Timor en Onderhoorigheden. Tijdschrift van het Koninklijk Nederlandsch Aardrijkskundig Genootschap 31. 70--102.

Baird, Louise, Marian Klamer, and Frantisek Kratochv\'il, ms. 

Baird, Louise. 2005. Doing the split-S in Klon, in: J. Doetjes and J. van der Weijer (eds), Linguistics in the Netherlands 22, 1-12. Amsterdam: John Benjamins. 

Baird, Louise. 2008. A grammar of Klon: A non-Austronesian language of Alor, Indonesia. Canberra: Pacific Linguistics.

Baird, Louise. 2010. Grammaticalisation of asymmetrical serial verb constructions in Klon, in: Michael Ewing and Marian Klamer (eds), Typological and areal analyses: Contributions from East Nusantara, 189-206. Canberra: Pacific Linguistics.

Baird, Louise. To appear. Kafoa. Papuan Languages of Timor-Alor-Pantar: Sketch grammars, ed. by A. Schapper. Canberra: Pacific Linguistics.

Baird, Louise, Marian Klamer and Frantisek Kratochv\'il. Ms. xxxxx Alor Malay paper

Barnes, Robert H. 1982. The Majapahit Dependency Galiyao, Bijdragen tot de Taal-, Land- en Volkenkunde 138:407-412.

Barnes, Robert H. 1996. Sea Hunters of Indonesia. Fishers and Weavers of Lamalera. Oxford: Clarendon Press.

Barnes, Robert H. 2001. Alliance and warfare in an Eastern Indonesian principality - K\'edang in the last half of the nineteenth century Bijdragen tot de Taal-, Land- en Volkenkunde 157(2):271-311.

Bellwood, Peter. 1997. The prehistory of the Indo-Pacific archipelago. 2nd edition. Hololulu: University of Hawaii Press.

Blust, Robert. 2009. The Austronesian languages. Canberra: Pacific Linguistics.

Chung, Siaw-Fong. 2010. Numeral classifier buah in Malay: A corpus-based study. Language and Linguistics 11.3: 553-577. 

Coolhaas, W. Ph. 1979. Generale missiven van gouverneurs-generaal en raden aan Heren XVII der Verenigde Oostindische Compagnie, Volume VII. s Gravenhage: Martinus Nijhoff.

de Josselin de Jong, J.P.B. 1937. Studies in Indonesian culture I: Oirata, a Timorese settlement on Kisar. Amsterdam.

Dietrich, Stefan. 1984. A Note on Galiyao and the Early History of the Solor-Alor Islands, Bijdragen tot de Taal-, Land- en Volkenkunde 140:317-326.

Donohue, Mark. 1996. Inverse in Tanglapui. Language and Linguistics in Melanesia 27(2).101-18.

Du Bois, Cora. 1960 [1944]. The People of Alor. New York: Harper Torchbooks.

Engelenhoven A.Th.P.G. van (2009), On derivational processes in Fataluku, a non-Austronesian language in East-Timor. In: Wetzels W.L. (Ed.) The Linguistics of Endangered Languages, Contributions to Morphology and Morpho-Syntax. Utrecht: Netherlands Graduate School of Linguistics. 331-362.

Engelenhoven A.Th.P.G. van (2010), Verb Serialisation in Fataluku. The case of take. In: Azeb A, V\"ollmin S, Rapold Ch., Zaug-Coretti S (Eds.) Converbs, Medial Verbs, Clause Chaining and Related Issues. K\"oln: R\"udiger K\"oppe Verlag. 185-211.

Fedden, Sebastian, Dunstan Brown, Greville Corbett, Gary Holton, Marian Klamer, Laura C. Robinson and Antoinette Schapper. 2013. Conditions on pronominal marking in the Alor-Pantar languages. Linguistic xxxx

Foley, William A. 1986. The Papuan Languages of New Guinea. Cambridge: Cambridge University Press.

Foley, William A. 2000. The languages of New Guinea. Annual Review of Anthropology.357-404.

Gaalen, G. A. M. van 1945. Memorie van Overgave van den Fundgeerend Controleur van Alor G.A.M. van Gaalen. Manuscript.

Grimes, Charles, Tom Therik, Barbara Dix Grimes, and Max Jacob. 1997. A guide to the people and languages of Nusa Tenggara. Kupang: Artha Wacana Press. 

Haan, Johnson. 2001. The grammar of Adang, a Papuan language spoken on the island of Alor, East Nusa Tenggara, Indonesia. PhD thesis, University of Sydney.

H\"agerdal, Hans. 2010a. Cannibals and pedlars. Indonesia and the Malay World 38, 111, 217--246.

H\"agerdal, Hans. 2010b. Van Galens memorandum on the Alor Islands in 1946. An annotated translation with an introduction. Part 1, HumaNetten 25: 14-44.

H\"agerdal, Hans. 2011. Van Galens memorandum on the Alor Islands in 1946. An annotated translation with an introduction. Part 2, HumaNetten 27: 53-96.

H\"agerdal, Hans. 2012. Lords of the Land, Lords of the Sea: Conflict and adaptation in early colonial Timor, 1600-1800. Leiden: KITLV Press.

Hajek, John. 2010. Towards a phonological overview of the vowel and consonant systems of East Nusantara. Typological and Areal Analyses: Contributions from East Nusantara, ed. by M. Ewing and M. Klamer, 25-46. Canberra: Pacific Linguistics.

Holton, Gary. 2008. The rise and fall of semantic alignment in North Halmahera, Indonesia. The Typology of Semantic Alignment, ed. by M. Donohue and S. Wichmann, 252-76. Oxford: Oxford University Press.

Holton, Gary. 2010a. Person-marking, verb classes, and the notion of grammatical alignment in Western Pantar (Lamma), in: Michael Ewing and Marian Klamer (eds), Typological and areal analyses: Contributions from East Nusantara, 101-121. Canberra: Pacific Linguistics.

Holton, Gary. 2010b. An etymology for Galiyao. Unpublished ms., University of Fairbanks. 

Holton, Gary. 2011. Landscape in Western Pantar, a Papuan outlier of southern Indonesia, in: David M. Mark, Andrew G. Turk, Niclas Burenhult and David Stea (eds), Landscape in Language (Culture and Language Use: Studies in Anthropological Linguistics), 143--166. Amsterdam: John Benjamins.

Holton, Gary. To appear a. Numeral classifiers and number in two Papuan outliers of East Nusantara. In Klamer, Marian and Frantisek Kratochvil (eds.), Number in East Nusantara. Proceedings of the panel at the International Conference of Austronesian Languages, Bali, 2012.

Holton, Gary. To apppear b. Western Pantar. Papuan Languages of Timor-Alor-Pantar: Sketch grammars, ed. by A. Schapper. Canberra: Pacific Linguistics.

Holton, Gary and Marian Klamer. To appear. The Papuan languages of East Nusantara. In Oceania, ed. by Bill Palmer. Berlin-New York: Mouton de Gruyter. 

Holton, Gary, and Mahalalel Lamma Koly. 2008. Kamus pengantar Bahasa Pantar Barat (Companion dictionary of Western Pantar). Kupang, Indonesia: UBB-GMIT.

Holton, Gary, Marian Klamer, Franti\v{s}ek Kratochv\'il, Laura C. Robinson and Antoinette Schapper. 2012. The historical relation of the Papuan languages of Alor and Pantar, Oceanic Linguistics 5(11): 86-122. 

Hopper, Paul J. 1986. Some discourse functions of classifiers in Malay. In Noun Classes and Categorization ed. Colette G. Craig, 309-325. Amsterdam: John Benjamins. 

Huber, Juliette. 2008. First steps towards a grammar of Makasae. M\"unchen: Lincom.

Huber, Juliette. 2011. A grammar of Makalero: A Papuan language of East Timor. Utrecht: LOT.

Jacob, June and Chareles E. Grimes.  2003.  Kamus pengantar Bahasa Kupang--Bahasa Indonesia (dengan daftar Indonesia--Kupang). Kupang: Artha Wacana Press.  

Klamer, Marian and Antoinette Schapper. 2012. The development of give constructions in the Papuan languages of Timor-Alor-Pantar. Linguistic Discovery 10.3: 174-207. 

Klamer, Marian and Franti\v{s}ek Kratochv\'il. 2006. The role of animacy in Teiwa and Abui (Papuan), Proceedings of BLS 32. Berkeley: Berkeley Linguistic Society.

Klamer, Marian and Franti\v{s}ek Kratochv\'il. 2010. Abui Tripartite Verbs: Exploring the limits of compositionality, in: Jan Wohlgemuth and Michael Cysouw (eds), Rara and rarissima: Documenting the fringes of linguistic diversity, 185-210. Berlin/New York: Mouton de Gruyter. 

Klamer, Marian and Michael Ewing. 2010. The languages of East Nusantara: an introduction. In East Nusantara: Typological and Areal Analyses ed. Michael Ewing and Marian Klamer, 1-24. Canberra: Pacific Linguistics.

Klamer, Marian, Ger P. Reesink and Miriam van Staden. 2008. East Nusantara as a linguistic area. In From Linguistic Areas to Areal Linguistics, ed. by Pieter Muysken. Amsterdam: John Benjamins, 95-150. 

Klamer, Marian. 2010. A grammar of Teiwa. Berlin/New York: Mouton de Gruyter.

Klamer, Marian. 2011. A short grammar of Alorese (Austronesian). M\"unchen: Lincom.

Klamer, Marian. 2012. Papuan-Austronesian language contact: Alorese from an areal perspective. In Melanesian Languages on the Edge of Asia: Challenges for the 21st Century, Special issue of Language Documentation and Conservation, ed. by Nicholas Evans and Marian Klamer, 72-108. 

Klamer, Marian. 2013. The history of numeral classifiers in Teiwa (Papuan). In Gerrit J. Dimmendaal and Anne Storch (eds.) Number: Constructions and Semantics. Case studies from Africa, India, Amazonia \& Oceania. Amsterdam: Benjamins.

Klamer, Marian. To appear a. Numeral classifiers in Papuan Alor Pantar: a comparative perspective. In Klamer, Marian and Franti\v{s}ek Kratochv\'il (eds.), Number in East Nusantara. Proceedings of the panel at the International Conference of Austronesian Languages, Bali, 2012. 

Klamer, Marian. To appear b. Kaera. Papuan Languages of Timor-Alor-Pantar: Sketch grammars, ed. by A. Schapper. Canberra: Pacific Linguistics.

Kratochv\'il, Franti\v{s}ek. 2007. A grammar of Abui: A Papuan language of Alor. Utrecht: LOT Publications.

Kratochv\'il, Franti\v{s}ek. 2011a. Discourse-structuring functions of Abui demonstratives, in: Foong Ha Yap and Janick Wrona (eds), Nominalization in Asian languages: Diachronic and typological perspectives, Volume 2: Korean, Japanese and Austronesian Languages, 761-792. Amsterdam: John Benjamins.

Kratochv\'il, Franti\v{s}ek. 2011b. Transitivity in Abui, Studies in Language 35(3): 588-635.

Kratochv\'il, Franti\v{s}ek. To appear. Sawila. Papuan Languages of Timor-Alor-Pantar: Sketch grammars, ed. by A. Schapper. Canberra: Pacific Linguistics.

Kratochv\'il, Franti\v{s}ek and Benidiktus Delpada. 2008. Kamus pengantar Bahasa Abui (Abui-Indonesian-English dictionary). Kupang, Indonesia: UBB-GMIT.

Lansing, Stephen J., Murray P. Cox, Therese A. de Vet, Sean S. Downey, Brian Hallmark, Herawati Sudoyo. 2011. An ongoing Austronesian expansion in Island Southeast Asia. Journal of Anthropological Archaeology 30, 262-272.

Le Roux, C.C.F.M. 1929. De Elcanos tocht door den Timorarchipel met Magalh\~aes schip Victoria. In Feestbundel, uitgegeven door het Koninklijk Bataviaasch Genootschap van Kunsten en Wetenschappen bij gelegenheid van zijn 150 jarig bestaan:1778-1928, 1-99. Weltevreden: Kolff.

Lemoine, Annie. 1969. Histoires de Pantar LHomme 9(4): 5-32.

Lewis, M. Paul. 2009. Ethnologue: Languages of the World, 16th edition. Dallas: SIL International.

Lynden, D.W. C Baron van. 1851. Bijdrage tot de kennis van Solor, Alor, Rotti, Savoe en Omliggende Eilanden, getrokken uit een verslag van de residentie Timor, opgemaakt door Mr. D.W.C. Baron van Linden, Resident van Timor. Natuurkundig tijdschrift voor Nederlandsch-Indi\"e 2. 317--336.

Maddieson, Ian. 2005. Consonant inventories. The World Atlas of Language Structures, ed. by M. Haspelmath, M.S. Dryer, D. Gil and B. Comrie. Oxford: Oxford University Press.

Mahirta. 2006. The prehistory of Austronesian dispersal to the southern islands of eastern Indonesia.In Austronesian diaspora and the ethnogeneses of people in Indonesian archipelago, ed. by Truman Simanjuntak, Ingrid H.E. Pojoh and Mohammad Hisyam. Jakarta: LIPI Press, 129-145.

Malchukov, Andrej, Martin Haspelmath and Bernard Comrie. 2010. Studies in Ditransitive Constructions: A comparative handbook. Berlin: Mouton de Gruyter.

Martis, Non, Wati Kurnatiawati, Buha Aritonang, Hidayatul Astar and Ferr Feirizal. 2000. Monografi kosakata dasar Swadesh di Kabupaten Alor. Jakarta: Pusat Bahasa, Departemen Pendidikan Nasional.

McWilliam, Andrew. 2007.  Austronesians in linguistic disguise: Fataluku cultural fusion in East Timor.  Journal of Southeast Asian Studies, 38, 2: 355-375.

Nicolspeyer, Martha Margaretha. 1940. De Sociale Structuur van een Aloreesche Bevolkingsgroep. Rijswijk: Kramers. 

Nieuw 7(3). 67--88.

Nieuwenkamp, W.O.J. 1922. Drie weken op Alor. Nederlands-Indi\"e Oud en

Nitbani, Semuel H., Jeladu Kosmas, Sisila Wona \& Hilda Naley. 2001. Struktur Bahasa Lamma. Jakarta: Pusat Bahasa, Departemen Pendidikan Nasional.

OConnor, Sue. 2003. Nine new painted rock art sites from East Timor in the context of the Western Pacific region, Asian Perspectives 42(1): 96-128.

OConnor, Sue. 2007. New evidence from East Timor contributes to our understanding of earliest modern human colonisation east of the Sunda shelf, Antiquity 81: 523-535.

Pawley, Andrew K. 2005. The chequered career of the Trans New Guinea hypothesis. In Papuan Pasts: Cultural, linguistic and biological histories of Papuan-speaking peoples, ed. by Andrew Pawley, Robert Attenborough, Jack Golson and Robin Hide. Canberra: Pacific Linguistics, 67-107.

Robinson, Laura and Gary Holton. 2012. Internal classification of the Alor-Pantar language family using computational methods applied to the lexicon. Language Dynamics and Change 2, 2: 1-27.

Robinson, Laura and John W. Haan. To appear. Adang. Papuan Languages of Timor-Alor-Pantar: Sketch grammars, ed. by A. Schapper. Berlin: Mouton de Gruyter.

Robinson, Laura C. 2012. The Alor-Pantar (Papuan) languages and Austronesian contact in East Nusantara. Paper presented at the International Conference on Austronesian Linguistics. Denpasar, Bali, July 2-6.

Rodemeier, Susanne. 1995. Local tradition on Alor and Pantar: An attempt at localizing Galiyao, Bijdragen tot de Taal-, Land- en Volkenkunde 151(3): 438-442.

Rodemeier, Susanne. 2006. Tutu kadire in Pandai-Munaseli. Passauer Betr\"age zur S\"udostasienkunde Bd. 12. Berlin: LIT Verlag.

Roever, Arend de. 2002. De jacht op sandelhout : de VOC en de tweedeling van Timor in de zeventiende eeuw. PhD dissertation Leiden U. Zutphen: Walburg Pers. 

Ross, Malcolm. 2005. Pronouns as preliminary diagnostic for grouping Papuan languages, In Papuan Pasts: Cultural, linguistic and biological histories of Papuan-speaking peoples, ed. Andrew Pawley, Robert Attenborough, Jack Golson \& Robin Hide. Canberra: Pacific Linguistics, 15-65.

Schapper, Antoinette. 2010. Bunaq, a Papuan language of central Timor. PhD thesis, The Australian National University, Canberra.

Schapper, Antoinette. To appear. Kamang. In Antoinette Schapper (ed.), Papuan languages of Timor-Alor-Pantar: Sketch grammars. 

Siewierska 2005 WALS

Spriggs, Matthew. 2011. Archeology and the Austronesian expansion: where are we now? Antiquity 85: 510-528.

Steenbrink, Karel. 2003. Catholics in Indonesia 1808-1942. A documented history. Leiden: KITLV Press.

Steinhauer, Hein. 1977. Going and Coming in the Blagar of Dolap (Pura, Alor, Indonesia) NUSA Miscellaneous Studies in Indonesian and Languages in Indonesia III, Jakarta: 38-48.

Steinhauer, Hein. 1991. Demonstratives in the Blagar language of Dolap (Pura, Alor, Indonesia), in: Tom Dutton (ed), Papers in Papuan linguistics, 177-221. Canberra: Pacific Linguistics.

Steinhauer, Hein. 1993. Sisters and potential wives: where linguists and anthropologists meet:  notes on kinship in Blagar (Alor), in: P. Haenen (ed.), Vrienden en verwanten, Liber Amicorum Alex van der Leeden, Leiden/Jakarta, DSALCUL/IRIS: 147-168.

Steinhauer, Hein. 1995. Two varieties of the Blagar language (Alor, Indonesia), in: Connie Baak, Mary Bakker, and Dick van der Meij (eds), Tales from a concave world: Liber amicorum Bert Voorhoeve, 269-296. Leiden: Projects Division, Department of Languages and Cultures of South-East Asia and Oceania.

Steinhauer, Hein. 1999. Bahasa Blagar Selayang Pandang, in: Bambang Kaswanti Purwo (ed), Panorama bahasa Nusantara, 71-102. Jakarta: Universitas Cenderawasih and Summer Institute of Linguistics.

Steinhauer, Hein. 2010. Pura when we were younger than today, in: Artem Fedorchuk and Svetlana Chlenova (eds). Studia Anthropologica. A Festschrift in Honour of Michael Chlenov /\textcyrillic{{\CYRS}{\cyrb}{\cyro}{\cyrr}{\cyrn}{\cyri}{\cyrk} {\cyrs}{\cyrt}{\cyra}{\cyrt}{\cyre}{\cyrishrt} {\cyrv}{\cyrch}{\cyre}{\cyrs}{\cyrt}{\cyri} {\CYRM}.{\CYRF}.{\CYRCH}{\cyrl}{\cyre}{\cyrn}{\cyro}{\cyrv}{\cyra}, 261-283. {\CYRM}{\cyro}{\cyrs}{\cyrk}{\cyrv}{\cyra}/ {\CYRI}{\cyre}{\cyrr}{\cyru}{\cyrs}{\cyra}{\cyrl}{\cyri}{\cyrm}: {\CYRM}{\cyro}{\cyrs}{\cyrt}{\cyrery} {\CYRK}{\cyru}{\cyrl}{\cyrsftsn}{\cyrt}{\cyru}{\cyrr}{\cyrery}/ Gesharim.}

Steinhauer, Hein. 2012. Deictic categories in three languages of Eastern Indonesia, in: Bahren Umar Siregar, P. Ari Subagyo, Yassir Nasanius (eds). Dari menapak jejak kata sampai menyigi tata bahasa. Persembahan untuk Prof. Dr. Bambang Kaswanti Purwo dalam rangka ulang tahunnya yang ke-60, 115-147. Jakarta: Pusat Kajian Bahasa dan Budaya Universitas Katolik Indonesia Atma Jaya.

Steinhauer, Hein. To appear. Blagar. In Antoinette Schapper (ed.), Papuan languages of Timor-Alor-Pantar: Sketch grammars. 

Stokhof, W. A. L. 1975. Preliminary notes on the Alor and Pantar languages (East Indonesia). (Pacific Linguistics B-43). Canberra: Australian National University.

Stokhof, W. A. L. 1977. Woisika I: An Ethnographic Introduction. (Pacific Linguistics D-19). Canberra: Australian National University.

Stokhof, W. A. L. 1978. Woisika text, in: Miscellaneous Studies in Indonesian and Languages in Indonesia 5: 34-57. Jakarta: NUSA. 

Stokhof, W. A. L. 1983. Names and Naming in Ateita and environments (Woisika, Alor), Lingua 61 (2/3): 179-207.

Stokhof, W. A. L. 1987. A short Kabola text. A World of Language: Papers Presented to Prof. S.A. Wurm on his 65th Birthday, ed. by D.C. Laycock and W. Winter, 631-48. (Pacific Linguistics C-100). Canberra: Australian National University.

Stokhof, W.A.L.  1987. A short Kabola Text (Alor, East Indonesia), in:  Donald C. Laycock  and Werner Winter (eds), A World of Language: Papers presented to Professor Stephen A. Wurm on his 65th Birthday, 631-648. Canberra: Pacific Linguistics.

Stokhof, W.A.L. 1979. Woisika II: Phonemics. Canberra: Pacific Linguistics.

Stokhof, W.A.L. 1982. Woisika riddles. Canberra: Pacific Linguistics.

Stokhof, W.A.L. 1984. Annotations to a text in the Abui language (Alor). Bijdragen tot de Taal-, Land- en Volkenkunde 140: 106-162.

Summerhayes, Glenn R. 2007. Island Melanesian Pasts: A view from archeology. In Genes, language and culture history in the Southwest Pacific, ed. by Jonathan S. Friedlaender. Oxford: Oxford University Press, 10-35.

Verheijen, Jilis A.J. 1986. The Sama/Bajau Language in the Lesser Sunda Islands.~Verheijen, Jilis A. J. 1986. Australian National University.

Williams, Nick and Mark Donohue. forthcoming. Kula. Papuan Languages of Timor-Alor-Pantar: Sketch grammars, ed. by A. Schapper. Canberra: Pacific Linguistics.

Wurm, Stephen A., C.L. Voorhoeve and Kenneth A. McElhanon. 1975. The Trans-New Guinea Phylum in general. In New Guinea Area Languages and Language Study, vol. I, Papuan Languages and the New Guinea Linguistic Scene, ed. by S.A. Wurm, 299-322. (Pacific Linguistics C-38). Canberra: Australian National University.

Xu, Shuhua, Irina Pugach, Mark Stoneking, Manfred Kayser, Li Jin, and The HUGO Pan-Asian Consortium. 2012. Genetic dating indicates that the Asian-Papuan admixture through Eastern Indonesia corresponds to the Austronesian expansion. PNAS 109, 12, 4574-4579.

Zubin, David A. and Mitsuaki Shimojo. 1993. How general are General Classifiers? With special reference to ko and tsu in Japanese. Proceedings of the Nineteenth Annual  Meeting of the Berkeley Linguistics Society: General Session and Parasession on Semantic Typology and Semantic Universals, 490-502.

